% Telekom osCompendium extract template
%
% (c) Karsten Reincke, Deutsche Telekom AG, Darmstadt 2011
%
% This LaTeX-File is licensed under the Creative Commons Attribution-ShareAlike
% 3.0 Germany License (http://creativecommons.org/licenses/by-sa/3.0/de/): Feel
% free 'to share (to copy, distribute and transmit)' or 'to remix (to adapt)'
% it, if you '... distribute the resulting work under the same or similar
% license to this one' and if you respect how 'you must attribute the work in
% the manner specified by the author ...':
%
% In an internet based reuse please link the reused parts to www.telekom.com and
% mention the original authors and Deutsche Telekom AG in a suitable manner. In
% a paper-like reuse please insert a short hint to www.telekom.com and to the
% original authors and Deutsche Telekom AG into your preface. For normal
% quotations please use the scientific standard to cite.
%
% [ File structure derived from 'mind your Scholar Research Framework' 
%   mycsrf (c) K. Reincke CC BY 3.0  http://mycsrf.fodina.de/ ]

%
% select the document class
% S.26: [ 10pt|11pt|12pt onecolumn|twocolumn oneside|twoside notitlepage|titlepage final|draft
%         leqno fleqn openbib a4paper|a5paper|b5paper|letterpaper|legalpaper|executivepaper openrigth ]
% S.25: { article|report|book|letter ... }
%
% oder koma-skript S.10 + 16
\documentclass[DIV=calc,BCOR=5mm,11pt,headings=small,oneside,abstract=true, toc=bib]{scrartcl}

%%% (1) general configurations %%%
\usepackage[utf8]{inputenc}

%%% (2) language specific configurations %%%
\usepackage[]{a4,ngerman}
\usepackage[english, german, ngerman]{babel}
\selectlanguage{ngerman}

%language specific quoting signs
%default for language emglish is american style of quotes
\usepackage{csquotes}

% jurabib configuration
\usepackage[see]{jurabib}
\bibliographystyle{jurabib}
% Telekom osCompendium German Jurabib Configuration Include Module 
%
% (c) Karsten Reincke, Deutsche Telekom AG, Darmstadt 2011
%
% This LaTeX-File is licensed under the Creative Commons Attribution-ShareAlike
% 3.0 Germany License (http://creativecommons.org/licenses/by-sa/3.0/de/): Feel
% free 'to share (to copy, distribute and transmit)' or 'to remix (to adapt)'
% it, if you '... distribute the resulting work under the same or similar
% license to this one' and if you respect how 'you must attribute the work in
% the manner specified by the author ...':
%
% In an internet based reuse please link the reused parts to www.telekom.com and
% mention the original authors and Deutsche Telekom AG in a suitable manner. In
% a paper-like reuse please insert a short hint to www.telekom.com and to the
% original authors and Deutsche Telekom AG into your preface. For normal
% quotations please use the scientific standard to cite.
%
% [ File structure derived from 'mind your Scholar Research Framework' 
%   mycsrf (c) K. Reincke CC BY 3.0  http://mycsrf.fodina.de/ ]

% the first time cite with all data, later with shorttitle
\jurabibsetup{citefull=first}

%%% (1) author / editor list configuration
%\jurabibsetup{authorformat=and} % uses 'und' instead of 'u.'
% therefore define your own abbreviated conjunction: 
% an 'and before last author explicetly written conjunction

% for authors in citations
\renewcommand*{\jbbtasep}{ u. } % bta = between two authors sep
\renewcommand*{\jbbfsasep}{, } % bfsa = between first and second author sep
\renewcommand*{\jbbstasep}{ u. }% bsta = between second and third author sep
% for editors in citations
\renewcommand*{\jbbtesep}{ u. } % bta = between two authors sep
\renewcommand*{\jbbfsesep}{, } % bfsa = between first and second author sep
\renewcommand*{\jbbstesep}{ u. }% bsta = between second and third author sep

% for authors in literature list
\renewcommand*{\bibbtasep}{ u. } % bta = between two authors sep
\renewcommand*{\bibbfsasep}{, } % bfsa = between first and second author sep
\renewcommand*{\bibbstasep}{ u. }% bsta = between second and third author sep
% for editors  in literature list
\renewcommand*{\bibbtesep}{ u. } % bte = between two editors sep
\renewcommand*{\bibbfsesep}{, } % bfse = between first and second editor sep
\renewcommand*{\bibbstesep}{ u. }% bste = between second and third editor sep

% use: name, forname, forname lastname u. forname lastname
\jurabibsetup{authorformat=firstnotreversed}
\jurabibsetup{authorformat=italic}

%%% (2) title configuration
% in every case print the title, let it be seperated from the 
% author by a colon and use the slanted font
\jurabibsetup{titleformat={all,colonsep}}
%\renewcommand*{\jbtitlefont}{\textit}

%%% (3) seperators in bib data
% separate bibliographical hints and page hints by a comma
\jurabibsetup{commabeforerest}

%%% (4) specific configuration of bibdata in quotes / footnote
% use a.a.O if possible
\jurabibsetup{ibidem=strict}

% replace ugly a.a.O. by ders., a.a.O. resp. ders., ebda.
% but if there are more than one author or girl writers?
\AddTo\bibsgerman{
  \renewcommand*{\ibidemname}{Ds., a.a.O.}
  \renewcommand*{\ibidemmidname}{ds., a.a.O.}
}
\renewcommand*{\samepageibidemname}{Ds., ebda.}
\renewcommand*{\samepageibidemmidname}{ds., ebda.}

%%% (5) specific configuration of bibdata in bibliography
% ever an in: before journal and collection/book-tiltes 
\renewcommand*{\bibbtsep}{in: }
%\renewcommand*{\bibjtsep}{in: }

% ever a colon after author names 
\renewcommand*{\bibansep}{: }
% ever a semi colon after the title 
\renewcommand*{\bibatsep}{; }
% ever a comma before date/year
\renewcommand*{\bibbdsep}{, }

% let jurabib insert the S. and p. information
% no S. necessary in bib-files and in cites/footcites
\jurabibsetup{pages=format}

% use a compressed literature-list using a small line indent
\jurabibsetup{bibformat=compress}
\setlength{\jbbibhang}{1em}

% which follows the design of the cites and offers comments
\jurabibsetup{biblikecite}

% print annotations into bibliography
\jurabibsetup{annote}
\renewcommand*{\jbannoteformat}[1]{{ \itshape #1 }}

%refine the prefix of url download
\AddTo\bibsgerman{\renewcommand*{\urldatecomment}{Referenzdownload: }}

% we want to have the year of articles in brackets
\renewcommand*{\bibaldelim}{(}
\renewcommand*{\bibardelim}{)}

%Umformatierung des Reihentitels und der Reihennummer
\DeclareRobustCommand{\numberandseries}[2]{%
\unskip\unskip%,
\space\bibsnfont{(=~#2}%
\ifthenelse{\equal{#1}{}}{)}{, [Bd./Nr.]~#1)}%
}%

% Local Variables:
% mode: latex
% fill-column: 80
% End:


% language specific hyphenation
% Telekom osCompendium osHyphenation Include Module
%
% (c) Karsten Reincke, Deutsche Telekom AG, Darmstadt 2011
%
% This LaTeX-File is licensed under the Creative Commons Attribution-ShareAlike
% 3.0 Germany License (http://creativecommons.org/licenses/by-sa/3.0/de/): Feel
% free 'to share (to copy, distribute and transmit)' or 'to remix (to adapt)'
% it, if you '... distribute the resulting work under the same or similar
% license to this one' and if you respect how 'you must attribute the work in
% the manner specified by the author ...':
%
% In an internet based reuse please link the reused parts to www.telekom.com and
% mention the original authors and Deutsche Telekom AG in a suitable manner. In
% a paper-like reuse please insert a short hint to www.telekom.com and to the
% original authors and Deutsche Telekom AG into your preface. For normal
% quotations please use the scientific standard to cite.
%
% [ File structure derived from 'mind your Scholar Research Framework' 
%   mycsrf (c) K. Reincke CC BY 3.0  http://mycsrf.fodina.de/ ]
%


\hyphenation{rein-cke}

% Local Variables:
% mode: latex
% fill-column: 80
% End:


%%% (3) layout page configuration %%%

% select the visible parts of a page
% S.31: { plain|empty|headings|myheadings }
%\pagestyle{myheadings}
\pagestyle{headings}

% select the wished style of page-numbering
% S.32: { arabic,roman,Roman,alph,Alph }
\pagenumbering{arabic}
\setcounter{page}{1}

% select the wished distances using the general setlength order:
% S.34 { baselineskip| parskip | parindent }
% - general no indent for paragraphs
\setlength{\parindent}{0pt}
\setlength{\parskip}{1.2ex plus 0.2ex minus 0.2ex}


%%% (4) general package activation %%%
%\usepackage{utopia}
%\usepackage{courier}
%\usepackage{avant}
\usepackage[dvips]{epsfig}

% graphic
\usepackage{graphicx,color}
\usepackage{array}
\usepackage{shadow}
\usepackage{fancybox}

%- start(footnote-configuration)
%  flush the cite numbers out of the vertical line and let
%  the footnote text directly start in the left vertical line
\usepackage[marginal]{footmisc}
%- end(footnote-configuration)

\begin{document}

%% use all entries of the bliography

%%-- start(titlepage)
\titlehead{Literaturexzerpt}
\subject{Autor(en): Bernd Suchomski}
\title{Titel: Proprietäres Patentrecht beim Einsatz von Open Source
Software}
\subtitle{Jahr: 2011 }
\author{K. Reincke% Telekom osCompendium License Include Module
%
% (c) Karsten Reincke, Deutsche Telekom AG, Darmstadt 2011
%
% This LaTeX-File is licensed under the Creative Commons Attribution-ShareAlike
% 3.0 Germany License (http://creativecommons.org/licenses/by-sa/3.0/de/): Feel
% free 'to share (to copy, distribute and transmit)' or 'to remix (to adapt)'
% it, if you '... distribute the resulting work under the same or similar
% license to this one' and if you respect how 'you must attribute the work in
% the manner specified by the author ...':
%
% In an internet based reuse please link the reused parts to www.telekom.com and
% mention the original authors and Deutsche Telekom AG in a suitable manner. In
% a paper-like reuse please insert a short hint to www.telekom.com and to the
% original authors and Deutsche Telekom AG into your preface. For normal
% quotations please use the scientific standard to cite.
%
% [ File structure derived from 'mind your Scholar Research Framework' 
%   mycsrf (c) K. Reincke CC BY 3.0  http://mycsrf.fodina.de/ ]
%
\footnote{
This text is licensed under the Creative Commons Attribution-ShareAlike 3.0 Germany
License (http://creativecommons.org/licenses/by-sa/3.0/de/): Feel free \enquote{to
share (to copy, distribute and transmit)} or \enquote{to remix (to
adapt)} it, if you \enquote{[\ldots] distribute the resulting work under the
same or similar license to this one} and if you respect how \enquote{you
must attribute the work in the manner specified by the author(s)
[\ldots]}):
\newline
In an internet based reuse please mention the initial authors in a suitable
manner, name their sponsor \textit{Deutsche Telekom AG} and link it to
\texttt{http://www.telekom.com}. In a paper-like reuse please insert a short
hint to \texttt{http://www.telekom.com}, to the initial authors, and to their
sponsor \textit{Deutsche Telekom AG} into your preface. For normal citations
please use the scientific standard.
\newline
{ \tiny \itshape [based on myCsrf (= mind your Scholar Research Framework) 
\copyright K. Reincke CC BY 3.0  https://github.com/kreincke/mycsrf/)] }}

% Local Variables:
% mode: latex
% fill-column: 80
% End:
}

%\thanks{den Autoren von KOMA-Script und denen von Jurabib}
\maketitle
%%-- end(titlepage)
%\nocite{*}

\begin{abstract}
\noindent
Das Werk / The work\footcite[][]{Suchomski2011a} \\
\noindent \itshape
\ldots analysiert, in welchem Sinne man Patente auf der Basis von Open Source
Software 'gewinnt' resp. 'verliert': Patente können in Deutschland auch aus Open
Source Software heraus angemeldet werden, ihre Erteilung kann über den Verweis
auf OS Software kaum verhindert werden. Allerdings erteilt jede Firma bei der
Weitergabe der Software ein Nutzungsrecht auch an den Patenten: bei OSS ohne
Copyleft-Klausel implizit über das Einräumen der Nutzungsrechte, bei Software
mit Copyleft-Klauseln indirekt über die Pflicht zur Freigabe der Veränderungen -
und bei OSS mit Patentklauseln das alles explizit.
\\
\noindent
\ldots analyzes the meaning of 'gaining' and 'losing' a patent based on OSS:
First, (in Germany) patents can be registers on the base of OSS. But mostly
it's difficult to prevent their registration with a reference to an already
existing OS software. Second, in general a company distributes the right to use
its' patents with distributing the OSS: In case of Non-Copyleft software the
right is granted implicitly by granting the right to use the software. In case
of copyleft software the patent must also be given free because of the
obligation to publish the changings. And in case of OSS patent clauses all this
is done
explicitly.
\end{abstract}
\footnotesize
%\tableofcontents
\normalsize

\section{Line of Thought}

Dieses Buch - eine Magisterarbeit - analysiert, in welchem Sinne man Patente auf
der Basis von Open Source Software 'gewinnt' resp. 'verliert':

\subsection{OS basierte Patentanmeldung}

These: \textbf{Patente können in Deutschland auch aus Open Source Software
heraus angemeldet werden}

Zuerst entsteht die Frage, ob man als Lizenznehmer einer Open Source Software
basierend auf dieser Software überhaupt ein Patent anmelden könne, weil es ja
gerade der Lizenzgeber sei, der die urspüngliche 'Idee' programmiert
hat\footcite[vgl.][21]{Suchomski2011a}. Dies wird in Deutschland über das
\enquote{First-to-File-Prinzip}
ermöglicht\footcite[vgl.][21]{Suchomski2011a} - verbunden mit der Tatsache, dass
in Deutschland bei einer Patentanmeldung nur das technische Prinzip dargestellt
und der Code gar nicht vorgelegt werden
müsse\footcite[vgl.][26]{Suchomski2011a}. Und eben darauf beruhe (wenigstens
teilweise) \enquote{rechtswissenschftlich und praktisch} gesehen die
\enquote{[\ldots] oft geäußerte Angst der OSS Comnmunity vor
Softwarepatenten}\footcite[vgl.][27]{Suchomski2011a}: 

\begin{quote}
\enquote{Durch die Verbreitung des Quellcodes ohne gleichzeitige Patentanmeldung
wird ein Anreiz auf für unrechtmäßige Patentgesuche Dritter geschaffen.
}\footcite[vgl.][27]{Suchomski2011a}
\end{quote}

\subsection{OS basierte Patentbekämpfung}
Auch umgekehrt sei es mit der Wirksamkeit von OSS nicht so gut bestellt:
Grundsätzlich gelten \enquote{Quellen aus dem Internet [\ldots] als
manipulierbar und können die gesetztlichen Vermutung zugunsten der
Neuheit der Erfindung [\ldots] nicht
überwinden}\footcite[vgl.][27]{Suchomski2011a}. Das bedeutet, dass ein
Hinweis auf existierende Open Source Software die Annahme eines Patentantrages
nicht verhindern könne, weil - nach Argumentation des Bundespatentgerichts -
\enquote{[\ldots] digitale Informationen - insbesondere aus dem Internet
als Hauptverbreitungsmedium - nur selten als Beweis für die
Neuheitsschädlichkeit zugelassen werden, das etwaige Zeitangaben
manipulierbar (seien)}\footcite[vgl.][33]{Suchomski2011a}.

\subsection{Erteilung der Nutzungsrechte}
Allerdings erteilt jede Firma bei der Weitergabe der Software ein Nutzungsrecht
auch an den Patenten, wobei \enquote{[\ldots] grundsätzlich von einer
negativen Patentlizenz [\ldots] auszugehen (sei)}, sobald \enquote{[\ldots]
eine OSS-Lizenz durch einen derzeitigen oder zukünftigen Patentinhaber
angenommen wird}\footcite[vgl.][79]{Suchomski2011a}. Bei einer
'negativen Patentlizenz' \enquote{[\ldots] duldet der Patentinhaber die
Nutzungen durch priviligierte Personengruppen, soweit sie über die
Scgranken des Patentrechts hinaustreten}. Und sie beinhaltet
\enquote{einen vertraglichen Verzicht auf
Klage}\footcite[vgl.][69]{Suchomski2011a}

\subsubsection{bei OSS ohne Copyleft-Klausel}
bei OSS ohne Copyleft-Klausel implizit über das Einräumen der
Nutzungsrechte\footcite[vgl.][117]{Suchomski2011a}: Hier wird  natürlich
nur der rechtliche interessante Fall analysiert, dass jemand OSS nutzt =
ein Lizenznehmer ist, und gleichzeitig ein Patent besitzt oder anmeldet, dass
der Nutzung dieser Software entgegenstünde: \enquote{Durch die intendierte
Gleichbehandlung aller Lizenznehmer [\ldots]}, wie sie in den OS Lizenzen
ausgedrückt wird, \enquote{[\ldots] wird auf objektiv ersichtlich, dass
dieser Rahmen allen Entwicklern - Patentinhabern und Nicht-Patentinhabern
- zustehen soll}\footcite[vgl.][118]{Suchomski2011a}: 

\begin{quote}
\enquote{Als Lizenznehmer darf der Patentinhaber diesen Vetragszweck daher im Wege
von Treu und Glauben nicht konterkarieren
}\footcite[][118]{Suchomski2011a}
\end{quote}

Das ist eine gute Nachricht: Es ist mithin nicht rechtens, bestehende Open
Source Software zu nehmen, ein Patent darauf anzumelden, die Software
anschließend möglichst weit zu distribuieren, nur um deren Nutzer dann
patentrechtlich zu verklagen.

\subsubsection{bei OSS mit Copyleft-Klausel}
bei Software mit Copyleft-Klauseln indirekt über die Pflicht zur Freigabe der
Veränderungen, was sich aus der Pflicht zu Veröffentlichung ergibt:
\enquote{Die hiermit verbundene konkludenten Einräumung der negativen
Patentlizenz an seine Abnehmer gestaltet sich [\ldots] wie bei den
Non-Copyleft-Lizenzen}\footcite[vgl.][118]{Suchomski2011a}

\subsubsection{bei OSS mit Patentklauseln}
und bei OSS mit Patentklauseln das alles explizit, denn \enquote{was
ansonsten auch ohne diese Regelgung für die Verbreitung von Software
durch die Patentinhaber gegolten hätte, wird damit nochmals
festgehalten}\footcite[vgl.][128]{Suchomski2011a}

\small
\bibliography{../bibfiles/oscResourcesDe}

\end{document}
