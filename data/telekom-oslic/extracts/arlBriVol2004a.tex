% Telekom osCompendium extract template
%
% (c) Karsten Reincke, Deutsche Telekom AG, Darmstadt 2011
%
% This LaTeX-File is licensed under the Creative Commons Attribution-ShareAlike
% 3.0 Germany License (http://creativecommons.org/licenses/by-sa/3.0/de/): Feel
% free 'to share (to copy, distribute and transmit)' or 'to remix (to adapt)'
% it, if you '... distribute the resulting work under the same or similar
% license to this one' and if you respect how 'you must attribute the work in
% the manner specified by the author ...':
%
% In an internet based reuse please link the reused parts to www.telekom.com and
% mention the original authors and Deutsche Telekom AG in a suitable manner. In
% a paper-like reuse please insert a short hint to www.telekom.com and to the
% original authors and Deutsche Telekom AG into your preface. For normal
% quotations please use the scientific standard to cite.
%
% [ File structure derived from 'mind your Scholar Research Framework' 
%   mycsrf (c) K. Reincke CC BY 3.0  http://mycsrf.fodina.de/ ]

%
% select the document class
% S.26: [ 10pt|11pt|12pt onecolumn|twocolumn oneside|twoside notitlepage|titlepage final|draft
%         leqno fleqn openbib a4paper|a5paper|b5paper|letterpaper|legalpaper|executivepaper openrigth ]
% S.25: { article|report|book|letter ... }
%
% oder koma-skript S.10 + 16
\documentclass[DIV=calc,BCOR=5mm,11pt,headings=small,oneside,abstract=true, toc=bib]{scrartcl}

%%% (1) general configurations %%%
\usepackage[utf8]{inputenc}

%%% (2) language specific configurations %%%
\usepackage[]{a4,ngerman}
\usepackage[english, german, ngerman]{babel}
\selectlanguage{ngerman}

%language specific quoting signs
%default for language emglish is american style of quotes
\usepackage[german=quotes]{csquotes}

% jurabib configuration
\usepackage[see]{jurabib}
\bibliographystyle{jurabib}
% Telekom osCompendium German Jurabib Configuration Include Module 
%
% (c) Karsten Reincke, Deutsche Telekom AG, Darmstadt 2011
%
% This LaTeX-File is licensed under the Creative Commons Attribution-ShareAlike
% 3.0 Germany License (http://creativecommons.org/licenses/by-sa/3.0/de/): Feel
% free 'to share (to copy, distribute and transmit)' or 'to remix (to adapt)'
% it, if you '... distribute the resulting work under the same or similar
% license to this one' and if you respect how 'you must attribute the work in
% the manner specified by the author ...':
%
% In an internet based reuse please link the reused parts to www.telekom.com and
% mention the original authors and Deutsche Telekom AG in a suitable manner. In
% a paper-like reuse please insert a short hint to www.telekom.com and to the
% original authors and Deutsche Telekom AG into your preface. For normal
% quotations please use the scientific standard to cite.
%
% [ File structure derived from 'mind your Scholar Research Framework' 
%   mycsrf (c) K. Reincke CC BY 3.0  http://mycsrf.fodina.de/ ]

% the first time cite with all data, later with shorttitle
\jurabibsetup{citefull=first}

%%% (1) author / editor list configuration
%\jurabibsetup{authorformat=and} % uses 'und' instead of 'u.'
% therefore define your own abbreviated conjunction: 
% an 'and before last author explicetly written conjunction

% for authors in citations
\renewcommand*{\jbbtasep}{ u. } % bta = between two authors sep
\renewcommand*{\jbbfsasep}{, } % bfsa = between first and second author sep
\renewcommand*{\jbbstasep}{ u. }% bsta = between second and third author sep
% for editors in citations
\renewcommand*{\jbbtesep}{ u. } % bta = between two authors sep
\renewcommand*{\jbbfsesep}{, } % bfsa = between first and second author sep
\renewcommand*{\jbbstesep}{ u. }% bsta = between second and third author sep

% for authors in literature list
\renewcommand*{\bibbtasep}{ u. } % bta = between two authors sep
\renewcommand*{\bibbfsasep}{, } % bfsa = between first and second author sep
\renewcommand*{\bibbstasep}{ u. }% bsta = between second and third author sep
% for editors  in literature list
\renewcommand*{\bibbtesep}{ u. } % bte = between two editors sep
\renewcommand*{\bibbfsesep}{, } % bfse = between first and second editor sep
\renewcommand*{\bibbstesep}{ u. }% bste = between second and third editor sep

% use: name, forname, forname lastname u. forname lastname
\jurabibsetup{authorformat=firstnotreversed}
\jurabibsetup{authorformat=italic}

%%% (2) title configuration
% in every case print the title, let it be seperated from the 
% author by a colon and use the slanted font
\jurabibsetup{titleformat={all,colonsep}}
%\renewcommand*{\jbtitlefont}{\textit}

%%% (3) seperators in bib data
% separate bibliographical hints and page hints by a comma
\jurabibsetup{commabeforerest}

%%% (4) specific configuration of bibdata in quotes / footnote
% use a.a.O if possible
\jurabibsetup{ibidem=strict}

% replace ugly a.a.O. by ders., a.a.O. resp. ders., ebda.
% but if there are more than one author or girl writers?
\AddTo\bibsgerman{
  \renewcommand*{\ibidemname}{Ds., a.a.O.}
  \renewcommand*{\ibidemmidname}{ds., a.a.O.}
}
\renewcommand*{\samepageibidemname}{Ds., ebda.}
\renewcommand*{\samepageibidemmidname}{ds., ebda.}

%%% (5) specific configuration of bibdata in bibliography
% ever an in: before journal and collection/book-tiltes 
\renewcommand*{\bibbtsep}{in: }
%\renewcommand*{\bibjtsep}{in: }

% ever a colon after author names 
\renewcommand*{\bibansep}{: }
% ever a semi colon after the title 
\renewcommand*{\bibatsep}{; }
% ever a comma before date/year
\renewcommand*{\bibbdsep}{, }

% let jurabib insert the S. and p. information
% no S. necessary in bib-files and in cites/footcites
\jurabibsetup{pages=format}

% use a compressed literature-list using a small line indent
\jurabibsetup{bibformat=compress}
\setlength{\jbbibhang}{1em}

% which follows the design of the cites and offers comments
\jurabibsetup{biblikecite}

% print annotations into bibliography
\jurabibsetup{annote}
\renewcommand*{\jbannoteformat}[1]{{ \itshape #1 }}

%refine the prefix of url download
\AddTo\bibsgerman{\renewcommand*{\urldatecomment}{Referenzdownload: }}

% we want to have the year of articles in brackets
\renewcommand*{\bibaldelim}{(}
\renewcommand*{\bibardelim}{)}

%Umformatierung des Reihentitels und der Reihennummer
\DeclareRobustCommand{\numberandseries}[2]{%
\unskip\unskip%,
\space\bibsnfont{(=~#2}%
\ifthenelse{\equal{#1}{}}{)}{, [Bd./Nr.]~#1)}%
}%

% Local Variables:
% mode: latex
% fill-column: 80
% End:


% language specific hyphenation
% Telekom osCompendium osHyphenation Include Module
%
% (c) Karsten Reincke, Deutsche Telekom AG, Darmstadt 2011
%
% This LaTeX-File is licensed under the Creative Commons Attribution-ShareAlike
% 3.0 Germany License (http://creativecommons.org/licenses/by-sa/3.0/de/): Feel
% free 'to share (to copy, distribute and transmit)' or 'to remix (to adapt)'
% it, if you '... distribute the resulting work under the same or similar
% license to this one' and if you respect how 'you must attribute the work in
% the manner specified by the author ...':
%
% In an internet based reuse please link the reused parts to www.telekom.com and
% mention the original authors and Deutsche Telekom AG in a suitable manner. In
% a paper-like reuse please insert a short hint to www.telekom.com and to the
% original authors and Deutsche Telekom AG into your preface. For normal
% quotations please use the scientific standard to cite.
%
% [ File structure derived from 'mind your Scholar Research Framework' 
%   mycsrf (c) K. Reincke CC BY 3.0  http://mycsrf.fodina.de/ ]
%


\hyphenation{rein-cke}

% Local Variables:
% mode: latex
% fill-column: 80
% End:


%%% (3) layout page configuration %%%

% select the visible parts of a page
% S.31: { plain|empty|headings|myheadings }
%\pagestyle{myheadings}
\pagestyle{headings}

% select the wished style of page-numbering
% S.32: { arabic,roman,Roman,alph,Alph }
\pagenumbering{arabic}
\setcounter{page}{1}

% select the wished distances using the general setlength order:
% S.34 { baselineskip| parskip | parindent }
% - general no indent for paragraphs
\setlength{\parindent}{0pt}
\setlength{\parskip}{1.2ex plus 0.2ex minus 0.2ex}


%%% (4) general package activation %%%
%\usepackage{utopia}
%\usepackage{courier}
%\usepackage{avant}
\usepackage[dvips]{epsfig}

% graphic
\usepackage{graphicx,color}
\usepackage{array}
\usepackage{shadow}
\usepackage{fancybox}

%- start(footnote-configuration)
%  flush the cite numbers out of the vertical line and let
%  the footnote text directly start in the left vertical line
\usepackage[marginal]{footmisc}
%- end(footnote-configuration)

\begin{document}

%% use all entries of the bliography

%%-- start(titlepage)
\titlehead{Literaturexzerpt}
\subject{Autor(en): Arl etc.}
\title{Titel: BSD- und Mozilla-Lizenzen}
\subtitle{Jahr: 200X }
\author{K. Reincke% Telekom osCompendium License Include Module
%
% (c) Karsten Reincke, Deutsche Telekom AG, Darmstadt 2011
%
% This LaTeX-File is licensed under the Creative Commons Attribution-ShareAlike
% 3.0 Germany License (http://creativecommons.org/licenses/by-sa/3.0/de/): Feel
% free 'to share (to copy, distribute and transmit)' or 'to remix (to adapt)'
% it, if you '... distribute the resulting work under the same or similar
% license to this one' and if you respect how 'you must attribute the work in
% the manner specified by the author ...':
%
% In an internet based reuse please link the reused parts to www.telekom.com and
% mention the original authors and Deutsche Telekom AG in a suitable manner. In
% a paper-like reuse please insert a short hint to www.telekom.com and to the
% original authors and Deutsche Telekom AG into your preface. For normal
% quotations please use the scientific standard to cite.
%
% [ File structure derived from 'mind your Scholar Research Framework' 
%   mycsrf (c) K. Reincke CC BY 3.0  http://mycsrf.fodina.de/ ]
%
\footnote{
This text is licensed under the Creative Commons Attribution-ShareAlike 3.0 Germany
License (http://creativecommons.org/licenses/by-sa/3.0/de/): Feel free \enquote{to
share (to copy, distribute and transmit)} or \enquote{to remix (to
adapt)} it, if you \enquote{[\ldots] distribute the resulting work under the
same or similar license to this one} and if you respect how \enquote{you
must attribute the work in the manner specified by the author(s)
[\ldots]}):
\newline
In an internet based reuse please mention the initial authors in a suitable
manner, name their sponsor \textit{Deutsche Telekom AG} and link it to
\texttt{http://www.telekom.com}. In a paper-like reuse please insert a short
hint to \texttt{http://www.telekom.com}, to the initial authors, and to their
sponsor \textit{Deutsche Telekom AG} into your preface. For normal citations
please use the scientific standard.
\newline
{ \tiny \itshape [based on myCsrf (= mind your Scholar Research Framework) 
\copyright K. Reincke CC BY 3.0  https://github.com/kreincke/mycsrf/)] }}

% Local Variables:
% mode: latex
% fill-column: 80
% End:
}

%\thanks{den Autoren von KOMA-Script und denen von Jurabib}
\maketitle
%%-- end(titlepage)
%\nocite{*}

\begin{abstract}
Das Werk / The work\footcite[][]{ArlBriVol2004a} \\
\noindent \itshape
\ldots Dies Kapitel beschreibt den juristischen Gedankengang der BSD Lizenz und
der Mozilla-Lizenzen, wobei letztere als Copyleft-Lizenzen mit anderem Ansatz
als die GPL dargestellt werden. \\
\noindent
\ldots This chapter of the book describes the thoughts of the BSD and the
Mozilla licenses. The Mozilla licenses are taken as Copyleft licenses which
differ from the GPL.
\end{abstract}
\footnotesize
%\tableofcontents
\normalsize

\section{Line of Thought}

Dieses Kapitel \enquote{'BSD'- und 'Mozilla'-artige Lizenzen} des Buches

\begin{itemize}
  \item \ldots skizziert sehr kurz die KLizenzgeschichte von BSD und
  Mozilla\footcite[vgl.][318ff]{ArlBriVol2004a}
  \item \ldots beschreibt danach die \enquote{Rechteeinräumung und
  -umfang}\footcite[vgl.][323ff]{ArlBriVol2004a}
  \item \ldots geht danach auf die Pflichten ein, die von BSD und
  Mozilla-Lizenzen evoziert werden\footcite[vgl.][342ff]{ArlBriVol2004a}
  \item und stellt schließlich Lizenzen gleichen Typs zusammen, sodass von
  \enquote{MPL-artigen Lizenezn}\footcite[vgl.][361ff]{ArlBriVol2004a} und
  von \enquote{BSD-artigen Lizenzen}\footcite[vgl.][364ff]{ArlBriVol2004a}
  gesprochen wird.
\end{itemize}

\subsection{Rechte}

Behauptet wird, dass sich BSD und Mozilla-Lizenzen \enquote{hinsichtlich
ihres Schutzumfangs [\ldots] grundsätzlich nhicht von der GPL (unterscheiden)
}\footcite[vgl.][323]{ArlBriVol2004a}, was sich auf ebenso auf die
\enquote{einfachen Nutzungsrechte}\footcite[vgl.][324]{ArlBriVol2004a} wie
auf das Recht der \enquote{Weiterverbreitung}, der \enquote{Bearbeitung}
und der \enquote{Vervielfältigung}
beziehe\footcite[vgl.][325]{ArlBriVol2004a}.

Und betont wird, dass auch BSD- und Mozilla-Lizenzen eben nicht auf
Urheberrechte verzichten, \enquote{[\ldots] sondern [\ldots] sich gerade
des Urheberrechtsd (bedienen), indem spezifische Nutzungsrechte
eingeräumt werden}\footcite[vgl.][324]{ArlBriVol2004a}

Ein leichte Unterschiedung zwischen GPL und Mozilla gibt es hinsichtlich der
Rechte insofern, als Mozilla das \enquote{[\ldots] Recht zur
Unterlizenzierung (einräume)} und so die Übetragung der Rechte und
Pflichten an dritte un vierte gewährleistet, während die GPL von einer
automatischen Lizenzierung des ursprünglichen Urhebers
ausgeht\footcite[vgl.][329]{ArlBriVol2004a}, was sozusagen unter dem Terminus
\enquote{Fingierter Ersterwerb in der GPL} erfasst
wird\footcite[vgl.][331]{ArlBriVol2004a}

Des weiteren zeichnet sich die MPL durch ihren Umgang mit den Patenten aus,
deren Nutzung für die Nutzung der SDpfwtare getsattet
ist\footcite[vgl.][330]{ArlBriVol2004a}  [Achtung Vorsicht: hier wird von
\enquote{Patentansprüchen Dritter} gefaselt. Ich glaube nicht, dass die MPL
sich Recht anmaßt, die sie nicht haben kann.]


\subsection{Pflichten: BSD}

Bei der Weitergabe einer BSD-Software sei der Weitergebende in der Pflicht,
\enquote{[\ldots] den Urhebervermerk, den Lizenztext selsbt sowie den
Haftungs- und Gewährleistungsausschluss
(beizubehalten)} bei der
Distribution in Form des Quellcodes einfach durch das Belassen, bei der
Distribution in Form von Binärcode durch entsprechende Formulierungen in der
Dokumentation.\footcite[vgl.][343]{ArlBriVol2004a}:

Für die Weitergabe von modifizierten Versionen konstatieren die Autoren eine
\enquote{missverständliche}Formulierung\footcite[vgl.][343]{ArlBriVol2004a}:
Tatsächlich bezieht sich das Gebot des Lizenzerhaltes vom Lizenztext her nur auf
den unveränderten Quellcode: \enquote{Bei der Verwertung modifizierter
Versionen ist der Nutzer hingegen nicht verpflichtet, Urhebervermerk,
Gewährleistungs- und Haftungsausschluss oder den Lizenztext hinzuzufügen},
sodass eben gerade keine >Copyleft-Konstruktion
entsteht\footcite[vgl.][344]{ArlBriVol2004a}

\subsection{Pflichten: MPL}

Gravierender sollen die Unterschiede zwischen MPL zu BSD und GPL dadagen bzgl.
der MPL eigenen Art des Copylefts sein.

Die anderen Pflichten wie \enquote{Beilegung des Lizenztextes}, die
\enquote{Einbeziehung von Urheberechtsvermerken} und der Umgang mit Gebühren
etc. gleichen in etwa denen der GPL\footcite[vgl.][346f]{ArlBriVol2004a}.

Bzgl. der Pflichten in Sachen Weitergabe modifizierter Versionen sprechen die
Autoren dagegen vom \enquote{beschränkten Copyleft-Effekt der
MPL}\footcite[vgl.][348]{ArlBriVol2004a}. Dies leiten sie wie folgt her:

\begin{itemize}
  \item Grundsätzlich gäbe es durch die §§ 3.1 und 3.2 der MPL einen der MPL
  \enquote{inhärenten Copyleft-Effekt}, sodass \enquote{[\ldots]
  Bearbeitungrn frt >Software grundsätzlich von der Lizenz erfasst
  (seien)}, was besage, dass \enquote{[\ldots] der Quellcode
  inklusive der vorgenommenen Änderungen daher allein unter der MPL
  vertrieben werden (könne)}\footcite[vgl.][349]{ArlBriVol2004a}. Dies
  gelte dann auch bei der Verbreitung in Maschinencode, bei der der modifizierte
  Quellcode eben auch \enquote{zugänglich} gemacht werden
  müsse\footcite[vgl.][349]{ArlBriVol2004a}, allerdings nicht 'dauerhaft', wie
  bei der GPL, sondern nur für \enquote{längstens} 12 MOnate.
  \item Eine kleine besonderheit betrifft bei der MPL die Modifikation, auf die
  hier extra hingewiesen werden müssen und die gesondert dokumentiert werden
  müssen, sodass Original und Ableitung auseinanderzuhalten
  ist\footcite[vgl.][350]{ArlBriVol2004a}
  \item Die nächste Besonderheit der MPL betrifft die Erlaubnis, dass der
  Objektcode selbst unter einer beliebigen Lizenz vertrieben werden dürfe,
  sofern diese\enquote{[\ldots] nicht die Rechte des Quelltextes
  (beschränke)} und sofern angegeben wird, wo und wie der Quellcode
  erhältnich ist\footcite[vgl.][350]{ArlBriVol2004a}. Damit - so die Autoren -
 \enquote{dürften die praktischen Auswirkungen dieser 'Aufweichung' des
 Pflichtenprogramms eher als gering einzustufen
 sein}\footcite[vgl.][351]{ArlBriVol2004a}
 \item Der Hauptunterschied betrifft die \enquote{Kombination proprietärer
 Software mit Open Source Software unter der MPL-Lizenz in einem
 Gesamtpaket, einem sog, 'Larger Work'}: Zwar müsse für das eiegntliche
 MPL lizenziert Werk die MPL-Lizenzbedingung eingehalten werden. Aber
 \enquote{die mit diesem gemeinsam vertriebenen Module müssen dagegen nicht
 der MPL unterstellt werden, sodass weder deren Objekt- noch deren
 Quellcode zugänglich gemacht werden
 müssen}\footcite[vgl.][352]{ArlBriVol2004a}. Das meint explizit, dass
 Änderungen, Weglassungen und Erweiterung am bestehenden Code eben als
 \enquote{Bearbeitung} aufgefasst
 werden\footcite[vgl.][354]{ArlBriVol2004a}: Auch wenn der MPL-Code um das
 doppelkte anwüchse, läge immer noch kein 'Larger Work' vor:
 \enquote{Voraussetzung} für eine Larger Work, über dass die Autnomie der
 eigen Zutat begründet und der ZUgriff auf dess Code verweigert wird, ist
 \enquote{[\ldots] dass die neu hinzugefügten Module in eigenständigen
 Dateien vorliegen, die selbst keine Bestandteile des Originalcodes
 enthalten}\footcite[vgl.][354]{ArlBriVol2004a}
 \item Exkurs zur GPL: Die Autoren unterstreichenh sodann, dass \enquote{[\ldots]
 bestimmte Foirmen des gemeinsamen Vertriebes proprietärer Software und
 GPL-lizenzierter Software [\ldots] auch nach der GPL denkbar seien},
 sofern es sich dabei um identifizierbare Werke handele, \enquote{[\ldots]
 welche nicht vom Originalprogramm abgeleitet (seiern) und
 vernünftigerweise slebst als unabhängige und eigenständige Werke
 betrachtet werden können}\footcite[vgl.][354]{ArlBriVol2004a}. Allerdings
 gäbe es bei der GPL eben das Postulat der \enquote{Eigenständigkeit} oder
 \enquote{Unabhängikeit}, das bei der \enquote{[\ldots] MPL keine
 Entsprechung (finde)}\footcite[vgl.][355]{ArlBriVol2004a}
\end{itemize}



[\ldots drüber nachdenken und nachsehen: ist in MPL wirklich von Modulen die
Rede? Wenn ja, füge ich in den MPL-Code eine Modulschnittstelle ein müsste ich
den Code freigeben, greife ich dagegen vom revidierten Code auf meine
gesonderten Module 'zu' und linke die dazu, wäre das das umgekewhrte LGPL
verfahren. ]

\small
\bibliography{../bibfiles/oscResourcesDe}

\end{document}
