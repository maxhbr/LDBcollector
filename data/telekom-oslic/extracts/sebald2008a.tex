% Telekom osCompendium extract template
%
% (c) Karsten Reincke, Deutsche Telekom AG, Darmstadt 2011
%
% This LaTeX-File is licensed under the Creative Commons Attribution-ShareAlike
% 3.0 Germany License (http://creativecommons.org/licenses/by-sa/3.0/de/): Feel
% free 'to share (to copy, distribute and transmit)' or 'to remix (to adapt)'
% it, if you '... distribute the resulting work under the same or similar
% license to this one' and if you respect how 'you must attribute the work in
% the manner specified by the author ...':
%
% In an internet based reuse please link the reused parts to www.telekom.com and
% mention the original authors and Deutsche Telekom AG in a suitable manner. In
% a paper-like reuse please insert a short hint to www.telekom.com and to the
% original authors and Deutsche Telekom AG into your preface. For normal
% quotations please use the scientific standard to cite.
%
% [ File structure derived from 'mind your Scholar Research Framework' 
%   mycsrf (c) K. Reincke CC BY 3.0  http://mycsrf.fodina.de/ ]

%
% select the document class
% S.26: [ 10pt|11pt|12pt onecolumn|twocolumn oneside|twoside notitlepage|titlepage final|draft
%         leqno fleqn openbib a4paper|a5paper|b5paper|letterpaper|legalpaper|executivepaper openrigth ]
% S.25: { article|report|book|letter ... }
%
% oder koma-skript S.10 + 16
\documentclass[DIV=calc,BCOR=5mm,11pt,headings=small,oneside,abstract=true, toc=bib]{scrartcl}

%%% (1) general configurations %%%
\usepackage[utf8]{inputenc}

%%% (2) language specific configurations %%%
\usepackage[]{a4,ngerman}
\usepackage[english, german, ngerman]{babel}
\selectlanguage{ngerman}

%language specific quoting signs
%default for language emglish is american style of quotes
\usepackage{csquotes}

% jurabib configuration
\usepackage[see]{jurabib}
\bibliographystyle{jurabib}
% Telekom osCompendium German Jurabib Configuration Include Module 
%
% (c) Karsten Reincke, Deutsche Telekom AG, Darmstadt 2011
%
% This LaTeX-File is licensed under the Creative Commons Attribution-ShareAlike
% 3.0 Germany License (http://creativecommons.org/licenses/by-sa/3.0/de/): Feel
% free 'to share (to copy, distribute and transmit)' or 'to remix (to adapt)'
% it, if you '... distribute the resulting work under the same or similar
% license to this one' and if you respect how 'you must attribute the work in
% the manner specified by the author ...':
%
% In an internet based reuse please link the reused parts to www.telekom.com and
% mention the original authors and Deutsche Telekom AG in a suitable manner. In
% a paper-like reuse please insert a short hint to www.telekom.com and to the
% original authors and Deutsche Telekom AG into your preface. For normal
% quotations please use the scientific standard to cite.
%
% [ File structure derived from 'mind your Scholar Research Framework' 
%   mycsrf (c) K. Reincke CC BY 3.0  http://mycsrf.fodina.de/ ]

% the first time cite with all data, later with shorttitle
\jurabibsetup{citefull=first}

%%% (1) author / editor list configuration
%\jurabibsetup{authorformat=and} % uses 'und' instead of 'u.'
% therefore define your own abbreviated conjunction: 
% an 'and before last author explicetly written conjunction

% for authors in citations
\renewcommand*{\jbbtasep}{ u. } % bta = between two authors sep
\renewcommand*{\jbbfsasep}{, } % bfsa = between first and second author sep
\renewcommand*{\jbbstasep}{ u. }% bsta = between second and third author sep
% for editors in citations
\renewcommand*{\jbbtesep}{ u. } % bta = between two authors sep
\renewcommand*{\jbbfsesep}{, } % bfsa = between first and second author sep
\renewcommand*{\jbbstesep}{ u. }% bsta = between second and third author sep

% for authors in literature list
\renewcommand*{\bibbtasep}{ u. } % bta = between two authors sep
\renewcommand*{\bibbfsasep}{, } % bfsa = between first and second author sep
\renewcommand*{\bibbstasep}{ u. }% bsta = between second and third author sep
% for editors  in literature list
\renewcommand*{\bibbtesep}{ u. } % bte = between two editors sep
\renewcommand*{\bibbfsesep}{, } % bfse = between first and second editor sep
\renewcommand*{\bibbstesep}{ u. }% bste = between second and third editor sep

% use: name, forname, forname lastname u. forname lastname
\jurabibsetup{authorformat=firstnotreversed}
\jurabibsetup{authorformat=italic}

%%% (2) title configuration
% in every case print the title, let it be seperated from the 
% author by a colon and use the slanted font
\jurabibsetup{titleformat={all,colonsep}}
%\renewcommand*{\jbtitlefont}{\textit}

%%% (3) seperators in bib data
% separate bibliographical hints and page hints by a comma
\jurabibsetup{commabeforerest}

%%% (4) specific configuration of bibdata in quotes / footnote
% use a.a.O if possible
\jurabibsetup{ibidem=strict}

% replace ugly a.a.O. by ders., a.a.O. resp. ders., ebda.
% but if there are more than one author or girl writers?
\AddTo\bibsgerman{
  \renewcommand*{\ibidemname}{Ds., a.a.O.}
  \renewcommand*{\ibidemmidname}{ds., a.a.O.}
}
\renewcommand*{\samepageibidemname}{Ds., ebda.}
\renewcommand*{\samepageibidemmidname}{ds., ebda.}

%%% (5) specific configuration of bibdata in bibliography
% ever an in: before journal and collection/book-tiltes 
\renewcommand*{\bibbtsep}{in: }
%\renewcommand*{\bibjtsep}{in: }

% ever a colon after author names 
\renewcommand*{\bibansep}{: }
% ever a semi colon after the title 
\renewcommand*{\bibatsep}{; }
% ever a comma before date/year
\renewcommand*{\bibbdsep}{, }

% let jurabib insert the S. and p. information
% no S. necessary in bib-files and in cites/footcites
\jurabibsetup{pages=format}

% use a compressed literature-list using a small line indent
\jurabibsetup{bibformat=compress}
\setlength{\jbbibhang}{1em}

% which follows the design of the cites and offers comments
\jurabibsetup{biblikecite}

% print annotations into bibliography
\jurabibsetup{annote}
\renewcommand*{\jbannoteformat}[1]{{ \itshape #1 }}

%refine the prefix of url download
\AddTo\bibsgerman{\renewcommand*{\urldatecomment}{Referenzdownload: }}

% we want to have the year of articles in brackets
\renewcommand*{\bibaldelim}{(}
\renewcommand*{\bibardelim}{)}

%Umformatierung des Reihentitels und der Reihennummer
\DeclareRobustCommand{\numberandseries}[2]{%
\unskip\unskip%,
\space\bibsnfont{(=~#2}%
\ifthenelse{\equal{#1}{}}{)}{, [Bd./Nr.]~#1)}%
}%

% Local Variables:
% mode: latex
% fill-column: 80
% End:


% language specific hyphenation
% Telekom osCompendium osHyphenation Include Module
%
% (c) Karsten Reincke, Deutsche Telekom AG, Darmstadt 2011
%
% This LaTeX-File is licensed under the Creative Commons Attribution-ShareAlike
% 3.0 Germany License (http://creativecommons.org/licenses/by-sa/3.0/de/): Feel
% free 'to share (to copy, distribute and transmit)' or 'to remix (to adapt)'
% it, if you '... distribute the resulting work under the same or similar
% license to this one' and if you respect how 'you must attribute the work in
% the manner specified by the author ...':
%
% In an internet based reuse please link the reused parts to www.telekom.com and
% mention the original authors and Deutsche Telekom AG in a suitable manner. In
% a paper-like reuse please insert a short hint to www.telekom.com and to the
% original authors and Deutsche Telekom AG into your preface. For normal
% quotations please use the scientific standard to cite.
%
% [ File structure derived from 'mind your Scholar Research Framework' 
%   mycsrf (c) K. Reincke CC BY 3.0  http://mycsrf.fodina.de/ ]
%


\hyphenation{rein-cke}
\hyphenation{Rein-cke}
\hyphenation{OS-LiC}
\hyphenation{ori-gi-nal}
\hyphenation{bi-na-ry}
\hyphenation{Li-cence}
\hyphenation{li-cence}

% Local Variables:
% mode: latex
% fill-column: 80
% End:


%%% (3) layout page configuration %%%

% select the visible parts of a page
% S.31: { plain|empty|headings|myheadings }
%\pagestyle{myheadings}
\pagestyle{headings}

% select the wished style of page-numbering
% S.32: { arabic,roman,Roman,alph,Alph }
\pagenumbering{arabic}
\setcounter{page}{1}

% select the wished distances using the general setlength order:
% S.34 { baselineskip| parskip | parindent }
% - general no indent for paragraphs
\setlength{\parindent}{0pt}
\setlength{\parskip}{1.2ex plus 0.2ex minus 0.2ex}


%%% (4) general package activation %%%
%\usepackage{utopia}
%\usepackage{courier}
%\usepackage{avant}
\usepackage[dvips]{epsfig}

% graphic
\usepackage{graphicx,color}
\usepackage{array}
\usepackage{shadow}
\usepackage{fancybox}

%- start(footnote-configuration)
%  flush the cite numbers out of the vertical line and let
%  the footnote text directly start in the left vertical line
\usepackage[marginal]{footmisc}
%- end(footnote-configuration)

\begin{document}

%% use all entries of the bliography

%%-- start(titlepage)
\titlehead{Literaturexzerpt}
\subject{Autor(en): Sebald}
\title{Titel: Offene Wissensökonomie}
\subtitle{Jahr: 2008 }
\author{K. Reincke% Telekom osCompendium License Include Module
%
% (c) Karsten Reincke, Deutsche Telekom AG, Darmstadt 2011
%
% This LaTeX-File is licensed under the Creative Commons Attribution-ShareAlike
% 3.0 Germany License (http://creativecommons.org/licenses/by-sa/3.0/de/): Feel
% free 'to share (to copy, distribute and transmit)' or 'to remix (to adapt)'
% it, if you '... distribute the resulting work under the same or similar
% license to this one' and if you respect how 'you must attribute the work in
% the manner specified by the author ...':
%
% In an internet based reuse please link the reused parts to www.telekom.com and
% mention the original authors and Deutsche Telekom AG in a suitable manner. In
% a paper-like reuse please insert a short hint to www.telekom.com and to the
% original authors and Deutsche Telekom AG into your preface. For normal
% quotations please use the scientific standard to cite.
%
% [ File structure derived from 'mind your Scholar Research Framework' 
%   mycsrf (c) K. Reincke CC BY 3.0  http://mycsrf.fodina.de/ ]
%
\footnote{
This text is licensed under the Creative Commons Attribution-ShareAlike 3.0 Germany
License (http://creativecommons.org/licenses/by-sa/3.0/de/): Feel free \enquote{to
share (to copy, distribute and transmit)} or \enquote{to remix (to
adapt)} it, if you \enquote{[\ldots] distribute the resulting work under the
same or similar license to this one} and if you respect how \enquote{you
must attribute the work in the manner specified by the author(s)
[\ldots]}):
\newline
In an internet based reuse please mention the initial authors in a suitable
manner, name their sponsor \textit{Deutsche Telekom AG} and link it to
\texttt{http://www.telekom.com}. In a paper-like reuse please insert a short
hint to \texttt{http://www.telekom.com}, to the initial authors, and to their
sponsor \textit{Deutsche Telekom AG} into your preface. For normal citations
please use the scientific standard.
\newline
{ \tiny \itshape [based on myCsrf (= mind your Scholar Research Framework) 
\copyright K. Reincke CC BY 3.0  https://github.com/kreincke/mycsrf/)] }}

% Local Variables:
% mode: latex
% fill-column: 80
% End:
}

%\thanks{den Autoren von KOMA-Script und denen von Jurabib}
\maketitle
%%-- end(titlepage)
%\nocite{*}

\begin{abstract}
\noindent
Das Werk / The work\footcite[][]{Sebald2008a} \\
\noindent \itshape
\ldots Vom praktischen Standpunkt aus ein etwas abgehobenes soziologisches Buch:
Es versucht die Bedingungen zur Möglichkeit zu ermitteln, dass sich eine offene
Wissensökonomie in einer kapitalistischen Umgebung etablieren kann. OS scheint
auf GPL reduziert zu sein. Lizenzfragen werden auf 3 Seiten verhandelt.
Allerdings skizziert das Buch die Genese der Idee von 'Freier Software'
erfolgreich. Und es unterstreicht, dass die Vorläuferform der GPL - die
emacs-Lizenz - noch gefordert hatte, alle Veränderungen öffentlich zu machen,
also auch die privatesten, wohingegen die GPL diesen Anspruch später aufgegeben
hat.\\
\noindent
\ldots From the practical viewpoint a withdrawn socialogical book: tries to
discuss the conditions for the possibility that an open knowledge economy
establishes itself in a captalist environment. OS seems to be reduced to GPL.
License questions are discussed on only 3 pages. But at least the book outlines
the evolution of the idea 'Free Software'. And it highlights that the forerunner
of the GPL - the emacs license - had still required to publish all changings -
even the most private improvements - whereas later on the GPL gave up this
condition. 
\end{abstract}
\footnotesize
%\tableofcontents
\normalsize

\section{Summarynachweise}

Vom praktischen Standpunkt aus ein etwas abgehobenes soziologisches Buch:
Es versucht die Bedingungen zur Möglichkeit zu ermitteln, dass sich eine offene
Wissensökonomie in einer kapitalistischen Umgebung etablieren
kann\footcite[vgl.][17]{Sebald2008a}. OS scheint auf GPL reduziert zu
sein\footcite[vgl.][63ff]{Sebald2008a}. Lizenzfragen werden auf 3 Seiten
verhandelt und betreffen die GNU Lizenzen\footcite[vgl.][75ff]{Sebald2008a}.
Allerdings skizziert das Buch dabei die Genese der Idee von 'Freier Software'
erfolgreich\footcite[vgl.][83ff]{Sebald2008a}. Und es unterstreicht, dass die
Vorläuferform der GPL - die emacs-Lizenz - noch gefordert hatte, alle
Veränderungen öffentlich zu machen, also auch die privatesten, wohingegen die
GPL diesen Anspruch später aufgegeben hat\footcite[vgl.][76]{Sebald2008a}.

\section{Line of Thought}

Ziel des Buch es ist es, \enquote{[\ldots] nach den Bedingungen der Möglichkeit
einer [\ldots] 'offenen Wissensökonomie' (zu fragen), die in einem
kapitalistischen Umfeld entsteht und sich stabilisiert, das eigentlich eine
bedingungslose Verwertung nahelegt}\footcite[vgl.][17]{Sebald2008a}. Dabei
bezieht sich der Begriff \enquote{offene Wissensökonomie} gerade auf das
Phänomen 'offene Software', bei der das
\enquote{explizite Wissen}, wie es in der \enquote{Software}
respektive im \enquote{Quellcode} enthalten ist, \enquote{[\ldots] offen und
öffentlich potentiell für alle verfügbar
(bleibt)}\footcite[vgl.][17]{Sebald2008a}. Den die Frage nach den
Bedingungen zur Mööglichkeit evozierenden Befund charakterisiert Sebald
entwaffnend einfach:

\begin{quotation}
  \enquote{Das Erstaunlichste an der F/OSS ist meines Erachtens nach wie
  vor, daß diese offnene Wissensökonomie sich gesellschaftlich etabliert
  hat, obwohl dieses Feld der Produktion von einer funktionierenden und
  florierenden kapitalistisch organisierten Ökonomie beackert
  wird.}\footcite[vgl.][23]{Sebald2008a}
\end{quotation} 

Die Antwort auf diese Frage im Kapitel \enquote{Finis opera} fällt eher dünn
aus: So sei es das Wesen der \enquote{Software} selbst, die \enquote{[\ldots] in ihrer
formalisierten, dekontextuierten Schriftlichkeit die entscheidenden
Rahmenbedingungen setzt für die neue selbstorganisierte Wissensproduktion
von selbstbestimmten Programmierern}\footcite[vgl.][236]{Sebald2008a}.
\section{Specific Aspects}. Allerdings fordere dies einen Preis, eine
Eingangsvoraussetzung, sozusagen eine notwendige, aber nicht hinreichende
Bedingung für die Möglichkeit überhaupt: Es wird eine
\enquote{Wissensschwelle}, ohne deren Überwindung eine \enquote{[\ldots]
mögliche Beteiligun eingeschränkt (bliebe)}, nämlich die
\enquote{Computeralphabetisierung}\footcite[vgl.][236]{Sebald2008a}.
Zudem habe \enquote{in Bezug auf die zentrale
Fragestellung nach der Möglichkeit einer nichtkapitalisitisch organisierten Produktionsweise,
die parallel zu einer kapitalisitschen besteht, [\ldots] empirisch
gezeigt, daß die Abgrenzung sich seitens der F/OSS nur auf den Bereich
der selbstbestimmten, [!SW!] technischen Entscheidung und auf die
Sicherung des gemeinsamen Wissenspools
zielt}\footcite[vgl.][237f]{Sebald2008a}

\section{Special Aspects}

\subsection{Begriffsdefinition F/OSS}

Sebald bietet eine etwas merkwürdige Auflösung der Kürzel für Freien Software:
So bezeichnet er \enquote{Free/Open Source-Softwareentwicklung} als
\enquote{F/OSS} und grenzt dieses gegen das Kürzel \enquote{F/OS} ab,
das für \enquote{Free/Open Source} stehen
soll\footcite[vgl.][15]{Sebald2008a}. 

Das ist merkwürdig insofern, als Open Source Software gern als OSS abgekürzt
wird. Geschuldet ist das natürlich der nicht guten Verbindung von FSS, die es
nicht gibt, sondern nur FS. So wäre F/OSS eigentlich ungefähr zu lesen als
F/(OS)S.

Immerhin betont Seblad richtig, dass \enquote{der Begriff 'libre software'
[\ldots]  für die Free/Open Source-Entwickler selbst kaum eine Rolle
(spiele)}, obwohl er \enquote{[\ldots] für dieses Phänomen in Europa
häufiger verwendet (werde)}\footcite[vgl.][15, Anm.7]{Sebald2008a}

\subsection{Zwangsweise Veröffentlichung von (privaten) Verbesserungen}

Sebald erwähnt, dass Stallmann zunächst mit der \enquote{GNU Emacs License
von 1985} operiert hat, bevor er die GPL entwicklet
habe\footcite[vgl.][76]{Sebald2008a}. Dabei habe es zur 'Emacs-License' gehört,
dass das \enquote{[\ldots] Programm auch verändert werden durfte, solange
die Änderungen veröffentlicht wurden}\footcite[vgl.][76]{Sebald2008a}.
Pikanterweise unterstreicht Selbald in einer Fußnote dazu, dass
eine solche \enquote{[\ldots] Veröffentlichung auch (dann) vorgeschrieben (war),
wenn die Änderung nur für den privaten Gebrauch gemacht wurde}. Und er
betont ebenda unter Fückgriff auf Williams 2002, S. 127, dass
\enquote{dieser 'Big Brother aspect [\ldots] in den späteren Versionen
fallen gelassen (wurde)}\footcite[vgl.][76 Anm. 85]{Sebald2008a}

Wenn das stimmt, wäre das ein wichtiger zusätzlicher Belege über den reinen
Lizenztext hinaus, dass die Veröffentlichung eben nicht in allen Fällen
erzwungen ist.

\small
\bibliography{../bibfiles/oscResourcesDe}

\end{document}
