% Telekom osCompendium extract template
%
% (c) Karsten Reincke, Deutsche Telekom AG, Darmstadt 2011
%
% This LaTeX-File is licensed under the Creative Commons Attribution-ShareAlike
% 3.0 Germany License (http://creativecommons.org/licenses/by-sa/3.0/de/): Feel
% free 'to share (to copy, distribute and transmit)' or 'to remix (to adapt)'
% it, if you '... distribute the resulting work under the same or similar
% license to this one' and if you respect how 'you must attribute the work in
% the manner specified by the author ...':
%
% In an internet based reuse please link the reused parts to www.telekom.com and
% mention the original authors and Deutsche Telekom AG in a suitable manner. In
% a paper-like reuse please insert a short hint to www.telekom.com and to the
% original authors and Deutsche Telekom AG into your preface. For normal
% quotations please use the scientific standard to cite.
%
% [ File structure derived from 'mind your Scholar Research Framework' 
%   mycsrf (c) K. Reincke CC BY 3.0  http://mycsrf.fodina.de/ ]

%
% select the document class
% S.26: [ 10pt|11pt|12pt onecolumn|twocolumn oneside|twoside notitlepage|titlepage final|draft
%         leqno fleqn openbib a4paper|a5paper|b5paper|letterpaper|legalpaper|executivepaper openrigth ]
% S.25: { article|report|book|letter ... }
%
% oder koma-skript S.10 + 16
\documentclass[DIV=calc,BCOR=5mm,11pt,headings=small,oneside,abstract=true, toc=bib]{scrartcl}

%%% (1) general configurations %%%
\usepackage[utf8]{inputenc}

%%% (2) language specific configurations %%%
\usepackage[]{a4,ngerman}
\usepackage[english, german, ngerman]{babel}
\selectlanguage{ngerman}

%language specific quoting signs
%default for language emglish is american style of quotes
\usepackage{csquotes}

% jurabib configuration
\usepackage[see]{jurabib}
\bibliographystyle{jurabib}
% Telekom osCompendium German Jurabib Configuration Include Module 
%
% (c) Karsten Reincke, Deutsche Telekom AG, Darmstadt 2011
%
% This LaTeX-File is licensed under the Creative Commons Attribution-ShareAlike
% 3.0 Germany License (http://creativecommons.org/licenses/by-sa/3.0/de/): Feel
% free 'to share (to copy, distribute and transmit)' or 'to remix (to adapt)'
% it, if you '... distribute the resulting work under the same or similar
% license to this one' and if you respect how 'you must attribute the work in
% the manner specified by the author ...':
%
% In an internet based reuse please link the reused parts to www.telekom.com and
% mention the original authors and Deutsche Telekom AG in a suitable manner. In
% a paper-like reuse please insert a short hint to www.telekom.com and to the
% original authors and Deutsche Telekom AG into your preface. For normal
% quotations please use the scientific standard to cite.
%
% [ File structure derived from 'mind your Scholar Research Framework' 
%   mycsrf (c) K. Reincke CC BY 3.0  http://mycsrf.fodina.de/ ]

% the first time cite with all data, later with shorttitle
\jurabibsetup{citefull=first}

%%% (1) author / editor list configuration
%\jurabibsetup{authorformat=and} % uses 'und' instead of 'u.'
% therefore define your own abbreviated conjunction: 
% an 'and before last author explicetly written conjunction

% for authors in citations
\renewcommand*{\jbbtasep}{ u. } % bta = between two authors sep
\renewcommand*{\jbbfsasep}{, } % bfsa = between first and second author sep
\renewcommand*{\jbbstasep}{ u. }% bsta = between second and third author sep
% for editors in citations
\renewcommand*{\jbbtesep}{ u. } % bta = between two authors sep
\renewcommand*{\jbbfsesep}{, } % bfsa = between first and second author sep
\renewcommand*{\jbbstesep}{ u. }% bsta = between second and third author sep

% for authors in literature list
\renewcommand*{\bibbtasep}{ u. } % bta = between two authors sep
\renewcommand*{\bibbfsasep}{, } % bfsa = between first and second author sep
\renewcommand*{\bibbstasep}{ u. }% bsta = between second and third author sep
% for editors  in literature list
\renewcommand*{\bibbtesep}{ u. } % bte = between two editors sep
\renewcommand*{\bibbfsesep}{, } % bfse = between first and second editor sep
\renewcommand*{\bibbstesep}{ u. }% bste = between second and third editor sep

% use: name, forname, forname lastname u. forname lastname
\jurabibsetup{authorformat=firstnotreversed}
\jurabibsetup{authorformat=italic}

%%% (2) title configuration
% in every case print the title, let it be seperated from the 
% author by a colon and use the slanted font
\jurabibsetup{titleformat={all,colonsep}}
%\renewcommand*{\jbtitlefont}{\textit}

%%% (3) seperators in bib data
% separate bibliographical hints and page hints by a comma
\jurabibsetup{commabeforerest}

%%% (4) specific configuration of bibdata in quotes / footnote
% use a.a.O if possible
\jurabibsetup{ibidem=strict}

% replace ugly a.a.O. by ders., a.a.O. resp. ders., ebda.
% but if there are more than one author or girl writers?
\AddTo\bibsgerman{
  \renewcommand*{\ibidemname}{Ds., a.a.O.}
  \renewcommand*{\ibidemmidname}{ds., a.a.O.}
}
\renewcommand*{\samepageibidemname}{Ds., ebda.}
\renewcommand*{\samepageibidemmidname}{ds., ebda.}

%%% (5) specific configuration of bibdata in bibliography
% ever an in: before journal and collection/book-tiltes 
\renewcommand*{\bibbtsep}{in: }
%\renewcommand*{\bibjtsep}{in: }

% ever a colon after author names 
\renewcommand*{\bibansep}{: }
% ever a semi colon after the title 
\renewcommand*{\bibatsep}{; }
% ever a comma before date/year
\renewcommand*{\bibbdsep}{, }

% let jurabib insert the S. and p. information
% no S. necessary in bib-files and in cites/footcites
\jurabibsetup{pages=format}

% use a compressed literature-list using a small line indent
\jurabibsetup{bibformat=compress}
\setlength{\jbbibhang}{1em}

% which follows the design of the cites and offers comments
\jurabibsetup{biblikecite}

% print annotations into bibliography
\jurabibsetup{annote}
\renewcommand*{\jbannoteformat}[1]{{ \itshape #1 }}

%refine the prefix of url download
\AddTo\bibsgerman{\renewcommand*{\urldatecomment}{Referenzdownload: }}

% we want to have the year of articles in brackets
\renewcommand*{\bibaldelim}{(}
\renewcommand*{\bibardelim}{)}

%Umformatierung des Reihentitels und der Reihennummer
\DeclareRobustCommand{\numberandseries}[2]{%
\unskip\unskip%,
\space\bibsnfont{(=~#2}%
\ifthenelse{\equal{#1}{}}{)}{, [Bd./Nr.]~#1)}%
}%

% Local Variables:
% mode: latex
% fill-column: 80
% End:


% language specific hyphenation
% Telekom osCompendium osHyphenation Include Module
%
% (c) Karsten Reincke, Deutsche Telekom AG, Darmstadt 2011
%
% This LaTeX-File is licensed under the Creative Commons Attribution-ShareAlike
% 3.0 Germany License (http://creativecommons.org/licenses/by-sa/3.0/de/): Feel
% free 'to share (to copy, distribute and transmit)' or 'to remix (to adapt)'
% it, if you '... distribute the resulting work under the same or similar
% license to this one' and if you respect how 'you must attribute the work in
% the manner specified by the author ...':
%
% In an internet based reuse please link the reused parts to www.telekom.com and
% mention the original authors and Deutsche Telekom AG in a suitable manner. In
% a paper-like reuse please insert a short hint to www.telekom.com and to the
% original authors and Deutsche Telekom AG into your preface. For normal
% quotations please use the scientific standard to cite.
%
% [ File structure derived from 'mind your Scholar Research Framework' 
%   mycsrf (c) K. Reincke CC BY 3.0  http://mycsrf.fodina.de/ ]
%


\hyphenation{rein-cke}

% Local Variables:
% mode: latex
% fill-column: 80
% End:


%%% (3) layout page configuration %%%

% select the visible parts of a page
% S.31: { plain|empty|headings|myheadings }
%\pagestyle{myheadings}
\pagestyle{headings}

% select the wished style of page-numbering
% S.32: { arabic,roman,Roman,alph,Alph }
\pagenumbering{arabic}
\setcounter{page}{1}

% select the wished distances using the general setlength order:
% S.34 { baselineskip| parskip | parindent }
% - general no indent for paragraphs
\setlength{\parindent}{0pt}
\setlength{\parskip}{1.2ex plus 0.2ex minus 0.2ex}


%%% (4) general package activation %%%
%\usepackage{utopia}
%\usepackage{courier}
%\usepackage{avant}
\usepackage[dvips]{epsfig}

% graphic
\usepackage{graphicx,color}
\usepackage{array}
\usepackage{shadow}
\usepackage{fancybox}

%- start(footnote-configuration)
%  flush the cite numbers out of the vertical line and let
%  the footnote text directly start in the left vertical line
\usepackage[marginal]{footmisc}
%- end(footnote-configuration)

\begin{document}

%% use all entries of the bliography

%%-- start(titlepage)
\titlehead{Literaturexzerpt}
\subject{Autor(en): Edward Viesel}
\title{Titel: Freiheit statt Freibier}
\subtitle{Jahr: 2006 }
\author{K. Reincke% Telekom osCompendium License Include Module
%
% (c) Karsten Reincke, Deutsche Telekom AG, Darmstadt 2011
%
% This LaTeX-File is licensed under the Creative Commons Attribution-ShareAlike
% 3.0 Germany License (http://creativecommons.org/licenses/by-sa/3.0/de/): Feel
% free 'to share (to copy, distribute and transmit)' or 'to remix (to adapt)'
% it, if you '... distribute the resulting work under the same or similar
% license to this one' and if you respect how 'you must attribute the work in
% the manner specified by the author ...':
%
% In an internet based reuse please link the reused parts to www.telekom.com and
% mention the original authors and Deutsche Telekom AG in a suitable manner. In
% a paper-like reuse please insert a short hint to www.telekom.com and to the
% original authors and Deutsche Telekom AG into your preface. For normal
% quotations please use the scientific standard to cite.
%
% [ File structure derived from 'mind your Scholar Research Framework' 
%   mycsrf (c) K. Reincke CC BY 3.0  http://mycsrf.fodina.de/ ]
%
\footnote{
This text is licensed under the Creative Commons Attribution-ShareAlike 3.0 Germany
License (http://creativecommons.org/licenses/by-sa/3.0/de/): Feel free \enquote{to
share (to copy, distribute and transmit)} or \enquote{to remix (to
adapt)} it, if you \enquote{[\ldots] distribute the resulting work under the
same or similar license to this one} and if you respect how \enquote{you
must attribute the work in the manner specified by the author(s)
[\ldots]}):
\newline
In an internet based reuse please mention the initial authors in a suitable
manner, name their sponsor \textit{Deutsche Telekom AG} and link it to
\texttt{http://www.telekom.com}. In a paper-like reuse please insert a short
hint to \texttt{http://www.telekom.com}, to the initial authors, and to their
sponsor \textit{Deutsche Telekom AG} into your preface. For normal citations
please use the scientific standard.
\newline
{ \tiny \itshape [based on myCsrf (= mind your Scholar Research Framework) 
\copyright K. Reincke CC BY 3.0  https://github.com/kreincke/mycsrf/)] }}

% Local Variables:
% mode: latex
% fill-column: 80
% End:
}

%\thanks{den Autoren von KOMA-Script und denen von Jurabib}
\maketitle
%%-- end(titlepage)
%\nocite{*}

\begin{abstract}
Das Werk / The work\footcite[][]{Viesel2006a} \\
\noindent \itshape
\ldots Die zweite Häfte des Buches führt allgemein in GNU/Linux ein, ein
weiteres Viertel in DRM, Formate und verwandte kollaborative Projekte. Und das
erste Viertel skizziert die Geschichte von Open Source und die Idee einiger OS
Lizenzen, durchaus mit interessanten, wenn auch nicht im einzelnen belegten
Details.
\\
\noindent
\ldots The second half of the book introduces into GNU/Linux, another quarter
discusses the problem of DRM, formats, and linked collaborative projects. The
first quarter summarizes the history of Open Source and some ideas of OS
licenses and mentions special aspects - unfortunately without referring to its
sources.
\end{abstract}
\footnotesize
%\tableofcontents
\normalsize

\section{Line of Thought}

Das möchte möchte erkennbar in alle relevanten Felder des Linux-Users einführen:
\begin{itemize}
  \item Als erstes skizziert Viesel die \enquote{Geschichte der Freien
  Software}\footcite[vgl.][11ff]{Viesel2006a}
  \item Dann beschreibt er kurz die Unterschiede zwischen dem
  \enquote{kontinentaleuropäischen
  Autorenrecht}\footcite[vgl.][46ff]{Viesel2006a} und dem
  \enquote{angloamerikanischen Copyright}\footcite[vgl.][53ff]{Viesel2006a}
  \item Sodann spezifziert er verschiedenen
  Lizenzen\footcite[vgl.][58ff]{Viesel2006a}, vornehmlich die
  verschiedenen GPL Versionen und das \enquote{Prinzip des
  Copyleft}\footcite[vgl.][60ff]{Viesel2006a}, die
  BSD-Lizenz\footcite[vgl.][67f]{Viesel2006a} und die Nicht-Software-Lizenzen
  \enquote{Gnu Free Documentatione
  License}\footcite[vgl.][68f]{Viesel2006a} und das Gefüge
  \enquote{Creative-Commons-Lizenzen}\footcite[vgl.][69ff]{Viesel2006a}.
  \item Ferner ergänzt Viesel den theoretischen Teil mit einer
  Spezifikation von DRM, also über \enquote{Kopiersperren und
  Kontrollsystemen}\footcite[vgl.][69ff]{Viesel2006a} \ldots
  \item \ldots und einer Auflistung genereller freier nicht unbedingt auf
  Software ausgelegter Projekte wie Wikipedia oder
  Gutenberg\footcite[vgl.][88ff]{Viesel2006a}
  \item Schließlich spezifiziert Viesel noch verschiedene
  Datenformate\footcite[vgl.][113ff]{Viesel2006a} \ldots
  \item \ldots um dann den Linux-Computer an sich und seine
  Killerapplikationen zu erläutern\footcite[vgl.][141ff]{Viesel2006a}
  \item 
\end{itemize}
Insgesamt möchte dieses Buch eine allumfassende Einführung sein. Leider
verzichtet Viesel durchgängig darauf, seine Aussagen zu belegen. Das macht seine
Arbeit nur begrenzt verwertbar.

\section{Specific Aspects}

\subsection{Zur Genese der Freien Software}

\subsubsection{Ausgangspunkt I}
Als Ausgangspunkt sieht auch Viesel die Kultur des freien Austausches von
Informationen und Quellcode, wie es in den 1950er und 1960er Jahren in
universitäterer Tradition gepflegt wurde, will sagen: die
\enquote{Hackerkultur} als eine \enquote{[\ldots] Lebensphilosophie, die das
Miteinanderteilen,  Offenheit, Dezentralisierung und das Praxisgebot [\ldots] in
den Vordergrund (stellte)}\footcite[vgl.][11f]{Viesel2006a}

\subsubsection{Eingriffe I}

Mit den 1970er Jahren setzt Viesel eine 'Popularisierung' der bis dahin er
Spezialisten vorbehaltenen Idee 'Computer' an, wie sie sich in der allgemeinen
Bereitstellung und Verbreitung des
'PDP-10'\footcite[vgl.][14]{Viesel2006a} und des
'Altair'\footcite[vgl.][16]{Viesel2006a} manifestierte. Darauf aufsetzend resp.
anknüpfend beginnt - gewissermaßen als \enquote{Geburt der
Computerindustrie} die Geschichte von Bill Gates und Microsoft als Beigabe
von Basic zum Altair\footcite[vgl.][17f]{Viesel2006a} und die Geschichte der
Firma 'Apple'\footcite[vgl.][19]{Viesel2006a}. 

Was daraus resp. dahinein zu lesen ist, ist die These, dass Software einen
eigenständigen Wert bekommt.

\subsubsection{Eingriffe II}

Sodann erzählt Viesel - unter der Überschrift \enquote{Richard Stallmans
Offenbarungserlebnis} -, dass die \enquote{Geburt} der Idee einer
\enquote{politischen Bewegung für Freie Software} letztlich über einen
\enquote{Papierstau im Laserdrucker} am MIT in Cambridge bei Boston
ausgelöst worde sei\footcite[vgl.][20ff]{Viesel2006a}, infolge dessen Stallman
zur Behebung des Übels den Quellcode erhalten wollte, sozusagen auf kleinem
Diesntwege, \enquote{von Softwareentwickler zu
Softwareentwickler}\footcite[vgl.][21f]{Viesel2006a} - was allerdings zu
der berühmten Ablehnung geführt haben soll, die zu der Einsicht führte, dass
\enquote{[\ldots] Geheimhaltungsvereinbarungen ihre Opfer
haben}\footcite[vgl.][22]{Viesel2006a}, in diesem Fall RMS.

\subsubsection{Ausgangspunkt II}
Der zweite Ausgangspunkt ist auch hier Unix und die Sprache C, die zwischen ab
1969 und 1973 entwickelt\footcite[vgl.][25f]{Viesel2006a} worden sind, aber ob
der kartellrechtlichen Begrenzungen von AT\&T\footcite[vgl.][26]{Viesel2006a}
nicht kommerziell verwendet werden konnten. Deshalb sei ab 1975 \enquote{[\ldots]
Unix in der Version 6 mitsamt dem Quellcode für den Preis der Datenträger
einigen Universitäten zur Verfügung gestellt (worden)}, was insbesondere
an der Berkeley Universität auf fruchtbaren Boden
fiel\footcite[vgl.][26]{Viesel2006a}, sodass zuletzt in den 1990er Jahren mit
dem 386BSD und den Folge Systemen FreeBSD oder OpenBSD sogar gänzlich freie
Versionen daraus entwickelt werden konnten\footcite[vgl.][27]{Viesel2006a}.

\subsubsection{Eingriffe III}

Notwendig wurde die Nachentwicklung von freien Unix's durch die Entscheidung von
AT\&T zu Beginn der 1980er Jahre, Unix zu
proprietarisieren\footcite[vgl.][27]{Viesel2006a}.

\subsubsection{Eingriffe IV}

Als durchaus \enquote{kritische} Antwort auf die
\enquote{Entwicklungvon Unix} habe Stallman 1983 angekündigt, GNU als
\enquote{komplett freies Betriebnssystem} zu
entwickeln\footcite[vgl.][27]{Viesel2006a}. Voraus gegangen sei dem der
\enquote{Symbolics-Krieg}, bei dem Stallman proprietäre Neuerung an der
Sprache LISP - durchaus als \enquote{Kampf gegen
Windmühlen}\footcite[vgl.][31]{Viesel2006a} - nachprogrammiert hatte, was
letztlich aber über das Urheberrecht beendet werden
musste\footcite[vgl.][29f]{Viesel2006a}

\subsubsection{Antwort}

Damit hat Stallman die Proprietarisierung 3x persönlich als eine
Einschränkung getroffen, beim Druckertreiber, bei seiner gelibeten Sprache LISP
und durch das Unix-Umfeld. Die Antwort war das GNU Projekt, 1983
angekündigt\footcite[vgl.][27]{Viesel2006a}, ab 1984 um der eigenen
Unabhängigkeit willen privat vorangetrieben\footcite[vgl.][31]{Viesel2006a} und
1985 mit der Fertigstellung des Editors 'Gnu Emacs' und der Veröffentlichung des
'Gnu Manifesto' konkretisiert\footcite[vgl.][32]{Viesel2006a}

Die erste GNU Lizenz war 1985 noch die \enquote{Gnu Emacs License}, die
allerdings - verglichen mit der späteren GPL - noch schärferer Anfoderungen an
die Emacs User stellte\footcite[vgl.][33]{Viesel2006a}: \enquote{Die (Gnu Emacs)
Lizenz sah vor, dass jeder das Programm frei kopieren, es verändern und dieser
veränderte Version weiterverbreiten durfte, jedoch nur, wenn er auf das
Urheberrecht an der modifizierten Version verzichtete und die Veränderungen
öffentlich machte}\footcite[][33]{Viesel2006a}.

Aus dieser GNU Emacs Lizenz wurde ab 1986 die \enquote{GNU General Public
License} entwickelt und 1989 in der ersten Version
veröffentlicht\footcite[vgl.][34]{Viesel2006a}:

\begin{quote}
\enquote{Ein wesentlicher Unterschied zwischen der neuen Lizenz und den
Nutzungsbedingungen der Emacs-Kommune betraf vor allem die Rückmeldung von
Änderungen an eine zentrale Koordinierungsstelle, den Administrator. Stallman
war zum Schluß gekommen, dass eine zentrale Koordinierung mit dem Prinzip
'gleiche Rechte für alle' unvereinbar war. In frr General Public License gilt
folgendes Prinzip: So lange die Änderun gen jederzeit eingehesehen werden
können, gibt es keinerlei Verpflichtung, diese beispielsweise Stallmann zur
Verfügung zu stellen
}\footcite[][34]{Viesel2006a}
\end{quote}

[KR: So richtig der Wechsel in Sache Veröffentlichungspflicht ist, so falsch
ist der Ansazu der GPL hier ausgedrückt. Es gibt keinerlei unbedingte Pflicht
der Veröffentlichung von Veränderungen. Und schon gar nicht gibt es die Pflicht
zur 'Einsichtnahme'. Was es gibt ist die bedingte Pflicht der Veröffentlichung
unter denselben Rechten: bedingt ist diese Pflicht insofern, als sie an den Akt
der Weitergabe geknüpft ist!!!]

Historisch gesehen gab es dann nach GNU emacs und der Lizenz zwei weitere grosse
Meilensteine, den GNU C Compiler 1987\footcite[vgl.][34]{Viesel2006a}, durch
Tiemann binnen 6 Monaten zum C+(+?) Compiler
erweitert\footcite[vgl.][34 VBiesel schreibt nur vom C+-Compiler,
ist das richtig?]{Viesel2006a}, und den GNU Project Debugger
1990\footcite[vgl.][35]{Viesel2006a} bevor als Kernel Linux hinzugekommen
sei\footcite[vgl.][35]{Viesel2006a}, das um freie Programme ergänzt zu
Linux-Distributionen führtefootcite[vgl.][36]{Viesel2006a}, deren Popularität
Stallman dazu veränlasste, auf deren Bezeichnung als \enquote{GNU/Linux} zu
beharren\footcite[vgl.][37]{Viesel2006a}

Weitere Meilensteine des freien Betriebssystems seien dann Ende der 1990er
Jahren zur Jahtausendwende KDE, GNOME und vor allem die Freigabe von Staroffice
als 'Openoffice' gewesen\footcite[vgl.][38]{Viesel2006a} und ab 1998 auch die
Freigabe des Netscape Internetbrowser\footcite[vgl.][41]{Viesel2006a}, aus dem
dann später Firefox entstanden ist\footcite[vgl.][41]{Viesel2006a}.


\subsubsection{Von Free Software zu Open Source}

Im Jahre 1998 habe es dann einen \enquote{Freeware-Gipfel} gegeben,
\enquote{[\ldots] zu dem Richard Stallman wegen seiner mangelnden
Konpromissbereitschaft nicht eingeladen wurde [\ldots]}. Dort sei dann der
von \enquote{Christine Peterson} erfundene, und von \enquote{Eric
Raymond} propagierte Begriff \enquote{Open Source} als die bessere
Alternative sozusagen verabschiedet. Der Begriff wird aber von RMS abgelehnt,
\enquote{[\ldots] weil er, um sich Unternehmen anzudienen, die Freiheit in
den Hintergrund rücke}\footcite[vgl.][41]{Viesel2006a}.

Zum Beweis der Ablehnung wird RMS ohne Qiellenangabe zitiert: \enquote{Wir
müssen nicht verzweifelt versuchen, mit Unternehmen zusammenzuarbeiten,
weil wir gezeigt haben, was wir können; wir brauchen auch nicht unsere
Ziele zu verraten}\footcite[vgl.][41]{Viesel2006a}

Auf der Basis dea Open Source Begriffes ist dann jedenfalls 1998 die Open Source
Initiative von Eric Raymond \enquote{gestartet} worden, die mit der Open
Source Definition - erstellt von Bruce Perens -
arbeitet\footcite[vgl.][41]{Viesel2006a}. Allerdings habe die OSI
\enquote{[\ldots] nie die Bedeutung der Free Software Foundation erreicht}
und Bruce Perens habe die OSI sogar verlassen, \enquote{[\ldots] weil er
deren oppositionelle Einstellung zur Free Software Foundation nicht für
sinnvoll hielt}\footcite[vgl.][41]{Viesel2006a}.

\subsection{Urheberrecht versus Copyright}

Allgemein könne man sagen, dass sich \enquote{das Recht am geistigen
Eigentum [\ldots] in zwei eigenständigen Traditionen entwickelt (habe),
dem angloamerikanischen Copyright und dem kontinentaleuropäischen Droit
d'auteur, also dem Autorenrecht}\footcite[vgl.][45]{Viesel2006a}

\subsubsection{das kontinentaleuropäischen Autorenrecht}

Das europäische und insbesondere das deutsche Urheberrecht ist dadurch
gekennzeichnet, dass ein \enquote{[\ldots] unzertrennliches Band zwischen Urheber
und Werk bestehen (bleibt)}, geregelt durch das
\enquote{Urheberpersönlichkeitsrecht}: der Urheber bestimmt über Umfang und
Art der Verwendung. Und er verliert die Rechte \enquote{an seiner geistige
Schöpfung} selbst dann nicht, wenn er alle Verwertungsrechte abgetreten
hat\footcite[vgl.][48]{Viesel2006a}

\subsection{Lizenzen}

\subsubsection{GPL und Coypleft}

Viesel erzählt die Geschichte, derzufolge der begriff 'Copyleft' durch ein
Wortspiel von Don Hopkins auf einem Aufkleber entstanden sein soll: \enquote{Aus
Copyright (R), All Rights Reserved} wurde da \enquote{Aus
Copy\textbf{left} (\textbf{L}), All Rights
Re\textbf{v}er\textbf{s}ed}\footcite[vgl.][60]{Viesel2006a}

Coypleft hier wörtlich zitiert nach Stallmans Originaltext 'Was ist Copyleft'?
[KR Lieber das Oroginal nehmen]

Interessant der implizite Hinweis auf 'Paying by Doing': Viesel verweist auf den
Rechtsanwalt 'Mark Fischer', der an der Entstehung der GPL aus der GNU Emacs
License beteiligt war und der die Idee so zusammengefasst haben soll, dass die
\enquote{[\ldots] revolutionäre (Copyleft) Klausel von jedem, der ein Programm an
die eigenen Bedürfnisse anpassen (wolle), einen Preis}: \enquote{Statt aber
einen Preis in Geld zu verlangen wird als Gegenleistung die Veröffentlichung der
gemachten Änderung gefordert}\footcite[vgl.][61]{Viesel2006a}.
[Diese Veröffentlichungspflicht ist jedoch 'nur in abgeschäwchter Form in die
GPL eingegangen: nur wenn der modifizierte Code weitergegeben wird, muss der
Quellcode dem Empfänger unter denselben Rechten und Pflichten übergeben werden]

Nach Erwähnung der Veröffentlichungsdaten für die GPL - 1989 die Version
1.0, 1991 die Version 2.0\footcite[vgl.][61]{Viesel2006a} - erläutert Viesel
dann den Grundgedanken der juristischen
Konstruktion\footcite[vgl.][62ff]{Viesel2006a}: Die GPL \enquote{[\ldots]
(lehne) die totale Freiheit (ab)}, ein \enquote{Laisser-faire} sei
nicht vorgesehen; \enquote{die Software des Projektes GNU (sei) ganz
gezielt keine gemeinfreie Software, mit der jeder machen kann, was er
will}\footcite[vgl.][63]{Viesel2006a}. Vielmehr werde GPL Software zuerst
unters Copyright gestellt - ein im amerikanischen Coypright notwendiger Schritt,
um nicht der Public Domain anheim zu fallen -, um dann zur Sihcerung der Rechte
die Weitergabe und Modifikation an bestimmte Verpflichtung zu
knüpfen\footcite[vgl.][62f]{Viesel2006a}

Für die Frage nach der Auswirkung auf das abgeleitete Werk konstatiert Viesel,
dass die GPL Software \enquote{[\ldots] abwertend als 'viral' oder
'infektiös' beschrieben worden (sein)} - und zwar basierend auf der
Aussage: \enquote{Wird Freie Software als Bestandteil in ein größeres
Programm eingebunden, muss die gesamte resultierende Software unhter den
Bedingungen der GPL verbreitet werden, ansonsten erlischt automatisch das
Nutzungsrecht der eingebundenen Freien
Software}\footcite[vgl.][63]{Viesel2006a}.

\subsubsection{LGPL}
\begin{quote}\enquote{Im Unterschied zur GNU General Public License dürfen
Programme jedoch eine unter der LGPL lizenzierte Software vorübergehend
einbinden, ohne dass das ganze Programm deshalb unter die Freie Lizenz
gestellt werden müsse.}\footcite[vgl.][66]{Viesel2006a}
\end{quote}

\subsubsection{BSD}

Der Hauptunterschied zur GPL sei hier, \enquote{[\ldots] dass die
BSD-Lizenz keine Copyleft (kenne)}, dass man bei BSD lizenziertem Code
\enquote{[\ldots] nicht verpflichte (sei), den Quellcode seines veränderten
Programms zu veröffentlichen}: \enquote{Somit eignet sich unter der
BSD-Lizenz stehende Sogftware auch als Vorlage für kommerzielle oder
teilweise proprietäre Produkte}. Und das habe - so behauptet Viesel OHNE
NACHWEIS - habe Microsoft für sein Betriebssystem Windows genutzt, indem MS
\enquote{[\ldots] große Mengen Code aus dem Betriebssystem FreeBSD [dafür]
verwenden konnte}\footcite[vgl.][67]{Viesel2006a}.

WICHTIG: [Lizenzkompatibilität]

Es habe laut Viesel in der \enquote{ursprünglichen BSD Lizenz} die Pflicht
gegeben, \enquote{[\ldots] bei jeglicher Werbung darauf zu verweisen, dass
in dem beworbenen Produkt an der University of California in Berkeley
erstellte Software enthalten [Seitenumbruch] (sei)} - was als
\enquote{advertising clause} bezeichnet werde und inhaltlich mit der GPL
nicht vereinbar sei. Ab 1999 sei diese Klausel aufgehoben
worden.\footcite[vgl.][67f]{Viesel2006a}

\subsection{KDE versus GNOME}

Viesel erwähnt auch die anfänglichen Lizenzirritationen im KDE Projekt, die
sogar indirekt als Geburtshelfer für GNOME gewirkt haben: KDE beruht auf der
Bibliothek QT, erstellt von der \enquote{norwegischen Firma Trolltech}.
Diese Basisbibliothek wurde aber nicht als Freie Software bereitgestellt, so
dass die frei Oberfläche KDE (anfänglich) eben im Kern nicht frei gewesen sei.
Dieses habe dazu geführt, dass GNOME als echzte GPL basierte Alternative gepusht
worden sei. Erst im Jahre 2000 habe sich Trolltech entschieden, \enquote{[\ldots]
QT in einer freien Version unter der [\ldots] GPL zur Verfügung zu
stellen}, sodass auch KDE in freien Distributionen integriert werden
konnte\footcite[vgl.][165 s.a. 167f]{Viesel2006a}.

\small
\bibliography{../bibfiles/oscResourcesDe}

\end{document}
