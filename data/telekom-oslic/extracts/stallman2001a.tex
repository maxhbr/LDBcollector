% Telekom osCompendium extract template
%
% (c) Karsten Reincke, Deutsche Telekom AG, Darmstadt 2011
%
% This LaTeX-File is licensed under the Creative Commons Attribution-ShareAlike
% 3.0 Germany License (http://creativecommons.org/licenses/by-sa/3.0/de/): Feel
% free 'to share (to copy, distribute and transmit)' or 'to remix (to adapt)'
% it, if you '... distribute the resulting work under the same or similar
% license to this one' and if you respect how 'you must attribute the work in
% the manner specified by the author ...':
%
% In an internet based reuse please link the reused parts to www.telekom.com and
% mention the original authors and Deutsche Telekom AG in a suitable manner. In
% a paper-like reuse please insert a short hint to www.telekom.com and to the
% original authors and Deutsche Telekom AG into your preface. For normal
% quotations please use the scientific standard to cite.
%
% [ File structure derived from 'mind your Scholar Research Framework' 
%   mycsrf (c) K. Reincke CC BY 3.0  http://mycsrf.fodina.de/ ]

%
% select the document class
% S.26: [ 10pt|11pt|12pt onecolumn|twocolumn oneside|twoside notitlepage|titlepage final|draft
%         leqno fleqn openbib a4paper|a5paper|b5paper|letterpaper|legalpaper|executivepaper openrigth ]
% S.25: { article|report|book|letter ... }
%
% oder koma-skript S.10 + 16
\documentclass[DIV=calc,BCOR=5mm,11pt,headings=small,oneside,abstract=true, toc=bib]{scrartcl}

%%% (1) general configurations %%%
\usepackage[utf8]{inputenc}

%%% (2) language specific configurations %%%
\usepackage[]{a4,ngerman}
\usepackage[ngerman, german, english]{babel}
\selectlanguage{english}

%language specific quoting signs
%default for language emglish is american style of quotes
\usepackage{csquotes}

% jurabib configuration
\usepackage[see]{jurabib}
\bibliographystyle{jurabib}
% Telekom osCompendium English Jurabib Configuration Include Module 
%
% (c) Karsten Reincke, Deutsche Telekom AG, Darmstadt 2011
%
% This LaTeX-File is licensed under the Creative Commons Attribution-ShareAlike
% 3.0 Germany License (http://creativecommons.org/licenses/by-sa/3.0/de/): Feel
% free 'to share (to copy, distribute and transmit)' or 'to remix (to adapt)'
% it, if you '... distribute the resulting work under the same or similar
% license to this one' and if you respect how 'you must attribute the work in
% the manner specified by the author ...':
%
% In an internet based reuse please link the reused parts to www.telekom.com and
% mention the original authors and Deutsche Telekom AG in a suitable manner. In
% a paper-like reuse please insert a short hint to www.telekom.com and to the
% original authors and Deutsche Telekom AG into your preface. For normal
% quotations please use the scientific standard to cite.
%
% [ File structure derived from 'mind your Scholar Research Framework' 
%   mycsrf (c) K. Reincke CC BY 3.0  http://mycsrf.fodina.de/ ]

% the first time cite with all data, later with shorttitle
\jurabibsetup{citefull=first}

%%% (1) author / editor list configuration
%\jurabibsetup{authorformat=and} % uses 'und' instead of 'u.'
% therefore define your own abbreviated conjunction: 
% an 'and before last author explicetly written conjunction

% for authors in citations
\renewcommand*{\jbbtasep}{ a.\ } % bta = between two authors sep
\renewcommand*{\jbbfsasep}{, } % bfsa = between first and second author sep
\renewcommand*{\jbbstasep}{, a.\ }% bsta = between second and third author sep
% for editors in citations
\renewcommand*{\jbbtesep}{ a.\ } % bta = between two authors sep
\renewcommand*{\jbbfsesep}{, } % bfsa = between first and second author sep
\renewcommand*{\jbbstesep}{, a.\ }% bsta = between second and third author sep

% for authors in literature list
\renewcommand*{\bibbtasep}{ a.\ } % bta = between two authors sep
\renewcommand*{\bibbfsasep}{, } % bfsa = between first and second author sep
\renewcommand*{\bibbstasep}{, a.\ }% bsta = between second and third author sep
% for editors  in literature list
\renewcommand*{\bibbtesep}{ a.\ } % bte = between two editors sep
\renewcommand*{\bibbfsesep}{, } % bfse = between first and second editor sep
\renewcommand*{\bibbstesep}{, a.\ }% bste = between second and third editor sep

% use: name, forname, forname lastname u. forname lastname
\jurabibsetup{authorformat=firstnotreversed}
\jurabibsetup{authorformat=italic}

%%% (2) title configuration
% in every case print the title, let it be seperated from the 
% author by a colon and use the slanted font
\jurabibsetup{titleformat={all,colonsep}}
%\renewcommand*{\jbtitlefont}{\textit}

%%% (3) seperators in bib data
% separate bibliographical hints and page hints by a comma
\jurabibsetup{commabeforerest}

%%% (4) specific configuration of bibdata in quotes / footnote
% use a.a.O if possible
\jurabibsetup{ibidem=strict}
% replace ugly a.a.O. by translation of ders., a.a.O.
\AddTo\bibsgerman{
  \renewcommand*{\ibidemname}{Id., l.c.}
  \renewcommand*{\ibidemmidname}{id., l.c.}
}
\renewcommand*{\samepageibidemname}{Id., ibid.}
\renewcommand*{\samepageibidemmidname}{id., ibid.}


%%% (5) specific configuration of bibdata in bibliography
% ever an in: before journal and collection/book-tiltes 
\renewcommand*{\bibbtsep}{in: }
\renewcommand*{\bibjtsep}{in: }
% ever a colon after author names 
\renewcommand*{\bibansep}{: }
% ever a semi colon after the title
% \AddTo\bibsgerman{\renewcommand*{\urldatecomment}{Referenzdownload: }}
\renewcommand*{\bibatsep}{; }
% ever a comma before date/year
\renewcommand*{\bibbdsep}{, }

% let jurabib insert the S. and p. information
% no S. necessary in bib-files and in cites/footcites
\jurabibsetup{pages=format}

% use a compressed literature-list using a small line indent
\jurabibsetup{bibformat=compress}
\setlength{\jbbibhang}{1em}

% which follows the design of the cites and offers comments
\jurabibsetup{biblikecite}

% print annotations into bibliography
\jurabibsetup{annote}
\renewcommand*{\jbannoteformat}[1]{{ \itshape #1 }}

%refine the prefix of url download
\AddTo\bibsgerman{\renewcommand*{\urldatecomment}{reference download: }}

% we want to have the year of articles in brackets
\renewcommand*{\bibaldelim}{(}
\renewcommand*{\bibardelim}{)}

% in english version Nr. must be replaced by No.
\renewcommand*{\artnumberformat}[1]{\unskip,\space No.~#1}
\renewcommand*{\pernumberformat}[1]{\unskip\space No.~#1}%
\renewcommand*{\revnumberformat}[1]{\unskip\space No.~#1}%


%Reformatierung Seriestitels and Seriesnumber
\DeclareRobustCommand{\numberandseries}[2]{%
\unskip\unskip%,
\space\bibsnfont{(=~#2}%
\ifthenelse{\equal{#1}{}}{)}{, [Vol./No.]~#1)}%
}%


% Local Variables:
% mode: latex
% fill-column: 80
% End:


% language specific hyphenation
% Telekom osCompendium osHyphenation Include Module
%
% (c) Karsten Reincke, Deutsche Telekom AG, Darmstadt 2011
%
% This LaTeX-File is licensed under the Creative Commons Attribution-ShareAlike
% 3.0 Germany License (http://creativecommons.org/licenses/by-sa/3.0/de/): Feel
% free 'to share (to copy, distribute and transmit)' or 'to remix (to adapt)'
% it, if you '... distribute the resulting work under the same or similar
% license to this one' and if you respect how 'you must attribute the work in
% the manner specified by the author ...':
%
% In an internet based reuse please link the reused parts to www.telekom.com and
% mention the original authors and Deutsche Telekom AG in a suitable manner. In
% a paper-like reuse please insert a short hint to www.telekom.com and to the
% original authors and Deutsche Telekom AG into your preface. For normal
% quotations please use the scientific standard to cite.
%
% [ File structure derived from 'mind your Scholar Research Framework' 
%   mycsrf (c) K. Reincke CC BY 3.0  http://mycsrf.fodina.de/ ]
%


\hyphenation{rein-cke}
\hyphenation{Rein-cke}
\hyphenation{OS-LiC}
\hyphenation{ori-gi-nal}
\hyphenation{bi-na-ry}
\hyphenation{Li-cence}
\hyphenation{li-cence}

% Local Variables:
% mode: latex
% fill-column: 80
% End:


%%% (3) layout page configuration %%%

% select the visible parts of a page
% S.31: { plain|empty|headings|myheadings }
%\pagestyle{myheadings}
\pagestyle{headings}

% select the wished style of page-numbering
% S.32: { arabic,roman,Roman,alph,Alph }
\pagenumbering{arabic}
\setcounter{page}{1}

% select the wished distances using the general setlength order:
% S.34 { baselineskip| parskip | parindent }
% - general no indent for paragraphs
\setlength{\parindent}{0pt}
\setlength{\parskip}{1.2ex plus 0.2ex minus 0.2ex}


%%% (4) general package activation %%%
%\usepackage{utopia}
%\usepackage{courier}
%\usepackage{avant}
\usepackage[dvips]{epsfig}

% graphic
\usepackage{graphicx,color}
\usepackage{array}
\usepackage{shadow}
\usepackage{fancybox}

%- start(footnote-configuration)
%  flush the cite numbers out of the vertical line and let
%  the footnote text directly start in the left vertical line
\usepackage[marginal]{footmisc}
%- end(footnote-configuration)

\begin{document}

%% use all entries of the bliography

%%-- start(titlepage)
\titlehead{Literaturexzerpt}
\subject{Autor(en): Stallman / Stallman2001a}
\title{Titel: Free Software: Freedom and Cooperation}
\subtitle{Jahr: 2001 / 2002 }
\author{K. Reincke% Telekom osCompendium License Include Module
%
% (c) Karsten Reincke, Deutsche Telekom AG, Darmstadt 2011
%
% This LaTeX-File is licensed under the Creative Commons Attribution-ShareAlike
% 3.0 Germany License (http://creativecommons.org/licenses/by-sa/3.0/de/): Feel
% free 'to share (to copy, distribute and transmit)' or 'to remix (to adapt)'
% it, if you '... distribute the resulting work under the same or similar
% license to this one' and if you respect how 'you must attribute the work in
% the manner specified by the author ...':
%
% In an internet based reuse please link the reused parts to www.telekom.com and
% mention the original authors and Deutsche Telekom AG in a suitable manner. In
% a paper-like reuse please insert a short hint to www.telekom.com and to the
% original authors and Deutsche Telekom AG into your preface. For normal
% quotations please use the scientific standard to cite.
%
% [ File structure derived from 'mind your Scholar Research Framework' 
%   mycsrf (c) K. Reincke CC BY 3.0  http://mycsrf.fodina.de/ ]
%
\footnote{
This text is licensed under the Creative Commons Attribution-ShareAlike 3.0 Germany
License (http://creativecommons.org/licenses/by-sa/3.0/de/): Feel free \enquote{to
share (to copy, distribute and transmit)} or \enquote{to remix (to
adapt)} it, if you \enquote{[\ldots] distribute the resulting work under the
same or similar license to this one} and if you respect how \enquote{you
must attribute the work in the manner specified by the author(s)
[\ldots]}):
\newline
In an internet based reuse please mention the initial authors in a suitable
manner, name their sponsor \textit{Deutsche Telekom AG} and link it to
\texttt{http://www.telekom.com}. In a paper-like reuse please insert a short
hint to \texttt{http://www.telekom.com}, to the initial authors, and to their
sponsor \textit{Deutsche Telekom AG} into your preface. For normal citations
please use the scientific standard.
\newline
{ \tiny \itshape [based on myCsrf (= mind your Scholar Research Framework) 
\copyright K. Reincke CC BY 3.0  https://github.com/kreincke/mycsrf/)] }}

% Local Variables:
% mode: latex
% fill-column: 80
% End:
}

%\thanks{den Autoren von KOMA-Script und denen von Jurabib}
\maketitle
%%-- end(titlepage)
%\nocite{*}

\begin{abstract}
\noindent
\cite[(in:)][]{StaGay2002a} \\
\noindent
\cite[(ist:)][]{Stallman2001a} \\
Das Werk / The work\footcite[][]{Stallman2001a} \\
\noindent \itshape
\ldots  Fasst Aspekte der Idee 'Freie Software' zusammen und erzählt deren
Genese aus persönlicher Sicht von RMS.
\\
\noindent
\ldots This transcript summarizes the main ideas of Free Software and their
genesis: it tells the history of RMS.
\end{abstract}
\footnotesize
%\tableofcontents
\normalsize

\section{Line of Thought}

\subsection{the analogy of recipes}
First, RMS compares the writing and using of compupter codes with the noting
and using of recipes: He aims at the point that the exchange of recipes is an
internal and intionally included part of the art and the act of
cooking\footcite[cf][156]{Stallman2001a}: \enquote{These are the natural things to
do with functionally usefule recipes.}\footcite[][156]{Stallman2001a}

\begin{quote}
\enquote{So imagine what it would be like if recipes were packaged inside black
boxes. You couldn't see what ingredients they're using, let alone change them,
and imagine if you made a copy for a friend, they would call you a pirate a try
to put you in prison for years.
}\footcite[][157]{Stallman2001a}
\end{quote}

\subsection{The structure of his history}

\begin{itemize}
  \item In the beginning Stallman was \enquote{[\ldots] part of a community
  of programmers who shared software}, e.g. \enquote{[\ldots] any of
  it with anybody} as a \enquote{way of life} defined by
  \enquote{cooperation}\footcite[cf][157]{Stallman2001a}.
  \item In this context Stallman also tells the story of his earliest experience
  with a new spirit of thinking: He mentions the Xerox laser printer
  designed as net net printer which ran into a trap, stopped working and
  disturbed the work of many people\footcite[cf][157]{Stallman2001a}:
  \enquote{The printer jammed and nobody saw. So it stayed jammed for a
  long time}\footcite[][157]{Stallman2001a}. Now Stallman tried to help
  themselves, as he often might had done before: He tried to get the source code
  from Xerox for being able to improve the software of the printer. But that
  software \enquote{[\ldots] that ran that printer was not free
  software}\footcite[][158]{Stallman2001a}. Stallman tried to solve the
  problem by using the short an 'abbreviation'. He contacted one of those who
  had a copy of the misdesigned software and tried to get it from developer
  colleague to developer colleague. But - as RMS tells - he only got the answer,
  that his colleague had promised, not to give a copy to
  aanyone\footcite[cf][158]{Stallman2001a}. Stallman summarizes this promizes
  that \enquote{[\ldots] he had promised to refuse to cooperate with just about
  the entire population of the planet
  earth}\footcite[cf][158]{Stallman2001a} - or in other words: \enquote{he
  had signed a non-disclosure agreement}\footcite[][158]{Stallman2001a}:
  \begin{quote}
  \enquote{ Now, this was my first direct encounter with a non-discloure agreement,
  and it taught me an important lesson - a lesson that' important because most
  programmers never learn it. This was my first encounter with a non-disclosure
  agreement, and I was the victim. I, and my whole lab, were the victims. And the
  lesson it taught me was that non-disclosure agreements have victims
  }\footcite[][158]{Stallman2001a}
  \end{quote}
  Stallmans concluded that he refuses to use a piece of software for which he
  had to sign a non-disclosure agreement\footcite[cf][159]{Stallman2001a}
  \item But RMS' environment had not only changed by the growing of non free
  software, but also by the removal of the PDP-10 by more modern computers which
  was one of the \enquote{calamities} by which his \enquote{[\ldots]
  community was destroyed [\ldots and] ultimately [\ldots] wiped
  out}\footcite[cf][157 (richtige Seite!)]{Stallman2001a}. This change
  \enquote{[\ldots] gave (RMS) a moral dilemma}\footcite[cf][159 (richtige
  Seite!)]{Stallman2001a}: Either he had to accept the situation and had to jump
  into the uncooperative world of signed \enquote{non-discolsure
  agreements}\footcite[cf][159]{Stallman2001a} or he had to find an
  \enquote{alternative}\footcite[cf][160]{Stallman2001a} which was nothing
  else writing a new free operating system, called
  GNU\footcite[cf][161]{Stallman2001a}.
  \item Stallmans tells that he take the word GNU because it \enquote{[\ldots]
  is the funniest word in the English language}, pronounced like 'new' and
  therefore designed for good wordplays, like 'gnu' = new operation
  system\footcite[cf][161]{Stallman2001a}. And he tells, that he filled the
  m,eaning of GNU following the recursive tradition which had been established
  during the 60's and 70s by generating the editor TECO and its derivates like
  'Tint' standing for \enquote{Tint Is Not Teco} and which had been
  transferred to the history of 'Emacs' and his derivations like 'Fine' (
  \enquote{Fine Is Not Emacs} ) until 'MINCE' ( \enquote{Mince Is Not
  Complete Emacs} ) or anything else\footcite[cf][161]{Stallman2001a}
  \item Part of the history of RMS is also that he had quit his Job at MIT in
  1984\footcite[cf][162]{Stallman2001a} and that he had to \enquote{[\ldots]
  to make money through (his) work on free software
  [\ldots]}\footcite[cf][162]{Stallman2001a}. His solution was to sell
  \enquote{tapes} containing the software of \enquote{Emacs} - each for
  a price of 150\$\footcite[cf][162]{Stallman2001a}. Stallman himself
  explecitely says that this a an example for the fact that \enquote{free
  software} is \enquote{[\ldots] referring to freedom, not price}. And
  he add, that it is \enquote{[\ldots] not (his) goal [\ldots] to making
  sure programmers got less money}\footcite[cf][163]{Stallman2001a}
  \item Then RMS defines fewee software in the known sense of the four
  freedoms\footcite[cf][163ff]{Stallman2001a}
  \item And RMS again mentions the difference between the \enquote{open
  source movement} and the \enquote{free software movement}: RMS says,
  that the Open Source movement \enquote{[\ldots] only cite the practical
  benefits}, that they \enquote{[\ldots] deny that people are
  entitled to the freedom to share with their neighbour and to see what
  the program's doing and change it if thex don't like it}. Hen underlines
  that they only \enquote{[\ldots] go to companies and say them, 'You might
  make more money if you let people do this'}. Hence RMS concludes that
  the Open Source movement \enquote{[\ldots] lead people in a similar direction
  [as the Free Software movement do], but for totally different - for
  fundamentallly different philosophical reasons} - he states, that
  \enquote{[\ldots] philosophically, there's a tremendous
  disagreement}\footcite[cf][167]{Stallman2001a}. But on the other hand
  RMS doesn't lack to mention that \enquote{the open source movement has
  contributed substantially to our [free software KR;] community, and we work
  together [with them] on practical
  projects}\footcite[cf][167]{Stallman2001a}
  \item Again RMS tells the story of X Window, which had been developed on the
  MIT and released as free software before the software was overtaken,
  proprietarized more or less little changings and bundled into the unix
  packages of the overtaking companies.\footcite[cf][168f]{Stallman2001a} And
  RMS mentions that this was the reason why he invented 
  \enquote{copyleft}\footcite[cf][169]{Stallman2001a}.
  \item After having reffered the main topics of Copy Left as it known, he make
  an intersting proposition: \enquote{Whenever you distribute[sic!]
  anything that contains any piece of this [copylefted KR;] program, that
  whole program[sic!] must be distributed under these same terms, no
  more, no less}\footcite[cf][169]{Stallman2001a}.
  \item The limits of this last statement are again specified by an explanation
  of X: RMS explicitely describes, that he was happy that X came and he decided
  to use it, although it was not licensed under
  GNU\footcite[cf][171]{Stallman2001a}. [KR: Hence, setting something ontop of a
  copylefted program(lirbary)) requires, that the ontop set program must also be
  published as under the same License, but not the used library X-Window. the
  part]
\end{itemize}


\section{Specific Aspects}

\small
\bibliography{../bibfiles/oscResourcesEn}

\end{document}
