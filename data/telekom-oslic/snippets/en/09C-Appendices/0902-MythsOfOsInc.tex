% Telekom osCompendium 'for being included' snippet template
%
% (c) Karsten Reincke, Deutsche Telekom AG, Darmstadt 2011
%
% This LaTeX-File is licensed under the Creative Commons Attribution-ShareAlike
% 3.0 Germany License (http://creativecommons.org/licenses/by-sa/3.0/de/): Feel
% free 'to share (to copy, distribute and transmit)' or 'to remix (to adapt)'
% it, if you '... distribute the resulting work under the same or similar
% license to this one' and if you respect how 'you must attribute the work in
% the manner specified by the author ...':
%
% In an internet based reuse please link the reused parts to www.telekom.com and
% mention the original authors and Deutsche Telekom AG in a suitable manner. In
% a paper-like reuse please insert a short hint to www.telekom.com and to the
% original authors and Deutsche Telekom AG into your preface. For normal
% quotations please use the scientific standard to cite.
%
% [ File structure derived from 'mind your Scholar Research Framework' 
%   mycsrf (c) K. Reincke CC BY 3.0  http://mycsrf.fodina.de/ ]

%


%% use all entries of the bibliography
%\nocite{*}


\section{Some Widespread Open Source Myths}

From the viewpoint of an internet student we have to consider that the web
offers a mass of rumors concerning the nature of open source software
(Licenses). Here are some of the myths\footcite[At least one time even a
scientific legally discussing book is talking about the \enquote{myth around open
source licenses} -- although only as part of  the title: cf][1ff,
especially 209ff]{GuiOvd2006a} we met:
 
\textbf{BE CAREFUL: THIS SECTION MUST THOROUGHLY BE REVIEWED AND REWRITTEN. 
IT'S ONLY AN OUTLINE!!! Do not quote part of it. It must be verified.}

\begin{description}
  \item[open source tries to improve the world ethically] :- No, there's a clear
  ban to exclude persons, groups, purposes. Thus, there is no chance to exclude
  anyone from using open source software because he is an ethical or moralic
  malefactor.
  \item[Changed open source software must be re-published] :- No, in a double
  sense! There are OS licenses which allow the proprietarization of the
  modified code. And even the LGPL and the GPL, which clearly try to prevent
  the proprietarization, do not require generally that a modified code must be
  (re-)published. Only if you give your modfied (L)GPL licensed application as
  binary to anybody, then you have to handover the modified code, too.
  \item[Modified open source software must be given back to the whole community]
  :- No. Again, there are OS licenses which allow the proprietarization of the
  modified code. And even the LGPL and the GPL -- which clearly require, that you
  also publish the modified code, if you give the modified binary to anybody --
  do not require that you distribute your modification around the world. LGPL and
  GPL clearly say that you have to hand over the code to those persons you
  give the binary to. And if you only give your improvement only one person or a
  group of persons, then you must handover your code only to that persons or
  only to all members of that group.
  \item[Published open source software is open for ever] :- No, if this myth
  says that also all future versions will have to be distributed under an open
  source license. The copyright holder ever holds the copyright. They can change
  the licence of next release of its software -- but only for the following
  release, not for the current or for former versions. Those releases, which
  already have been distributed under an open source license, indeed remain
  open.
  \item[Software can either be open source software or proprietary software] :-
  No. The copyright holders themselves can additionally distribute the code
  under other conditions when ever they want to do it. That's not a question of
  the licence, but of the copyright.  
  \item[The opposite of open source software is commercial Software] :- No.
  First, you are also allowed to use the open source software in any commercial
  purpose. There's only one point which is excluded in OSS: you are not allowed
  to ask for a licence fee if you distribute 'open source software'. Second,
  there are many other forms like freeware, public domain software or anything
  else which is neither open source software nor Commercial Software. It's
  pointless to take the question of money as a criterion for distinguish open
  source software and its opposite. Moreover: Proprietary Software as opposite
  of open source software should be defined ex negativo: all kind of software,
  which does not fit the OSD is proprietary.
  \item[open source software prohibits to earn money] :- No, you are allowed to
  invent each business model you want. There's only one exception: you are not
  allowed to ask for a licence fee if you distribute open source software. This
  limitation is based on the open source definition which clearly states that a
  license -- which wants to become an open source license -- \enquote{shall not
  restrict any party from selling or giving away the software as a component of
  an aggregate software distribution containing programs from several different
  sources} and that the license under this circumstances \enquote{[\ldots] shall
  not require a royalty or other fee for such sale}\footcite[cf.][§1]{OSI2012a}.
  If you combine this constraint with the requirements that an open source
  license \enquote{[\ldots] must not restrict anyone from making use of the
  program [\ldots]}\footcite[cf.][§6]{OSI2012a} and that it \enquote{[\ldots]
  must allow distribution in source code as well as compiled form
  [\ldots]}\footcite[cf.][§2]{OSI2012a}, you can generally conclude that none of
  the open source licenses may require a fee for using and/or distributing the
  program. But being paid for the service to install the program, to collect
  and compile a customer specific version, and/or to monitor the environment is
  of course not excluded by this condition.
  
  Historically this mistake might be evoked by Debian: The GNU project missed
  its kernel while the Linux kernel was already distributed as part of
  collections which also include GNU software. Then, in 1983? Ian Murdock was
  supported by RMS and its FSF to build a really free distribution (Debian)
  containg GNU software and the Linux kernel. But Ian Murdock states also, that
  Debian does not want to earn money.
% TODO find sources for indirect citations
% TODO: check, whether OSD requires license fee free distribution
  \item[Modifications of open source software must be marked] :- No. This is not
  a defining postulation of the OSD. The OSD allows licenses to require the mark
  of modifications. But it does not require from all licenses to require the mark
  modifications for being an open source license.
  \item[Modifications of open source software must be marked by your personal
  data] :- No, it is only required to mark modifications so that a reader could
  distinguish the modifications from the original code. It's required for saving
  the integrity of the original author. And therefore it is not required as a
  constitutive criterion by the OSD. It might be that a license additionally
  requires your name. But that is not feature of open source software in general.
  And at least the licenses discussed by us do not require to insert your name.
% TODO: check whether any of our licenses reuire that you mark modifications by
% your personal data / real name  
  \item[The open source Definition determines the conditions to use open source
  software] :- No. The \emph{Open Source Definition} determines which licenses
  are open source licenses, nothing more. The OSD is a set of necessary
  conditions to be an open source license. It determines the freedom and the
  responsibilities of a user as a set of more or less abstract rules. But it
  does not constitute a set of sufficient tasks which a user has to perform for
  fulfilling any open source license. Open source licenses may differ by
  instantiating the OSD criteria. So, if you want to know what you have to do to
  fulfill a license, you have to go back to the real license of that software
  you are using.
\end{description}

%\bibliography{../bibfiles/oscResourcesEn}

% Local Variables:
% mode: latex
% fill-column: 80
% End:
