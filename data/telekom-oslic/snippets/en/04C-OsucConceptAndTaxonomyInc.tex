% Telekom osCompendium 'for being included' snippet template
%
% (c) Karsten Reincke, Deutsche Telekom AG, Darmstadt 2011
%
% This LaTeX-File is licensed under the Creative Commons Attribution-ShareAlike
% 3.0 Germany License (http://creativecommons.org/licenses/by-sa/3.0/de/): Feel
% free 'to share (to copy, distribute and transmit)' or 'to remix (to adapt)'
% it, if you '... distribute the resulting work under the same or similar
% license to this one' and if you respect how 'you must attribute the work in
% the manner specified by the author ...':
%
% In an internet based reuse please link the reused parts to www.telekom.com and
% mention the original authors and Deutsche Telekom AG in a suitable manner. In
% a paper-like reuse please insert a short hint to www.telekom.com and to the
% original authors and Deutsche Telekom AG into your preface. For normal
% quotations please use the scientific standard to cite.
%
% [ Framework derived from 'mind your Scholar Research Framework' 
%   mycsrf (c) K. Reincke 2012 CC BY 3.0  http://mycsrf.fodina.de/ ]
%


%% use all entries of the bibliography
%\nocite{*}

\chapter{Open Source Use Cases: Concept and Taxonomy}\label{sec:OSUCdeduction}

\footnotesize \begin{quote}\itshape This chapter establishes our concept of
\emph{open source use cases} as a classification system for to-do lists. The
conditions of a specific license, in the context of a par\-ti\-cu\-lar
\emph{open source use case}, shall be satisfiable by following the corresponding
to-do list. Additionally this chapter introduces a taxonomy for these \emph{open
source use cases}. Later on, this taxonomy will organize the \emph{Open Source
Use Case Finder}.
\end{quote}
\normalsize{}

After all these introductory remarks, we can summarize our idea. We know that
the right to use open source software depends on the tasks required by the open
source licenses. As opposed to commercial licenses, you can not buy the right to
use a piece of open source software by paying money. It is embedded into the
\emph{Open Source Definition} that the right to use the software may not be
sold. The OSD states first that an open source license may \enquote{[\ldots]
not restrict any party from selling or giving away the software as a component
of (any) aggregate software distribution}, and adds second in the same context
that an open source license \enquote{[\ldots] shall not require a royalty or
other fee for such sale}\footcite[cf.][\nopage wp §1]{OSI2012a}.

However, it would be wrong to conclude that you are automatically allowed to use
open source software without any service in return: generally you have to do
something to gain the right to use the software. In other words: open source
software is covered by the idea of ’paying by doing’. Accordingly, open source
li\-cen\-ses describe specific circumstances under which the user must execute
some tasks in order to be compliant with the licenses. So, if we want to offer
to-do lists for fulfilling license conditions, we must consider these tasks and
circumstances.

In practice, such circumstances are not linear and simple. They contain
combinations of (sometimes context sensitive) conditions which can be grouped
into classes of tokens. Such a class of tokens might denote a feature of the
software itself---such as being an application or a library. Or it can refer to
the circumstances of using the software, such as 'using the software only for
yourself' or 'distributing the software also to third parties'.

At the end, we want to determine a set of specific OSUCs---the \emph{open source
use cases}. And we want to deliver for each of these OSUCs and for each of the
considered open source licenses one list of actions which fulfills the license
in that context\footnote{Fortunately, sometimes one task list fulfills the
conditions of more than one use case---a welcome reduction of complexity}.

Such an \emph{open source use case} shall be considered as a set of tokens
describing the circumstances of a specific usage. Hence, to begin, we must
specify the relevant classes of tokens, before we can determine the valid
combinations of these tokens---our \emph{open source use cases}. Finally, based
on the tokens, we generate a taxonomy in the form of a tree. This tree will
become the base of the \emph{Open Source Use Case Finder} which will be offered
in the next chapter, and which leads you to your specific OSUC by evaluating
just a few questions and answers.

There are only a handful of tokens which are relevant to the circumstances of
open source software licenses:

\label{OsucTokens}
\begin{itemize}
  \item The \textbf{\underline{type} of the open source software}: On the one
  hand, we regard code snippets, modules, libraries and plugins, and on the
  other hand, autonomous applications, programs and servers. We will take the
  word ’snimolis’ for the first set, and ’proapses’ for the second. This is
  necessary, as we are not only talking about libraries and applications in the
  everyday sense, but rather in the broadest sense\footnote{Of course, our newly
  introduced concepts of 'snimoli' and 'proapse' are not absolutely one of the
  most elegant words. So, initially we tried to talk about 'applications' and
  'libraries', although in our context these words should denote more, than they
  traditionally do. But we couldn't minimize the irritations of our
  interlocutors. Too often we had to remind them that we were not talking about
  applications and libraries in the strict sense of the words. Finally we
  decided to find our own words---and to stay open for better proposals ;-) }.
  More specifically, we will ask you, whether the open source software you want
  to use, is an includable code snippet, a linkable module or library, or a
  loadable plugin, or whether it is an autonomous application or server which
  can be executed or processed. In the first case, the answer should be 'it is a
  \underline{snimoli}', in the second 'it is a \underline{proapse}'.

  \item The \textbf{\underline{state} of the open source software}: It might be
  used exactly as one has received it. Or it can be modified, before being used.
  More specifically, we will ask you, whether you want to leave the open source
  software as you have received it, or whether you want to modify it before
  using and/or distributing it to 3rd parties. In the first case, the answer
  should be '\underline{unmodified}', in the second '\underline{modified}'.
  
  \item The \textbf{usage \underline{context} of the open source
  software}: On the one hand you might use the received open source software as a
  readily prepared application. On the other hand you might embed the received
  open source into a larger application as one of its components. More
  specifically, we will ask you, whether you are using the open source
  software as an autonomous piece of software, or whether you are using it as an
  embedded part of a larger, more complex piece of software. In the first case,
  the answer should be '\underline{independent}', in the second
  '\underline{embedded}'.
  
  \item The \textbf{\underline{recipient} of the open source software}:
  Sometimes you might wish to use the received open source software only for
  yourself. In other cases you might intend to hand over the software (also) to
  other people. More specifically, we will ask you, whether you are going to use
  the open source software only for yourself, or whether you plan to
  (re)distribute it (also) to third parties. In the first case, the answer
  should be '\underline{4yourself}', in the second '\underline{2others}'.
 
  \item The \textbf{\underline{form} of the distributed files}: Many licenses
  also draw a distinction between distributing the software as sources and
  distributing the files as binaries. In this case, we will ask you, whether you
  want to distribute the software in the form of binaries or as source code. In
  the first case, the answer should be '\underline{binaries}', in the second
  '\underline{sources}'
  
  \item The kind of the \textbf{\underline{ioAccess} of the executed program}:
  At least one license draws a distinction between an open source based work
  offering only local access to its io data and an open source based work
  distributing its io data via internet. In the first case, the answer should be
  '\underline{onlyLocally}', in the second '\underline{viaInternet}'
\end{itemize}

From a more programmatic point-of-view, we can summarize these tokens as
follows:

\begin{itemize}
  \item \texttt{type::snimoli} \emph{or} \texttt{type::proapse}
  \item \texttt{state::unmodified} \emph{or} \texttt{state::modified}
  \item \texttt{context::independent} \emph{or} \texttt{context::embedded}
  \item \texttt{recipient::4yourself} \emph{or} \texttt{recipient::2others}
  \item \texttt{form::binaries} \emph{or} \texttt{form::sources}
  \item \texttt{ioAccess::onlyLocally} \emph{or} \texttt{ioAccess::viaInternet}
\end{itemize}

We have already defined the \emph{open source use case} as the combination of
these tokens. If we simply combine all these tokens of all these classes with
all the tokens of the other classes\footnote{in the sense of the cross product
TYPE $\times$ STATE $\times$ CONTEXT $\times$ RECIPIENT $\times$ FORM $\times$
IOACCESS. In some earlier versions of the \oslic{}, we also asked whether you
are going to combine or to embed the open source software with other software
components by linking them statically or dynamically, or by textually including
(parts of) the open source software into your larger product. Meanwhile, we
clearly discovered that it is unnecessary to increase the complexity by the
results of this question. For Details $\rightarrow$ \oslic{} p.\
\pageref{sec:LinkingSecondary}}, we get $2 \cdot 2 \cdot 2 \cdot 2 \cdot 2 \cdot
2 = 62$ sets of tokens---or 62 \emph{open source use cases}. Fortunately, some
of the generated sets are invalid from an empirical or logical view, and some of
these sets are context sensitive:

\begin{enumerate}
  \label{InvalidFinderTokenCombinations}
  \item If you already have specified that the used open source software is a
  \emph{proapse}---an autonomous program, an application, or a server---then
  your answer implies that the software is used independently and is not
  embedded with other components into a larger unit. But if you have specified
  that the used open source software is a \emph{snimoli}---a snippet of
  code, a module, a plugin, or a library---then it can indeed be used as an
  embedded component of a constructed larger application or server, or it can be
  used independently in case you 'only' re-distribute it to 3rd. parties.
  
  \item If you already have specified that the used open source software is a
  \emph{snimoli}---a snippet of code, a module, a plugin, or a library---and
  that this \emph{snimoli} shall be used only by yourself (not distributed to
  other 3rd.\ parties) then your answer must also imply that this \emph{snimoli}
  is used in combination, as an embedded part of a larger unit. A library can
  not be used autonomously, without using it as a component of another
  application. In this case, it would simply sit on the disk and would do
  nothing more than occupying space.
  
  \item To enquire the \emph{form} of the distributed files is only relevant if
  you have decided to distribute the software to other recipients
  \emph{2others}.
  
  \item With respect to the one license using the type of ioAccess as a
  discriminator, it is only relevant to enquire the type of the ioAccess if you
  either locally execute a \emph{modified} open source program \emph{4yourself}
  or if you locally execute a program \emph{4yourself}, which uses an
  \emph{embedded} open source component, regardless whether it has been modified
  or not.
  
\end{enumerate}

Does this sound complex? We thought so, too. We spent much time explaining these
constraints to ourselves, and only when we had transposed all the combinations
and rules into a tree, the situation became clearer. The following diagram
summarizes the main results of our investigation\footnote{ Each of the invalid
use cases (= sets of tokens) [for details s. p.\
\pageref{InvalidFinderTokenCombinations}] is marked by an \lightning{} and leads
to an empty set (= $\varnothing$). We are using the word 'invalid' a little
ambigiuosly: A combination of values is invalid, if it is empirically
impossible, to combine the features or if it is irrelavant to subclassify a
concept by the added features. Particularly:
\begin{itemize}
  \item A proapse can not be embedded into another software unit, also
  containing a main-function.
  \item Using a software library only for yourself and independently (not in
  combination with larger software unit), is like having an unused heap of bytes
  on your disc.
  \item To discriminate between sources and binaries is only valid in case of
  distributing software.
  \item To discriminate between an executed program with an only locally based
  io access and that with an internet based io access is only relevant, if you
  are using the software for yourself what implies to execute it.
\end{itemize}
 . 

}::

\tikzstyle{StartNode} = [font=\tiny, ellipse, draw, fill=gray!5,
text width=1em, text centered, minimum height=1em]

\tikzstyle{TypeNode} = [font=\tiny, rectangle, draw, fill=gray!10,
text width=1cm, text centered, rounded corners, minimum height=1em]

\tikzstyle{VLabelNode} = [font=\tiny, draw, rectangle,  text
width=1.4cm, fill=white]

\tikzstyle{VNibelNode} = [font=\tiny, draw, rectangle,  text
width=1cm, ]

\tikzstyle{VdTupelNode} = [font=\tiny, draw, rectangle, dotted, text
width=1.4cm, ]

\tikzstyle{VmTupelNode} = [font=\tiny, draw, rectangle, dotted, text
width=2.3cm, ]

\tikzstyle{VQuadrupelNode} = [font=\tiny, draw, rectangle, dotted, text
width=3cm, ]

\tikzstyle{VsQuadrupelNode} = [font=\tiny, draw, rectangle, dotted, text
width=2.7cm, ]


\tikzstyle{VlTupelNode} = [font=\tiny, draw, rectangle, dotted, text
width=3.8cm, ]


\tikzstyle{VmTupelLeaf} = [font=\tiny, draw, rectangle, dashed, text
width=2.7cm, ]

\tikzstyle{VlTupelLeaf} = [font=\tiny, draw, rectangle, dashed, text
width=3.8cm, ]

\tikzstyle{OsucNode} = [font=\tiny, rectangle, draw, fill=gray!20,
text width=1.4cm, text centered, rounded corners, minimum height=1em]



\tikzstyle{arrow} = [draw, -latex']
\tikzstyle{edge} = [draw]


\begin{tikzpicture}

% drwa lines at first:




% classification types and their values

\node[TypeNode] (tForm) at (2,7.4) {\textit{form?}};

\node[VNibelNode] (vSources) at (3,6.7) {sources};
\node[VNibelNode] (vBinaries) at (2.5,6.2) {binaries};

\node[TypeNode] (tType) at (0,15.6) {\textit{type?}};
\node[VNibelNode] (vProapse) at (2.8,15.9) {proapse};
\node[VNibelNode] (vSnimoli) at (2.8,15.4) {snimoli};

\node[TypeNode] (tState) at (0.8,12.6) {\textit{state?}};
\node[VLabelNode] (vUnmodified) at (2.6,12.9) {unmodified};
\node[VLabelNode] (vModified) at (2.6,12.4) {modified};

\node[TypeNode] (tContext) at (1.0,10.6) {\textit{context?}};
\node[VLabelNode] (vIndependent) at (2.7,10.9) {independent};
\node[VLabelNode] (vEmbedded) at (2.7,10.4) {embedded};

\node[TypeNode] (tRecipient) at (0.8,8.6) {\textit{recipient?}};
\node[VNibelNode] (v4yourself) at (2.3,8.9) {4yourself};
\node[VNibelNode] (v2others) at (2.3,8.4) {2others};

\node[TypeNode] (tIoAccess) at (0,1.6) {\textit{ioAccess?}};
\node[VLabelNode] (vViaInternet) at (2,1.9) {viaInternet};
\node[VLabelNode] (vOnlyLocal) at (2,1.4) {onlyLocal};

\node[StartNode] (vStart) at (0,10) {\textbf{\#}};

% value collictions defining the osucs

\node[OsucNode] (o1) at (12,19.8) {OSUC-01};
\node[VlTupelLeaf, color=blue] (vo1) at (15,19.8)
{\{proapse, independent,\\4yourself, unmodified\}};

\node[OsucNode] (o2s) at (12,19) {OSUC-02S};
\node[VlTupelLeaf, color=blue] (vo2s) at (15,19)
{\{proapse, independent,\\2others, unmodified, sources\}};

\node[OsucNode] (o2b) at (12,18.2) {OSUC-02B};
\node[VlTupelLeaf, color=blue] (vo2b) at (15,18.2)
{\{proapse, independent,\\2others, unmodified, binaries\}};

\node[OsucNode] (o3l) at (12,17.4) {OSUC-03L};
\node[VlTupelLeaf, color=blue] (vo3l) at (15,17.4)
{\{proapse, independent,\\4yourself, modified, onlyLocal\}};

\node[OsucNode] (o3n) at (12,16.6) {OSUC-03N};
\node[VlTupelLeaf, color=blue] (vo3n) at (15,16.6)
{\{proapse, independent,\\4yourself, modified, viaInternet\}};

\node[OsucNode] (o4s) at (12,15.8) {OSUC-04S};
\node[VlTupelLeaf, color=blue] (vo4s) at (15,15.8)
{\{proapse, independent,\\2others, modified, sources\}};

\node[OsucNode] (o4b) at (12,15) {OSUC-04B};
\node[VlTupelLeaf, color=blue] (vo4b) at (15,15)
{\{proapse, independent,\\2others, modified, binaries\}};

\node[OsucNode] (o5s) at (13,12.2) {OSUC-05S};
\node[VmTupelLeaf, color=blue] (vo5s) at (15.5,12.2)
{\{snimoli, independent,\\2others, unmodified,\\sources\}};

\node[OsucNode] (o5b) at (13,11.1) {OSUC-05B};
\node[VmTupelLeaf, color=blue] (vo5b) at (15.5,11.1)
{\{snimoli, independent,\\2others, unmodified,\\binaries\}};

\node[OsucNode] (o6l) at (13,10) {OSUC-06L};
\node[VmTupelLeaf, color=blue] (vo6l) at (15.5,10)
{\{snimoli, embedded,\\4yourself, unmodified,\\onlyLocal\}};

\node[OsucNode] (o6n) at (13,8.9) {OSUC-06N};
\node[VmTupelLeaf, color=blue] (vo6l) at (15.5,8.9)
{\{snimoli, embedded,\\4yourself, unmodified,\\viaInternet\}};

\node[OsucNode] (o7s) at (13,7.8) {OSUC-07S};
\node[VmTupelLeaf, color=blue] (vo7s) at (15.5,7.8)
{\{snimoli, embedded,\\2others, unmodified,\\sources\}};

\node[OsucNode] (o7b) at (13,6.7) {OSUC-07B};
\node[VmTupelLeaf, color=blue] (vo7b) at (15.5,6.7)
{\{snimoli, embedded,\\2others, unmodified,\\binaries\}};

\node[OsucNode] (o8s) at (13,5.6) {OSUC-08S};
\node[VmTupelLeaf, color=blue] (vo8s) at (15.5,5.6)
{\{snimoli, independent,\\2others, modified,\\sources\}};

\node[OsucNode] (o8b) at (13,4.4) {OSUC-08B};
\node[VmTupelLeaf, color=blue] (vo8b) at (15.5,4.4)
{\{snimoli, independent,\\2others, modified,\\binaries\}};

\node[OsucNode] (o9l) at (13,3.3) {OSUC-09L};
\node[VmTupelLeaf, color=blue] (vo9l) at (15.5,3.3)
{\{snimoli, embedded,\\4yourself, modified,\\onlyLocal\}};

\node[OsucNode] (o9n) at (13,2.2) {OSUC-09N};
\node[VmTupelLeaf, color=blue] (vo9n) at (15.5,2.2)
{\{snimoli, embedded,\\4yourself, modified,\\viaInternet\}};

\node[OsucNode] (o10s) at (13,1.1) {OSUC-010S};
\node[VmTupelLeaf, color=blue] (vo10s) at (15.5,1.1)
{\{snimoli, embedded,\\2others, modified,\\sources\}};

\node[OsucNode] (o10b) at (13,0) {OSUC-010B};
\node[VmTupelLeaf, color=blue] (vo10b) at (15.5,0)
{\{snimoli, embedded,\\2others, modified,\\binaries\}};

% concepts directly referring osucs

\node[VQuadrupelNode] (v4osuc1) at (5.4,19.8)
{\{proapse, 4yourself,\\ independent, unmodified\}};

\node[VQuadrupelNode] (v4osuc2) at (7,18.7)
{\{proapse, 2others,\\ independent, unmodified\}};

\node[VsQuadrupelNode] (v4osuc3) at (8.3,17.6)
{\{proapse, 4yourself,\\ independent, modified\}};

\node[VsQuadrupelNode] (v4osuc4) at (9.6,16.5)
{\{proapse, 2others,\\ independent, modified\}};

\node[VlTupelLeaf, color=red] (vLightning1) at (10.4,14.3)
{\{proapse, 4yourself, embedded,\\ \{unmodified, modified\}\} \lightning{}};

\node[VlTupelLeaf, color=red] (vLightning2) at (10.4,13.6)
{\{proapse, 2others, embedded,\\ \{unmodified, modified\}\} \lightning{}};

\node[VlTupelLeaf, color=red] (vLightning3) at (10.4,12.9)
{\{snimoli, 4yourself, independent,\\ \{unmodified, modified\}\} \lightning{}};

\node[VQuadrupelNode] (v4osuc5) at (10.2,11.8)
{\{snimoli, 2others,\\ independent, unmodified\}};

\node[VsQuadrupelNode] (v4osuc6) at (9.5,10.8)
{\{snimoli, 4yourself,\\ embedded, unmodified\}};

\node[VsQuadrupelNode] (v4osuc7) at (9.2,9.8)
{\{snimoli, 2others,\\ embedded, unmodified\}};

\node[VsQuadrupelNode] (v4osuc8) at (8.2,8.8)
{\{snimoli, 4yourself,\\ embedded, modified\}};

\node[VsQuadrupelNode] (v4osuc9) at (7.3,7.6)
{\{snimoli, 2others,\\ independent, modified\}};

\node[VsQuadrupelNode] (v4osuc10) at (6.2,6.4)
{\{snimoli, 2others,\\ embedded, modified\}};

%meta nodes reffering concepts reffering osucs


\node[VmTupelNode] (v2otherSources) at (6,5.4) {\{2others, sources\}};
\node[VmTupelNode] (v2otherBinaries) at (6,4.8) {\{2others, binaries\}};

\node[VlTupelNode] (v4proapseViaInternet) at (7.2,4.0)
  {\{proapse, 4yourself,\\ modified, viaInternet\}};
\node[VlTupelNode] (v4proapseOnlyLocal) at (7.2,3.2)
  {\{proapse, 4yourself,\\ modified, onlyLocal\}};

\node[VlTupelLeaf, color=red] (vLightning4) at (7.2,2.4)
  {\{proapse, 4yourself, unmodified,\\ \{viaInternet, onlyLocal\} \} \lightning{}};

\node[VlTupelNode] (v4snimoliViaInternet) at (7.2,1.6)
  {\{snimoli, 4yourself,\\ embedded, viaInternet\}};
\node[VlTupelNode] (v4snimoliOnlyLocal) at (7.2,0.8)
  {\{snimoli, 4yourself,\\ embedded, onlyLocal\}};

\node[VlTupelLeaf, color=red] (vLightning5) at (7.2,0)
  {\{snimoli, 4yourself, independent,\\ \{viaInternet, onlyLocal\} \} \lightning{}};

% entry paths

\path [edge] (vStart) -- (tType);
\path [edge] (tType) -- (vProapse);
\path [edge] (tType) -- (vSnimoli);

\path [edge] (vStart) -- (tState);
\path [edge] (tState) -- (vUnmodified);
\path [edge] (tState) -- (vModified);

\path [edge] (vStart) -- (tContext);
\path [edge] (tContext) -- (vIndependent);
\path [edge] (tContext) -- (vEmbedded);

\path [edge] (vStart) -- (tRecipient);
\path [edge] (tRecipient) -- (v4yourself);
\path [edge] (tRecipient) -- (v2others);

\path [edge] (v2others) -- (tForm);
\path [edge] (tForm) -- (vSources);
\path [edge] (tForm) -- (vBinaries);

\path [edge] (vStart) -- (tIoAccess);
\path [edge] (tIoAccess) -- (vViaInternet);
\path [edge] (tIoAccess) -- (vOnlyLocal);


% middle paths
\path [edge] (vProapse) -- (v4osuc1);
\path [edge] (vProapse) -- (v4osuc2);
\path [edge] (vProapse) -- (v4osuc3);
\path [edge] (vProapse) -- (v4osuc4);
\path [edge] (vProapse) -- (vLightning1);
\path [edge] (vProapse) -- (vLightning2);
\path [edge] (vSnimoli) -- (vLightning3);
\path [edge] (vSnimoli) -- (v4osuc5);
\path [edge] (vSnimoli) -- (v4osuc6);
\path [edge] (vSnimoli) -- (v4osuc7);
\path [edge] (vSnimoli) -- (v4osuc8);
\path [edge] (vSnimoli) -- (v4osuc9);
\path [edge] (vSnimoli) -- (v4osuc10);

\path [edge] (vUnmodified) -- (v4osuc1);
\path [edge] (vUnmodified) -- (v4osuc2);
\path [edge] (vModified) -- (v4osuc3);
\path [edge] (vModified) -- (v4osuc4);
\path [edge] (vUnmodified) -- (vLightning1);
\path [edge] (vUnmodified) -- (vLightning2);
\path [edge] (vUnmodified) -- (vLightning3);
\path [edge] (vUnmodified) -- (v4osuc5);
\path [edge] (vUnmodified) -- (v4osuc6);
\path [edge] (vUnmodified) -- (v4osuc7);
\path [edge] (vModified) -- (v4osuc8);
\path [edge] (vModified) -- (v4osuc9);
\path [edge] (vModified) -- (v4osuc10);

\path [edge] (vIndependent) -- (v4osuc1);
\path [edge] (vIndependent) -- (v4osuc2);
\path [edge] (vIndependent) -- (v4osuc3);
\path [edge] (vIndependent) -- (v4osuc4);
\path [edge] (vIndependent) -- (vLightning3);
\path [edge] (vEmbedded) -- (vLightning1);
\path [edge] (vEmbedded) -- (vLightning2);
\path [edge] (vIndependent) -- (v4osuc5);
\path [edge] (vEmbedded) -- (v4osuc6);
\path [edge] (vEmbedded) -- (v4osuc7);
\path [edge] (vEmbedded) -- (v4osuc8);
\path [edge] (vIndependent) -- (v4osuc9);
\path [edge] (vEmbedded) -- (v4osuc10);

\path [edge] (v4yourself) -- (v4osuc1);
\path [edge] (v2others) -- (v4osuc2);
\path [edge] (v4yourself) -- (v4osuc3);
\path [edge] (v2others) -- (v4osuc4);
\path [edge] (v2others) -- (v4osuc5);
\path [edge] (v4yourself) -- (v4osuc6);
\path [edge] (v2others) -- (v4osuc7);
\path [edge] (v4yourself) -- (v4osuc8);
\path [edge] (v2others) -- (v4osuc9);
\path [edge] (v4yourself) -- (v4osuc10);


\draw[->] (vSources) to [out=0,in=180] (v2otherSources);
\draw[->] (vBinaries) to [out=0,in=180] (v2otherBinaries);

\draw[->] (vViaInternet) to [out=0,in=180] (v4proapseViaInternet);
\draw[->] (vViaInternet) to [out=0,in=180] (vLightning4);
\draw[->] (vViaInternet) to [out=0,in=180] (v4snimoliViaInternet);
\draw[->] (vViaInternet) to [out=0,in=180] (vLightning5);

\draw[->] (vOnlyLocal) to [out=0,in=180] (v4proapseOnlyLocal);
\draw[->] (vOnlyLocal) to [out=0,in=180] (vLightning4);
\draw[->] (vOnlyLocal) to [out=0,in=180] (v4snimoliOnlyLocal);
\draw[->] (vOnlyLocal) to [out=0,in=180] (vLightning5);

\draw[->] (v4yourself) to [out=0,in=180] (v4proapseViaInternet);
\draw[->] (v4yourself) to [out=0,in=180] (v4proapseOnlyLocal);
\draw[->] (v4yourself) to [out=0,in=180] (vLightning4);
\draw[->] (v4yourself) to [out=0,in=180] (v4snimoliViaInternet);
\draw[->] (v4yourself) to [out=0,in=180] (v4snimoliOnlyLocal);
\draw[->] (v4yourself) to [out=0,in=180] (vLightning5);

\draw[->] (vModified) to [out=0,in=180] (v4proapseViaInternet);
\draw[->] (vModified) to [out=0,in=180] (v4proapseOnlyLocal);
\draw[->] (vUnmodified) to [out=0,in=180] (vLightning4);
\draw[->] (vEmbedded) to [out=0,in=180] (v4snimoliViaInternet);
\draw[->] (vEmbedded) to [out=0,in=180] (v4snimoliOnlyLocal);
\draw[->] (vIndependent) to [out=0,in=180] (vLightning5);





\path [arrow, color=magenta] (v4osuc1) -- (o1);
\path [arrow, color=magenta] (v4osuc2) -- (o2s);
\path [arrow, color=magenta] (v4osuc2) -- (o2b);
\path [arrow, color=magenta] (v4osuc3) -- (o3l);
\path [arrow, color=magenta] (v4osuc3) -- (o3n);
\path [arrow, color=magenta] (v4osuc4) -- (o4s);
\path [arrow, color=magenta] (v4osuc4) -- (o4b);
\path [arrow, color=magenta] (v4osuc5) -- (o5s);
\path [arrow, color=magenta] (v4osuc5) -- (o5b);
\path [arrow, color=magenta] (v4osuc6) -- (o6l);
\path [arrow, color=magenta] (v4osuc6) -- (o6n);
\path [arrow, color=magenta] (v4osuc7) -- (o7s);
\path [arrow, color=magenta] (v4osuc7) -- (o7b);
\path [arrow, color=magenta] (v4osuc8) -- (o8s);
\path [arrow, color=magenta] (v4osuc8) -- (o8b);
\path [arrow, color=magenta] (v4osuc9) -- (o9l);
\path [arrow, color=magenta] (v4osuc9) -- (o9n);
\path [arrow, color=magenta] (v4osuc10) -- (o10s);
\path [arrow, color=magenta] (v4osuc10) -- (o10b);

\draw[->, color=violet] (v4snimoliOnlyLocal) to [out=0,in=180] (o9l);
\draw[->, color=violet] (v4snimoliViaInternet) to [out=0,in=180] (o9n);
\draw[->, color=violet] (v4snimoliOnlyLocal) to [out=0,in=180] (o6l);
\draw[->, color=violet] (v4snimoliViaInternet) to [out=0,in=180] (o6n);

\draw[->, color=violet] (v4proapseOnlyLocal) to [out=0,in=180] (o3l);
\draw[->, color=violet] (v4proapseViaInternet) to [out=0,in=180] (o3n);


   \draw[->, color=blue] (v2otherSources) to [out=0,in=224] (o2s);
   \draw[->, color=blue] (v2otherBinaries) to [out=0,in=220] (o2b);
   \draw[->, color=blue] (v2otherSources) to [out=0,in=216] (o4s);
   \draw[->, color=blue] (v2otherBinaries) to [out=0,in=212] (o4b);
   \draw[->, color=blue] (v2otherSources) to [out=0,in=208] (o5s);
   \draw[->, color=blue] (v2otherBinaries) to [out=0,in=204] (o5b);
   \draw[->, color=blue] (v2otherSources) to [out=0,in=200] (o7s);
   \draw[->, color=blue] (v2otherBinaries) to [out=0,in=196] (o7b);
   \draw[->, color=blue] (v2otherSources) to [out=0,in=192] (o8s);
   \draw[->, color=blue] (v2otherBinaries) to [out=0,in=188] (o8b);
   \draw[->, color=blue] (v2otherSources) to [out=0,in=184] (o10s);
   \draw[->, color=blue] (v2otherBinaries) to [out=0,in=180] (o10b);

% 


\end{tikzpicture}



