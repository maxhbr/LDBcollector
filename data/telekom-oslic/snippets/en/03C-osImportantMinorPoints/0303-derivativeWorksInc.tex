% Telekom osCompendium 'for being included' snippet template
%
% (c) Karsten Reincke, Deutsche Telekom AG, Darmstadt 2011
%
% This LaTeX-File is licensed under the Creative Commons Attribution-ShareAlike
% 3.0 Germany License (http://creativecommons.org/licenses/by-sa/3.0/de/): Feel
% free 'to share (to copy, distribute and transmit)' or 'to remix (to adapt)'
% it, if you '... distribute the resulting work under the same or similar
% license to this one' and if you respect how 'you must attribute the work in
% the manner specified by the author ...':
%
% In an internet based reuse please link the reused parts to www.telekom.com and
% mention the original authors and Deutsche Telekom AG in a suitable manner. In
% a paper-like reuse please insert a short hint to www.telekom.com and to the
% original authors and Deutsche Telekom AG into your preface. For normal
% quotations please use the scientific standard to cite.
%
% [ Framework derived from 'mind your Scholar Research Framework' 
%   mycsrf (c) K. Reincke 2012 CC BY 3.0  http://mycsrf.fodina.de/ ]
%

% Apache 
% ----------------
% "Derivative Works" shall mean any work, whether in Source or Object
% form, that is based on (or derived from) the Work and for which the
% editorial revisions, annotations, elaborations, or other
% modifications represent, as a whole, an original work of
% authorship. For the purposes of this License, Derivative Works shall
% not include works that remain separable from, or merely link (or
% bind by name) to the interfaces of, the Work and Derivative Works
% thereof. [Preamble]
%
% BSD
% ----------------
% with or without modification
%
% MPL 2.0
% ----------------
% 1.10. “Modifications”
%
%   means any of the following:
%
%   a.  any file in Source Code Form that results from an addition to,
%       deletion from, or modification of the contents of Covered
%       Software; or
%
%   b.  any new file in Source Code Form that contains any Covered
%       Software.
%
% CDDL
% ----------------
% 1.6. Larger Work means a work which combines Covered Software or
%      portions thereof with code not governed by the terms of this
%      License. 
%
%  1.9. Modifications means the Source Code and Executable form of any of the following:
%
%   A. Any file that results from an addition to, deletion from or
%      modification of the contents of a file containing Original
%      Software or previous Modifications; 
%
%   B. Any new file that contains any part of the Original Software or
%      previous Modification; or 
%
%   C. Any new file that is contributed or otherwise made available
%      under the terms of this License. 
%
% EPL
% ----------------
% Contributions do not include additions to the Program which: (i) are
% separate modules of software distributed in conjunction with the
% Program under their own license agreement, and (ii) are not
% derivative works of the Program.  
%
% GPL
% ----------------
% §0. The "Program", below, refers to any such program or work, and a
% "work based on the Program" means either the Program or any
% derivative work under copyright law: that is to say, a work
% containing the Program or a portion of it, either verbatim or with
% modifications and/or translated into another language.
%
% §2. Thus, it is not the intent of this section to claim rights or
% contest your rights to work written entirely by you; rather, the
% intent is to exercise the right to control the distribution of
% derivative or collective works based on the Program. 
%
% GPL 3
% ----------------
% §0. To “modify” a work means to copy from or adapt all or part of
% the work in a fashion requiring copyright permission, other than the 
% making of an exact copy. The resulting work is called a “modified
% version” of the earlier work or a work “based on” the earlier work. 
%
% §5c. You must license the entire work, as a whole, under this
% License to anyone who comes into possession of a copy. This License
% will therefore apply, along with any applicable section 7 additional
% terms, to the whole of the work, and all its parts, regardless of how
% they are packaged.
%
% §5 at the end:
% A compilation of a covered work with other separate and independent
% works, which are not by their nature extensions of the covered work,
% and which are not combined with it such as to form a larger program,
% in or on a volume of a storage or distribution medium, is called an
% “aggregate” if the compilation and its resulting copyright are not
% used to limit the access or legal rights of the compilation's users
% beyond what the individual works permit. Inclusion of a covered work
% in an aggregate does not cause this License to apply to the other
% parts of the aggregate.
%
% http://www.gnu.org/licenses/gpl-faq.html#MereAggregation

%% use all entries of the bibliography
%\nocite{*}

{
\newcommand{\softbreak}{\hspace{0pt plus 1cm}}
\newcommand{\sourceNeeded}{\footnote{cite the sources}}

\section{Excursion: What is a 'Derivative Work' - the basic idea of open source}
\footnotesize \begin{quote}\itshape This chapter briefly discusses aspects of
being a derivated pieces of software which have to be known for using open
source software compliantly. As usually, the \oslic{}
only tries to find one safe interpretation. The authors know that there
exist many other ways to consider this topic. So, if you feel, that the
viewpoint of the \oslic{} does not fit the specific circumstances of your
particular case, do not hesitate to ask your own lawyer. But if you agree with
the \oslic{}, be aware that you dealing with this topic from the viewpoint of a
benevolent user.
\end{quote}
\normalsize
Let us outline the argumentation:

\begin{description}
  \item[The meaning `derivative work' must be known!]\softbreak
    Many open source licenses use the term `derivative work,'\sourceNeeded
    either directly or indirectly in form of the word `modification.'\sourceNeeded
    [Write a table as survey] 
    And nearly all licenses that are using the term `derivative work' etc., 
    are linking tasks that must be executed to comply with the corresponding
    license, to the precondition that something is a derivative work. 
    [table survey] 
    \textbf{Hence, for acting in accordance with such a license, it has to be
    known what a derivate work is.}  

  \item[Unfortunately the meaning is not clearly fixed.]\softbreak
    There exist different readings of the term `derivative work.' 
    [specify the differences and cite the sources] 
    \textbf{Hence, it is not as clear what a derivative work is as one could wish}

  \item[So, let us argue from the viewpoint of a benevolent developer:]\softbreak
    Open source licenses are written for software developers, mostly to preserve
    their freedom to develop software. And sometimes these licenses are also
    written by software developers---or at least with their assistance. So, one
    should be able to answer the question under which circumstances a piece of
    software is a `derivative work' of another piece of software based on two
    principles: 
  \begin{itemize}
    \item Let us argue from the viewpoint of a benevolent neutral software
      developer without hidden interests or a hidden agenda.
    \item In case of doubts let us preferably assume that the two pieces
      interrelate as source and derivative work---so that the \oslic{} rather
      recommends to perform the required tasks.
  \end{itemize}
\end{description}

We generalize a specific viewpoint of the LGPL. It uses three terms:

\begin{description}
  \item[\enquote{library}] is defined as \enquote{a collection of software
  functions and/or data prepared so as to be conveniently linked with
  application programs.}\footcite[cf.][\nopage wp §0]{Lgpl21OsiLicense1999a}
  \item[\enquote{work based on the library}] is defined as \enquote{either the
  library or any derivative work.}\footcite[cf.][\nopage wp
  §0]{Lgpl21OsiLicense1999a}
  \item[\enquote{work that uses the library}] is defined as something which
  initially \enquote{[\ldots] is not a derivative work of the library [\ldots]}
  but can become a derivative work by being combined / linked to the library it
  uses.\footcite[cf.][\nopage wp §5]{Lgpl21OsiLicense1999a}
\end{description}

Following these specifications, one has to conclude that 
\emph{derivative works} of the library can be drieved in two different ways: 
First, the library itself can be enhanced without changing the character of 
being a library. Then, of course, the resulting library is a derivative work 
of the initial library.  Second, an overaching program can use the library by 
calling functions, methods, or data offered by the library. In this case, the 
overarching program functionally depends on the library and is a derivative work 
(as soon as it is linked to the library).

This viewpoint can be generalized: snippets, modules, plugins can be
enhanced and used by overarching programs or even by more complex libraries.
Based on this viewpoint---which should finally be formulated as the viewpoint of
a benevolent impartial developer---the \oslic{} uses the following rules by which
the \oslic{} decides to take something as derivative Work:
\label{sec:BenevolentDerivativeWorkUnderstanding}

\begin{description}
  \item[Copy-Case] Copying a piece of code from a source file and pasting it
  into a target file makes the target file a derivatve work of the source
  file.\footnote{Be careful: this case must still be distinguished from the case
  of an automatic inclusion (header files, script libraries) during the
  compilation / execution: Inclusion of header files alone should not create a derivative
  work.}
  \item[Modify-Case] Inserting any new content or deleting any existing content
  of a source file makes the resulting target file a derivate work of the
  source file.
  %% RPD: what is the rationale for the call-case?
  \item[Call-Case] Inserting the call of function which is defined inside of and 
  delivered by a sourcefile into a target file makes that target file
  depending on the source file and therefore a derivative work of the delivering
  source file.
\end{description}

And here are some applications of these rules:

\begin{itemize}
  \item \textbf{Enlarging an existing source file by an external text creates a
  derivative work!} Why? \emph{Because you are going to reuse the
  external code for simplifying our life.} [see 'Copy Case']
  \item \textbf{Reducing a source file creates a derivative work!} Why?
  \emph{Because you are going to prepare the given file(s) for a better reuse.}
  [see 'Modify-Case']
  \item \textbf{Replacing something in a source file creates a derivative work!}
  Why? \emph{Because you are going to reuse parts of the existing code for
  simplifying your life.} [see 'Modify Case']
  \item \textbf{Integrating a foreign snippet into an existing source code
  creates a derivative work!} Why? \emph{Because you are going to simplify
  your life by reusing both, the foreign snippet and the original file.} [see
  'Copy Case' and 'Modify-Case']
  %% RPD The following example claims that if you split a file A into a
  %% library part L (offering a function that was part of A)  and a client 
  %% part C (containing everything in A that is not in L) then
  %% - C and L are derived from A (obviuosly)
  %% - C is derived from L (call-case; but this makes it clear, that there 
  %%   is no such thing as a call case.)
  \item \textbf{Refactoring a given work by extracting a function / method into
  an autonomous file creates a derivative work in two respects!} Why?
  \emph{Because, first, all modified / generated files depend
  on the original file and, second, because those function calls in the files
  introduce a dependecy on the file defining the function itself.}
  [see 'Modify-Case' and 'Call-Case]
  \item \textbf{Calling a function - served by a defining module - lets the
  calling file become a derivative work of the serving module!} Why?
  \emph{Because you are going to simplify your life by reusing an already
  prepared work (often offered by other developers).} [see 'Call-Case']
  \item \textbf{By calling elements - served by a defining library - the
  calling file becomes a derivative work of the serving library!} Why? 
  \emph{Because you are going to simplify your life by reusing an already
  prepared work (often offered by other developers).} [see
  'Call-Case']\footnote{In this context, you may sometimes read that one has to
  differentiate the defining file (for example the C-code) and the declaring
  file (for example the C-Header). But in our view, it is not so important to
  make such a difference: The using file, which includes the declaring header
  file depends on the defining source code file ('Call-Case'). So, one can
  ignore the formal dependance on the declaring header file ('Copy-Case').}
\end{itemize}

And now some additional 'ideas' which might invite to be discussed:

\begin{itemize}
  \item \textbf{Does a plugin depend on its framework? No.} Why? \emph{
  Because it is like a module: it offers a function (normally without using
  a function, offered by the framework itself).}
  \item \textbf{Does a framework depend on its plugin? Let us try to answer:
  Sometimes yes, sometimes no.} Why? \emph{If the framwork crashes when it is
  missing its plugin, then it clearly depends on the plugin. No doubt. It is
  simply not autonomous. But if it does not crash, then it perfectly does for
  which it has been designed: it is offering a place which might be filled by
  the plugin, but not necessarily. This kind of a framework is like an
  application listing to a port for getting data which it shall process and
  which are served by another application.}
  \item \textbf{Does a program using inter process communication depend on its
  IO-partners? Definitely not!} Why? \emph{Because, otherwise, we need not discuss
  all these cases, every thing would depend on everything---in each running system.}
\end{itemize}

[\ldots TBD \ldots]

}
%\bibliography{../../../bibfiles/oscResourcesEn}

% Local Variables:
% mode: latex
% fill-column: 80
% End:
