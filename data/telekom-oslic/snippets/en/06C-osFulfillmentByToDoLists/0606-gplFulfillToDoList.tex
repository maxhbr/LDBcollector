% Telekom osCompendium 'for being included' snippet template
%
% (c) Karsten Reincke, Deutsche Telekom AG, Darmstadt 2011
%
% This LaTeX-File is licensed under the Creative Commons Attribution-ShareAlike
% 3.0 Germany License (http://creativecommons.org/licenses/by-sa/3.0/de/): Feel
% free 'to share (to copy, distribute and transmit)' or 'to remix (to adapt)'
% it, if you '... distribute the resulting work under the same or similar
% license to this one' and if you respect how 'you must attribute the work in
% the manner specified by the author ...':
%
% In an Internet based reuse please link the reused parts to www.telekom.com and
% mention the original authors and Deutsche Telekom AG in a suitable manner. In
% a paper-like reuse please insert a short hint to www.telekom.com and to the
% original authors and Deutsche Telekom AG into your preface. For normal
% quotations please use the scientific standard to cite.
%
% [ Framework derived from 'mind your Scholar Research Framework' 
%   mycsrf (c) K. Reincke 2012 CC BY 3.0  http://mycsrf.fodina.de/ ]
%


%% use all entries of the bibliography
%\nocite{*}

\section{GPL licensed software}

Both versions of the GNU General Public License explicitly distinguish the
distribution of the source code from that of the binaries: On the one hand, the
GPL-2.0 mainly talks about copying and distributing the source
code,\citeGPLtwo{§1, §2} but also mentions the specific conditions for
\enquote{[\ldots] (copying) and (distributing) the Program [\ldots] in object
code or executable form [\ldots]}\citeGPLtwo{§3} On the other hand, the GPL-3.0
describes the \enquote{Basic Permissions} and the conditions for
\enquote{Conveying Verbatim Copies} or for \enquote{Conveying Modified Source
Versions}\citeGPLtwo{§2, §4, §5} before it explains the rules for
\enquote{Conveying Non-Source-Forms}.\citeGPLtwo{§2, §4, §5}  

GPL-2.0 and GPL-3.0 mainly talk about copying \emph{and} distributing the
software; private use is nearly completely unspecified: The GPL-2.0 lists its
`restrictions' only with respect to the act of copying \emph{and} distributing
\enquote{copies of the program}\citeGPLtwo{§1, §2, §4 et passim; emphasize by
KR} while the GPL-3.0 explicitly specifies that one \enquote{[\ldots] may
make, run and propagate covered works that (one does) not convey, without
conditions so long as (the) license otherwise remains in
force.}\citeGPLthree{§2} 

As licenses with a strong copyleft, they require that any application that
contains a GPL-licensed library must itself be licensed under the same
conditions as the library.
 
Finally, the GPL-2.0 and the GPL-3.0 aim for the same results and share the
same spirit by requiring nearly the same task to be performed for fulfilling the
license conditions.  Therefore it is appropriate to cover both versions in the
same chapter and to offer a common specialized GPL open source use case
structure for quickly finding the appropriate task list.%
  \footnote{For details of the general OSUC finder $\rightarrow$ \oslic,
    pp.\ \pageref{OsucTokens} and \pageref{OsucDefinitionTree}}
However, the task lists themselves will be kept separate.

In the following diagram, GPL-*-C1 (GPL-*-C2, \ldots, GPL-*-CB) is either
GPL-2.0-C1 (and so forth), if you are looking for the GPL-2.0 use case, or
GPL-3.0-C1, \ldots for the GPL-3.0 use case.

%% ============================================================================= 
%% Use-Case Finder

\gplUseCaseFinder{GPL}{2.0}{3.0}

%% ============================================================================= 
%% Common Building Blocks

\newcommand{\useCaseOne}{%
  \gtbUseCaseOne{GPL-\ver}
  \gtbCoversOne{GPL-\ver}}

\newcommand{\useCaseTwo}{%
  \gtbUseCaseTwo{GPL-\ver}
  \gtbCoversTwo{GPL-\ver}}

\newcommand{\useCaseThree}{%
  \gtbUseCaseThree{GPL-\ver}
  \gtbCoversThree{GPL-\ver}}

\newcommand{\useCaseFour}{%
  \gtbUseCaseFour{GPL-\ver}{a}
  \gtbCoversFour{GPL-\ver}}

\newcommand{\useCaseFive}{%
  \gtbUseCaseFive{GPL-\ver}{a}
  \gtbCoversFive{GPL-\ver}}

\newcommand{\useCaseSix}{%
  \gtbUseCaseSix{GPL-\ver}{a}
  \gtbCoversSix{GPL-\ver}}

\newcommand{\useCaseSeven}{%
  \gtbUseCaseSeven{GPL-\ver}{a}
  \gtbCoversSeven{GPL-\ver}}

\newcommand{\useCaseEight}{%
  \gtbUseCaseEight{GPL-\ver}{a}
  \gtbCoversEight{GPL-\ver}}

\newcommand{\useCaseNine}{%
  \gtbUseCaseNine{GPL-\ver}{a}
  \gtbCoversNine{GPL-\ver}}

\newcommand{\useCaseA}{%
  \gtbUseCaseA{GPL-\ver}{a}
  \gtbCoversA{GPL-\ver}}

\newcommand{\useCaseB}{%
  \gtbUseCaseB{GPL-\ver}{a}
  \gtbCoversB{GPL-\ver}}

% ------------------------------------------------------------------------------
% Common Text Blocks from 0600-commomn-text-blocks.tex

\newcommand{\keepLicenseElements}{\gtbKeepLicenseElements{GPL-\ver}}
\newcommand{\addToDocumentation}{\gtbAddToDocumentation{GPL-\ver}}
\newcommand{\giveLicense}{\gtbGiveLicense{GPL-\ver}}
\newcommand{\retainCopyrightNotices}{\gtbKeepCopyrightNotices{GPL-\ver}}
\newcommand{\describeHowToGetSource}{\gtbDescribeHowToGetSource{GPL-\ver}}
\newcommand{\createChangelog}{\gtbCreateChangelog{GPL-\ver}}
\newcommand{\markEmbeddedModifications}{\gtbMarkEmbeddedModifications{GPL-\ver}}
\newcommand{\markLibraryModifications}{\gtbMarkLibraryModifications{GPL-\ver}}
\newcommand{\markProgramModifications}{\gtbMarkProgramModifications{GPL-\ver}}
\newcommand{\gpltwoEnsureCopyrightNoticeSource}{\gtbVTwoCopyrightNotice{GPL-2.0}{source code}}
\newcommand{\gpltwoEnsureCopyrightNoticeBinary}{\gtbVTwoCopyrightNotice{GPL-2.0}{binary}}
\newcommand{\gplthreeEnsureCopyrightNoticeSource}{\gtbVThreeCopyrightNotice{GPL-3.0}{source code}}
\newcommand{\gplthreeEnsureCopyrightNoticeBinary}{\gtbVThreeCopyrightNotice{GPL-3.0}{binary}}
\newcommand{\makeUnmodifiedSourceAvailable}{\gtbMakeUnmodifiedSourceAvailable{GPL-\ver}} 
\newcommand{\makeModifiedSourceAvailable}{\gtbMakeModifiedSourceAvailable{GPL-\ver}} 
\newcommand{\makeAllSourcesAvailable}{\gtbMakeAllSourcesAvailable{GPL-\ver}}
\newcommand{\arrangeProgramChanges}{\gtbArrangeProgramChanges{GPL-\ver}}
\newcommand{\arrangeLibraryChanges}{\gtbArrangeLibraryChanges{GPL-\ver}}
\newcommand{\arrangeEmbeddedChanges}{\gtbArrangeEmbeddedChanges{GPL-\ver}}
\newcommand{\howToApplyTheseTerms}{\gtbHowToApplyTheseTerms{GPL-\ver}}
\newcommand{\noPatentLitigation}{\gtbNoPatentLitigation{GPL-\ver}}
\newcommand{\addToCopyrightDialogLib}{\gtbAddToCopyrightDialogStrongCopyleft{GPL-\ver}}
\newcommand{\addToCopyrightDialogApp}{\gtbAddToCopyrightDialogApp{GPL-\ver}}

% ------------------------------------------------------------------------------
% Make sure, licensing statements apply to enclosing program

\newcommand{\auxArrange}[1]{Arrange the #1 in a way
  that they are covered by the GPL-\ver{} licensing statements.} 

\newcommand{\arrangeEnclosingBinaries}{%
  \auxArrange{the binaries of the on-top development}}

\newcommand{\arrangeEnclosingSources}{%
  \auxArrange{the sources of the on-top development}}

%% ============================================================================= 
%% GPL-2.0 Use Cases

\newcommand{\ver}{2.0}

\begin{license}{GPL2} 
\licensename{GPL-2.0}
\licensespec{General Public License Version 2}
\licenseabbrev{GPL}
%\licenseversion{2.0}

% ------------------------------------------------------------------------------
\subsection{GPL-\ver-C1: Using the software only for yourself}
\begin{lsuc}{GPL-\ver-C1}
  \linkosuc{01}
  \linkosuc{03L} 
  \linkosuc{03N} 
  \linkosuc{06L}
  \linkosuc{06N}
  \linkosuc{09L}
  \linkosuc{09N}

  \useCaseOne

  \begin{lsucrequiresnothing}
    \lsucitem{You are allowed to use any kind of GPL-\ver software in any sense
      and in any context without being obliged to do anything as long as you do
      not give the software to third parties.}
  \end{lsucrequiresnothing}

  \lsucprohibitsnothing
\end{lsuc}

% ------------------------------------------------------------------------------
\subsection{GPL-\ver-C2: Passing the unmodified software as independent sources}
\begin{lsuc}{GPL-\ver-C2}
  \linkosuc{02S}
  \linkosuc{05S}

  \useCaseTwo

  \begin{lsucrequires}
    \lsucmandatory{\keepLicenseElements}
    \lsucmandatory{\gpltwoEnsureCopyrightNoticeSource}
    \lsucmandatory{\giveLicense}\passingFilesCorrectly
    \lsucmandatory{\retainCopyrightNotices}
    \lsucoptional{\addToDocumentation}
  \end{lsucrequires}

  \lsucprohibitsnothing
\end{lsuc}

% ------------------------------------------------------------------------------
\subsection{GPL-\ver-C3: Passing the unmodified software as independent binaries} 
\begin{lsuc}{GPL-\ver-C3}
  \linkosuc{02B} 
  \linkosuc{05B}

  \useCaseThree

  \begin{lsucrequires}
    \lsucmandatory{\keepLicenseElements}
    \lsucmandatory{\gpltwoEnsureCopyrightNoticeBinary}
    \lsucmandatory{\giveLicense}\passingFilesCorrectly  
    \lsucmandatory{\makeUnmodifiedSourceAvailable}
    \lsucmandatory{\describeHowToGetSource}
    \lsucmandatory{\retainCopyrightNotices}
    \lsucsourcedist{GPL-\ver-C2}
    \lsucoptional{\addToDocumentation}
  \end{lsucrequires}

  \lsucprohibitsnothing
\end{lsuc}

% ------------------------------------------------------------------------------
\subsection{GPL-\ver-C4: Passing the unmodified library as embedded sources}
\begin{lsuc}{GPL-\ver-C4}
  \linkosuc{07S} 

  \useCaseFour

  \begin{lsucrequires}
    \lsucmandatory{\keepLicenseElements}
    \lsucmandatory{\gpltwoEnsureCopyrightNoticeSource}
    \lsucmandatory{\giveLicense}\passingFilesCorrectly
    \lsucmandatory{\retainCopyrightNotices}
    \lsucmandatory{\addToCopyrightDialogLib}
    \lsucmandatory{\arrangeEnclosingSources}
    \lsucoptional{\addToDocumentation}
  \end{lsucrequires}

  \lsucprohibitsnothing
\end{lsuc}

% ------------------------------------------------------------------------------
\subsection{GPL-\ver-C5: Passing the unmodified library as embedded binaries} 
\begin{lsuc}{GPL-\ver-C5}
  \linkosuc{07B} 

  \useCaseFive

  \begin{lsucrequires}
    \lsucmandatory{\keepLicenseElements}
    \lsucmandatory{\gpltwoEnsureCopyrightNoticeBinary}
    \lsucmandatory{\giveLicense}\passingFilesCorrectly
    \lsucmandatory{\makeAllSourcesAvailable}
    \lsucmandatory{\describeHowToGetSource}
    \lsucmandatory{\addToCopyrightDialogLib}
    \lsucmandatory{\arrangeEnclosingBinaries}
    \lsucmandatory{\retainCopyrightNotices}
    \lsucsourcedist{GPL-\ver-C4}
    \lsucoptional{\addToDocumentation}
  \end{lsucrequires}

  \lsucprohibitsnothing
\end{lsuc}

% ------------------------------------------------------------------------------
\subsection{GPL-\ver-C6: Passing a modified program as source code}
\begin{lsuc}{GPL-\ver-C6}
  \linkosuc{04S} 

  \useCaseSix

  \begin{lsucrequires}
    \lsucmandatory{\keepLicenseElements}
    \lsucmandatory{\gpltwoEnsureCopyrightNoticeSource}
    \lsucmandatory{\giveLicense}\passingFilesCorrectly
    \lsucmandatory{\retainCopyrightNotices}
    \lsucmandatory{\addToCopyrightDialogApp}
    \lsucmandatory{\markProgramModifications}
    \lsucmandatory{\arrangeProgramChanges}\howToApplyTheseTerms
    \lsucoptional{\createChangelog}
    \lsucoptional{\addToDocumentation}
  \end{lsucrequires}

  \lsucprohibitsnothing
\end{lsuc}

% ------------------------------------------------------------------------------
\subsection{GPL-\ver-C7: Passing a modified program as binary}
\begin{lsuc}{GPL-\ver-C7}
  \linkosuc{04B}

  \useCaseSeven

  \begin{lsucrequires}
    \lsucmandatory{\keepLicenseElements}
    \lsucmandatory{\gpltwoEnsureCopyrightNoticeBinary}
    \lsucmandatory{\giveLicense}\passingFilesCorrectly
    \lsucmandatory{\retainCopyrightNotices}
    \lsucmandatory{\markProgramModifications}
    \lsucmandatory{\addToCopyrightDialogApp}
    \lsucmandatory{\arrangeProgramChanges}\howToApplyTheseTerms
    \lsucmandatory{\makeModifiedSourceAvailable}
    \lsucmandatory{\describeHowToGetSource}
    \lsucsourcedist{GPL-\ver-C6}
    \lsucoptional{\createChangelog}
    \lsucoptional{\addToDocumentation}
  \end{lsucrequires}

  \lsucprohibitsnothing
\end{lsuc}

% ------------------------------------------------------------------------------
\subsection{GPL-\ver-C8: Passing a modified library as independent source code}
\begin{lsuc}{GPL-\ver-C8}
  \linkosuc{08S}

  \useCaseEight

  \begin{lsucrequires}
    \lsucmandatory{\keepLicenseElements}
    \lsucmandatory{\gpltwoEnsureCopyrightNoticeSource}
    \lsucmandatory{\giveLicense}\passingFilesCorrectly
    \lsucmandatory{\retainCopyrightNotices}
    \lsucmandatory{\markLibraryModifications}
    \lsucmandatory{\arrangeLibraryChanges}\howToApplyTheseTerms
    \lsucoptional{\createChangelog}
    \lsucoptional{\addToDocumentation}
  \end{lsucrequires}

  \lsucprohibitsnothing
\end{lsuc}

% ------------------------------------------------------------------------------
\subsection{GPL-\ver-C9: Passing a modified library as independent binary}
\begin{lsuc}{GPL-\ver-C9}
  \linkosuc{08B}

  \useCaseNine

  \begin{lsucrequires}
    \lsucmandatory{\keepLicenseElements}
    \lsucmandatory{\gpltwoEnsureCopyrightNoticeSource}  
    \lsucmandatory{\giveLicense}\passingFilesCorrectly
    \lsucmandatory{\retainCopyrightNotices}
    \lsucmandatory{\makeModifiedSourceAvailable}
    \lsucmandatory{\describeHowToGetSource}
    \lsucsourcedist{GPL-\ver-C8}
    \lsucmandatory{\markLibraryModifications}
    \lsucmandatory{\arrangeLibraryChanges}\howToApplyTheseTerms
    \lsucoptional{\createChangelog}
    \lsucoptional{\addToDocumentation}
  \end{lsucrequires}

  \lsucprohibitsnothing
\end{lsuc}

% ------------------------------------------------------------------------------
\subsection{GPL-\ver-CA: Passing a modified library as embedded source code}
\begin{lsuc}{GPL-\ver-CA}
  \linkosuc{10S}

  \useCaseA

  \begin{lsucrequires}
    \lsucmandatory{\keepLicenseElements}
    \lsucmandatory{\gpltwoEnsureCopyrightNoticeSource}
    \lsucmandatory{\giveLicense}\passingFilesCorrectly
    \lsucmandatory{\retainCopyrightNotices}
    \lsucmandatory{\addToCopyrightDialogLib}
    \lsucmandatory{\markEmbeddedModifications}
    \lsucmandatory{\arrangeEmbeddedChanges}\howToApplyTheseTerms
    \lsucmandatory{\arrangeEnclosingSources}
    \lsucoptional{\createChangelog}
    \lsucoptional{\addToDocumentation}
  \end{lsucrequires}

  \lsucprohibitsnothing
\end{lsuc}

% ------------------------------------------------------------------------------
\subsection{GPL-\ver-CB: Passing a modified library as embedded binary}
\begin{lsuc}{GPL-\ver-CB}
  \linkosuc{10B}

  \useCaseB

  \begin{lsucrequires}
    \lsucmandatory{\keepLicenseElements}
    \lsucmandatory{\gpltwoEnsureCopyrightNoticeBinary}
    \lsucmandatory{\giveLicense}\passingFilesCorrectly
    \lsucmandatory{\retainCopyrightNotices}
    \lsucmandatory{\makeAllSourcesAvailable}
    \lsucmandatory{\describeHowToGetSource}
    \lsucsourcedist{GPL-\ver-CA}
    \lsucmandatory{\addToCopyrightDialogLib}
    \lsucmandatory{\markEmbeddedModifications}
    \lsucmandatory{\arrangeEmbeddedChanges}\howToApplyTheseTerms
    \lsucmandatory{\arrangeEnclosingBinaries}
    \lsucoptional{\createChangelog}
    \lsucoptional{\addToDocumentation}
  \end{lsucrequires}

  \lsucprohibitsnothing
\end{lsuc}

% ------------------------------------------------------------------------------
\end{license}

%% =============================================================================
%% GPL-3.0 Use Cases

\renewcommand{\ver}{3.0}

\begin{license}{GPL3} 
\licensename{GPL-3.0}
\licensespec{General Public License Version 3}
\licenseabbrev{GPL}
%\licenseversion{3.0}

% ------------------------------------------------------------------------------
\subsection{GPL-\ver-C1: Using the software only for yourself}
\begin{lsuc}{GPL-\ver-C1}
  \linkosuc{01}
  \linkosuc{03L} 
  \linkosuc{03N} 
  \linkosuc{06L}
  \linkosuc{06N}
  \linkosuc{09L}
  \linkosuc{09N}

  \useCaseOne

  \begin{lsucrequiresnothing}
    \lsucitem{You are allowed to use any kind of GPL software in any sense and in
      any context without being obliged to do anything as long as you do not
      give the software to third parties.}
  \end{lsucrequiresnothing}

  \begin{lsucprohibits}
    \lsucitem{\noPatentLitigation}
  \end{lsucprohibits}
\end{lsuc}

% ------------------------------------------------------------------------------
\subsection{GPL-\ver-C2: Passing the unmodified software as independent sources}
\begin{lsuc}{GPL-\ver-C2}
  \linkosuc{02S}
  \linkosuc{05S}

  \useCaseTwo

  \begin{lsucrequires}
    \lsucmandatory{\keepLicenseElements}
    \lsucmandatory{\gplthreeEnsureCopyrightNoticeSource}
    \lsucmandatory{\giveLicense}\passingFilesCorrectly
    \lsucmandatory{\retainCopyrightNotices}
    \lsucoptional{\addToDocumentation}
  \end{lsucrequires}

  \begin{lsucprohibits}
    \lsucitem{\noPatentLitigation}
  \end{lsucprohibits}
\end{lsuc}

% ------------------------------------------------------------------------------
\subsection{GPL-\ver-C3: Passing the unmodified software as independent binaries} 
\begin{lsuc}{GPL-\ver-C3}
  \linkosuc{02B} 
  \linkosuc{05B}

  \useCaseThree

  \begin{lsucrequires}
    \lsucmandatory{\keepLicenseElements}
    \lsucmandatory{\gplthreeEnsureCopyrightNoticeBinary}
    \lsucmandatory{\giveLicense}\passingFilesCorrectly  
    \lsucmandatory{\makeUnmodifiedSourceAvailable}
    \lsucmandatory{\describeHowToGetSource}
    \lsucmandatory{\retainCopyrightNotices}
    \lsucsourcedist{GPL-\ver-C2}
    \lsucoptional{\addToDocumentation}
  \end{lsucrequires}

  \begin{lsucprohibits}
    \lsucitem{\noPatentLitigation}
  \end{lsucprohibits}
\end{lsuc}

% ------------------------------------------------------------------------------
\subsection{GPL-\ver-C4: Passing the unmodified library as embedded sources}
\begin{lsuc}{GPL-\ver-C4}
  \linkosuc{07S} 

  \useCaseFour

  \begin{lsucrequires}
    \lsucmandatory{\keepLicenseElements}
    \lsucmandatory{\gplthreeEnsureCopyrightNoticeSource}
    \lsucmandatory{\giveLicense}\passingFilesCorrectly
    \lsucmandatory{\retainCopyrightNotices}
    \lsucmandatory{\addToCopyrightDialogLib}
    \lsucmandatory{\arrangeEnclosingSources}
    \lsucoptional{\addToDocumentation}
  \end{lsucrequires}

  \begin{lsucprohibits}
    \lsucitem{\noPatentLitigation}
  \end{lsucprohibits}
\end{lsuc}

% ------------------------------------------------------------------------------
\subsection{GPL-\ver-C5: Passing the unmodified library as embedded binaries} 
\begin{lsuc}{GPL-\ver-C5}
  \linkosuc{07B} 

  \useCaseFive

  \begin{lsucrequires}
    \lsucmandatory{\keepLicenseElements}
    \lsucmandatory{\gplthreeEnsureCopyrightNoticeBinary}
    \lsucmandatory{\giveLicense}\passingFilesCorrectly
    \lsucmandatory{\makeAllSourcesAvailable}
    \lsucmandatory{\describeHowToGetSource}
    \lsucmandatory{\addToCopyrightDialogLib}
    \lsucmandatory{\arrangeEnclosingBinaries}
    \lsucmandatory{\retainCopyrightNotices}
    \lsucsourcedist{GPL-\ver-C4}
    \lsucoptional{\addToDocumentation}
  \end{lsucrequires}

  \begin{lsucprohibits}
    \lsucitem{\noPatentLitigation}
  \end{lsucprohibits}
\end{lsuc}

% ------------------------------------------------------------------------------
\subsection{GPL-\ver-C6: Passing a modified program as source code}
\begin{lsuc}{GPL-\ver-C6}
  \linkosuc{04S} 

  \useCaseSix

  \begin{lsucrequires}
    \lsucmandatory{\keepLicenseElements}
    \lsucmandatory{\gplthreeEnsureCopyrightNoticeSource}
    \lsucmandatory{\giveLicense}\passingFilesCorrectly
    \lsucmandatory{\retainCopyrightNotices}
    \lsucmandatory{\addToCopyrightDialogApp}
    \lsucmandatory{\markProgramModifications}
    \lsucmandatory{\arrangeProgramChanges}\howToApplyTheseTerms
    \lsucoptional{\createChangelog}
    \lsucoptional{\addToDocumentation}
  \end{lsucrequires}

  \begin{lsucprohibits}
    \lsucitem{\noPatentLitigation}
  \end{lsucprohibits}
\end{lsuc}

% ------------------------------------------------------------------------------
\subsection{GPL-\ver-C7: Passing a modified program as binary}
\begin{lsuc}{GPL-\ver-C7}
  \linkosuc{04B}

  \useCaseSeven

  \begin{lsucrequires}
    \lsucmandatory{\keepLicenseElements}
    \lsucmandatory{\gplthreeEnsureCopyrightNoticeBinary}
    \lsucmandatory{\giveLicense}\passingFilesCorrectly
    \lsucmandatory{\retainCopyrightNotices}
    \lsucmandatory{\markProgramModifications}
    \lsucmandatory{\addToCopyrightDialogApp}
    \lsucmandatory{\arrangeProgramChanges}\howToApplyTheseTerms
    \lsucmandatory{\makeModifiedSourceAvailable}
    \lsucmandatory{\describeHowToGetSource}
    \lsucsourcedist{GPL-\ver-C6}
    \lsucoptional{\createChangelog}
    \lsucoptional{\addToDocumentation}
  \end{lsucrequires}

  \begin{lsucprohibits}
    \lsucitem{\noPatentLitigation}
  \end{lsucprohibits}
\end{lsuc}

% ------------------------------------------------------------------------------
\subsection{GPL-\ver-C8: Passing a modified library as independent source code}
\begin{lsuc}{GPL-\ver-C8}
  \linkosuc{08S}

  \useCaseEight

  \begin{lsucrequires}
     \lsucmandatory{\keepLicenseElements}
    \lsucmandatory{\gplthreeEnsureCopyrightNoticeSource}
    \lsucmandatory{\giveLicense}\passingFilesCorrectly
    \lsucmandatory{\retainCopyrightNotices}
    \lsucmandatory{\markLibraryModifications}
    \lsucmandatory{\arrangeLibraryChanges}\howToApplyTheseTerms
    \lsucoptional{\createChangelog}
    \lsucoptional{\addToDocumentation}
  \end{lsucrequires}

  \begin{lsucprohibits}
    \lsucitem{\noPatentLitigation}
  \end{lsucprohibits}
\end{lsuc}

% ------------------------------------------------------------------------------
\subsection{GPL-\ver-C9: Passing a modified library as independent binary}
\begin{lsuc}{GPL-\ver-C9}
  \linkosuc{08B}

  \useCaseNine

  \begin{lsucrequires}
    \lsucmandatory{\keepLicenseElements}
    \lsucmandatory{\gplthreeEnsureCopyrightNoticeBinary}
    \lsucmandatory{\giveLicense}\passingFilesCorrectly
    \lsucmandatory{\retainCopyrightNotices}
    \lsucmandatory{\makeModifiedSourceAvailable}
    \lsucmandatory{\describeHowToGetSource}
    \lsucsourcedist{GPL-\ver-C8}
    \lsucmandatory{\markLibraryModifications}
    \lsucmandatory{\arrangeLibraryChanges}\howToApplyTheseTerms
    \lsucoptional{\createChangelog}
    \lsucoptional{\addToDocumentation}
  \end{lsucrequires}

  \begin{lsucprohibits}
    \lsucitem{\noPatentLitigation}
  \end{lsucprohibits}
\end{lsuc}

% ------------------------------------------------------------------------------
\subsection{GPL-\ver-CA: Passing a modified library as embedded source code}
\begin{lsuc}{GPL-\ver-CA}
  \linkosuc{10S}

  \useCaseA

  \begin{lsucrequires}
    \lsucmandatory{\keepLicenseElements}
    \lsucmandatory{\gplthreeEnsureCopyrightNoticeSource}
    \lsucmandatory{\giveLicense}\passingFilesCorrectly
    \lsucmandatory{\retainCopyrightNotices}
    \lsucmandatory{\addToCopyrightDialogLib}
    \lsucmandatory{\markEmbeddedModifications}
    \lsucmandatory{\arrangeEmbeddedChanges}\howToApplyTheseTerms
    \lsucmandatory{\arrangeEnclosingSources}
    \lsucoptional{\createChangelog}
    \lsucoptional{\addToDocumentation}
  \end{lsucrequires}

  \begin{lsucprohibits}
    \lsucitem{\noPatentLitigation}
  \end{lsucprohibits}
\end{lsuc}

% ------------------------------------------------------------------------------
\subsection{GPL-\ver-CB: Passing a modified library as embedded binary}
\begin{lsuc}{GPL-\ver-CB}
  \linkosuc{10B}

  \useCaseB

  \begin{lsucrequires}
    \lsucmandatory{\keepLicenseElements}
    \lsucmandatory{\gplthreeEnsureCopyrightNoticeBinary}
    \lsucmandatory{\giveLicense}\passingFilesCorrectly
    \lsucmandatory{\retainCopyrightNotices}
    \lsucmandatory{\makeAllSourcesAvailable}
    \lsucmandatory{\describeHowToGetSource}
    \lsucsourcedist{GPL-\ver-CA}
    \lsucmandatory{\addToCopyrightDialogLib}
    \lsucmandatory{\markEmbeddedModifications}
    \lsucmandatory{\arrangeEmbeddedChanges}\howToApplyTheseTerms
    \lsucmandatory{\arrangeEnclosingBinaries}
    \lsucoptional{\createChangelog}
    \lsucoptional{\addToDocumentation}
  \end{lsucrequires}

  \begin{lsucprohibits}
    \lsucitem{\noPatentLitigation}
  \end{lsucprohibits}
\end{lsuc}

% ------------------------------------------------------------------------------
\end{license}

%% =============================================================================
%% Discussion

\subsection{Discussions and Explanations}
\label{GPL2Discussion}%
\label{GPL3Discussion}

\newcommand{\gplTwoAndThree}[2]{\footnote{%
    For GPL-2.0 see \cite[cf.][\nopage wp.\ #1]{Gpl20OsiLicense1991a}.\par\noindent 
    For GPL-3.0 see \cite[cf.][\nopage wp.\ #2]{Gpl30OsiLicense2007a}.}} 

The GPL-2.0 allows to \enquote{[\ldots] copy and (to) distribute verbatim copies
of the Program's complete source code as you receive it [...] provided that
you [a] conspicuously and appropriately publish on each copy an appropriate
copyright notice and disclaimer of warranty; [b] keep intact all the notices
that refer to this License and to the absence of any warranty; and [c]
distribute a copy of this License along with the Program.}\citeGPLtwo{§1} The
GPL-2.0 also allows to \enquote{[\ldots] copy and distribute [\ldots]
modifications (of the Program or any portion of it) [\ldots] under the terms
of Section~1}\citeGPLtwo{§2} while it allows to distribute binaries
\enquote{under the terms of Sections 1 and~2}.\citeGPLtwo{§4} But the GPL-2.0
does not require any tasks if you are using the work only for yourself. Thus,
the quoted conditions of \enquote{Section~1} are mandatory for all use cases
concerning the distribution of an GPL-2.0 licensed work (GPL-2.0-C2 --
GPL-2.0-CB)
  
\label{Gpl3ConditionsDistri}
The GPL-3.0 uses a similar structure to establish the same requirements: In §4
it allows to \enquote{[\ldots] convey verbatim copies of the Program's source
code as you receive it [\ldots] provided that you conspicuously and
appropriately publish on each copy an appropriate copyright notice; keep
intact all notices stating that this License and any non-permissive terms
added in accord with section 7 apply to the code; keep intact all notices of
the absence of any warranty; and give all recipients a copy of this License
along with the Program}. §5 also allows to \enquote{[\ldots] convey [\ldots]
modifications [\ldots] under the terms of section 4 [\ldots]} and §6 gives
permission to \enquote{[\ldots] convey a covered work in object form under the
terms of sections of 4 and 5}.\citeGPLthree{§4, §5, §6} In contrast to the
GPL-2.0, the GPL-3.0 explicitly states that one \enquote{[\ldots] may make, run
and propagate covered works that (one) (does) not convey [distribute], without
conditions so long as (the GPL-3.0) license otherwise remains in
force.}\citeGPLthree{§2}
% TODO: rephrase, including hosting the software, and name the conditions:
% exclusively for 'you' and under 'your' direction and control, and no copies
% of your own copyrighted material
Moreover, giving a package to a third party for getting a modified version back
has not to be taken as a case of distribution if the modification has only been
executed on behalf and only for the purpose of the purchaser and if the modified
version is not distributed to any third party.\citeGPLthree{§2} If one collects
all these GPL-3.0 statements together, than one may conclude that the tasks
which fulfill the corresponding GPL-2.0 requirements together also fit the
GPL-3.0 conditions.
  
The GPL-2.0 allows to \enquote{[\ldots] copy and (to) distribute the Program (or
a work based on it [\ldots]) in object code or executable form [\ldots] provided
that you accompany it with the complete corresponding machine-readable source
code [\ldots] on a medium customarily used for software
interchange}.\citeGPLtwo{§3, §3a} As a substitution for this basic condition,
the GPL-2.0 allows to \enquote{accompany} the binary distribution package
\enquote{[\ldots] with a written offer, valid for at least three years, to give
any third party, for a charge no more than your cost of physically performing
source distribution, a complete machine-readable copy of the corresponding
source code [\ldots] on a medium customarily used for software
interchange}.\citeGPLtwo{§3b} The \oslic{} construes the common technique to
download files from the Internet as a distribution \emph{on a medium [being
today] customarily used for software interchange}. Therefore, the \oslic{} requires
for all open source use cases that refer to the distribution of binaries
(GPL-2.0-C3, GPL-2.0-C7, GPL-2.0-C9, GPL-2.0-CB) to make the corresponding
source code of the library itself accessible via an Internet repository under
your own control. 
  
\label{Gpl3CondCopyleft}
The GPL-3.0 also explicitly requires to make the source code accessible in case
of distributing binaries. But opposite to the GPL-2.0, the GPL-3.0 explicitly
offers the option of giving \enquote{[\ldots] access to copy the Corresponding
Source from a network server at no charge} as a means to fulfill the
conditions.\citeGPLthree{§6 and §6b} So again, the tasks which ensure to act in
accordance to the GPL-2.0 license in case of distributing binaries, also fulfill
the conditions of the GPL-3.0.

The weakness that in this case \enquote{third parties [which have received the
binaries] are not compelled to copy the source code [\ldots]} is a concession
made by the GPL-2.0.\citeGPLtwo{§3, at the end} But the necessity to offer the
source code via a repository controlled by yourself may generally not be
circumvented: The GPL-2.0 allows to redistribute a link to an external source
code repository only in case of \enquote{noncommercial
distributions}.\citeGPLtwo{§3c} 
  
Both, the GPL-2.0 and the GPL-3.0 allow you to \enquote{[\ldots] modify your
copy or copies of the Program or any portion of it [\ldots] and (to) copy and
distribute such modifications [\ldots]} only under very similar restrictions and
conditions:\citeGPLtwo{§2} 
\begin{itemize}
\item First, modified files must be marked as modifications and the date of the
  modification.\gplTwoAndThree{§2}{§5} These conditions must be respected by all
  open source use cases concerning the distribution of the modified work
  [GPL-2.0-C6/GPL-3.0C6 -- GPL-2.0-C9/GPL-3.0-C9], because even if one primarily
  intends to distribute binaries, one has also to deliver the source code. The
  \oslic{} captures this requirement in the mandatory condition to mark each
  modified file and the voluntary condition to update / generate a general
  changelog.
    
\item Second, both versions of the GPL require that all copies of the modified
  software which are using an interactive interface or a method to display
  messages must \enquote{[\ldots] print or display an announcement including an
  appropriate copyright notice and a notice that there is no warranty [\ldots]
  and that users may redistribute the program under these conditions, and
  telling the user how to view a copy of this License.}\gplTwoAndThree{§2c}{§5d}
  The \oslic{} rewrites this condition in the form that the work shall let its
  copyright dialog clearly reproduce the content of the existing copyright
  notices, the software name, a link to its homepage, the respective disclaimer
  of warranty, and a link to the GPL-2.0-file (or GPL-3.0-file, resp.), which
  has to be delivered together with the software.
  % TODO: actually, the task does not refer to the _file_
  These conditions have to be respected if one redistributes the received and
  then modified programs (GPL-2.0-C6, GPL-2.0-C7, GPL-3.0-C6, GPL-3.0-C7) or if
  one distributes one's own programs which are using (modified) libraries as
  embedded components (GPL-2.0-CA, GPL-2.0-CB, GPL-3.0-CA, GPL-3.0-CB). For
  those open source use cases that concern the redistribution of received and
  modified libraries, etc., the \oslic{} does not mention these requirements
  because libraries, plugins, or snippets normally do not have their own
  copyright dialogs.  
    
\item Third, the GPL requires to \enquote{ [\ldots] cause any work (being
  distributed or published), that in whole or in part contains or is derived
  from the Program or any part thereof, to be licensed as a whole at no charge
  to all third parties under the terms of this (GPL.)}\gplTwoAndThree{§2b}{§5c}
  This requirement does not depend of the form in which the software is
  distributed. The \oslic{} adopts this statement in the following way:
  \begin{itemize}
  \item For all open source use cases which concern the distribution (GPL-2.0-C2
    \ldots GPL-2.0-CB, GPL-3.0-C2 \ldots GPL-3.0-CB), the \oslic{} rewrites this
    condition as the mandatory requirement to retain all existing licensing
    elements.
      
  \item For all use cases which deal with the distribution of a modified version
    of the software (GPL-2.0-C6 \ldots GPL-2.0-CB, GPL-3.0-C6 \ldots
    GPL-3.0-CB), the OSliC adds the requirement to organize the modifications in
    a way that they are covered by the respective GPL-2.0 or GPL-3.0 licensing
    statements.
      
  \item For the use case which deal with the distribution of an embedded library
    (GPL-2.0-C4, GPL-2.0-C5, GPL-2.0-CA, GPL-2.0-CB, GPL-3.0-C4, GPL-3.0-C5,
    GPL-3.0-CA, GPL-3.0-CB) the \oslic{} requires also to license the on-top
    development under the terms of the respective GPL-2.0 or GPL-3.0 license.
    \end{itemize}
   
\item Finally, as parts of those task lists which concern the distribution in
  the form of binaries, the \oslic{} reminds the reader also to execute the
  corresponding source code use cases because distributing the binaries without
  making the corresponding sources accessible is not allowed by the GPL.
\end{itemize}

And a last issue should be addressed here. It concerns the problem of
granularity.

The GPL-3.0 allows \enquote{[\ldots] to convey a covered work in object code
form [\ldots] provided that [one] also conveys the [\ldots] Corresponding
Source [\ldots]}\citeGPLthree{§6}. For understanding the scope of the sources
one has to convey, one must known, what the term \emph{Corresponding Source}
means. Fortunately, the GPL-3.0 assists its readers to understand this term in
the right way:

\begin{itemize}
  \item  \enquote{The \enquote{Corresponding Source} for a work in object code
  form means all the source code needed to generate, install, and (for an
  executable work) run the object code and to modify the work, including scripts
  to control those activities.\footcite[cf.][\nopage wp.
  §1]{Gpl30OsiLicense2007a}} Thus, if one took this statements seriously, one
  would have to \enquote{provide access to} the complete software stack of the
  executed AGPL program, just down to the glibc. But the GPL does not want to
  be to greedy. Therefore it limits the scope:
  \item To limit the sope, the GPL states, that the \emph{Corresponding Source}
  \enquote{[\ldots] does not include the work's System Libraries, or
  general-purpose tools or generally available free programs which are used
  unmodified in performing those activities but which are not part of the
  work}\footcite[cf.][\nopage wp. §1]{Gpl30OsiLicense2007a}. Unfortunately, one
  now has to analyze, what the term \emph{System Libraries} means, if one wants
  to understand this rule correctly.
  \item Therefore, the GPl says also, that \enquote{the \enquote{System
  Libraries} of an executable work include anything, other than the work as a
  whole, that (a) is included in the normal form of packaging a Major Component,
  but which is not part of that Major Component, and (b) serves only to enable
  use of the work with that Major Component, or to implement a Standard
  Interface for which an implementation is available to the public in source
  code form.\footcite[cf.][\nopage wp. §1]{Gpl30OsiLicense2007a}}. And for
  understing this sentence adequately, one has to know, what a \emph{Major Component}
  is.
  \item So, finally, the GPL defines as \enquote{enquote{Major Component}
  [\ldots as] a major essential component (kernel, window system, and so on) of
  the specific operating system (if any) on which the executable work runs, or a
  compiler used to produce the work, or an object code interpreter used to run
  it\footcite[cf.][\nopage wp. §1]{Gpl30OsiLicense2007a}}.
\end{itemize}

Based on these specifications, one can give some rule of thumbs concerning the
question down to which level one has to give access to the corresponding source
code of an conveyed GPL binary program:
\begin{itemize}
  \item If one conveys a GPL licensed binary program, then one has
  also to deliver the code of
  \begin{itemize}
  \item the dlivered program itself
  \item every modified embedded component of that program
  \item every not freely accessible embedded component of that program
  \item all not freely accessible tools, scripts, data which are necessary to
  compile the sources of the program in a freely accessible compilation /
  developement environment
  \end{itemize}
  But it is not necessary to deliver the code of unmodified standard libraries,
  compilers, or tools which can freely be downloaded from their standard
  repository.
  \item If one conveys a GPL licensed script, then one has also to deliver the
  code of
  \begin{itemize}
  \item every modified embedded script component included by the main script
  \item every not freely accessible embedded script component included by the main script
  \item all not freely accessible tools, scripts, data which are necessary to
  to let that main script be executed by a freely accessible interpreter
  \item the interpreter itself if it is not freely accessible.
  \end{itemize}
  But it is not necessary to give access to unmodified standard script
  libraries, interpreters, or tools which can freely be downloaded from their
  standard repository.
\end{itemize}


  
%\bibliography{../../../bibfiles/oscResourcesEn}

% Local Variables:
% mode: latex
% fill-column: 80
% End:
