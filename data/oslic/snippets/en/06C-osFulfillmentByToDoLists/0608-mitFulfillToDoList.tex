% Telekom osCompendium 'for being included' snippet template
%
% (c) Karsten Reincke, Deutsche Telekom AG, Darmstadt 2011
%
% This LaTeX-File is licensed under the Creative Commons Attribution-ShareAlike
% 3.0 Germany License (http://creativecommons.org/licenses/by-sa/3.0/de/): Feel
% free 'to share (to copy, distribute and transmit)' or 'to remix (to adapt)'
% it, if you '... distribute the resulting work under the same or similar
% license to this one' and if you respect how 'you must attribute the work in
% the manner specified by the author ...':
%
% In an internet based reuse please link the reused parts to www.telekom.com and
% mention the original authors and Deutsche Telekom AG in a suitable manner. In
% a paper-like reuse please insert a short hint to www.telekom.com and to the
% original authors and Deutsche Telekom AG into your preface. For normal
% quotations please use the scientific standard to cite.
%
% [ Framework derived from 'mind your Scholar Research Framework' 
%   mycsrf (c) K. Reincke 2012 CC BY 3.0  http://mycsrf.fodina.de/ ]
%


%% use all entries of the bibliography
%\nocite{*}

\section{MIT licensed software}

\begin{license}{MIT}
\licensename{MIT}
\licensespec{MIT License}
\licenseabbrev{MIT}

The MIT license is known as one of the most permissive licenses. Thus, the
MIT specific finder can be simplified:

\tikzstyle{nodv} = [font=\small, ellipse, draw, fill=gray!10, 
    text width=2cm, text centered, minimum height=2em]

\tikzstyle{nods} = [font=\footnotesize, rectangle, draw, fill=gray!20, 
    text width=1.2cm, text centered, rounded corners, minimum height=3em]

\tikzstyle{nodb} = [font=\footnotesize, rectangle, draw, fill=gray!20, 
    text width=2.2cm, text centered, rounded corners, minimum height=3em]

\tikzstyle{nodx} = [font=\footnotesize, rectangle, draw, fill=gray!20, 
    text width=2.4cm, text centered, rounded corners, minimum height=3em]
    
\tikzstyle{leaf} = [font=\tiny, rectangle, draw, fill=gray!30, 
    text width=1.2cm, text centered, minimum height=6em]

\tikzstyle{edge} = [draw, -latex']

\begin{tikzpicture}[]

\node[nodv] (l61) at ( 2.4, 9.2) {MIT};

\node[nodb] (l51) at ( 0.0, 7.8) {\textit{recipient:} \\ \textbf{4yourself}};
\node[nodb] (l52) at ( 4.8, 7.8) {\textit{recipient:} \\ \textbf{2others}};

\node[nodb] (l41) at ( 2.5, 6.2) {\textit{state:} \\ \textbf{unmodified}};
\node[nodb] (l42) at ( 7.0, 6.2) {\textit{state:} \\ \textbf{modified}};

\node[nodb] (l31) at ( 5.0, 4.6) {\textit{type:} \\ \textbf{proapse}};
\node[nodb] (l32) at ( 9.0, 4.6) {\textit{type:} \\ \textbf{snimoli}};

\node[nodx] (l21) at ( 7.5, 2.8) {\textit{context:} \\ \textbf{independent}};
\node[nodx] (l22) at (10.5, 2.8) {\textit{context:} \\ \textbf{embedded}};

\node[leaf] (l11) at ( 0.0, 0.0) {\textbf{MIT-C1} \textit{using software only for yourself}};
\node[leaf] (l12) at ( 2.5, 0.0) {\textbf{MIT-C2} \textit{distributing unmodified package}};
\node[leaf] (l13) at ( 5.0, 0.0) {\textbf{MIT-C3} \textit{distributing modified program}};
\node[leaf] (l14) at ( 7.5, 0.0) {\textbf{MIT-C4} \textit{distributing modified library as independent package}};
\node[leaf] (l15) at (10.5, 0.0) {\textbf{MIT-C5} \textit{distributing modified library as embedded package}};


\path [edge] (l61) -- (l51);
\path [edge] (l61) -- (l52);
\path [edge] (l51) -- (l11);
\path [edge] (l52) -- (l41);
\path [edge] (l52) -- (l42);
\path [edge] (l41) -- (l12);
\path [edge] (l42) -- (l31);
\path [edge] (l42) -- (l32);
\path [edge] (l31) -- (l13);
\path [edge] (l32) -- (l21);
\path [edge] (l32) -- (l22);
\path [edge] (l21) -- (l14);
\path [edge] (l22) -- (l15);

\end{tikzpicture}

%%
%% Common building blocks
%%

% ------------------------------------------------------------------------------
% License elements must be preserved
\newcommand{\keepLicensingElements}{Ensure that the licensing elements
  (especially the MIT license text containing the specific copyright notices of
  the original author(s), the permission notices and the MIT disclaimer) are
  retained in your package in the form you have received them.}

% ------------------------------------------------------------------------------
% Add a link to the project home page
\newcommand{\linkToProject}{It's a good tradition to let the documentation of
  your distribution and/or your additional material also contain a link to the 
  original software (project) and its homepage.}

% ------------------------------------------------------------------------------
% Mark your modifications of the software
\newcommand{\markYourModifications}{Mark your modifications in the source code,
  regardless whether you want to distribute the code or not.}

% ------------------------------------------------------------------------------
% Add a copyright notice for your modifications 
\newcommand{\addYourCopyrightNotice}{You can augment an existing copyright 
  notice presented by the program with information about your own work or
  modifications.}

% ------------------------------------------------------------------------------
% Acknowledge the use of the OSS
\newcommand{\displayAcknowledement}{It is a good practice of the open source  
  community to let the copyright notice that is shown by the running program  
  also state that the program uses a component licensed under the MIT
  license. And it is a good tradition to insert links to the homepage or 
  download page of this component.} 

% ------------------------------------------------------------------------------

\subsection{MIT-C1: Using the software only for yourself}
\begin{lsuc}{MIT-C1}
  \linkosuc{01}
  \linkosuc{03L} 
  \linkosuc{03N} 
  \linkosuc{06L}
  \linkosuc{06N}
  \linkosuc{09L}
  \linkosuc{09N}
  
  \lsucmeans{that you received MIT licensed software, that you will use it
  only for yourself and that you do not hand it over to any 3rd party in any
  sense.}

  \lsuccovers{OSUC-01, OSUC-03L, OSUC-03N, OSUC-06L, OSUC-06N, OSUC-09L, and
  OSUC-09N\footnote{For details $\rightarrow$ \oslic, pp.\ \pageref{OSUC-01-DEF}
  - \pageref{OSUC-09N-DEF}}}

  \begin{lsucrequiresnothing}
    \lsucitem{You are allowed to use any kind of MIT licensed software in any
      sense and in any context without any obligations if you do not give the
      software to third parties and if you do not modify the existing copyright
      notices and the existing permission notice.}
  \end{lsucrequiresnothing}

  \lsucprohibitsnothing
\end{lsuc}

\subsection{MIT-C2: Passing the unmodified software}
\begin{lsuc}{MIT-C2}
  \linkosuc{02S} 
  \linkosuc{05S} 
  \linkosuc{07S} 
  \linkosuc{02B} 
  \linkosuc{05B} 
  \linkosuc{07B} 

  \lsucmeans{that you received MIT licensed software which you are now going to
  distribute to third parties in the form of unmodified binaries or as unmodifed
  source code files. In this case it makes no difference if you distribute a
  program, an application, a server, a snippet, a module, a library, or a plugin
  as an independent package.}

  \lsuccovers{OSUC-02S,  OSUC-02B, OSUC-05S, OSUC-05B, OSUC-07S,
    OSUC-07B\footnote{For details $\rightarrow$ \oslic,
      pp.\ \pageref{OSUC-02S-DEF} - \pageref{OSUC-07B-DEF}}} 

  \begin{lsucrequires}
    \lsucmandatory{\keepLicensingElements}
    \lsucoptional{\linkToProject}
  \end{lsucrequires}

  \lsucprohibitsnothing
\end{lsuc}


\subsection{MIT-C3: Passing a modified program}
\begin{lsuc}{MIT-C3}
  \linkosuc{04S}
  \linkosuc{04B}

  \lsucmeans{that you received an MIT licensed program, application, or server
  (proapse), that you modified it, and that you are now going to distribute this
  modified version to third parties in the form binaries or as source code
  files.}

  \lsuccovers{OSUC-04S, OSUC-04B\footnote{For details $\rightarrow$ \oslic,
      pp.\ \pageref{OSUC-04S-DEF}}}

  \begin{lsucrequires}
    \lsucmandatory{\keepLicensingElements}
    \lsucoptional{\markYourModifications}
    \lsucoptional{\linkToProject}
    \lsucoptional{\addYourCopyrightNotice} 
    \lsucoptional{\displayAcknowledement}
  \end{lsucrequires}

  \lsucprohibitsnothing
\end{lsuc}

\subsection{MIT-C4: Passing a modified library independently}
\begin{lsuc}{MIT-C4}
  \linkosuc{08S}
  \linkosuc{08B}

  \lsucmeans{that you received an MIT licensed code snippet, module, library, or
  plugin (snimoli), that you modified it, and that you are now going to
  distribute this modified version to third parties in the the form of binaries
  or as source code files, but without embedding it into another larger software
  unit.}

  \lsuccovers{OSUC-08S, OSUC-08B\footnote{For details $\rightarrow$ \oslic,
      pp.\ \pageref{OSUC-08S-DEF}}}

  \begin{lsucrequires}
    \lsucmandatory{\keepLicensingElements}
    \lsucoptional{\markYourModifications}
    \lsucoptional{\linkToProject}
  \end{lsucrequires}

  \lsucprohibitsnothing
\end{lsuc}


\subsection{MIT-C5: Passing a modified library as embedded component}
\begin{lsuc}{MIT-C5}
  \linkosuc{10S}
  \linkosuc{10B}

  \lsucmeans{that you received an MIT licensed code snippet, module, library, or
  plugin (snimoli), that you modified it, and that you are now going to
  distribute this modified version to third parties in the form of binaries or
  as source code files together with another larger software unit which contains
  this code snippet, module, library, or plugin as an embedded component,
  regardless whether you distribute it in the form of binaries or as source code
  files.}

  \lsuccovers{OSUC-10S, OSUC-10B\footnote{For details $\rightarrow$ \oslic,
      pp.\ \pageref{OSUC-10S-DEF}}}

  \begin{lsucrequires}
    \lsucmandatory{\keepLicensingElements}
    \lsucoptional{\markYourModifications}
    \lsucoptional{\displayAcknowledement}
    \lsucoptional{\linkToProject}

    \lsucoptional{Arrange your distribution so that the original licensing
      elements (especially the MIT license text containing the specific
      copyright notices of the original author(s), the permission notices and
      the MIT disclaimer) clearly refer only to the embedded library and do not
      disturb the licensing of your own overarching work. It's a good tradition
      to keep the libraries, modules, snippet, or plugins in separate
      directories, which contain also all licensing elements.}
  \end{lsucrequires}

  \lsucprohibitsnothing
\end{lsuc}

\subsection{Discussions and Explanations}
\label{MITDiscussion}
The MIT-License is known as one of the most permissive licenses. It is a very
short license containing (0) a copyright notice, (1) a paragraph saying that you
are allowed to do almost anything you want, followed (2) by the condition that
you have to \enquote{include} the existing copyright notes and the permission
notes \enquote{[\ldots] in all copies or substantial portions of the software},
and (3) closed by the well known disclaimer.\citeMIT{} But the license doesn't
talk about the difference of source code and object code. So, you have to find
the right way by yourself. 
Here are our readings:

\begin{itemize}
  \item If you do not modify the received MIT licensed application, neither for
    your own purposes, nor for handing over the program to 3rd parties, you can 
    conclude that all copyright notices and permission notices are already
    correct.
  \item Nevertheless, we added the hint not to modify these licensing elements
    in the context of the use case \emph{used by yourself}. This is implied by
    the MIT license itself. It requires explicitly that \enquote{the above
      copyright notice and this permission notice shall be included in all
      copies or substantial portions of the Software}\citeMIT---thus also into
    those copies you make for your own purposes on your own machines. 
  \item If you modify the MIT licensed application, regardless for which
    purpose, you are simply not allowed to erase or modify existing copyright
    notes and permission notices. You may add your own modifications under new
    conditions, but the old notices must survive. 
  \item We request that you also keep the MIT disclaimer. This is not
    explicitely required by the license. The permission notices, which is
    required to be preserved, most likely refers to the text \emph{between} the
    copyright notice and the disclaimer and, hence, does not include the latter.
    But another possible, although less likely interpretation is that the whole
    text of the license is what permission notice refers to.  
\end{itemize}

\end{license}

%\bibliography{../../../bibfiles/oscResourcesEn}

% Local Variables:
% mode: latex
% fill-column: 80
% End:
