% Telekom osCompendium 'for being included' snippet template
%
% (c) Karsten Reincke, Deutsche Telekom AG, Darmstadt 2011
%
% This LaTeX-File is licensed under the Creative Commons Attribution-ShareAlike
% 3.0 Germany License (http://creativecommons.org/licenses/by-sa/3.0/de/): Feel
% free 'to share (to copy, distribute and transmit)' or 'to remix (to adapt)'
% it, if you '... distribute the resulting work under the same or similar
% license to this one' and if you respect how 'you must attribute the work in
% the manner specified by the author ...':
%
% In an internet based reuse please link the reused parts to www.telekom.com and
% mention the original authors and Deutsche Telekom AG in a suitable manner. In
% a paper-like reuse please insert a short hint to www.telekom.com and to the
% original authors and Deutsche Telekom AG into your preface. For normal
% quotations please use the scientific standard to cite.
%
% [ Framework derived from 'mind your Scholar Research Framework' 
%   mycsrf (c) K. Reincke 2012 CC BY 3.0  http://mycsrf.fodina.de/ ]
%


%% use all entries of the bibliography
%\nocite{*}

\section{BSD licensed software}

As an approved open source license, the BSD license exists in two
versions%
  \footnote{Following the OSI, there is another `ancient' BSD
    license---containing a fourth clause known as advertising clause---which
    \enquote{(\ldots) officially was rescinded by the Director of the Office of
      Technology Licensing of the University of California on July 22nd, 1999}. 
    Because of that cancellation you can simply act according the  
    \cite[cf.][\nopage wp.]{BsdLicense3Clause} 
    if you have to fulfill the oldest of the BSD licenses.}  
The latest release is the \textit{BSD 2-Clause license,}\citeBSDsimple{}, the
older release is the \textit{BSD 3-Clause license.}\citeBSDnew{} The very little
differences between the two versions have to be respected exactly. 

All BSD open source licenses focus explicitely on the (re-)distribution
\textit{open source use cases,} which we have specified by our token
\textit{2others}. Conditions for the other use cases specified by the token
\textit{4yourself} can be derived.%
  \footnote{For details of the \textit{open source use case tokens} see
    p.\ \pageref{OsucTokens}. For details of the \textit{open source use cases}
    based on these token see p. \pageref{OsucDefinitionTree}} 
Additionally the BSD licenses distinguishes between different forms of distribution,
esp.\ whether the work is distributed as a (set of) source code file(s) or as a
set of binary file(s). Use the following tree to find the BSD license fulfilling
to-do lists. 

\tikzstyle{nodv} = [font=\small, ellipse, draw, fill=gray!10, 
    text width=2cm, text centered, minimum height=2em]


\tikzstyle{nods} = [font=\footnotesize, rectangle, draw, fill=gray!20, 
    text width=1.2cm, text centered, rounded corners, minimum height=3em]

\tikzstyle{nodb} = [font=\footnotesize, rectangle, draw, fill=gray!20, 
    text width=2.2cm, text centered, rounded corners, minimum height=3em]
    
\tikzstyle{leaf} = [font=\tiny, rectangle, draw, fill=gray!30, 
    text width=1.2cm, text centered, minimum height=6em]

\tikzstyle{edge} = [draw, -latex']

\begin{tikzpicture}[]

\node[nodv] (l81) at (4,11.8) {BSD};

\node[nodv] (l71) at (0,10.2) {3-Clause License};
\node[nodv] (l72) at (6.5,10.2) {2-Clause License};


\node[nodb] (l61) at (0,8.6) {\textit{recipient:} \\ \textbf{4yourself}};
\node[nodb] (l62) at (6.5,8.6) {\textit{recipient:} \\ \textbf{2others}};

\node[nodb] (l51) at (2.5,7) {\textit{state:} \\ \textbf{unmodified}};
\node[nodb] (l52) at (9.3,7) {\textit{state:} \\ \textbf{modified}};

\node[nods] (l41) at (1.8,5.4) {\textit{form:} \textbf{source}};
\node[nods] (l42) at (3.6,5.4) {\textit{form:} \textbf{binary}};
\node[nodb] (l43) at (6.5,5.4) {\textit{type:} \\ \textbf{proapse}};
\node[nodb] (l44) at (12,5.4) {\textit{type:} \\ \textbf{snimoli}};


\node[nods] (l31) at (5.4,3.8) {\textit{form:} \textbf{source}};
\node[nods] (l32) at (7.2,3.8) {\textit{form:} \textbf{binary}};
\node[nodb] (l33) at (10,3.8) {\textit{context:} \\ \textbf{independent}};
\node[nodb] (l34) at (13.5,3.8) {\textit{context:} \\ \textbf{embedded}};

\node[nods] (l21) at (9,2.2) {\textit{form:} \textbf{source}};
\node[nods] (l22) at (10.8,2.2) {\textit{form:} \textbf{binary}};
\node[nods] (l23) at (12.6,2.2) {\textit{form:} \textbf{source}};
\node[nods] (l24) at (14.4,2.2) {\textit{form:} \textbf{binary}};

\node[leaf] (l11) at (0,0) {
  \textbf{BSD2-C1} 
  \textbf{BSD3-C1} 
  \textit{using software only for yourself}};

\node[leaf] (l12) at (1.8,0) { 
  \textbf{BSD2-C2} 
  \textbf{BSD3-C2} 
  \textit{ distributing unmodified software as sources}};

\node[leaf] (l13) at (3.6,0) { 
  \textbf{BSD2-C3}  
  \textbf{BSD3-C3}  
  \textit{ distributing unmodified software as binaries}};

\node[leaf] (l14) at (5.4,0) { 
  \textbf{BSD2-C4}  
  \textbf{BSD3-C4}  
  \textit{ distributing modified program as sources}};

\node[leaf] (l15) at (7.2,0) { 
  \textbf{BSD2-C5}  
  \textbf{BSD3-C5}  
  \textit{ distributing modified program as binaries}};

\node[leaf] (l16) at (9,0) { 
  \textbf{BSD2-C6}  
  \textbf{BSD3-C6}  
  \textit{ distributing modified library as independent sources}};

\node[leaf] (l17) at (10.8,0) { 
  \textbf{BSD2-C7} 
  \textbf{BSD3-C7} 
  \textit{distributing modified library as independent binaries}};

\node[leaf] (l18) at (12.6,0) { 
  \textbf{BSD2-C8}  
  \textbf{BSD3-C8}  
  \textit{distributing modified library as embedded sources}};

\node[leaf] (l19) at (14.4,0) { 
  \textbf{BSD2-C9}  
  \textbf{BSD3-C9}  
  \textit{ distributing modified library as embedded binaries}};

\path [edge] (l81) -- (l71);
\path [edge] (l81) -- (l72);
\path [edge] (l71) -- (l61);
\path [edge] (l71) -- (l62);
\path [edge] (l72) -- (l61);
\path [edge] (l72) -- (l62);
\path [edge] (l61) -- (l11);
\path [edge] (l62) -- (l51);
\path [edge] (l62) -- (l52);
\path [edge] (l51) -- (l41);
\path [edge] (l51) -- (l42);
\path [edge] (l52) -- (l43);
\path [edge] (l52) -- (l44);
\path [edge] (l41) -- (l12);
\path [edge] (l42) -- (l13);
\path [edge] (l43) -- (l31);
\path [edge] (l43) -- (l32);
\path [edge] (l44) -- (l33);
\path [edge] (l44) -- (l34);
\path [edge] (l31) -- (l14);
\path [edge] (l32) -- (l15);
\path [edge] (l33) -- (l21);
\path [edge] (l33) -- (l22);
\path [edge] (l34) -- (l23);
\path [edge] (l34) -- (l24);
\path [edge] (l21) -- (l16);
\path [edge] (l22) -- (l17);
\path [edge] (l23) -- (l18);
\path [edge] (l24) -- (l19);

\end{tikzpicture}

%% ============================================================================= 
%% Common Building Blocks

% ------------------------------------------------------------------------------
% Common description of license specific use cases

\newcommand{\useCaseOne}{that you received BSD licensed software, that you will
  use it only for yourself and that you do not hand it over to any 3rd party in
  any sense.}

\newcommand{\useCaseTwo}{that you received BSD licensed software which you are
  now going to distribute to third parties in the form of unmodified source code
  files or as unmodified source code package. In this case it makes no
  difference if you distribute a program, an application, a server, a snippet, a
  module, a library, or a plugin as an independent or as an embedded unit.}

\newcommand{\useCaseThree}{that you received BSD licensed software which you are
  now going to distribute to third parties in the form of unmodified binary
  files or as unmodified binary package. In this case it does not matter if you
  distribute a program, an application, a server, a snippet, a module, a
  library, or a plugin as an independent or an embedded unit.}

\newcommand{\useCaseFour}{that you received a BSD licensed program, application,
  or server (proapse), that you modified it, and that you are now going to
  distribute this modified version to third parties in the form of source code
  files or as a source code package.}

\newcommand{\useCaseFive}{that you received a BSD licensed program, application,
  or server (proapse), that you modified it, and that you are now going to
  distribute this modified version to third parties in the form of binary files
  or as a binary package.}

\newcommand{\useCaseSix}{that you received a BSD licensed code snippet, module,
  library, or plugin (snimoli), that you modified it, and that you are now going
  to distribute this modified version to third parties in the form of source
  code files or as a source code package, but without embedding it into another
  larger software unit.}

\newcommand{\useCaseSeven}{that you received a BSD licensed code snippet,
  module, library, or plugin (snimoli), that you modified it, and that you are
  now going to distribute this modified version to third parties in the form of
  binary files or as a binary package but without embedding it into another
  larger software unit.}

\newcommand{\useCaseEight}{that you received a BSD licensed code snippet,
  module, library, or plugin (snimoli), that you modified it, and that you are
  now going to distribute this modified version to third parties in the form of
  source code files or as a source code package together with another larger
  software unit which contains this code snippet, module, library, or plugin as
  an embedded component.}

\newcommand{\useCaseNine}{that you received a BSD licensed code snippet, module,
  library, or plugin (snimoli), that you modified it, and that you are now going
  to distribute this modified version to third parties in the form of binary
  files or as a binary package together with another larger software unit which
  contains this code snippet, module, library, or plugin as an embedded
  component.}

% ------------------------------------------------------------------------------
% Common mapping to generic use cases

\newcommand{\coversOne}{\coversOsucs{OSUC-01, OSUC-03L, OSUC-03N, OSUC-06L,
OSUC-06N, OSUC-09L and OSUC-09N}{01}{09N}}
\newcommand{\coversTwo}{\coversOsucs{OSUC-02S, OSUC-05S, OSUC-07S}{02S}{07S}}
\newcommand{\coversThree}{\coversOsucs{OSUC-02B, OSUC-05B, OSUC-07B}{02B}{07B}}
\newcommand{\coversFour}{\mapsToOsuc{04S}}
\newcommand{\coversFive}{\mapsToOsuc{04B}}
\newcommand{\coversSix}{\mapsToOsuc{08S}}
\newcommand{\coversSeven}{\mapsToOsuc{08B}}
\newcommand{\coversEight}{\mapsToOsuc{10S}}
\newcommand{\coversNine}{\mapsToOsuc{10B}}

% ------------------------------------------------------------------------------
% Keep license elements

\newcommand{\keepLicenseElements}{Ensure that the licensing elements
  (particularly the BSD license text, the specific copyright notice of the
  original author(s), and the BSD disclaimer) are retained in your package in
  the form you have received them.}

\newcommand{\insertLicenseIntoBinary}{Ensure that your distribution contains the
  original copyright notice, the BSD license, and the BSD disclaimer in the form
  you have received them. 
  If you build the binary package from the source code package and if this does
  not automatically generate and integrate the licensing files then create the
  copyright notice, the BSD conditions, and the BSD disclaimer in the form found
  to the in the source code package and insert these files into your
  distribution manually.} 

% ------------------------------------------------------------------------------
% Include copyright notice and license conditions in documentation

\newcommand{\includeLicenseInDocumentation}{Let the documentation of your
  distribution or your additional material also contain the original copyright
  notice, the BSD conditions, and the BSD disclaimer.}

% ------------------------------------------------------------------------------
% Add name of the projecty and link to its homepage to copyright dialog

\newcommand{\addLicenseToCopyrightMessage}{%
  It is a good practice of the open source community to let the copyright
  message that is shown by the running program also state that the program is 
  licensed under the BSD license. 
  Because you are already modifying the program you can also add such a hint if
  the presented original copyright notice lacks such a statement.}

\newcommand{\addLibraryLicenseToCopyrightMessage}{%
  It is a good practice of the open source community to let the copyright
  message that is shown by the running program also state that it contains
  components licensed under the BSD license. 
  Because you are embedding this snimoli into a larger software unit, you are
  developing this larger unit. 
  Hence, you can also expand the copyright notice of this larger unit by such a
  hint to its BSD components.}

% ------------------------------------------------------------------------------
% Keep components separate so that it is clear which license applies
\newcommand{\auxKeepComponentsSeparate}[1]{Arrange your #1 distribution so that
  the licensing elements (particularly, the BSD license text, the copyright
  notice of the original author(s), and the BSD disclaimer) clearly refer only
  to the embedded library and do not affect the licensing of your own
  overarching work. 
  It's a good tradition to keep the embedded components like libraries, modules,
  snippets, or plugins in separate directories, which also contains all their
  licensing elements.}  

\newcommand{\keepSourcesSeparate}{\auxKeepComponentsSeparate{source code}}
\newcommand{\keepBinariesSeparate}{\auxKeepComponentsSeparate{binary}}

% ------------------------------------------------------------------------------
% No advertising using the authors' names (BSD3)

\newcommand{\dontUseAuthorNames}{to use the name of the licensing organization
  or the names of the licensing contributors to promote your own work.}

%% =============================================================================
%% BSD-3-Clause
%% =============================================================================

\begin{license}{BSD3} 
\licensename{BSD-3-Clause}
\licensespec{New BSD (3 Clauses)}
\licenseabbrev{BSD-3-Clause}

%% ============================================================================= 
%% Use Cases

\subsection{BSD-3-Clause-C1: Using the software only for yourself}
\begin{lsuc}{BSD-3-Clause-C1}
  \linkosuc{01}
  \linkosuc{03L} 
  \linkosuc{03N} 
  \linkosuc{06L}
  \linkosuc{06N}
  \linkosuc{09L}
  \linkosuc{09N}
  
  \lsucmeans{\useCaseOne} 
  \lsuccovers{\coversOne}

  \begin{lsucrequiresnothing}
    \lsucitem{You are allowed to use any kind of BSD software in any sense and
      in any context without any obligations as long as you do not give the
      software to 3rd parties.}
  \end{lsucrequiresnothing}

  \begin{lsucprohibits}
    \lsucitem{\dontUseAuthorNames}%
    \footnote{which may be, for example, an internet service based on this BSD
      software used in your own data center}. 
  \end{lsucprohibits}
\end{lsuc}

% ------------------------------------------------------------------------------
\subsection{BSD-3-Clause-C2: Passing the unmodified software as source code}
\begin{lsuc}{BSD-3-Clause-C2}
  \linkosuc{02S} 
  \linkosuc{05S} 
  \linkosuc{07S} 

  \lsucmeans{\useCaseTwo}
  \lsuccovers{\coversTwo}

  \begin{lsucrequires}
    \lsucmandatory{\keepLicenseElements}
    \lsucoptional{\includeLicenseInDocumentation}
  \end{lsucrequires}

  \begin{lsucprohibits}
    \lsucitem{\dontUseAuthorNames}%
  \end{lsucprohibits}
\end{lsuc}

% ------------------------------------------------------------------------------
\subsection{BSD-3-Clause-C3: Passing the unmodified software as binary}
\begin{lsuc}{BSD-3-Clause-C3}
  \linkosuc{02B} 
  \linkosuc{05B} 
  \linkosuc{07B} 

  \lsucmeans{\useCaseThree}
  \lsuccovers{\coversThree}

  \begin{lsucrequires}  
    \lsucmandatory{\insertLicenseIntoBinary}\passingFilesCorrectly
    \lsucmandatory{\includeLicenseInDocumentation}
  \end{lsucrequires}

  \begin{lsucprohibits}
    \lsucitem{\dontUseAuthorNames}%
  \end{lsucprohibits}
\end{lsuc}

% ------------------------------------------------------------------------------
\subsection{BSD-3-Clause-C4: Passing a modified program as source code}
\begin{lsuc}{BSD-3-Clause-C4}
  \linkosuc{04S}

  \lsucmeans{\useCaseFour}
  \lsuccovers{\coversFour}

  \begin{lsucrequires}
    \lsucmandatory{\keepLicenseElements}
    \lsucoptional{\includeLicenseInDocumentation}
    \lsucoptional{\addLicenseToCopyrightMessage}
  \end{lsucrequires}

  \begin{lsucprohibits}
    \lsucitem{\dontUseAuthorNames}%
  \end{lsucprohibits}
\end{lsuc}

% ------------------------------------------------------------------------------
\subsection{BSD-3-Clause-C5: Passing a modified program as binary}
\begin{lsuc}{BSD-3-Clause-C5}
  \linkosuc{04B}

  \lsucmeans{\useCaseFive}
  \lsuccovers{\coversFive}

  \begin{lsucrequires}
    \lsucmandatory{\insertLicenseIntoBinary}\passingFilesCorrectly
    \lsucmandatory{\includeLicenseInDocumentation}
    \lsucoptional{\addLicenseToCopyrightMessage}
  \end{lsucrequires}

  \begin{lsucprohibits}
    \lsucitem{\dontUseAuthorNames}%
  \end{lsucprohibits}
\end{lsuc}

% ------------------------------------------------------------------------------
\subsection{BSD-3-Clause-C6: Passing a modified library as independent source code}
\begin{lsuc}{BSD-3-Clause-C6}
  \linkosuc{08S}

  \lsucmeans{\useCaseSix}
  \lsuccovers{\coversSix}

  \begin{lsucrequires}
    \lsucmandatory{\keepLicenseElements}
    \lsucoptional{\includeLicenseInDocumentation}
  \end{lsucrequires}

  \begin{lsucprohibits}
    \lsucitem{\dontUseAuthorNames}%
  \end{lsucprohibits}
\end{lsuc}

% ------------------------------------------------------------------------------
\subsection{BSD-3-Clause-C7: Passing a modified library as independent binary}
\begin{lsuc}{BSD-3-Clause-C7}
  \linkosuc{08B}

  \lsucmeans{\useCaseSeven}
  \lsuccovers{\coversSeven}

  \begin{lsucrequires}
    \lsucmandatory{\insertLicenseIntoBinary}\passingFilesCorrectly
    \lsucmandatory{\includeLicenseInDocumentation}
  \end{lsucrequires}

  \begin{lsucprohibits}
    \lsucitem{\dontUseAuthorNames}%
  \end{lsucprohibits}
\end{lsuc}

% ------------------------------------------------------------------------------
\subsection{BSD-3-Clause-C8: Passing a modified library as embedded source code}
\begin{lsuc}{BSD-3-Clause-C8}
  \linkosuc{10S}

  \lsucmeans{\useCaseEight}
  \lsuccovers{\coversEight}

  \begin{lsucrequires}
    \lsucmandatory{\keepLicenseElements}
    \lsucoptional{\includeLicenseInDocumentation}
    \lsucoptional{\addLibraryLicenseToCopyrightMessage}
    \lsucoptional{\keepSourcesSeparate}
  \end{lsucrequires}

  \begin{lsucprohibits}
    \lsucitem{\dontUseAuthorNames}%
  \end{lsucprohibits}
\end{lsuc}

% ------------------------------------------------------------------------------
\subsection{BSD-3-Clause-C9: Passing a modified library as embedded binary}
\begin{lsuc}{BSD-3-Clause-C9}
  \linkosuc{10B}

  \lsucmeans{\useCaseNine}
  \lsuccovers{\coversNine}

  \begin{lsucrequires}
    \lsucmandatory{\insertLicenseIntoBinary}\passingFilesCorrectly
    \lsucmandatory{\includeLicenseInDocumentation}
    \lsucoptional{\addLibraryLicenseToCopyrightMessage}
    \lsucoptional{\keepBinariesSeparate}
  \end{lsucrequires}

  \begin{lsucprohibits}
    \lsucitem{\dontUseAuthorNames}%
  \end{lsucprohibits}
\end{lsuc}

\end{license}

%% =============================================================================
%% BSD-2-Clause
%% =============================================================================

\begin{license}{BSD2}
\licensename{BSD-2-Clause}
\licensespec{Simplified BSD (2 Clauses)}
\licenseabbrev{BSD-2-Clause}

%% =============================================================================
%% Use Cases

\subsection{BSD-2-Clause-C1: Using the software only for yourself}
\begin{lsuc}{BSD-2-Clause-C1}
  \linkosuc{01}
  \linkosuc{03L} 
  \linkosuc{03N} 
  \linkosuc{06L}
  \linkosuc{06N}
  \linkosuc{09L}
  \linkosuc{09N}
  
  \lsucmeans{\useCaseOne} 
  \lsuccovers{\coversOne}

  \begin{lsucrequiresnothing}
    \lsucitem{You are allowed to use any kind of BSD software in any sense and
      in any context without any obligations as long as you do not give the
      software to 3rd parties.}
  \end{lsucrequiresnothing}

  \lsucprohibitsnothing
\end{lsuc}

% ------------------------------------------------------------------------------
\subsection{BSD-2-Clause-C2: Passing the unmodified software as source code}
\begin{lsuc}{BSD-2-Clause-C2}
  \linkosuc{02S} 
  \linkosuc{05S} 
  \linkosuc{07S} 

  \lsucmeans{\useCaseTwo}
  \lsuccovers{\coversTwo}

  \begin{lsucrequires}
    \lsucmandatory{\keepLicenseElements}
    \lsucoptional{\includeLicenseInDocumentation}
  \end{lsucrequires}

  \lsucprohibitsnothing
\end{lsuc}

% ------------------------------------------------------------------------------
\subsection{BSD-2-Clause-C3: Passing the unmodified software as binary}
\begin{lsuc}{BSD-2-Clause-C3}
  \linkosuc{02B} 
  \linkosuc{05B} 
  \linkosuc{07B} 

  \lsucmeans{\useCaseThree}
  \lsuccovers{\coversThree}

  \begin{lsucrequires}  
    \lsucmandatory{\insertLicenseIntoBinary}\passingFilesCorrectly
    \lsucmandatory{\includeLicenseInDocumentation}
  \end{lsucrequires}

  \lsucprohibitsnothing
\end{lsuc}

% ------------------------------------------------------------------------------
\subsection{BSD-2-Clause-C4: Passing a modified program as source code}
\begin{lsuc}{BSD-2-Clause-C4}
  \linkosuc{04S}

  \lsucmeans{\useCaseFour}
  \lsuccovers{\coversFour}

  \begin{lsucrequires}
    \lsucmandatory{\keepLicenseElements}
    \lsucoptional{\includeLicenseInDocumentation}
    \lsucoptional{\addLicenseToCopyrightMessage}
  \end{lsucrequires}

  \lsucprohibitsnothing
\end{lsuc}

% ------------------------------------------------------------------------------
\subsection{BSD-2-Clause-C5: Passing a modified program as binary}
\begin{lsuc}{BSD-2-Clause-C5}
  \linkosuc{04B}

  \lsucmeans{\useCaseFive}
  \lsuccovers{\coversFive}

  \begin{lsucrequires}
    \lsucmandatory{\insertLicenseIntoBinary}\passingFilesCorrectly
    \lsucmandatory{\includeLicenseInDocumentation}
    \lsucoptional{\addLicenseToCopyrightMessage}
  \end{lsucrequires}

  \lsucprohibitsnothing
\end{lsuc}

% ------------------------------------------------------------------------------
\subsection{BSD-2-Clause-C6: Passing a modified library as independent source code}
\begin{lsuc}{BSD-2-Clause-C6}
  \linkosuc{08S}

  \lsucmeans{\useCaseSix}
  \lsuccovers{\coversSix}

  \begin{lsucrequires}
    \lsucmandatory{\keepLicenseElements}
    \lsucoptional{\includeLicenseInDocumentation}
  \end{lsucrequires}

  \lsucprohibitsnothing
\end{lsuc}

% ------------------------------------------------------------------------------
\subsection{BSD-2-Clause-C7: Passing a modified library as independent binary}
\begin{lsuc}{BSD-2-Clause-C7}
  \linkosuc{08B}

  \lsucmeans{\useCaseSeven}
  \lsuccovers{\coversSeven}

  \begin{lsucrequires}
    \lsucmandatory{\insertLicenseIntoBinary}\passingFilesCorrectly
    \lsucmandatory{\includeLicenseInDocumentation}
  \end{lsucrequires}

  \lsucprohibitsnothing
\end{lsuc}

% ------------------------------------------------------------------------------
\subsection{BSD-2-Clause-C8: Passing a modified library as embedded source code}
\begin{lsuc}{BSD-2-Clause-C8}
  \linkosuc{10S}

  \lsucmeans{\useCaseEight}
  \lsuccovers{\coversEight}

  \begin{lsucrequires}
    \lsucmandatory{\keepLicenseElements}
    \lsucoptional{\includeLicenseInDocumentation}
    \lsucoptional{\addLibraryLicenseToCopyrightMessage}
    \lsucoptional{\keepSourcesSeparate}
  \end{lsucrequires}

  \lsucprohibitsnothing
\end{lsuc}

% ------------------------------------------------------------------------------
\subsection{BSD-2-Clause-C9: Passing a modified library as embedded binary}
\begin{lsuc}{BSD-2-Clause-C9}
  \linkosuc{10B}

  \lsucmeans{\useCaseNine}
  \lsuccovers{\coversNine}

  \begin{lsucrequires}
    \lsucmandatory{\insertLicenseIntoBinary}\passingFilesCorrectly
    \lsucmandatory{\includeLicenseInDocumentation}
    \lsucoptional{\addLibraryLicenseToCopyrightMessage}
    \lsucoptional{\keepBinariesSeparate}
  \end{lsucrequires}

  \lsucprohibitsnothing
\end{lsuc}

\end{license}

%% =============================================================================
%% Discussion
%% =============================================================================

\subsection{Discussions and Explanations}
\label{BSD2Discussion}%
\label{BSD3Discussion}

The \textit{BSD 2-Clause license} has a simple structure: In the beginning, it
generally \enquote{(permits) redistribution and use in source and binary forms,
with or without modification, [\ldots]}, if one fulfills the two rules of the
license.\citeBSDsimple{} The first rule concerns the (re)distribution in the
form of source code, the second the (re)distribution of binary packages. Here are
some explanations why we translated the rules into different sets of executable
tasks:

\begin{itemize}
\item For the \enquote{redistribution of source code}, the license requires
  that the package must \enquote{ [\ldots] retain the above copyright notice,
  this list of conditions and the following disclaimer.}\citeBSDsimple{}
  Hence, you are not allowed to modify any of the copyright notes which are
  already embedded in the (source) files. And from a logical point of view,
  there must exist an explicit or implicit assertion that the software is
  licensed under the \textit{BSD 2-Clause license}%
  \footnote{The BSD license requires that a re-distributed software package must 
    contain the (package specific) copyright notice, the (license specific)
    conditions and the BSD disclaimer.\cite[cf.][\nopage wp]{BsdLicense2Clause} 
    You might ask, what you should do, if these elements are missing in the
    package you received. If so, the package you received had not been licensed
    adequately. Hence, you do not know reliably whether you have received it
    under a BSD license. In other words: If you have received a BSD licensed
    software package, it must contain sufficient license fulfilling elements, or
    it is not BSD licensed software.}. 
  This is often implemented by simply adding a copy of the license into the
  package. Hence, you are furthermore not allowed to modify these files or
  corresponding text snippets. For our purposes, we translated the bans into the
  following executable task: 

\begin{quote}\textit{\keepLicenseElements}\end{quote}

\item For the redistribution in the form of binary files, the license requires,
  that the licensing elements must be \enquote{[\ldots] (reproduced) in the
  documentation and/or other materials provided with the distribution.}%
  \citeBSDsimple{}
  Hence, this is not required as a necessary condition for the (re)distribution
  as source code package. But nevertheless, even for a distribution in the form
  of source code, it is often possible to fulfill this rule, too---e.g., if you
  offer your own download site for source code packages.  In such cases, it is a
  sign of respect to mention the licensing not only inside the packages, but
  also in the text of your site. Because of that, we added the following
  voluntary task for all BSD open source use cases which deal with the
  redistribution in the form of source code: 

\begin{quote}\textit{\includeLicenseInDocumentation}\end{quote}

\item Naturally, because the reproduction of the licensing elements \enquote{in
  the documentation and/or other materials provided with the distribution} is
  explicitly required for the \enquote{redistribution in binary form},%
  \citeBSDsimple{} 
  we had to rewrite the facultative task for a distribution in the form of
  source code as a mandatory task for all BSD open source use cases which deals
  with the redistribution in binary form.

\item In case of (re)distributing the program in the form of binary files, it is
  sometimes not enough, to pass the licensing elements as one has received them.
  If you compile the binary package from the source code, it is not necessarily
  true, that the licensing elements are also automatically generated and
  embedded into the `binary package.' But nevertheless, you have to add the
  copyright notice, the conditions and the disclaimer to this package for acting
  according to the BSD license. Therefore we chose the following form of an
  executable, license fulfilling task for all binary distributions:

\begin{quote}\textit{\insertLicenseIntoBinary}\end{quote}

\item Finally, we wished to insert a hint to the general (open source) tradition
  to mention the open source software used and their licenses as part of the
  `copyright widget' of an application. This is not required by the BSD
  license. But it is a general, good tradition. Naturally, because of the
  freedom to use and modify open source software and to redistribute a modified
  version of it, you are also allowed to insert such references, even if they
  are missing. Therefore we added a third voluntary task to honor this tradition
  for all relevant open source use cases.

\end{itemize}


%\bibliography{../../../bibfiles/oscResourcesEn}

% Local Variables:
% mode: latex
% fill-column: 80
% End:
