% Telekom osCompendium 'for being included' snippet template
%
% (c) Karsten Reincke, Deutsche Telekom AG, Darmstadt 2011
%
% This LaTeX-File is licensed under the Creative Commons Attribution-ShareAlike
% 3.0 Germany License (http://creativecommons.org/licenses/by-sa/3.0/de/): Feel
% free 'to share (to copy, distribute and transmit)' or 'to remix (to adapt)'
% it, if you '... distribute the resulting work under the same or similar
% license to this one' and if you respect how 'you must attribute the work in
% the manner specified by the author ...':
%
% In an internet based reuse please link the reused parts to www.telekom.com and
% mention the original authors and Deutsche Telekom AG in a suitable manner. In
% a paper-like reuse please insert a short hint to www.telekom.com and to the
% original authors and Deutsche Telekom AG into your preface. For normal
% quotations please use the scientific standard to cite.
%
% [ File structure derived from 'mind your Scholar Research Framework' 
%   mycsrf (c) K. Reincke CC BY 3.0  http://mycsrf.fodina.de/ ]
%

% Chapter Abstract
% ----------------
\chapter{Conclusion}

During the last 4 years, we have developed this \textbf{O}pen \textbf{S}ource
\textbf{Li}cense \textbf{C}ompendium. We had the honor and the pleasure to
discuss our ideas with many open source experts, for example with those, who
visit the European Legal and Licensing Workshop, organized by the FSFE. We were
invited to present our work on different conferences, in Germany, in Europe, and
even in Asia. We got a very encouring feedback. Today we know what we only
supposed when we started: We could indeed close an important gap by offering a
simple and reliable way to ascertain what one has to do for using open source
software compliantly. We are proud of having gone this long way. And we pride
ourselves on the fact that -- today -- the OSLiC is officially listed by the OSI
as one of those tools by which one can manage the open source
compliance\footnote{$\rightarrow$
http://osi.xwiki.com/bin/Projects/Process+and+Compliance+Resources}.

But, we also got adjusting feedback: Namely our initial premise was justifiably
not really accepted by the community. We were told that the software developers
themselves would never use our OSLiC. They would never read a book of more than
300 pages full of lists and tables -- as long as this book was not a
specification of a computer language. The OSLiC would be too large and too
complex for simplifying the daily life of the open source users. It would be an
excellent foundation for becoming an open source license expert -- but not a
tool for the desk. And indeed, it was simply silly to assume that software
developers, project managers, or IT managers can directly understand and use the
OSLiC: reading the OSLiC the first time has a discouraging shock effect. Today,
also we know this.

Nevertheless, it was very important for us to fall for the charme of this
illusion. Without this error, we never would have started the development of the
OSLiC. And thus, we never would have find the idea to organize the issue in form
of finders and a 5 question form. Without this error, we today would never have
a work which justifies and proves each single assertion by quoting the licenses
and the experts. And without this frightening feedback we received, we never
would have got one of our best and encouraging experiences: 

When we had accepted the feedback, we directly decided to develop an online
version of the OSLiC, the Open Source Compliance Advisor, also know as
OSCAd\footnote{$\rightarrow$ http://opensource.telekom.net/oscad/}.
We distributed it under the terms of the AGPL. Then, the company Amadeus decided
to take over the development of this online tool. We, on our side, inserted an
export interface into the OSLiC. They, on their side, rewrote the OSCAd and
integrated an import interface. So -- finally -- we both were able to focus on
only one specific aspect:  they took the responsibility for computing and
maintaining the online tool\footnote{$\rightarrow$
https://github.com/AmadeusITGroup/oscad}, we took the responsibility maintaining
for the fundamental analysis of the open source licenses\footnote{$\rightarrow$
https://github.com/dtag-dbu/oslic/}.

Thus, we concretely experienced the advantages of sharing ideas and sources,
which were so often emphazied: Playing the open source game actively means
giving a bit and getting back a lot. Playing the open source game actively means
saving the own resources.
 
Therefore, you may also take the fact that we finally could indeed publish the
version 1.0 of the OSLiC as a thankful profound curtsey to the open source
community!
