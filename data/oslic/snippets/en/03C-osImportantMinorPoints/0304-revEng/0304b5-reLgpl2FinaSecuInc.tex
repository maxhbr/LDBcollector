% Telekom osCompendium 'for being included' snippet template
%
% (c) Karsten Reincke, Deutsche Telekom AG, Darmstadt 2011
%
% This LaTeX-File is licensed under the Creative Commons Attribution-ShareAlike
% 3.0 Germany License (http://creativecommons.org/licenses/by-sa/3.0/de/): Feel
% free 'to share (to copy, distribute and transmit)' or 'to remix (to adapt)'
% it, if you '... distribute the resulting work under the same or similar
% license to this one' and if you respect how 'you must attribute the work in
% the manner specified by the author ...':
%
% In an internet based reuse please link the reused parts to www.telekom.com and
% mention the original authors and Deutsche Telekom AG in a suitable manner. In
% a paper-like reuse please insert a short hint to www.telekom.com and to the
% original authors and Deutsche Telekom AG into your preface. For normal
% quotations please use the scientific standard to cite.
%
% [ Framework derived from 'mind your Scholar Research Framework' 
%   mycsrf (c) K. Reincke 2012 CC BY 3.0  http://mycsrf.fodina.de/ ]
%

So far, we have done a lot of work: At first, we unfolded and dissolved some
stylisch condensed formulations of the original LGPL2-RevEng-Sentence by their
linguistically explicit version. At second, we explicated the logical structure
of the sentence. At third, we empirically carved out the real meaning of the
sentence. And finally we mapped the triggering part of that rule to some
verifiable facts. Indeed, a lot of work for understanding only one sentence
correctly\footnote{Here, some readers might ask why the original authors have
encapsulated their clear ideas in such a sophisticate sentence. Here are two
answers: First, this question is practically irrelevant: The authors of the
LGPL-v2 did, what they have done. And many developers have already licensed
their works under the terms of the LGPL-v2. Thus, we simply have to live with
the results -- just until the last software being published under the terms of
the LGPL-v2 is relicensed by a better version. Probably this won't happen during
our life time. Secondly, we appreciate the foresight of the LGPL-v2 authors.
They wrote a license which have successfully worked for more than twenty years.
They chosed a formulation which had also to cover 'uninvented' techniques. So,
it is not so surprizing, that we -- today -- have to do a lot of work to
understand all details the original authors want to be understood.}. So, it is a
good securing to verify that the derived result fits the spirit and the goals of
the LGPL-v2 perfectly. 

For that purpose, let us fist discuss a little (semi-) official article --
written by David Turner and published by the FSF\footcite[cf.][\nopage
wp.]{Turner2004a} -- which deals with the compliant use of LGPL licensed Java
libraries. Turner refers to the \enquote{FSF's position} which - as he says -
\enquote{[\ldots] has remained constant throughout}:

\begin{quote}\noindent\emph{\enquote{[\ldots] the LGPL works as intended with
all known programming languages, including Java. Applications which link to LGPL
libraries need not be released under the LGPL. Applications need only follow the
requirements in section 6 of the LGPL: allow new versions of the library to be
linked with the application; and allow reverse engineering to debug
this.}\footcite[cf.][\nopage wp]{Turner2004a}}\end{quote}

Then he describes, that Java libraries are \enquote{[\ldots] distributed as a
separate JAR (Java Archive) file} and that applications \enquote{[\ldots] use
Java's \enquote{import} functionality to access classes from these libraries}.
Moreover, he also explains, that the process of compilation \enquote{creates}
and integrates \enquote{links} into the compiled application which let become
the application a \enquote{derivative work} of the library.
Finally he states, that not only the LGPL permits to distribute such derivative
works, but that \enquote{[\ldots] it is easy to comply with the LGPL} if one
indeed wants to \enquote{[\ldots] distribute a Java application that imports
LGPL libraries}: \enquote{Your application's license needs to allow users to
modify the library, and reverse engineer your code to debug these
modifications.}\footcite[cf.][\nopage wp.]{Turner2004a}

So, we might state, that even this semi-official article argues very similarly
to us. There is only one little phrase in this text which differs a little:
Summarizing the \enquote{section 6 of the LGPL} by the statement
\enquote{\emph{[\ldots] allow new versions of the library to be linked with the
application; and allow reverse engineering to debug this}} does not consider
that the first sentence of the section 6 of the LGPL contains a complex
condition. The \emph{LGPL2-RefEng-Sentence} means -- as we could prove -- that
\emph{one may distribute (a) \textbf{work containing portions of the Library}
only if one's license permit reverse engineering for debugging
modifications}\footnote{$\rightarrow$ p.
\pageref{RevEngEssentialLgplSection6Meaning}}. But -- as we could also show --
for determining wether an application really contains portions of the Library,
one has additionally to consider the limits defined by section 5 of the
LGPL\footnote{$\rightarrow$ p. \pageref{RevEngLgplSection5Derivation}}: the 
application's license needs to allow to reverse engineer the application only if
it contains more elements of the Library than §5 of the LGPL-v2 has specified as
limit.

That our analysis fits the spirit of the LGPL, can also be shown by considering
the LGPL directly:

The LGPL-v2 clearly describes its goals. It wants to enable the community to let
an LGPL Library \enquote{[\ldots] become a de-facto standard}. And the LGPL
knows, that \enquote{to achieve this [goal], non-free programs must be allowed
to use the library}, because the \enquote{[\ldots] permission to use a
particular library in non-free programs enables a greater number of people to
use a large body of free software}. But the LGPL also asserts in this context,
that \enquote{although the Lesser General Public License is Less protective of
the users' freedom, it does ensure that the user of a program that is linked
with the Library has the freedom and the wherewithal to run that program using a
modified version of the Library}\footcite[cf.][\nopage wp., preamble, emphasis
KR]{Lgpl21OsiLicense1999a}.

So -- as a last check of our derivation -- we can analyze, whether our derived
result violates this goal. If it does, then we probably made a tremendous fault;
if not, then we are allowed to trust in the consistence our analysis:

If you receive a work using the Library in form of a discrete (set of)
dynamically linkable or combinable file(s) and if -- hence -- your provider
assumed that the files he delivers will be linked on your target machine which
-- therefore -- has to provide a linker and the the necessary dynamically
linkable Libraries, than you systematically have the freedom to replace the
dynamically linked Libraries by their updated versions\footnote{In GNU/Linux --
for example -- you must (only) copy the dynamically linkable new version of the
Library into the lib/-directory and replace the existing link by a version
pointing to the newer version. Sometimes you should additionally verify the
ld.so.conf files and call ldconfig tool.}. And as long as the newer versions of
the Libraries preserve the defined and declared interfaces, you can do that
successfully. That's, what the LGPL-v2 wants to ensure.

In all other cases, you must have the permission of reverse engineering or you
have a direct access to the source code. So, you can use the corresponding tools
and techniques to replace the embedded version of the Library by a newer
version; especially if you have received a statically linked package. Hence,
also the second part of our interpretation respects the spirit of the LGPL-v2.

So, finally we can say, everything is fine: The LGPL2-RevEng-Rule -- together
with the meaning of being a portion of a Library -- does not only verifiably
exeplicate the meaning of the LGPL2-RevEng-Sentence, but also fits the spirit
and the purpose of the LGPL-v2 as it has been announced by its preamble.


%% use all entries of the bibliography
%\nocite{*}

