% Telekom osCompendium 'for being included' snippet template
%
% (c) Karsten Reincke, Deutsche Telekom AG, Darmstadt 2011
%
% This LaTeX-File is licensed under the Creative Commons Attribution-ShareAlike
% 3.0 Germany License (http://creativecommons.org/licenses/by-sa/3.0/de/): Feel
% free 'to share (to copy, distribute and transmit)' or 'to remix (to adapt)'
% it, if you '... distribute the resulting work under the same or similar
% license to this one' and if you respect how 'you must attribute the work in
% the manner specified by the author ...':
%
% In an internet based reuse please link the reused parts to www.telekom.com and
% mention the original authors and Deutsche Telekom AG in a suitable manner. In
% a paper-like reuse please insert a short hint to www.telekom.com and to the
% original authors and Deutsche Telekom AG into your preface. For normal
% quotations please use the scientific standard to cite.
%
% [ Framework derived from 'mind your Scholar Research Framework' 
%   mycsrf (c) K. Reincke 2012 CC BY 3.0  http://mycsrf.fodina.de/ ]
%


%% use all entries of the bibliography
%\nocite{*}

Beyond any doubt, the LGPL mentions \enquote{reverse engineering}
literally\footnote{For the LGPL-v2 \cite[cf.][\nopage wp., 
§6]{Lgpl21OsiLicense1999a}; for the LGPL-v3 \cite[cf.][\nopage wp., 
§4]{Lgpl30OsiLicense2007a} } for indicating that reverse engineering in any
respect must be allowed to use and distribute LGPL software compliantly:

\begin{quote}\noindent\emph{\enquote{[\ldots] you may [\ldots] distribute a work
(containing portions of the Library) under terms of your choice, provided that
the terms permit [\ldots] \emph{reverse engineering} [\ldots]}
\footnote{\cite[cf.][\nopage wp, §6]{Lgpl21OsiLicense1999a}. The LGPL-v2 uses
the capitalized word \enquote{Library} for denoting a library which
\enquote{[\ldots] has been distributed under (the) terms} of the LGPL-v2.
(\cite[cf.][\nopage wp, §0]{Lgpl21OsiLicense1999a}). Inside of our LGPL
chapter(s) we follow this interpretation. } }
\end{quote}

There are three strategies for dealing with such provisions: one can try to
fully honor its meaning, one can mitigate its meaning, or one can avoid to
discuss this requirement altogether:

A first group of well known open source experts take the sentence of the LGPL-v2
as a strict rule which requires that one has to allow reverse engineering of the
whole software product if one embeds any LGPL licensed component into that
product\footnote{For example, a very trustworthy German expert states that the
LGPL-2.1 generally requires that a distributor of a program which accesses a
LGPL-2.1 licensed library, must grant his customer also the right to modify the
accessing program and hence also the right to execute reverse engineering.
Literally the German text says:
\begin{quote}\enquote{Ziffer 6 LGPLv2.1 knüpft die Erlaubnis, das zugreifende
Programm unter beliebigen Lizenzbestimmungen verbreiten zu drüfen, an eine Reihe
von Verpflichtungen, die in der Praxis oft übersehen werden: Zunächst muss dem
Kunden, dem die Software geliefert wird, die Veränderung des zugreifenden
Programms gestattet werden und zu diesem Zweck auch ein Reverse Engineering zur
Fehlerbehebung. \emph{Dies dürfte alle Formen des Debugging und das
Dekompilieren des zugreifenden Programms umfassen}.} (\cite[cf.][81; emphasis
KR]{JaeMet2011a}).\end{quote} At first glance, also \enquote{copyleft.org} --
the \enquote{[...] collaborative project to create and disseminate useful
information, tutorial material, and new policy ideas regarding all forms of
copyleft licensing} (\cite[cf.][\nopage wp.]{CopyLeftOrg2014a}) -- could be
taken as another witness for such an attitude of strict reading: Some of its
contributors elucidate in a chapter dealing with \enquote{special topics in
compliance} that \enquote{the license of the whole work must [sic!] permit
\enquote{reverse engineering for debugging such modifications} to the library}
and that one therefore \enquote{ should take care that the EULA used for the
Application does not contradict this
permission}(\cite[cf.][86]{KuhSebGin2014a}}.

A second group of well known and knowledgeable open source experts signify that
the LGPL-v2 indeed literally contains a strict rule, but that this rule actually
is not meant as it sounds: For example, two of these experts explain that
\enquote{these requirements on the licensed combination require that the license
chosen not prohibit the customer’s modification and reverse engineering for
debugging these modifications in the work as a whole}. But then they directly
add the limitation, that \enquote{in practice, enforcement history suggests, it
means that the license terms chosen may not prohibit modification and reverse
engineering for debugging of modification in the LGPL’d code included in the
combination}\footnote{\cite[cf.][\nopage wp., chapter LGPLv2.1, section
6]{MogCho2014a}. Such a mitigation can also be found in the tutorial of
copyleft.org: After they have summarized the LGPL-v2 sentence as a strict rule,
they directly continue, that one \enquote{[\ldots] must refrain from license
terms on works based on the licensed work that prohibit replacement of the
licensed components of the larger non-LGPL'd work, or prohibit decompilation or
reverse engineering in order to enhance or fix bugs in the LGPL'd components}
(\cite[cf.][86]{KuhSebGin2014a}). This added specification indicates, that one
only has to facilitate the modification of the library and that reverse
engineering can be ignored as long as there are other ways to improve the
embedded library.}.

Finally, a third group of experts prefers not to discuss the problem of reverse
engineering, although this technique is literally mentioned in the license and
although they want explain how to use GPL/LGPL licensed software
compliantly\footnote{An article of Terry J. Ilardi might be taken as a first
witness of this third strategy: he profoundly explains the essence of the LGPL,
he especially discusses §6, and he delivers applicable rules like \enquote{DO
NOT statically link to LGPL [\ldots] code if you wish to keep your program
proprietary}. But he does not discuss \emph{reverse engineering}
(\cite[cf.][5f]{Ilardi2010a}). Similarily argues Rosen
(\cite[cf.][121ff]{Rosen2005a}). And -- despite their comments on reverse
engineering in the specific chapter \emph{special topics in compliance} -- the
copyleft.org document can also be taken as an instance of this attitude:
Although its authors recommend to \enquote{study chapter 10 carefully} for
establishing an adequate \enquote{compliance with LGPLv2.1}
(\cite[cf.][86]{KuhSebGin2014a}), this chapter 10 -- dedicated to the meaning of
the \enquote{Lesser GPL} -- does not deal with reverse engineering, although it
discusses the §6 of the LGPLv2.1 in depth (\cite[cf.][56ff, esp.
60f]{KuhSebGin2014a}).}.

This situation must bother companies and people who want to use open source
software compliantly and who therefore are looking for guidance. Particularly it
disturbs those who want to protect their business relevant software. At the end,
they might consider that this sentence is not consistently understood by the
open source community itself. And -- as far as we know -- at least some of these
companies preventively prohibit their developers to embed LGPL licensed
components into programs which contain business relevant techniques.
Unfortunately, this consequence does not only obstruct the access to a large set
of well written free software, but it is scarcely possible to obey such an
interdiction consequently: The glibc, which enables the programs to talk with
the kernel of the GNU/Linux system\footnote{cf.
http://www.gnu.org/software/libc/}, is licensed under the LGPL\footnote{cf.
http://en.wikipedia.org/wiki/GNU\_C\_Library}. And hence, this library is
indirectly linked to or combined with any program running on the GNU/Linux
system. So, if the LGPL-v2 indeed required, that reverse engineering of every
program must be allowed, which contains portions of any LGPL Library, then every
GNU/Linux user would be allowed to examine every program running on GNU/Linux by
\emph{reverse engineering}, simply, because finally every 'GNU/Linux program' is
linked to or combined with the glibc\footnote{This conclusion might surprise.
But it is inferred with exactly the same arguments as the conclusion, that
without a licence offering a weaker copyleft every program would have been
licensed under the GPL. The copyleft.org document explains this argumentation in
great detail (\cite[cf.][56f]{KuhSebGin2014a}).}. In other words: if the LGPL
indeed required the permission of reverse engineering, then
every program executed on GNU/Linux may be reverse engineered.

But an exhaustive reading of the LGPL-v2 strongly indicates, that there must be
another valid, more 'liquid' understanding of the LGPL: The preamble explains
the reason for offering another weaker license beside the GPL. It says that
\enquote{[\ldots] on rare occasions, there may be a special need to encourage
the widest possible use of a certain library, so that it becomes a de-facto
standard} and that therefore it could be strategicly necessary to \enquote{allow
[\ldots] non-free programs [\ldots] to use the library} without enforcing that
these programs become free software too\footcite[cf.][\nopage wp,
§preamble]{Lgpl21OsiLicense1999a}.

So, if the LGPL had indeed determined that every program linked to or combined
with any LGPL library may be reverse engineered, then the LGPL would have an
effect contrary to its own intention. It would have introduced something like
\emph{'security by obscurity'}: First, the LGPL would allow to protect the
internals of your own work against investigation by allowing to keep the code of
the non-free program using the library a scecret\footnote{The weak copyleft has
been introduced for encouring the widest possible use of the library.}. But then
-- in the end -- the LGPL would also allow the user to reverse engineer the
received binary and hence would enable him to nevertheless discover all
internals\footnote{It would only cost a little more effort - as security by
obscurity indicates.}. Hence, finally the LGPL-v2 would undermine its own
raison d'$\grave{e}$tre introduced by its inventors: under such circumstances
there probably would have been less hope that any LGPL library could have become
a defacto standard.

We know that the inventors of the GNU licenses and GNU software are very
sophisticated experts. They never would have published such an inconsistent
document. So, this dissent read in(to) the document is a strong indicator for
the fact, that there must be a better way to understand the license. And thus,
it is up to us, the followers, to explicate a more adequate interpretation. Of
course, such an interpretation must be grounded on the written text. No doubt:
we, the scholars, are not allowed to add our own wishes. We must read the
license very strictly. We have to deduce 'understandings' only by matching the
interpretations explicitly and reasonably back to the license text itself.

\label{RevEngOslicOsLisences}
Encouraged by the indication that a better understanding of the LGPL may exist
and contrary to the other strategies, we are going to prove that there is a
valid way to compliantly distribute any open source based software without
permitting reverse engineering: We want to show that none of the main open
source licenses\footnote{Just as the OSLiC, also this part focuses only on the
most important open source licenses (cf.
https://www.blackducksoftware.com/resources/data/top-20-open-source-licenses
wp.): the Apache license (\cite[cf.][\nopage wp.]{Apl20OsiLicense2004a}), the
BSD licenses (\cite[cf.][\nopage wp.]{BsdLicense3Clause} and \cite[cf.][\nopage
wp.]{BsdLicense2Clause})), the MIT license (\cite[cf.][\nopage
wp.]{MitLicense2012a}), the MS-PL (\cite[cf.][\nopage
wp.]{MsplOsiLicense2013a}), the PostgreSXQL (\cite[cf.][\nopage
wp.]{PglOsiLicense2013a}), the PHP license (\cite[cf.][\nopage
wp.]{Php30OsiLicense2013a}), the EPL (\cite[cf.][\nopage
wp.]{Epl10OsiLicense2005a}), the EUPL (\cite[cf.][\nopage
wp.]{Eupl11OsiLicense2007a}), the MPL (\cite[cf.][\nopage
wp.]{Mpl20OsiLicense2013a}), the LGPLs (\cite[cf.][\nopage
wp.]{Lgpl21OsiLicense1999a} and \cite[cf.][\nopage wp.]{Lgpl30OsiLicense2007a}),
the GPLs (\cite[cf.][\nopage wp.]{Gpl20OsiLicense1991a} and \cite[cf.][\nopage
wp.]{Gpl30OsiLicense2007a}) and the AGPL (\cite[cf.][\nopage
wp.]{Agpl30OsiLicense2007a})} requires reverse engineering if the work using
the open source Library is distributed in form of dynamically linkable files. In
particular, we are going to prove that one even has not to allow reverse
engineering of the work using an LGPL Library, if one distributes that work as
dynamically linkable files. And we want to show that in all other cases one has
at least to fear that one has implicitly allowed the reverse engineering of the
work using the open source Library if one distribute that work. In particular,
we want to prove that one has to fear this implicitly given permission even if
one distributes a work using a library licensed under any permissive
license\footnote{By the way, our analysis should also provide proof that the
LGPL is not something like a 'poisoned' license containing \enquote{an
imprenetrable maze of technology babble} which \enquote{[\ldots] should not be
in a general-purpose software license} (\cite[cf.][124]{Rosen2005a}). The
challenge of the today's descendants is to understand the former inventors of
the GNU licenses and their way to think about computing - including all the
hassle the computing language C might provoke.}.

In general, we hope that our analysis, grounded on the license text itself, will
support companies and people to compliantly use open source software more often
and with less hesitation due to the fear that they have to deliver themselves
unclear textual aspects.

But -- with respect to the discussion about this text in the OSI and Free
Sofwtare Mailing lists -- we have to add a disclaimer here: The license text
alone is not all. In the concrete situation of using free and open source
software, it is the intention of the licensor which has to be respected. Or in
the words of Eben Moglen:

\begin{quote}\noindent\emph{A license is, by definition, a unilateral permission
to make use of the property or intangible rights of another. The measure of the
permission is the intention of the party giving it.}\footnote{Eben
Moglen, eMail to ftf-legal-bounces@fsfeurope.org, 2015-03-06}\end{quote}

Nevertheless, we believe that each text firstly has its own inherent independent
meaning and message. But -- of course, in the specific situation of legally
contending about the practical consequences of a license, one has indeed to
consider what the specfic licensor really had had in his mind, when he released
his work. One has to consider his intention.

So, the pure textual meaning of the license might be overloaded or overwritten
by some external facts, traditions or understandings, not founded on the license
text itself. The problem with this legal fact is, that in a concrete legal case,
one has to prove what the licensor really had in his mind. As long as we do not
have direct insights into the brain of our fellow human beings, this can again
only be done by referring to other orally uttered or written words and texts.
Therefore, we indeed believe, that it is firstly important to know what the text
itself says and means.

Hence, let us prove our position 'bottom up'. Let us firstly show that it is
true for the LGPL-v2 -- by explicating the license text lingually, then
logically, and finally empirically, before we infer the correct understanding.
Then let us show that it is also true for the LGPL-v3. And in the end let us
show that it is true for all other licenses\footnote{analysed by the OSLiC:
$\rightarrow$ p. \pageref{RevEngOslicOsLisences}.}.

