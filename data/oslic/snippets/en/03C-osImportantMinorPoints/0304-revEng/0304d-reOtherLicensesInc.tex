% Telekom osCompendium 'for being included' snippet template
%
% (c) Karsten Reincke, Deutsche Telekom AG, Darmstadt 2011
%
% This LaTeX-File is licensed under the Creative Commons Attribution-ShareAlike
% 3.0 Germany License (http://creativecommons.org/licenses/by-sa/3.0/de/): Feel
% free 'to share (to copy, distribute and transmit)' or 'to remix (to adapt)'
% it, if you '... distribute the resulting work under the same or similar
% license to this one' and if you respect how 'you must attribute the work in
% the manner specified by the author ...':
%
% In an internet based reuse please link the reused parts to www.telekom.com and
% mention the original authors and Deutsche Telekom AG in a suitable manner. In
% a paper-like reuse please insert a short hint to www.telekom.com and to the
% original authors and Deutsche Telekom AG into your preface. For normal
% quotations please use the scientific standard to cite.
%
% [ Framework derived from 'mind your Scholar Research Framework' 
%   mycsrf (c) K. Reincke 2012 CC BY 3.0  http://mycsrf.fodina.de/ ]
%

The rest of our way is simple: First, we can ascertain, that none of the other
open source licenses we consider\footnote{$\rightarrow$ p.
\pageref{RevEngOslicOsLisences} }, contain the phrase 'reverse engineering'.
Moreover, they even do not contain one of the single words\footnote{One can
verify this negative statement by (a) loading down all licenses from the OSI
homepage (http://opensource.org/licenses/alphabetical) and by (b) executing the
command \texttt{grep -i "engineering" *} respectively \texttt{grep -i "reverse"
*} in the directory into which the license files have been stored: grep will
find the words \emph{reverse} and \emph{engineering} only in the texts of the
LGPLs.}. So, we may infer, that these most important other open source licenses
could at most indirectly require the permission of reverse engineering. Second,
we know already that distributing script code let the allowance to reverse
engineer, become irrelevant: script code can directly be read and understood, if
one knows the script language\footnote{$\rightarrow$ p.
\pageref{RevEngDistributeScripts}}.
Third, from the definition of strong copleft we may derive, that distributing
software licensed under a strong copyleft license let the permission of reverse
engineering become unimportant, because the source code of the work using the
libraries licensed under a copleft license, must also be made
accessible\footcite[cf.][\nopage wp]{Stallman1996c}.

So -- overally -- we may conclude, that we have only to consider those cases,
where a piece of software is distributed in form of binaries or bytecode, which
uses libraries licensed under permissive open source licenses or under weak
copyleft licenses.

From the definition of being a permissive license or a weak copyleft license we
know already that the licenses of the open source components do not directly
influence the permission or interdiction to use the overarching work which uses
the open source software components\footcite[cf.][20ff.]{Reincke2015a}.

So, if we distribute such a work in form of dynamically linkable, but still not
linked binaries or bytecode files, then there is no way to reasonably derive
that the work using the components, may be reverse engineered: The permissive or
weak copyleft open source licenses mainly concern the open source components,
not the work using the components. On the one side, these licenses indeed
require that we add the license texts and the copyright lines of all the open
source components our work wants to use, to the distributed package containing
our work. And the lisenses prohibit to modify the licensing assertions being
integrated into the open source components our work wants to use\footnote{These
requirements are part of all the open source licenses we consider here. For
details \cite[cf.][chapter 6.]{Reincke2015a}}. But -- on the other side and in
accordance to the permissive or weak-copyleft licenses -- the freedom to use, to
study, to modify, or to distribute the software, which is established by these
open source licenses, concerns only the open source components themselves, not
the work using the open source components. So, as long as these components still
are not linked to or combined with the using work in accordance to the standard
compilation and computation methods, they can indeed be studied or modified
without the need to study or modify the work which uses these
components\footnote{The only way to infer that the licenses of the components
operates also on the using work, is to argue that the using work must at least
contain elements (identifiers etc.) of the interfaces declared (but not defined)
by the libraries and that therefore at least these elements may be investigated
or modified. This challenge is explicitly addressed by the
LGPL\footnote{$\rightarrow$ p.
\pageref{RevEngDistributeDynamicallyLinkedCode}}. Fortunately, it is a general
feature of software libraries that they must and shall be used in accordance to
the interfaces, the developers of the libraries have designed to make their
libraries practically usable. So, if the licenses -- in contrary to the LGPLs --
do not explicitly address the issue of implicitly included portions of the
library in case of unlinked binaries or bytecode files which have been compiled
in accordance to the standard methods and which therefore use open source
software by reffering to their standard interfaces, then one has to infer from
the nature of computation, that the developers have implictly allowed without
any requirements such an integration of declared, but not defined interface
elements, because they have designed the interface as they did and because they
have licensed their work as they did. If they had not wished to use these
elements without any requirements, hey had designed another interface. And if
they had wished to incorporate any copyleft effect or permission of reverse
engineering, then they would have selected another license. But again: this
conclusion holds only for the standard methods to use a software library.}.

On the other side, if we compliantly distribute the work using the components,
as a statically linked binary or bytecode file -- which therefore already
contains all the necessary components\footnote{instead of only the declared
interface elements!} and can directly be executed --, then we are also obliged
to add all the open source license texts and all the copyright lines to our
package, and we are not allowed to modify one of the licensing assertions
integrated into the original open source components\footcite[cf.][chapter
6.]{Reincke2015a}. Thus, one might conclude, that the freedom to use and to
modify the open source components themselves, survive if we distribute software
statically linked to or combined with the open source components. So, the
receiver of the statically linked work probably is allowed to modify the
embedded open source components - even if he had to edit the binary or bytecode
files. Methods to develop binary files reversely, are known as reverse
engineering. Hence, if we distribute a statically linked work using open source
licensed components, we have at least to fear that our receivers indirectly have
also got the permission to reverse engineer our complete product. And we have to
fear so even if the statically linked libraries are licensed under any
permissive or weak copleft license.

So, again, we can summarize the result in the following form:

\begin{itemize}
  \item \emph{With respect to a Library licensed under any permissive or weak
  copyleft license, you are not required to allow reverse engineering, if you
  [A] develop your work using the Library, on the base of a standard version of
  the Library containing the interfaces as the original developers have designed it,
  if you [B] compile your work using this Library, as a discrete (set of)
  dynamically linkable or combinable file(s), if you [C] use only the standard
  compilation methods which preserve the upstream approved interfaces, and if
  you [D] distribute the produced unlinked object code or bytecode files before
  they are linked as an executable.}
  \item \emph{In all other cases of distributing a work using such a Library,
  you have at least to fear that you are implictly allowing reverse engineering
  of the work using this Library -- especially, \ldots}
  \begin{itemize}
    \item \emph{if you distribute the work using the Library and the Library
    together as a statically linked program or as an integrated package
    containing both parts, the work using the library and the Library
    itself\footnote{This holds also if you distribute a script language based
    program or package, notwithstanding the fact, that one does not need the
    permission of reverse engineering to understand script language based
    applications}.}
    \item \emph{if you distribute a work containing manually copied portions of
    the Library.}
  \end{itemize}
\end{itemize}


%% use all entries of the bibliography
%\nocite{*}

