% Telekom osCompendium 'for being included' snippet template
%
% (c) Karsten Reincke, Deutsche Telekom AG, Darmstadt 2011
%
% This LaTeX-File is licensed under the Creative Commons Attribution-ShareAlike
% 3.0 Germany License (http://creativecommons.org/licenses/by-sa/3.0/de/): Feel
% free 'to share (to copy, distribute and transmit)' or 'to remix (to adapt)'
% it, if you '... distribute the resulting work under the same or similar
% license to this one' and if you respect how 'you must attribute the work in
% the manner specified by the author ...':
%
% In an internet based reuse please link the reused parts to www.telekom.com and
% mention the original authors and Deutsche Telekom AG in a suitable manner. In
% a paper-like reuse please insert a short hint to www.telekom.com and to the
% original authors and Deutsche Telekom AG into your preface. For normal
% quotations please use the scientific standard to cite.
%
% [ Framework derived from 'mind your Scholar Research Framework' 
%   mycsrf (c) K. Reincke 2012 CC BY 3.0  http://mycsrf.fodina.de/ ]
%

Unfortunately, there are more than one software developing scenarios, which must
be considered for answering this question in detail. We see three general types
of developing computer software:

\begin{enumerate}
  \item You can produce software by using script languages. Source files which
  contain script language commands are distributed and executed by an
  interpreter without priorly being transformed into another 'more' machine
  specific language.
  \item You can develop software by using languages which are designed for being
  compiled into a machine independent bytecode. Later on, this independent
  bytecode is executed by a machine specific virtual machine.
  \item You can write traditional software files. Sometimes, these files are
  remastered by a preprocessor before the real process starts. The traditional
  sources themselves or the output files of the preprocessor are then compiled
  and linked as machine specific binary file(s).
\end{enumerate}
  
You may take 'php' is an example for the first environment, 'Java' an example
for the second, and 'C/C++' an example for the third.

Fortunately, the nature of these environments simplifies the answer to the
question under which conditions the work using the Library contains portions of
the Library:

\paragraph{Distributing works with manually copied portions of the Library
evokes the copyleft effect:}
\label{RevEngCopyCodeManually}
Manually copying code from the sources of the Library into the overarching
work that uses the Library, is not the standard way of combining both
components, neither in the world of script programming, nor in the world of
bytecode programming, nor in the world of programming machine specific code:

Normally, the work which uses the Library is joined to the intended Library by
an include statement, an input statement, an import statement, a package
statement, or anything else. These *join-statements are inserted into the code
of the work. They denote the file(s) which deliver(s) the used functions,
methods, classes, or data. It is an integrated feature of the normal development
tools that inserting such *join-statements does not directly augment the work
using the library by some code of the Library: The development processes are
designed to offer an automatic augmention as part of the standard compilation
which is started after the actual development loop has been terminated.

Nevertheless, developers can circumvent these standard methods for using a
Library. Technically, they can directly copy code from the Library into their
own work. Consequently, these manually copied extensions of the code will be
compiled and/or executed together with the 'own' code of the work. Thus, it is
clear that in this case the work that uses the Library already contains
portions of the Library, particularly before the normal *join-processes of the
environment are executed.

Hence, if you are going to distribute works that contain literal copies of the
Library source code, then you have to allow reverse engineering, even if they
have already been compiled (but still not linked) on the base of such augmented
files\footnote{This directly follows from the LGPL2-RefEng-Rule by Modus Ponens.
But nevertheless, one might reply here, that even the result of manually copying
code from the Library to the work using the Library is covered by the limits of
tolerance, introduced by the LGPL-v2-§5. Formally, this argument seems to be
appropriate. And indeed, also we have to consider these limits of
tolerance later on. But in the context of copying code from the Library into the
work manually, a closer look reduces its impact. You have to discriminate three
cases:
\begin{enumerate}
  \item Developers can  \textbf{manually copy / transfer some or at most all
  elements of the Library header files into the code of the work} which the
  preprocessor itself would copy / transfer into that code automatically. But
  developers will not do that. Some simple include commands would cause the same
  effect. And developers want to save resources, especially their own working
  time. So, why should they manually do what they can delegate to the standard
  development process. Thus, it is reasonable to assume that developers, who
  nevertheless copy portions from Library into their work, do not want to repeat
  the service of the preprocessor manually, but to transfer more than only these
  elements. Hence, it is reasonable to assume, that their work is covered by the
  LGPL2-RefEng-Rule.
  \item Developers can \textbf{manually copy / transfer more than only the
  elements of the Library header files from the Library sources into the code of
  the work using the Library and they can nevertheless let the work being linked
  to the Library}. But again developers will not do that, because -- again --
  some simple include and linking commands would cause the same effect. So it is
  reasonable to start from the premise that copying developers in fact do more
  than this. Thus, it is reasonable to assume that their work is covered by the
  LGPL2-RefEng-Rule.
  \item Developers can \textbf{manually copy / transfer more than only the
  elements of the Library header files from the Library sources into the code of
  the work using the Library without linking it to the Library}. This is a
  reasonable step of work, because it spares the developers to link their work
  to the Library. But -- by definition -- such an augmented work contains more elements
  of the Library than LGPL-v2-§5 tolerates. Thus it is -- again -- reasonable to
  assume, that such a work is covered by the LGPL2-RefEng-Rule.
\end{enumerate}
Hence -- overall and from a practical point of view -- we can indeed say that
manually copying code from the Library into the work using the Library
requires to allow reverse engineering.}.

But, if we manually copy code from the Library to our work using the Library, we
also have to consider that the LGPL-v2 directly regulates this kind of using
the Library: It says, that \enquote{you may modify your copy or copies of the
Library or any portion of it [\ldots] provided that you also [\ldots] cause the
whole of the work to be licensed [\ldots] under the terms of this
License}\footcite[cf.][\nopage wp., §2, escpcially §2c]{Lgpl21OsiLicense1999a}.
Thus, there are strong arguments for the proposition, that the LGPL causes the
copyleft effect in case of literally copying code from the Library into the work
using the Library: The code of the work using the library has to be made
accessible, as well.

So, overall, we might say, that 'manually' copying code from an LGPL-v2 Library
into a work using that Library as a bypass of the standard software combining
processes and distributing the result indeed requires to additionally permit its
reverse engineering -- even if this permission is probably not very important
for the recipient, because he probably must have a direct access to the code.

\paragraph{Distributing scripts does not need reverse engineering:}
\label{RevEngDistributeScripts}
Computer programs written in a script language are distributed as they have been
developed. They are not transformed into another kind of code\footnote{Java
script is often offered as compressed code. Roughly spoken, this means that at
first all white space signs have been replaced by blanks and then all rows of
blanks have been reduced to at most one single blank. So, even then, the code
itself is directly readable and comprehensible -- even if only for very
sophisticated experts.}. The interpreter takes the script file as it is and
directly executes it. Thus, there is no special technique of reverse engineering
for understanding these kind of software: you can directly read it if you know
the script language.

So, again, we might conclude, that a script using a script Library perhaps
requires to permit its reverse engineering -- but probably this permission is
not very important for the recipient, because he can directly read the code.

\paragraph{Distributing statically combined bytecode requires the
permission of reverse engineering:}
\label{RevEngDistributeStaticallyCombinedByteCode}
In Java -- the prototype for languages which are compiled to machine independent
portable bytecode -- each class is compiled as a separate class file. These
class files have to be stored somewhere in the classpath. Aside from that,
classes can also be collected and distributed in form of packages which then can
be used like 'traditional' Libraries. These packages must also be stored
somewhere in the classpath. A single class is made known to the work that wants
to use it by an import statement which contains the class name; a whole Java
library is made usable by integrating a package statement into the code.

The code which follows such import- or package statements, can then use the
definitions offered by the classes. It denotes the elements of the classes by
the (qualified) names of its public or protected member variables or methods.
Thus, -- from a strict viewpoint -- the code of such a Java work using a Library
indeed contains portions of that library, even if these portions are only
identifying names or data structures containing identifying names. The Java
compilation process which generates the bytecode, preserves these denoting
names. It does not replace the referring names by the referred code of the
methods and so on. Only just at the end, when the java virtual machine itself
tries to execute the work using the Library, it collects all necessary commands
of all 'joined' classes.

So, one might tend to argue that answering the question whether a distributed
java bytecode already contains portions of the used Library depends on the
interpretation whether a denoting identifier of a Library indeed is a portion
of the Library. We will discuss this case together with the corresponding
C/C++-Case. 

But there is another Java specific aspect, which has to be considered as well.
As already mentioned, in Java you can also join your work containing the
denoting identifiers and the denoted Library by building a new package, which
then contains both, the work using the Library and the used Library. Hence, one
can say, that this package is quasi statically linked: if you distribute such an
integrated package, then you are distributing both components together. Thus, if
you distribute a complete package, in other words: a quasi statically linked
work containing the work using the Library and (all portions of) the Library,
then you have to permit reverse engineering\footnote{This directly follows from
the LGPL2-RefEng-Rule by Modus Ponens}.

So, preliminarily we conclude that, with respect to Java programming you (a)
have to permit reverse engineering, if you distribute your work using the
Library and the Library itself as a (statically linked) integrated
package\footnote{This follows from the LGPL2-RevEng-Rule by Modus Ponens.} and
that (b) in all other cases your obligation to permit reverse engineering
depends on the interpretation whether the identifiers declared by a Library are
indeed portions of the Library.

Fortunately, we can reasonably decide the issue of case (b) soon.

\paragraph{Distributing statically combined binaries require the
permission of reverse engineering:}
\label{RevEngDistributeStaticallyLinkedBinaries}
Similar to Java, in C/C++ -- the prototype of those languages, which are
compiled as machine specific code -- a C/C++ Library is also explicitly made
known to the work that wants to use it, namely by some include statements. These
include statements denote the header files offered by and distributed with the
Library. They contain the declarations of those elements which the Library wants
to publish. Or briefly worded: the Library contains the definitions in form of
code, the header files the corresponding declarations.

The C/C++ code following such include statements can refer to the definitions
offered by the Library by using the declarations anounced by the header files.
So, again, -- from a strict viewpoint -- the code of such a C/C++ work using the
Library indeed contains portions of the library, even if these portions are only
identifying names or data structures published by the header files.

Beyond that conceptual relation, the C/C++ development process finally compiles
the work using the library as an object file containing machine specific code.
Just as the Java compilation, this process does not replace the referring
names by the referred code of the Library; it still preserves the denoting
names. The resulting file, which has been compiled into machine specific code,
but still contains the denoting identifiers, is also known as 'object code
file'.

The C/C++ compilation process is (mostly) managed by a make file, which is
executed by the make command\footnote{Sometimes there additionally exists a
complete meta environment which generates such make files. The GNU build system
for example offers a complex set of configure scripts and make file templates
(cf. http://en.wikipedia.org/wiki/GNU\_build\_system, wp.).}. This development
tool calls the compiler for each source file, makes known the directories which
contain the compiled target object files, and finally calls the linker.
The linker recursively scans the compiled object files and replaces each
embedded identifier by a truly executable jump command into that set of Library
commands which are denoted by the identifier and which shall be executed as part
of the work using the Library. So, only at the end, the linker collects all
necessary commands of all 'joined' object files and Libraries and produces the
really executable work.

But -- notwithstanding the above -- the linker can either be called as
integrated step of the development process itself, or the linker can
be called separatedly, especially on another machine. In the first case, the
development process generates a \emph{statically linked executable} which
already contains all necessary portions of all used Libraries. In the second
case, the development process generates a \emph{dynamically linkable program} by
collecting the (set of) still unlinked object code file(s) as a distributable
package. Thus, if you distribute a statically linked executable, it definitely
contains 'portions' of the library; if you distribute a dynamically linkable
program you have to decide whether the embedded identifying names of a Library
have indeed to be interpreted as portions of the Library.

Unfortunately, we still have to consider a little complication, based on the
nature of the a C/C++ development process: contrary to the Java development
environment, a C/C++ development process inherently uses a preprocessor engine.
This engine takes the header files delivered by the Library, verifies the
syntactically correct use of the Library and can indeed replace some tokens of
the work using the Library by commands and/or lines from the Library. This
technique is known as \emph{inline functions} or \emph{macros}. They have been
invented for those cases where expanding the stack of commands during the
compilation by a real function call is more expensive than writing the embedded
commands of the function more than one time into the whole code. Hence, in the
C/C++ development process the compiled object files can indeed contain more than
only the referring names which denote portions of the Library: beside the
denoting identifiers, they can also already contain real, functionally relevant
portions of the Library.
 
Thus, -- again and similar to Java compilation -- we may conclude, that with
respect to C/C++ programming you (a) have to permit reverse engineering, if you
distribute your work using the Library together with the Library as a statically
linked program\footnote{This follows from the LGPL2-RevEng-Rule by Modus
Ponens.} and that (b) in all other cases your obligation to permit reverse
engineering depends on the interpretation whether the used identifiers or
dissolved inline functions and macros, which have been declared by the Library
and which therefore have automatically and standard conformably been embedded
into an object file, are indeed portions of the Library.

Obviously, it is time to answer this crucial question:
\label{RevEngLgplSection5Derivation}
\paragraph{Distributing dynamically combinable bytecode and linkable object code
does not require the permission of reverse engineering:} 
\label{RevEngDistributeDynamicallyLinkedCode}
Of course, there is only one instance that can answer the question whether
identifiers and dissolved inline-functions or macros, which are -- according to
the development standard -- embedded into a work using the Library, indeed are
portions of the Library. This instance is the LGPL-v2 itself. And -- fortunately
-- this license supports us in a very clear way to answer this question, even if
not by its §6 which deals with the reverse engineering, but by its §5:

The LGPL simply specifies that \enquote{linking a \enquote{work that uses the
Library} with the Library creates an executable that is a derivative of the
Library (because it contains portions of the Library) [\ldots]} and that
\enquote{the executable is therefore covered by this
License}\footcite[cf.][\nopage wp. §5]{Lgpl21OsiLicense1999a}. Additionally, it
talks about compiled, but still unlinked \enquote{object files}, which therefore
are not executables. Such an unexecutable \enquote{object file} -- for example
that of the \enquote{work using the Library} --, which \enquote{[\ldots] uses
only numerical parameters, data structure layouts and accessors, and small
macros and small inline functions (ten lines or less in length)} shall
practically not be covered by the license of the Library, because
\enquote{[\ldots] the use of the object file is unrestricted regardless of
whether it is legally a derivative work}\footcite[cf.][\nopage wp.
§5]{Lgpl21OsiLicense1999a} - as long as it does not exceed the given limits.

Obviously, the answer of the LGPL to our question is this: (a) yes, such
object files containing names and snippets offered by the used Library, could
contain portions of the Library. But it is not necessary to clarify the details,
because (b) -- up to a specific limit of sizes -- these kind of 'little'
portions being embedded into the object file by the standard compilation
processes do not evoke any requirements: they especially do not evoke the
obligation to allow reverse engineering. In other words: these little portions
of a Library which are embedded by the standard development process and which do
not contain more than the specified size of code may be regarded as another type
of portions compared to the normal, real portions which indeed evoke the
obligation to allow reverse engineering. From the viewpoint of the LGPL, they
are \emph{pseudo portions} of the Library, because they do not restrict the
containg object file in any respect.

So, from the LGPL-RevEng-Rule we can now indirectly conclude, that distributing
dynamically linkable or combinable bytecode or object code files which contain
\enquote{only numerical parameters, data structure layouts and accessors, and
small macros and small inline functions (ten lines or less in length)} being
delivered by a Library does not require to allow reverse
engineering\footnote{From the decision not to allow reverse engineering follows
by Modus Tollens applied to the LGPL2-RevEng-Rule, that the distribution of the
work using the Library must not contain real portions of the Library. From
LGPL-v2-§5 and the limit of the standard proccesses follows that here the work
using the Library does not contain normal, real portions. So, we know, that this
case is not covered by the LGPL2-RevEng-Rule and thus we are allowed to
distribute a work using the Library without allowing its reverse engineering.}.

Unfortunately, there might be a practical objection which seems to disturb our
simple result: For applying this rule correctly, we apparently have to assure
that a compiled work that uses the Library but is still not *joined to it,
indeed has only been expanded by \enquote{small macros or small inline functions
(ten lines or less in length)}. Thus, seemingly, we have to study all header
files of all used Libraries in detail, if we want to compliantly distribute a
work using a Library without permitting reverse engineering. This could be a lot
of work -- up to a bulk which practically can not be managed.

Fortunately, there is a simple solution for this challenge, a rule of thumb,
based on the principle \enquote{trust the upstream}\footnote{On the ELLW 2013,
we were told about this principle for the first time. We do not know, whether
Armijn Hemel invented it. But we can respectfully affirm that he has
persuasively explained the spirit and purpose of the principle \enquote{trust
the upstream}.}:

The Library developers of course publish the header files or the public members
and functions of the classes in exactly that form they want these elements to be
used. And they want their Library to be used as an LGPL library, otherwise they
would have chosen another License. So, they wish that improvements of the
Libraries shall be made accessible as well, but that the works using the Library
shall not necessarily be published in form of source code\footnote{The meaning
of the weak copyleft.}. Thus, as long as we use a Library exactly in that form,
the original authors have published, as long as we download the Library from
the official repository, and as long as we do not modify the intended interfaces
defined and published by the original header and class files, we may justifiably
assume that we are using the Libraries just as their copyright owners want them
to be used. And thus, -- in other words: as long as we trust the upstream -- we
might assume that the header and class files of our Libraries fit the
restrictions of the LGPL-v2.

\paragraph{LGPL-v2 compliance with or without permitting reverse engineering:} 
\label{RevEngLgpl2ComplianceByRenverseEngine}

Now, we have reached our target. Our last clarification can directly be applied
to the both open cases: to the case of distributing Java bytecode as well, as to
the case of distribution C/C++ object code. We now know, that the LGPL-v2
wishes, that not all portions of a Library covered by a work using the Library,
trigger the permission of reverse engineering. And we now know that the limits
-- given by the LGPL-v2-§5 -- up to which such pseudo portions indeed do not
trigger the obligation to permit reverse engineering, are respected, if we use
\emph{'upstream approved'} C/C++ and Java libraries in standard development
environments. Thus, we indeed finally may conclude, that the
LGPL-RevEng-Sentence

\begin{quote}\noindent\emph{\enquote{[\ldots] you may [\ldots] combine
\textbf{or} link a \enquote{work that uses the Library} with the Library to
produce a work containing portions of the Library \textbf{and} distribute that
work under terms of your choice, \textbf{provided that} the terms permit
modification of the work for the customer's own use \textbf{and} \emph{reverse
engineering} for debugging such modifications.}\footcite[cf.][\nopage wp., §6, 
emphasis KR.]{Lgpl21OsiLicense1999a}}
\end{quote}

means 'nothing else' than

\begin{itemize}
  \item \emph{With respect to a LGPL-v2 licensed Library, you are not required
  to allow reverse engineering, if you [A] develop your work using the Library,
  on the base of a standard version of the Library containing the interfaces as
  the original developers have designed it, if you [B] compile your work using
  this Library, as a discrete (set of) dynamically linkable or combinable
  file(s), if you [C] use only the standard compilation methods which preserve
  the upstream approved interfaces\footnote{and which therefore do not to exceed
  the LGPL-v2 limits!}, and if you [D] distribute the produced unlinked object
  code or bytecode files before they are linked as an executable.}
  \item \emph{In all other cases of distributing a work using such a Library,
  you are required to allow reverse engineering of the work using this Library
  -- especially, \ldots}
  \begin{itemize}
    \item \emph{if you distribute the work using the Library and the Library
    together as a statically linked program or as an integrated package
    containing both parts, the work using the library and the Library
    itself\footnote{This holds also if you distribute a script language based
    program or package, notwithstanding the fact, that one does not need the
    permission of reverse engineering to understand script language based
    applications.}.}
    \item \emph{if you distribute a work containing manually copied portions of
    the Library.}
  \end{itemize}
\end{itemize}

%% use all entries of the bibliography
%\nocite{*}

