% Telekom osCompendium 'for beeing included' snippet template
%
% (c) Karsten Reincke, Deutsche Telekom AG, Darmstadt 2011
%
% This LaTeX-File is licensed under the Creative Commons Attribution-ShareAlike
% 3.0 Germany License (http://creativecommons.org/licenses/by-sa/3.0/de/): Feel
% free 'to share (to copy, distribute and transmit)' or 'to remix (to adapt)'
% it, if you '... distribute the resulting work under the same or similar
% license to this one' and if you respect how 'you must attribute the work in
% the manner specified by the author ...':
%
% In an internet based reuse please link the reused parts to www.telekom.com and
% mention the original authors and Deutsche Telekom AG in a suitable manner. In
% a paper-like reuse please insert a short hint to www.telekom.com and to the
% original authors and Deutsche Telekom AG into your preface. For normal
% quotations please use the scientific standard to cite.
%
% [ File structure derived from 'mind your Scholar Research Framework' 
%   mycsrf (c) K. Reincke CC BY 3.0  http://mycsrf.fodina.de/ ]

%


%% use all entries of the bibliography
%\nocite{*}

\textbf{Erzähltopos für Open Sourcewerke: Mit GPL als Prototyp}

Three things constitute the topos to tell 'Open Source':
\begin{enumerate}
  \item First,  one highlights the increasing importance opne Open
  Source\footnote{\cite[cf.][1ff]{Oberhem2008a}}.
  \item Second, one tells the genesis of open
  source\footnote{\cite[cf.][9]{Oberhem2008a}} and discusses the meaning of some
  general concepts\footnote{\cite[cf.][6ff and 17ff]{Oberhem2008a}}
  \item And last but not least one explains the Open Source
  Definition\footnote{\cite[cf.][10ff]{Oberhem2008a}}
\end{enumerate}

The next - and as I want to say: wrong - step of the line of thoughts is often
the focusing on the GPL: This license is declared as \enquote{Grundtypus}
and is therefore taken as the core of the license
discussion\footnote{\cite[cf.][33]{Oberhem2008a}}, sometimes additionally
underlined by the comment, that differences between GPL version 2 und GPL
version 3 are discussed in the footnotes\footnote{\cite[cf.][34]{Oberhem2008a}}.

This kind of reflecting the Open Source licenses is supported by the hope, that
talking about the GPL covers all other cases. That's invalid.

Unmittelbare Folgereferenz\footnote{\cite[cf.][S.2]{Oberhem2008a}}.


%\bibliography{../bibfiles/oscResourcesEn}

% Local Variables:
% mode: latex
% fill-column: 80
% End:
