% Telekom osCompendium extract file
%
% (c) Karsten Reincke, Deutsche Telekom AG, Darmstadt 2011
%
% This LaTeX-File is licensed under the Creative Commons Attribution-ShareAlike
% 3.0 Germany License (http://creativecommons.org/licenses/by-sa/3.0/de/): Feel
% free 'to share (to copy, distribute and transmit)' or 'to remix (to adapt)'
% it, if you '... distribute the resulting work under the same or similar
% license to this one' and if you respect how 'you must attribute the work in
% the manner specified by the author ...':
%
% In an internet based reuse please link the reused parts to www.telekom.com and
% mention the original authors and Deutsche Telekom AG in a suitable manner. In
% a paper-like reuse please insert a short hint to www.telekom.com and to the
% original authors and Deutsche Telekom AG into your preface. For normal
% quotations please use the scientific standard to cite.
%
% [ File structure derived from 'mind your Scholar Research Framework' 
%   mycsrf (c) K. Reincke CC BY 3.0  http://mycsrf.fodina.de/ ]

%
% select the document class
% S.26: [ 10pt|11pt|12pt onecolumn|twocolumn oneside|twoside notitlepage|titlepage final|draft
%         leqno fleqn openbib a4paper|a5paper|b5paper|letterpaper|legalpaper|executivepaper openrigth ]
% S.25: { article|report|book|letter ... }
%
% oder koma-skript S.10 + 16
\documentclass[DIV=calc,BCOR=5mm,11pt,headings=small,oneside,abstract=true, toc=bib]{scrartcl}

%%% (1) general configurations %%%
\usepackage[utf8]{inputenc}

%%% (2) language specific configurations %%%
\usepackage[]{a4,ngerman}
\usepackage[english,ngerman]{babel}
\selectlanguage{ngerman}

%language specific quoting signs
%default for language emglish is american style of quotes
\usepackage[german=quotes]{csquotes}

% jurabib configuration
\usepackage[see]{jurabib}
\bibliographystyle{jurabib}
% Telekom osCompendium German Jurabib Configuration Include Module 
%
% (c) Karsten Reincke, Deutsche Telekom AG, Darmstadt 2011
%
% This LaTeX-File is licensed under the Creative Commons Attribution-ShareAlike
% 3.0 Germany License (http://creativecommons.org/licenses/by-sa/3.0/de/): Feel
% free 'to share (to copy, distribute and transmit)' or 'to remix (to adapt)'
% it, if you '... distribute the resulting work under the same or similar
% license to this one' and if you respect how 'you must attribute the work in
% the manner specified by the author ...':
%
% In an internet based reuse please link the reused parts to www.telekom.com and
% mention the original authors and Deutsche Telekom AG in a suitable manner. In
% a paper-like reuse please insert a short hint to www.telekom.com and to the
% original authors and Deutsche Telekom AG into your preface. For normal
% quotations please use the scientific standard to cite.
%
% [ File structure derived from 'mind your Scholar Research Framework' 
%   mycsrf (c) K. Reincke CC BY 3.0  http://mycsrf.fodina.de/ ]

% the first time cite with all data, later with shorttitle
\jurabibsetup{citefull=first}

%%% (1) author / editor list configuration
%\jurabibsetup{authorformat=and} % uses 'und' instead of 'u.'
% therefore define your own abbreviated conjunction: 
% an 'and before last author explicetly written conjunction

% for authors in citations
\renewcommand*{\jbbtasep}{ u. } % bta = between two authors sep
\renewcommand*{\jbbfsasep}{, } % bfsa = between first and second author sep
\renewcommand*{\jbbstasep}{ u. }% bsta = between second and third author sep
% for editors in citations
\renewcommand*{\jbbtesep}{ u. } % bta = between two authors sep
\renewcommand*{\jbbfsesep}{, } % bfsa = between first and second author sep
\renewcommand*{\jbbstesep}{ u. }% bsta = between second and third author sep

% for authors in literature list
\renewcommand*{\bibbtasep}{ u. } % bta = between two authors sep
\renewcommand*{\bibbfsasep}{, } % bfsa = between first and second author sep
\renewcommand*{\bibbstasep}{ u. }% bsta = between second and third author sep
% for editors  in literature list
\renewcommand*{\bibbtesep}{ u. } % bte = between two editors sep
\renewcommand*{\bibbfsesep}{, } % bfse = between first and second editor sep
\renewcommand*{\bibbstesep}{ u. }% bste = between second and third editor sep

% use: name, forname, forname lastname u. forname lastname
\jurabibsetup{authorformat=firstnotreversed}
\jurabibsetup{authorformat=italic}

%%% (2) title configuration
% in every case print the title, let it be seperated from the 
% author by a colon and use the slanted font
\jurabibsetup{titleformat={all,colonsep}}
%\renewcommand*{\jbtitlefont}{\textit}

%%% (3) seperators in bib data
% separate bibliographical hints and page hints by a comma
\jurabibsetup{commabeforerest}

%%% (4) specific configuration of bibdata in quotes / footnote
% use a.a.O if possible
\jurabibsetup{ibidem=strict}

% replace ugly a.a.O. by ders., a.a.O. resp. ders., ebda.
% but if there are more than one author or girl writers?
\AddTo\bibsgerman{
  \renewcommand*{\ibidemname}{Ds., a.a.O.}
  \renewcommand*{\ibidemmidname}{ds., a.a.O.}
}
\renewcommand*{\samepageibidemname}{Ds., ebda.}
\renewcommand*{\samepageibidemmidname}{ds., ebda.}

%%% (5) specific configuration of bibdata in bibliography
% ever an in: before journal and collection/book-tiltes 
\renewcommand*{\bibbtsep}{in: }
%\renewcommand*{\bibjtsep}{in: }

% ever a colon after author names 
\renewcommand*{\bibansep}{: }
% ever a semi colon after the title 
\renewcommand*{\bibatsep}{; }
% ever a comma before date/year
\renewcommand*{\bibbdsep}{, }

% let jurabib insert the S. and p. information
% no S. necessary in bib-files and in cites/footcites
\jurabibsetup{pages=format}

% use a compressed literature-list using a small line indent
\jurabibsetup{bibformat=compress}
\setlength{\jbbibhang}{1em}

% which follows the design of the cites and offers comments
\jurabibsetup{biblikecite}

% print annotations into bibliography
\jurabibsetup{annote}
\renewcommand*{\jbannoteformat}[1]{{ \itshape #1 }}

%refine the prefix of url download
\AddTo\bibsgerman{\renewcommand*{\urldatecomment}{Referenzdownload: }}

% we want to have the year of articles in brackets
\renewcommand*{\bibaldelim}{(}
\renewcommand*{\bibardelim}{)}

%Umformatierung des Reihentitels und der Reihennummer
\DeclareRobustCommand{\numberandseries}[2]{%
\unskip\unskip%,
\space\bibsnfont{(=~#2}%
\ifthenelse{\equal{#1}{}}{)}{, [Bd./Nr.]~#1)}%
}%

% Local Variables:
% mode: latex
% fill-column: 80
% End:


% language specific hyphenation
% Telekom osCompendium osHyphenation Include Module
%
% (c) Karsten Reincke, Deutsche Telekom AG, Darmstadt 2011
%
% This LaTeX-File is licensed under the Creative Commons Attribution-ShareAlike
% 3.0 Germany License (http://creativecommons.org/licenses/by-sa/3.0/de/): Feel
% free 'to share (to copy, distribute and transmit)' or 'to remix (to adapt)'
% it, if you '... distribute the resulting work under the same or similar
% license to this one' and if you respect how 'you must attribute the work in
% the manner specified by the author ...':
%
% In an internet based reuse please link the reused parts to www.telekom.com and
% mention the original authors and Deutsche Telekom AG in a suitable manner. In
% a paper-like reuse please insert a short hint to www.telekom.com and to the
% original authors and Deutsche Telekom AG into your preface. For normal
% quotations please use the scientific standard to cite.
%
% [ File structure derived from 'mind your Scholar Research Framework' 
%   mycsrf (c) K. Reincke CC BY 3.0  http://mycsrf.fodina.de/ ]
%


\hyphenation{rein-cke}

% Local Variables:
% mode: latex
% fill-column: 80
% End:


%%% (3) layout page configuration %%%

% select the visible parts of a page
% S.31: { plain|empty|headings|myheadings }
%\pagestyle{myheadings}
\pagestyle{headings}

% select the wished style of page-numbering
% S.32: { arabic,roman,Roman,alph,Alph }
\pagenumbering{arabic}
\setcounter{page}{1}

% select the wished distances using the general setlength order:
% S.34 { baselineskip| parskip | parindent }
% - general no indent for paragraphs
\setlength{\parindent}{0pt}
\setlength{\parskip}{1.2ex plus 0.2ex minus 0.2ex}


%%% (4) general package activation %%%
%\usepackage{utopia}
%\usepackage{courier}
%\usepackage{avant}
\usepackage[dvips]{epsfig}

% graphic
\usepackage{graphicx,color}
\usepackage{array}
\usepackage{shadow}
\usepackage{fancybox}

%- start(footnote-configuration)
%  flush the cite numbers out of the vertical line and let
%  the footnote text directly start in the left vertical line
\usepackage[marginal]{footmisc}
%- end(footnote-configuration)

\begin{document}

%% use all entries of the bliography

%%-- start(titlepage)
\titlehead{Literaturexzerpt}
\subject{Autor: Rouven Buchtala}
\title{Titel: Determinatente der Open Source Software}
\subtitle{Jahr: 2006 }
\author{K. Reincke% Telekom osCompendium License Include Module
%
% (c) Karsten Reincke, Deutsche Telekom AG, Darmstadt 2011
%
% This LaTeX-File is licensed under the Creative Commons Attribution-ShareAlike
% 3.0 Germany License (http://creativecommons.org/licenses/by-sa/3.0/de/): Feel
% free 'to share (to copy, distribute and transmit)' or 'to remix (to adapt)'
% it, if you '... distribute the resulting work under the same or similar
% license to this one' and if you respect how 'you must attribute the work in
% the manner specified by the author ...':
%
% In an internet based reuse please link the reused parts to www.telekom.com and
% mention the original authors and Deutsche Telekom AG in a suitable manner. In
% a paper-like reuse please insert a short hint to www.telekom.com and to the
% original authors and Deutsche Telekom AG into your preface. For normal
% quotations please use the scientific standard to cite.
%
% [ File structure derived from 'mind your Scholar Research Framework' 
%   mycsrf (c) K. Reincke CC BY 3.0  http://mycsrf.fodina.de/ ]
%
\footnote{
This text is licensed under the Creative Commons Attribution-ShareAlike 3.0 Germany
License (http://creativecommons.org/licenses/by-sa/3.0/de/): Feel free \enquote{to
share (to copy, distribute and transmit)} or \enquote{to remix (to
adapt)} it, if you \enquote{[\ldots] distribute the resulting work under the
same or similar license to this one} and if you respect how \enquote{you
must attribute the work in the manner specified by the author(s)
[\ldots]}):
\newline
In an internet based reuse please mention the initial authors in a suitable
manner, name their sponsor \textit{Deutsche Telekom AG} and link it to
\texttt{http://www.telekom.com}. In a paper-like reuse please insert a short
hint to \texttt{http://www.telekom.com}, to the initial authors, and to their
sponsor \textit{Deutsche Telekom AG} into your preface. For normal citations
please use the scientific standard.
\newline
{ \tiny \itshape [based on myCsrf (= mind your Scholar Research Framework) 
\copyright K. Reincke CC BY 3.0  https://github.com/kreincke/mycsrf/)] }}

% Local Variables:
% mode: latex
% fill-column: 80
% End:
}
%\thanks{den Autoren von KOMA-Script und denen von Jurabib}
\maketitle
%%-- end(titlepage)
%\nocite{*}

\begin{abstract}
\noindent
Die Arbeit/The work\footcite[][]{Buchtala2007a} \\
\noindent \itshape
\ldots will auf Basis spieltheoretischer Aspekte ermitteln, was die
Auswahl von OS-Lizenzen beeinflusst. Dazu werden diese als 'permissiv',
'restriktiv' und 'hochrestriktiv' klassifiziert. Das erhellt inhaltliche
Bedingungen der Lizenzerfüllung. Die Namen der Kategorien sind jedoch aus Sicht
einer Firma gewählt, die sich möglichst uneingeschränkt an freiem Code bedienen
will. Die Intention der OS-Lizenzgeber, gewährte Freiheiten gegen ihre
'Rücknahme' abzusichern, entfällt dabei. \\
\noindent
\ldots tries to detect why specific OS-licenses are chosen. Game
theoretic aspects shall help to answer the question. Thereto OS-licenses are
classified as 'permissive', 'restrictive' and 'highly restrictive' licenses.
This differentiation highlights some necessary constraints for respecting a
license. But the names of these categories are generated with respect to a
company which wants to be able to (re)use and (re)sell OS code with minimal
restrictions. The intention of the OS licensors to defend the licensed freedom
for revocations is not covered by these names.
\end{abstract}
\footnotesize
%\tableofcontents
\normalsize

\section{Ausrichtung und Hauptargumentationen}

\subsection{Summary}
Die Arbeit möchte die Frage beantworten, "`[\ldots] welche Faktoren [\ldots] die
Wahl zwischen den unterschiedlichen OSS-Lizenzen (beeinflussen)
"\footcite[cf.][25]{Buchtala2007a}:

\begin{quotation}\noindent\enquote{Mit Hilfe eines spieltheoretischen Modells lassen
sich die Bedingungen charakterisieren, unter denen ein privater oder ein kommerzieller
Entwickler sich für eine permissive, restriktive oder hoch restriktive Lizenz
entschieden wird.}\footcite[][26]{Buchtala2007a}
\end{quotation}

Mit der Arbeit geht so der Anspruch einer
'Erklärungsadäquatheit'\footnote{Begriff von mir, KR} einher: Am Anfang eines
jeden spieltheoretischen Zugriffs steht eine Modellierung der Wirklichkeit als Set von 'Annahmen'. Da, so der Gedankengang im Groben,
\enquote{die Vorhersagen des Modells [\ldots] dabei im Einklang mit bisher
erbrachten empirischen Ergebnissen (stehen) [\ldots]}, dürfen die
modellierten Annahmen \enquote{[\ldots] als eine mögliche Erklärungen von diesen
[den Ergebnissen, KR] angesehen werden}\footcite[][26]{Buchtala2007a}.

\subsection{Details}

Ein Kernpunkt der Gesamtargumentation ist die Klassifikation resp. Gruppierung
bestehender Open Source Lizenzen in \enquote{permissive, restriktive und
hochrestriktive Lizenzen}\footcite[cf.][55]{Buchtala2007a}: Als Open
Source Lizenzen seien sie durch die 10 Kriterien der Open Source Initiative zu
einer Gruppe zusammengefasst\footcite[cf.][53ff]{Buchtala2007a}. In sich
ließe sich diese Gruppe allerdings weiter clustern. So stünde der Gruppe von
Lizenzen, die \enquote{\ldots es dem Nutzer explizit (erlauben), dass abgeleitete
Werke proprietarisiert werden können}, diejenigen gegenüber, die dieses
verbieten\footcite[cf.][55]{Buchtala2007a}. Und diese Gruppe der die
'Proprietarisierung' verbietenden Lizenzen könne, so Buchtala, in Anlehnung an
\enquote{Lerner und Tirole (2005)} weiter in zwei Subgruppen unterteilt
werden, in die Gruppe, bei der \enquote{Modifikationen \ldots im Distributionsfall
offengelegt werden (müssen)}, und in die andere Gruppe, die für
abgeleitete Werke eine Publikation \enquote{unter derselben Lizenz}
erfordere\footcite[cf.][56]{Buchtala2007a}.

So kommt Buchtala für die per definitionem quelloffenen Open Source Lizenzen zu
den Subtypen \enquote{Permissiv} (\enquote{Quellcode ist einsehbar, frei
verteilbar und modifiziertbar}), \enquote{Restriktiv}
(\enquote{Quellcode von Modifikationen muss im Distributionsfall offen gelegt
werden (Copyleft)}) und \enquote{Hochrestriktiv} (\enquote{Abgeleitete
Werke dürfen nur mit Software verbunden werden, die unter derselben Lizenz
veröffentlicht ist (Reziprozität)})\footcite[cf.][57 (Typo im
Original)]{Buchtala2007a}. Die dazu gehörende Grafik signalisiert, dass alle
restriktiven Lizenzen auch permissiv sind, aber nicht umgekehrt, und dass alle
restriktiven Lizenzen hoch hochrestriktiv sind, aber nicht
umgekehrt\footcite[cf.][57]{Buchtala2007a}. Explizit bestätigt wird diese
Teilmengen basierte Clusterung durch Buchtalas Merkmalsmatrix: Alle behandelten
Lizenzen (GPL, LGPL, BSD, Artistic License, Apache License (APL), MIT, Mozilla
Publuc License (MPL)) seien im \enquote{Quellcode [\ldots] einsehbar, frei
verteilbar und modifizierbar}, bei GPL, LGPL und MPL müssten
\enquote{Modifikationen [\ldots] im Distributionsfall offengelegt werden}
und nur bei der GPL dürften \enquote{abgleitete Werke [\ldots] nur mit Software
verbunden werden, die unter derselben Lizenz veröffentlicht
ist}\footcite[cf.][62]{Buchtala2007a}

In seiner Sichtung der Forschungsliteratur kommt Buchtala sodann zu dem Schluss,
dass zwar wegen des 'statistischen Primats' der hochrestriktiven Lizenzen - also
der GPL - die \enquote{[\ldots] Akteure bei der Auswahl der von ihnen verwendeten
Lizenzen umso stärker auf (hoch)restriktive Lizenzen zurück greifen würden, je
wichtiger die Zusammenarbeit mit anderen (hoch)restriktiven Lizenzen bzw. mit
der OSS-Community ist[\ldots]}, dass mit diesem Faktum aber noch nicht die
Frage beantwortet sei, \enquote{[\ldots] wovon die ursprüngliche
Lizenzentscheidung determiniert
wird}\footcite[cf.][156]{Buchtala2007a}. Einen solchen Grund für die
ursprünglichen Wahl 'nachzureichen', ist dann, wie angekündigt, Hauptziel des
Buches.

So konstatiert Buchtala, dass \enquote{[\ldots] die oftmals wahrgenommene
Verbundenheit zwischen dem OSS-Gedanken und der GPL keineswegs unbegründet oder
zufällig entstanden (sei)}\footcite[cf.][168]{Buchtala2007a}. Dazu
argumentiert er so: Zum einen könne \enquote{Code}, der \enquote{[\ldots]
unter einer hochrestriktiven Lizenz veröffentlicht wurde, [\ldots] nur dann in
einem anderen Projekt legal weiterverwendet werden, falls dieses ebenfalls eine
hochrestriktive Lizenz (verwende)}. Andererseits dürfe
\enquote{permissiv} lizenzierter Code \enquote{[\ldots] unter praktisch jeder
anderen Lizenz, sowohl einer hochrestriktiven als auch einer proprietären,
veröffentlicht werden}. Daraus ergäbe sich dann \enquote{ein inhärenter
Vorteil für hochrestriktive Lizenzen und im speziellen für die GPL}: Da
'hochrestriktiv lizenzierter Code' auf das Ergebnis einer größeren Anzahl von
zuarbeitenden OS-Projekten zugreifen könn, nämlich auf die mit permissiven
Lizenzen UND auf die mit (hoch)restriktiven Lizenzen, während 'permissiv
lizenzierter Code' eben nur auf das Ergebnis solcher Projekte, die
selbst permissiv orientiert seien\footcite[cf.][165]{Buchtala2007a}. Buchtala
fixiert das in folgendem Befund:
\begin{quotation}\noindent
\enquote{Ein Projekt, dass unter einer hochrestriktiven Lizenz veröffentlicht
wird, kann Code von sämtlichen OSS Projekten verwenden, unabhängig von deren Lizenz. Ein
Projekt, das unter einer permissiven Lizenz veröffentlicht wird, kann nur den
Code von Projekten verwenden, die ebenfalls unter einer permissiven Lizenz
veröffentlicht wurden.}\footcite[][165]{Buchtala2007a}.
\end{quotation}

Anm.: Die Fakten widersprechehn der Deduktion. Buchtala kennt nur eine
hochrestriktive Lizenz, nämlich die
\enquote{GPL}\footcite[cf.][62]{Buchtala2007a}. Gerade das GNU Projekt resp. die
FSF listet explizit auf, dass und mit welchen anderen Lizenzen die GPL gerade
nicht kompatibel ist. Und darunter befinden sich durchaus auch - um mit Buchtala
zu sprechen - restriktive oder permissive. Insofern wird dieser irrtümlich
angesetzte und doch in die spieltheoretische Modellierung eingegangene
\enquote{inhärente Vorteil für hochrestriktive
Lizenzen}\footcite[cf.][168]{Buchtala2007a} das Ergebnis
verfälschen.

\section{Spezifische Diskussionen}

\subsection{Veröffentlichungspflicht}
Buchtala unterstreicht, dass bei OS-Lizenzen die Veröffentlichungspflicht von
Modifikationen nur dann greife, \enquote{[\ldots] wenn die modifizierte Software
auch weiter vertrieben wird}. Sofern Software nur \enquote{[\ldots] für den
internen Gebrauch weiterentwickelt (werde)}, sei die Publikation der
Modifikationen nicht geboten. Außerdem - betont Buchtala weiter - impliziere das
Veröffentlichungsgebot eben nicht, dass die Modifikationen \enquote{[\ldots] für
alle einsichtig zu machen (seien)[\ldots]}. Vielmehr müsse der revidierte
Code nur konkret denjenigen gegenüber offengelegt werden, an die \enquote{[\ldots]
das Produkt distribuiert (werde)}\footcite[cf.][168 - B. beruft sich
selbst auf Henkel 2003 u. 2004b]{Buchtala2007a}.

\subsection{(rein) permissive Lizenzen}
Als (rein) permissive Lizenzen gelten Buchtala die BSD-License, die Artistic
License, die Apache License (APL) und die
MIT-License\footcite[cf.][62]{Buchtala2007a}.

Anm-1: Für die Apache Lizenz scheint das auf den ersten Blick zu stimmen, der
zweite Blick in den Lizenztext sollte stutzig machen. Dort steht, dass man das
Recht hat, Modifikationen und Derivationen als Sourcecode oder Binärcode frei zu
vertreiben. Mann kann den modifizierten Code also weitergeben, muss es aber
nicht. In jedem Fall aber, also auch wenn man seine eigene Weiterentwicklung -
proprietär - nur als Binärcode weitergibt, muss man die Apache-Lizenz dazu legen
und mit weitergeben. Damit unterläge also auch der distribuierte Binärcode der
Apache-Lizenz, die ihrerseits Modifikationen frei erlaubt, also auch das
Reengineering des Codes. Ein vollständige Proprietärisierung ist demanch mit der
APL gar nicht möglich.

Anm-2: Für die Artistic License skizziert Buchtala ein Argument gegen die
Deutung als permissive Lizenz: Diese fordere im Normalfall zwar, dass
Modifikation im Distributionsfall offengelegt werden müssen, allerdings erlaube
sie Ausnahmen. Deshalb werde die Artistic License \enquote{[\ldots]
überlicherweise als permissiv angesehen}\footcite[cf.][61 B. beruft
sich auf Perens, 1999 und Lerner und Tirole 2005, jeweils ohne nähere
Seitenangabe]{Buchtala2007a}

\subsection{Klassifikation}

Es gibt mindestens 2 Schwächen in der Art, wie B. die Lizenzen klassifiziert.
Die eine betrifft die Teilmengenrelation der definierenden Features, die zweite
den Namensgebeung der Kategorien:

\subsubsection{Problem: Teilmengen begründete Klassendefinition}

Wenn alle restriktiven Lizenzen auch permissiv
sind, nicht aber umgekehrt, und wenn alle hochrestriktiven Lizenzenh auch
restriktiv sind, nicht aber umgekehrt\footcite[cf.][62]{Buchtala2007a},
dann macht es keinen Sinn mehr, von den 'permissiven Lizenzen' oder den
'restriktiven Lizenzen' zu sprechen, wenn man die einen (BSD, Artistic, (APL)
MIT) gegen die anderen (MPL, GPL, LGPL) abgrenzen will.

\subsubsection{Problem: Klassenbennung}

ANM-1: Wenn alle restriktiven Lizenzen auch permissiv sind, nicht aber
umgekehrt, dann müsste man also zur Kennzeichnung der einen von den
permissiv-restriktiven Lizenzen sprechen. Das führt zumindest zu sprachlichen
Unschärfen.

ANM-2: Die Namenswahl wird offensichtlich Firmen gerecht, die sich zu ihren
eigenen Zwecken möglichst ohne Einschränkungen an Open Source Software bedienen
wollen (permissiv) und die daran durch die Lizenzen gehindert (restriktiv) oder
gar stark gehindert (hochrestiktiv) werden.  Die Intention der OS-Lizenzgeber -
wie sie gerade in der GPL explizit als Grund für eine Einschränkung genannt wird
- , gewährte Freiheiten gegen ihre 'Rücknahme' abzusichern, entfällt dabei.

\subsection{Reziprozitäts-Missverständnis (?)}

B. erklärt, dass bei \enquote{hochrestiktive Lizenzen} - und er kennt da nur
eine, nämlich die GPL - von einem so lizenzierten Werk davon \enquote{abgeleitete
Werke [\ldots] nur mit Software verbunden werden (dürfen), die unter derselben
Lizenz veröffentlicht ist
(Reziprozität)}\footcite[cf.][62]{Buchtala2007a}. Dies geht am Kern der
GPL-Vorstellung vorbei. Das, was die GPL von dem abgeleiteten Werk fordert, ist,
dass es selbst ebenfalls unter der GPL veröffentlicht wird. Mit welchen anderen
Lizenzen das abgeleiete Werk dann verbunden werden darf, hängt von der GPL resp.
von deren Interpretation ab und ist nur ein Randaspekt. Kern der GPL ist, dass
die GPL-Rechte auch für das abgeleitete Werk geelten soll.

\small
\bibliography{../bibfiles/oscResourcesDe}

\end{document}
