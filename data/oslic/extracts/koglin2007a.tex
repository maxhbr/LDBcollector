% Telekom osCompendium extract template
%
% (c) Karsten Reincke, Deutsche Telekom AG, Darmstadt 2011
%
% This LaTeX-File is licensed under the Creative Commons Attribution-ShareAlike
% 3.0 Germany License (http://creativecommons.org/licenses/by-sa/3.0/de/): Feel
% free 'to share (to copy, distribute and transmit)' or 'to remix (to adapt)'
% it, if you '... distribute the resulting work under the same or similar
% license to this one' and if you respect how 'you must attribute the work in
% the manner specified by the author ...':
%
% In an internet based reuse please link the reused parts to www.telekom.com and
% mention the original authors and Deutsche Telekom AG in a suitable manner. In
% a paper-like reuse please insert a short hint to www.telekom.com and to the
% original authors and Deutsche Telekom AG into your preface. For normal
% quotations please use the scientific standard to cite.
%
% [ File structure derived from 'mind your Scholar Research Framework' 
%   mycsrf (c) K. Reincke CC BY 3.0  http://mycsrf.fodina.de/ ]

%
% select the document class
% S.26: [ 10pt|11pt|12pt onecolumn|twocolumn oneside|twoside notitlepage|titlepage final|draft
%         leqno fleqn openbib a4paper|a5paper|b5paper|letterpaper|legalpaper|executivepaper openrigth ]
% S.25: { article|report|book|letter ... }
%
% oder koma-skript S.10 + 16
\documentclass[DIV=calc,BCOR=5mm,11pt,headings=small,oneside,abstract=true, toc=bib]{scrartcl}

%%% (1) general configurations %%%
\usepackage[utf8]{inputenc}

%%% (2) language specific configurations %%%
\usepackage[]{a4,ngerman}
\usepackage[english, german, ngerman]{babel}
\selectlanguage{ngerman}

%language specific quoting signs
%default for language emglish is american style of quotes
\usepackage{csquotes}

% jurabib configuration
\usepackage[see]{jurabib}
\bibliographystyle{jurabib}
% Telekom osCompendium German Jurabib Configuration Include Module 
%
% (c) Karsten Reincke, Deutsche Telekom AG, Darmstadt 2011
%
% This LaTeX-File is licensed under the Creative Commons Attribution-ShareAlike
% 3.0 Germany License (http://creativecommons.org/licenses/by-sa/3.0/de/): Feel
% free 'to share (to copy, distribute and transmit)' or 'to remix (to adapt)'
% it, if you '... distribute the resulting work under the same or similar
% license to this one' and if you respect how 'you must attribute the work in
% the manner specified by the author ...':
%
% In an internet based reuse please link the reused parts to www.telekom.com and
% mention the original authors and Deutsche Telekom AG in a suitable manner. In
% a paper-like reuse please insert a short hint to www.telekom.com and to the
% original authors and Deutsche Telekom AG into your preface. For normal
% quotations please use the scientific standard to cite.
%
% [ File structure derived from 'mind your Scholar Research Framework' 
%   mycsrf (c) K. Reincke CC BY 3.0  http://mycsrf.fodina.de/ ]

% the first time cite with all data, later with shorttitle
\jurabibsetup{citefull=first}

%%% (1) author / editor list configuration
%\jurabibsetup{authorformat=and} % uses 'und' instead of 'u.'
% therefore define your own abbreviated conjunction: 
% an 'and before last author explicetly written conjunction

% for authors in citations
\renewcommand*{\jbbtasep}{ u. } % bta = between two authors sep
\renewcommand*{\jbbfsasep}{, } % bfsa = between first and second author sep
\renewcommand*{\jbbstasep}{ u. }% bsta = between second and third author sep
% for editors in citations
\renewcommand*{\jbbtesep}{ u. } % bta = between two authors sep
\renewcommand*{\jbbfsesep}{, } % bfsa = between first and second author sep
\renewcommand*{\jbbstesep}{ u. }% bsta = between second and third author sep

% for authors in literature list
\renewcommand*{\bibbtasep}{ u. } % bta = between two authors sep
\renewcommand*{\bibbfsasep}{, } % bfsa = between first and second author sep
\renewcommand*{\bibbstasep}{ u. }% bsta = between second and third author sep
% for editors  in literature list
\renewcommand*{\bibbtesep}{ u. } % bte = between two editors sep
\renewcommand*{\bibbfsesep}{, } % bfse = between first and second editor sep
\renewcommand*{\bibbstesep}{ u. }% bste = between second and third editor sep

% use: name, forname, forname lastname u. forname lastname
\jurabibsetup{authorformat=firstnotreversed}
\jurabibsetup{authorformat=italic}

%%% (2) title configuration
% in every case print the title, let it be seperated from the 
% author by a colon and use the slanted font
\jurabibsetup{titleformat={all,colonsep}}
%\renewcommand*{\jbtitlefont}{\textit}

%%% (3) seperators in bib data
% separate bibliographical hints and page hints by a comma
\jurabibsetup{commabeforerest}

%%% (4) specific configuration of bibdata in quotes / footnote
% use a.a.O if possible
\jurabibsetup{ibidem=strict}

% replace ugly a.a.O. by ders., a.a.O. resp. ders., ebda.
% but if there are more than one author or girl writers?
\AddTo\bibsgerman{
  \renewcommand*{\ibidemname}{Ds., a.a.O.}
  \renewcommand*{\ibidemmidname}{ds., a.a.O.}
}
\renewcommand*{\samepageibidemname}{Ds., ebda.}
\renewcommand*{\samepageibidemmidname}{ds., ebda.}

%%% (5) specific configuration of bibdata in bibliography
% ever an in: before journal and collection/book-tiltes 
\renewcommand*{\bibbtsep}{in: }
%\renewcommand*{\bibjtsep}{in: }

% ever a colon after author names 
\renewcommand*{\bibansep}{: }
% ever a semi colon after the title 
\renewcommand*{\bibatsep}{; }
% ever a comma before date/year
\renewcommand*{\bibbdsep}{, }

% let jurabib insert the S. and p. information
% no S. necessary in bib-files and in cites/footcites
\jurabibsetup{pages=format}

% use a compressed literature-list using a small line indent
\jurabibsetup{bibformat=compress}
\setlength{\jbbibhang}{1em}

% which follows the design of the cites and offers comments
\jurabibsetup{biblikecite}

% print annotations into bibliography
\jurabibsetup{annote}
\renewcommand*{\jbannoteformat}[1]{{ \itshape #1 }}

%refine the prefix of url download
\AddTo\bibsgerman{\renewcommand*{\urldatecomment}{Referenzdownload: }}

% we want to have the year of articles in brackets
\renewcommand*{\bibaldelim}{(}
\renewcommand*{\bibardelim}{)}

%Umformatierung des Reihentitels und der Reihennummer
\DeclareRobustCommand{\numberandseries}[2]{%
\unskip\unskip%,
\space\bibsnfont{(=~#2}%
\ifthenelse{\equal{#1}{}}{)}{, [Bd./Nr.]~#1)}%
}%

% Local Variables:
% mode: latex
% fill-column: 80
% End:


% language specific hyphenation
% Telekom osCompendium osHyphenation Include Module
%
% (c) Karsten Reincke, Deutsche Telekom AG, Darmstadt 2011
%
% This LaTeX-File is licensed under the Creative Commons Attribution-ShareAlike
% 3.0 Germany License (http://creativecommons.org/licenses/by-sa/3.0/de/): Feel
% free 'to share (to copy, distribute and transmit)' or 'to remix (to adapt)'
% it, if you '... distribute the resulting work under the same or similar
% license to this one' and if you respect how 'you must attribute the work in
% the manner specified by the author ...':
%
% In an internet based reuse please link the reused parts to www.telekom.com and
% mention the original authors and Deutsche Telekom AG in a suitable manner. In
% a paper-like reuse please insert a short hint to www.telekom.com and to the
% original authors and Deutsche Telekom AG into your preface. For normal
% quotations please use the scientific standard to cite.
%
% [ File structure derived from 'mind your Scholar Research Framework' 
%   mycsrf (c) K. Reincke CC BY 3.0  http://mycsrf.fodina.de/ ]
%


\hyphenation{rein-cke}

% Local Variables:
% mode: latex
% fill-column: 80
% End:


%%% (3) layout page configuration %%%

% select the visible parts of a page
% S.31: { plain|empty|headings|myheadings }
%\pagestyle{myheadings}
\pagestyle{headings}

% select the wished style of page-numbering
% S.32: { arabic,roman,Roman,alph,Alph }
\pagenumbering{arabic}
\setcounter{page}{1}

% select the wished distances using the general setlength order:
% S.34 { baselineskip| parskip | parindent }
% - general no indent for paragraphs
\setlength{\parindent}{0pt}
\setlength{\parskip}{1.2ex plus 0.2ex minus 0.2ex}


%%% (4) general package activation %%%
%\usepackage{utopia}
%\usepackage{courier}
%\usepackage{avant}
\usepackage[dvips]{epsfig}

% graphic
\usepackage{graphicx,color}
\usepackage{array}
\usepackage{shadow}
\usepackage{fancybox}

%- start(footnote-configuration)
%  flush the cite numbers out of the vertical line and let
%  the footnote text directly start in the left vertical line
\usepackage[marginal]{footmisc}
%- end(footnote-configuration)

\begin{document}

%% use all entries of the bliography

%%-- start(titlepage)
\titlehead{Literaturexzerpt}
\subject{Autor(en): Koglin}
\title{Titel: Opensourcerecht}
\subtitle{Jahr: 2007 }
\author{K. Reincke% Telekom osCompendium License Include Module
%
% (c) Karsten Reincke, Deutsche Telekom AG, Darmstadt 2011
%
% This LaTeX-File is licensed under the Creative Commons Attribution-ShareAlike
% 3.0 Germany License (http://creativecommons.org/licenses/by-sa/3.0/de/): Feel
% free 'to share (to copy, distribute and transmit)' or 'to remix (to adapt)'
% it, if you '... distribute the resulting work under the same or similar
% license to this one' and if you respect how 'you must attribute the work in
% the manner specified by the author ...':
%
% In an internet based reuse please link the reused parts to www.telekom.com and
% mention the original authors and Deutsche Telekom AG in a suitable manner. In
% a paper-like reuse please insert a short hint to www.telekom.com and to the
% original authors and Deutsche Telekom AG into your preface. For normal
% quotations please use the scientific standard to cite.
%
% [ File structure derived from 'mind your Scholar Research Framework' 
%   mycsrf (c) K. Reincke CC BY 3.0  http://mycsrf.fodina.de/ ]
%
\footnote{
This text is licensed under the Creative Commons Attribution-ShareAlike 3.0 Germany
License (http://creativecommons.org/licenses/by-sa/3.0/de/): Feel free \enquote{to
share (to copy, distribute and transmit)} or \enquote{to remix (to
adapt)} it, if you \enquote{[\ldots] distribute the resulting work under the
same or similar license to this one} and if you respect how \enquote{you
must attribute the work in the manner specified by the author(s)
[\ldots]}):
\newline
In an internet based reuse please mention the initial authors in a suitable
manner, name their sponsor \textit{Deutsche Telekom AG} and link it to
\texttt{http://www.telekom.com}. In a paper-like reuse please insert a short
hint to \texttt{http://www.telekom.com}, to the initial authors, and to their
sponsor \textit{Deutsche Telekom AG} into your preface. For normal citations
please use the scientific standard.
\newline
{ \tiny \itshape [based on myCsrf (= mind your Scholar Research Framework) 
\copyright K. Reincke CC BY 3.0  https://github.com/kreincke/mycsrf/)] }}

% Local Variables:
% mode: latex
% fill-column: 80
% End:
}

%\thanks{den Autoren von KOMA-Script und denen von Jurabib}
\maketitle
%%-- end(titlepage)
%\nocite{*}

\begin{abstract}
\noindent
Das Werk / The work\footcite[][]{Koglin2007a} \\
\noindent \itshape
\ldots Sehr gründliche Analyse der Gültigkeit und des Umfangs der GPL bezogen
auf das allgemeine deutsche Recht, das Urheberrecht und das Vertragsrecht.
Behandelt zudem detailliert und facettenreich die einzelnen GPL Bestimmung.
Andere Open Source Lizenzen werden jedoch nicht in derselben Art behandelt.\\
\noindent
\ldots Thoroughly this book analyzes the validity of GPL with respect to the
general German right, the German copyright ('Urheberrecht') and the German law
of contract. Additionally it discusses all paragraphs of the GPL and their
concret meaning as part of the German right. Other Open Source licenses are not
outlined in the same manner.
\end{abstract}
\footnotesize
%\tableofcontents
\normalsize

\section{Line of Thought}

Das Buch untersucht zunächst die \enquote{Rechtsnatur der
GPL}\footcite[vgl.][21ff]{Koglin2007a}. Sodann diskutiert es die
ültigkeit der GPL im Rahmen des \enquote{deutschen
Urheberrechts}\footcite[vgl.][65ff]{Koglin2007a} und des deutschen
\enquote{Vertragsrechts}\footcite[vgl.][135ff]{Koglin2007a}. Und schließlich
analysiert es die GPL und ihre Einzelanweisung Klausel für
Klausel\footcite[vgl.][185ff]{Koglin2007a}.

\subsection{Rechtsnatur der GPL}

Grosso modi konstatiert Koglin, dass \enquote{Open-Source-Lizenzen [\ldots] als
Softwarelinzverträge einen eigenen Vertragstyp (darstellen)}.
Insbesondere sei \enquote{das Schenkungsrecht [\ldots] auf Open-Source-Lizenzen nicht
anwendbar, da die Nichtigkeit bei Missachtung der Formvorschrift zu
unbilligen Ergebnissen (führe)}. Umgekehrt aber dürfe der
\enquote{schenkungsrechtliche Haftungsmaßstab} als \enquote{nicht zwingend
unbillig} angesehen werden\footcite[vgl.][63]{Koglin2007a}:

\begin{quote}\enquote{Die Inkompabilität zwischen Schenkungsrecht und
Open-Source-Lizenzierung betrifft jedoch nur das regide Formerfordernis.
Die Anwendung des schenkungsrechtlichen Haftungsmaßstabs trifft hingegen
auf keine Bedenken.}\footcite[][62 - Typo(?) im Original, gemeint ist
wahrscheinlich die Inkompatibilität]{Koglin2007a}
\end{quote}

Koglin respektiert, dass in der juristischen Forschungsliteratur andere
Standpunkte existieren, etwa der von Metzger und Jager, denenzufolge - so Koglin
- \enquote{[\ldots] die GPL den Tatbestand des §516 Abs. 1 BGB (erfülle)
und [\ldots] daher eine Schenkung
(darstelle)}\footcite[vgl.][38]{Koglin2007a}: Obwohl dieser Aufassung
partiell gefolgt werde, könne die Interpretation als Schenkung kritisiert
werden, weil es fraglich sei, \enquote{[\ldots] ob die Zuwendung aus dem
Vermögen des Schenkers stamme} und weil eine (geforderte)
\enquote{Gegenleistung} des 'Beschenkten' \enquote{[\ldots] eine
Schenkung ausschließe [\ldots]}\footcite[vgl.][39]{Koglin2007a}. Das
Problem mit der Deutung als Schenkung bestünde zudem darin, dass \enquote{[\ldots]
die Folge eine Qualifizierung der GPL als Schenkung [\ldots] die volle
Anwendung des Schenkungsrechts (wäre)}, was Rücktrittsrechte und
\enquote{Formbedürfnisse} etc. beträfe und zu Verwerfungen
führe\footcite[vgl.][52ff]{Koglin2007a}.

Für diesen Punkt formuliert Koeglin das Fazit:

\begin{quote}\enquote{Die Regelungen sind vom Lizenzgeber gestellte Allgemeine
Geschäftsbedingungen, die auch gegenüber Verbrauchern regelmäßig
[//Seitenwechsel//] wirksam eingebunden werden. Dabei ist insbesondere von
Bedeutung, dass Nutzer die GPL typischerweise nicht bereits beim Download oder
dem ersten Ausführen des Programms annehmen, sondern erst zum Vervielfältigen,
Verbreiten oder Bearbeiten des
Programms.}\footcite[vgl.][228]{Koglin2007a}
\end{quote}

\subsection{Urheberrechtliche Aspekte}

Zentral ist hier die Einsicht, dass bei Open Source Software, \enquote{[\ldots]
die im Wege der Basar-Methode von einer Vielzahl voneinander unabhängigen
Entwickler hergestellt wird, [\ldots] somit eine Vielzahl von
Lizenzgebern und folglich ein Bündel von Lizenzverträgen
(existiere)}\footcite[vgl.][225]{Koglin2007a}. Allerdings würden das
Nutzungs-, Verbreitungs- und Modifikationsrecht immer wechselseitig bei der
GPLisierung übertragen, sodass sich explizite Lizenzierung aller Urheber bei der
Weitergabe erübrige\footcite[vgl.][133]{Koglin2007a}


\subsection{Vertragsrechtliche Aspekte}

Als 'Vertrag' etabliere die GPL zunächst nur \enquote{ein unbefristetes und
unwiderrufliches Angebot}, das auch von \enquote{Dritten als Bote}
weitergeben werden dürfe. Zudem etabliere der erste faktisch
\enquote{abgeschlossene Softwarelizenzvertrag} einen \enquote{Vertrag
zugunsten Dritter}, will sagen: der Urheber unterwerfe sich damit der
Pflicht, mit allen dritten, vierten usw., die das Angebot annehmen, einen
ebensolchen Lizenzvertrag einzugehen\footcite[vgl.][182]{Koglin2007a}.

Das Angebot selbst müsse sodann noch angenommen werden. Dieses geschehe
\enquote{[\ldots] typischerweise nicht bereits mit dem Erhalt oder dem
Ausführen des Programms, sondern erst bei der Vervielfältigung oder bei
einer anderen über die bloße Nutzung hinausgehenden
Verwertungshandlung}\footcite[vgl.][182]{Koglin2007a}.

Diese 'nachgezogene Lizenzannahme' liefert sodann auch die Begründung dafür,
dass eine englisch sprachige Lizenz in einem deutschen Vertrag dennoch gültig
sein könne. Wer Software weitergebe oder gar modifiziere, sei \enquote{[\ldots]
in aller Regel im Softwarebereich versiert [\ldots]}. Und da ihm
außerdem im Moment der Annahme, also bei der Weitergabe oder der Modifikation
die GPL vorliege, \enquote{[\ldots] (werde) die GPL trotz der Abfassung in
der englischen Sprache wirksam in den Softwarelöizenzvertrag
eingebunden}\footcite[vgl.][182f]{Koglin2007a}

Für diesen Punkt formuliert Koeglin das Fazit:

\begin{quote}\enquote{Als zweiseitiger Softwarelizenzvertrag kommt die GPL
durch Angebot und Annahme zustande, wobei der Lizenzgeber auf den Zugang
der Annahmeerklärung verzichtet. [\ldots] Typische Nutzer nehmen das
Angebot noch nicht beim Download oder Ausführen eines GPL-Programms,
sondern erst bei [\ldots] einer Verbreitung oder der Bearbeitung des
Programms an.}\footcite[][227]{Koglin2007a}
\end{quote}

\subsection{GPL Klauseln}

\subsubsection{C-Klauseln und Weitergabe Lizenztext}
Koeglin betont, dass es bei der GPL - neben den bekannten Pflichten zur
Weitergabe auch des (modifizierten) Codes - insbesondere auch die Pflicht gebe,
die Copyright-Vermerke und Haftungshinweise bei Kopien anzubringen resp.
bestehen zu lassen und \enquote{[\ldots] im Moment der Übergabe des Programms
den Text der GPL mitzugeben}\footcite[vgl.][192]{Koglin2007a}

\subsubsection{Honorarfrage}

Koeglin erklärt, dass die Höhe des Entgelts von der GPL nicht beschränkt werde.
Stattdessen werde nur festgelegt, das sich dieses Entgeld zar beispielsweise auf
den Service der Distribution beziehen dürfe, \enquote{[\ldots] nicht jedoch
für den rechtlichen Vorgang des Einräumens von Nutzungsrechten verlangt
werden (dürfe)}\footcite[vgl.][198]{Koglin2007a}.

\subsubsection{Programmcode- und Quellcodeweitergabe}

Koeglin unterstreicht, dass nach GPL die getrennte Übergabe von Programm -
beispielsweise auf DVD - und Quellcode - beispielsweise per Download - schlicht
unzulässig sei: der \enquote{[\ldots] Wortlaut der GPL (sei) hier
eindeutig}\footcite[vgl.][203]{Koglin2007a}. Und auch wenn das gängige
Praxis sei, stelle es \enquote{[\ldots] gleichwohl einen Verstoß gegen die GPL
dar}\footcite[vgl.][204]{Koglin2007a}

Zudem gebe es insofern eine Lücke in der GPL, als ihre Methode, Quelltext und
Programm für drei Jahre zum Download bereithalten zu können und die
Distribution als Linkausübergabe organisieren zu drüfen, sich für die
Bereitstellung jeweils nur auf die Übergabe von a nach b beziehe, für die
Linkweitergabe aber auch auf b und c. Wenn b das Programm an c aber weitergäbe,
beginne die 3 Jahresfrist trptzdem nicht von neuem anzulaufen. Damit sei eben
\enquote{[\ldots] durch die GPL nicht sichergestellt, dass der Lizenznehmer
[c ; KR] den Quellcode [defacto] erhalten
kann}\footcite[vgl.][205]{Koglin2007a}

\subsubsection{Beginn der Weitergabepflicht}

Koeglin unterstreicht, dass die GPL eine \enquote{spezielle Bestimmung}
enthalte, derzufolge \enquote{[\ldots] die Pflicht zur Lizenzierung der
eigenen Bearbeitung [\ldots] erst mit der Verbreitung oder
Veröffentlichung [\ldots] eintreten solle}. Gemeint sei der GPL-Text unter
Sec. 2 Abs 1 lit. b\footcite[vgl.][208]{Koglin2007a}. Das Fazit dazu lautet:

\begin{quote}\enquote{So lange die Bearbeitung weder verbreitet noch
veröffentlicht wird, besteht keine Verpflichtung, das Werk der GPL zu
unterstellen. Eine solche Nutzungsrecht-Erteilung wäre ohnehin für Dritte
nutzlos, da diese ohne Verbreitung oder Veröffentlichung das Programm nicht
erhalten können[..]}\footcite[][209]{Koglin2007a}
\end{quote}

Oder noch schärfer: \enquote{Erst mit der Verbreitung oder Veröffentlichung
ergeben sich Verpflichtungen aus dem
Copyleft}\footcite[][211]{Koglin2007a}

So gesehen sei die Behauptungen, \enquote{[\ldots] Bearbeitungen müssten
zwangsläufig der GPL unterliegen} oder müssten gar auch veröffentlicht
werden, zumindest Missverständnisse\footcite[][209]{Koglin2007a}.

Entscheidend ist dann aber, wann der Akt der Weitergabe rechtlich einsetzt.
Hierzu gäbe es Gerichtsentscheidungen, denenzufolge dies innerhalb kleinerer,
wohl abgegrenzter Kreise rechtlich noch nicht der Fall
sei\footcite[vgl.][211]{Koglin2007a}. Allerdings sei es jedoch
\enquote{umstritten} geblieben, \enquote{[\ldots] ob die weite
Verteilung an zahllose Empfänger innerhalb eines großen Unternehmens oder
einer Behörde als Verbreitung zu bewerten
ist}\footcite[vgl.][211]{Koglin2007a}. Hier stellt sich Koeglin auf den
Standpunkt, dass nach dem Günstigkeitsprinzip das \enquote{unternehms- oder
behördeninterne Weiterreichen einer Kopie der Bearbeitung} keine
Distribution sei. Mithin \enquote{[\ldots] könne ein Unternehmen oder eine
andere Organisation ein der GPL unterliegendes Programm intern bearbeiten
und die neue Version an alle internen Stellen weitergeben und ausgiebig
nutzen, ohne dass eine Pflicht zur Lizenzierung an jedermann
bestehe}\footcite[vgl.][212]{Koglin2007a}.

[Hinweis; KR: Fraglich wird das aber wieder für Unternehmen, bei denen die
Weitergabe über Legaleinheiten erfolgt.]

Für diesen Punkt formuliert Koeglin das Fazit:

\begin{quote}\enquote{Die Verpflichtung aus dem Copyleft greift erst, wenn
Bearbeitungen über den nicht-öffentlichen, insbesondere privaten oder
unternehmesinternen Bereich hinaus verbreitet werden. Somit besteht aus dem
Copyleft keine Verpflichtung, jeglich - insbesondere unveröffentlichte -
Bearbeitungen an Dritte zu lizenzieren. Auich zwingt die GPL den Bearbeiter
nicht, die Bearbeitung Dritten zur Verfügung zu
stellen.}\footcite[vgl.][2227]{Koglin2007a}
\end{quote}

\section{Specific Aspects}

\subsection{Vertragsbeziehung}

Koglin betont, dass bei der Weitergabe und der Lizenzierung des Weitergegeben
ein \enquote{Drei-Personen-Verhältnis} entsteht: Einerseits räumt der
\enquote{Urheber} dem \enquote{Distributor} das Recht der
Vervielfältigung und der Weitergabe ein\footcite[vgl.][32]{Koglin2007a}.
Andererseits räumt der \enquote{Urheber} - und nicht der Distributor -
beiden, dem \enquote{Empfänger} und dem \enquote{Distributor} das Recht
der Nutzung und der Modifikation ein\footcite[vgl.][33]{Koglin2007a}. Zwischen
\enquote{Distributor} und \enquote{Empfänger} können zudem andere
eigenständiug Verträge entstehen, etwa, wenn der Distributor zusätzlich
\enquote{[\ldots] Informationen über das Programm (gibt), es (empfiehlt) oder
freiwillig eine Haftung (anbietet)}\footcite[vgl.][32]{Koglin2007a}.

\subsection{auch nachträgliche Dual-Lizenzierung / Geldverdienen}

Koelgin unterstreicht, dass die GPL nur den Lizenznehmer zur
Beachtung des Copylefts verpflichte, nicht aber den \enquote{ursprünglichen
Lizenzgeber}: er sei \enquote{[\ldots] - insbesondere bezüglich des
Lizenzgebührenverbots und des Copylefts - nicht als Lizenznehmer aus der
GPL verpflichtet}\footcite[vgl.][185f]{Koglin2007a}. Daraus ist zu
folgern, dass der ursprüngliche Lizenzgeber auch nachträglich eine Dual-Lizenzierung
verfügen kann und dass er für die initiale Weitergabe als GPL Software sehr wohl
Geld verlangen könnte (selbst wenn er damit kaum ein dauerhaftes
Geschäftsmodell haben dürfte).

Das Geldverdienen mit der GPL kann sogar noch schärfer gesehen werden: So
\enquote{[\ldots] entstehe mit der Veröffentlichung oder Verbreitung nur
die Pflicht zur Verschaffung einer kostenlosen Lizenz des Programms,
nicht aber des Programms selbst}. Das bedeute umgekehrt, dass für die
Weitergabe des Programms und des Codes sehr wohl \enquote{hohe Entgelte}
verlangt werden dürfen\footcite[vgl.][213]{Koglin2007a}. [Allerdings nochmals
meine Anmerldung: das Geschäftsmodell dürfte dauerhaft nicht tragen]

Für diesen Punkt formuliert Koeglin das Fazit:
\begin{quote}\enquote{[\ldots] der Lizenznehmer (darf) für eine freiwillig
von ihm gewährte Garantie oder für das Übertragen einer Kopie des
Programms ein Entgeld in unbeschränkter Höhe verlangen. Es ist ihm
lediglich verboten, für die Einräumung von Nutzungsrechten an dem
Programm eine Gegenleistung zu
verlangen.}\footcite[vgl.][228]{Koglin2007a}
\end{quote}
 
\subsection{implizite Lizenzierung bei Weitergabe}
Koeglin verweist darauf, dass nach Verständnis der GPL Dritte, an die eine Kopie
des Programms weitergeben worden ist, die Lizenzierung automatisch erhalten.
Dies führt er auf die GPL Sec 6 Satz 1 zurück, in der es heiße, dass bei einer
Weitergabe diese die Rechte 'automatically'
erhielten\footcite[vgl.][186]{Koglin2007a}.

\small
\bibliography{../bibfiles/oscResourcesEn}

\end{document}
