% Telekom osCompendium extract file
%
% (c) Karsten Reincke, Deutsche Telekom AG, Darmstadt 2011
%
% This LaTeX-File is licensed under the Creative Commons Attribution-ShareAlike
% 3.0 Germany License (http://creativecommons.org/licenses/by-sa/3.0/de/): Feel
% free 'to share (to copy, distribute and transmit)' or 'to remix (to adapt)'
% it, if you '... distribute the resulting work under the same or similar
% license to this one' and if you respect how 'you must attribute the work in
% the manner specified by the author ...':
%
% In an internet based reuse please link the reused parts to www.telekom.com and
% mention the original authors and Deutsche Telekom AG in a suitable manner. In
% a paper-like reuse please insert a short hint to www.telekom.com and to the
% original authors and Deutsche Telekom AG into your preface. For normal
% quotations please use the scientific standard to cite.
%
% [ File structure derived from 'mind your Scholar Research Framework' 
%   mycsrf (c) K. Reincke CC BY 3.0  http://mycsrf.fodina.de/ ]

%
% select the document class
% S.26: [ 10pt|11pt|12pt onecolumn|twocolumn oneside|twoside notitlepage|titlepage final|draft
%         leqno fleqn openbib a4paper|a5paper|b5paper|letterpaper|legalpaper|executivepaper openrigth ]
% S.25: { article|report|book|letter ... }
%
% oder koma-skript S.10 + 16
\documentclass[DIV=calc,BCOR=5mm,11pt,headings=small,oneside,abstract=true, toc=bib]{scrartcl}

%%% (1) general configurations %%%
\usepackage[utf8]{inputenc}

%%% (2) language specific configurations %%%
\usepackage[]{a4,ngerman}
\usepackage[english, german, ngerman]{babel}
\selectlanguage{ngerman}

%language specific quoting signs
%default for language emglish is american style of quotes
\usepackage{csquotes}

% jurabib configuration
\usepackage[see]{jurabib}
\bibliographystyle{jurabib}
% Telekom osCompendium German Jurabib Configuration Include Module 
%
% (c) Karsten Reincke, Deutsche Telekom AG, Darmstadt 2011
%
% This LaTeX-File is licensed under the Creative Commons Attribution-ShareAlike
% 3.0 Germany License (http://creativecommons.org/licenses/by-sa/3.0/de/): Feel
% free 'to share (to copy, distribute and transmit)' or 'to remix (to adapt)'
% it, if you '... distribute the resulting work under the same or similar
% license to this one' and if you respect how 'you must attribute the work in
% the manner specified by the author ...':
%
% In an internet based reuse please link the reused parts to www.telekom.com and
% mention the original authors and Deutsche Telekom AG in a suitable manner. In
% a paper-like reuse please insert a short hint to www.telekom.com and to the
% original authors and Deutsche Telekom AG into your preface. For normal
% quotations please use the scientific standard to cite.
%
% [ File structure derived from 'mind your Scholar Research Framework' 
%   mycsrf (c) K. Reincke CC BY 3.0  http://mycsrf.fodina.de/ ]

% the first time cite with all data, later with shorttitle
\jurabibsetup{citefull=first}

%%% (1) author / editor list configuration
%\jurabibsetup{authorformat=and} % uses 'und' instead of 'u.'
% therefore define your own abbreviated conjunction: 
% an 'and before last author explicetly written conjunction

% for authors in citations
\renewcommand*{\jbbtasep}{ u. } % bta = between two authors sep
\renewcommand*{\jbbfsasep}{, } % bfsa = between first and second author sep
\renewcommand*{\jbbstasep}{ u. }% bsta = between second and third author sep
% for editors in citations
\renewcommand*{\jbbtesep}{ u. } % bta = between two authors sep
\renewcommand*{\jbbfsesep}{, } % bfsa = between first and second author sep
\renewcommand*{\jbbstesep}{ u. }% bsta = between second and third author sep

% for authors in literature list
\renewcommand*{\bibbtasep}{ u. } % bta = between two authors sep
\renewcommand*{\bibbfsasep}{, } % bfsa = between first and second author sep
\renewcommand*{\bibbstasep}{ u. }% bsta = between second and third author sep
% for editors  in literature list
\renewcommand*{\bibbtesep}{ u. } % bte = between two editors sep
\renewcommand*{\bibbfsesep}{, } % bfse = between first and second editor sep
\renewcommand*{\bibbstesep}{ u. }% bste = between second and third editor sep

% use: name, forname, forname lastname u. forname lastname
\jurabibsetup{authorformat=firstnotreversed}
\jurabibsetup{authorformat=italic}

%%% (2) title configuration
% in every case print the title, let it be seperated from the 
% author by a colon and use the slanted font
\jurabibsetup{titleformat={all,colonsep}}
%\renewcommand*{\jbtitlefont}{\textit}

%%% (3) seperators in bib data
% separate bibliographical hints and page hints by a comma
\jurabibsetup{commabeforerest}

%%% (4) specific configuration of bibdata in quotes / footnote
% use a.a.O if possible
\jurabibsetup{ibidem=strict}

% replace ugly a.a.O. by ders., a.a.O. resp. ders., ebda.
% but if there are more than one author or girl writers?
\AddTo\bibsgerman{
  \renewcommand*{\ibidemname}{Ds., a.a.O.}
  \renewcommand*{\ibidemmidname}{ds., a.a.O.}
}
\renewcommand*{\samepageibidemname}{Ds., ebda.}
\renewcommand*{\samepageibidemmidname}{ds., ebda.}

%%% (5) specific configuration of bibdata in bibliography
% ever an in: before journal and collection/book-tiltes 
\renewcommand*{\bibbtsep}{in: }
%\renewcommand*{\bibjtsep}{in: }

% ever a colon after author names 
\renewcommand*{\bibansep}{: }
% ever a semi colon after the title 
\renewcommand*{\bibatsep}{; }
% ever a comma before date/year
\renewcommand*{\bibbdsep}{, }

% let jurabib insert the S. and p. information
% no S. necessary in bib-files and in cites/footcites
\jurabibsetup{pages=format}

% use a compressed literature-list using a small line indent
\jurabibsetup{bibformat=compress}
\setlength{\jbbibhang}{1em}

% which follows the design of the cites and offers comments
\jurabibsetup{biblikecite}

% print annotations into bibliography
\jurabibsetup{annote}
\renewcommand*{\jbannoteformat}[1]{{ \itshape #1 }}

%refine the prefix of url download
\AddTo\bibsgerman{\renewcommand*{\urldatecomment}{Referenzdownload: }}

% we want to have the year of articles in brackets
\renewcommand*{\bibaldelim}{(}
\renewcommand*{\bibardelim}{)}

%Umformatierung des Reihentitels und der Reihennummer
\DeclareRobustCommand{\numberandseries}[2]{%
\unskip\unskip%,
\space\bibsnfont{(=~#2}%
\ifthenelse{\equal{#1}{}}{)}{, [Bd./Nr.]~#1)}%
}%

% Local Variables:
% mode: latex
% fill-column: 80
% End:


% language specific hyphenation
% Telekom osCompendium osHyphenation Include Module
%
% (c) Karsten Reincke, Deutsche Telekom AG, Darmstadt 2011
%
% This LaTeX-File is licensed under the Creative Commons Attribution-ShareAlike
% 3.0 Germany License (http://creativecommons.org/licenses/by-sa/3.0/de/): Feel
% free 'to share (to copy, distribute and transmit)' or 'to remix (to adapt)'
% it, if you '... distribute the resulting work under the same or similar
% license to this one' and if you respect how 'you must attribute the work in
% the manner specified by the author ...':
%
% In an internet based reuse please link the reused parts to www.telekom.com and
% mention the original authors and Deutsche Telekom AG in a suitable manner. In
% a paper-like reuse please insert a short hint to www.telekom.com and to the
% original authors and Deutsche Telekom AG into your preface. For normal
% quotations please use the scientific standard to cite.
%
% [ File structure derived from 'mind your Scholar Research Framework' 
%   mycsrf (c) K. Reincke CC BY 3.0  http://mycsrf.fodina.de/ ]
%


\hyphenation{rein-cke}

% Local Variables:
% mode: latex
% fill-column: 80
% End:


%%% (3) layout page configuration %%%

% select the visible parts of a page
% S.31: { plain|empty|headings|myheadings }
%\pagestyle{myheadings}
\pagestyle{headings}

% select the wished style of page-numbering
% S.32: { arabic,roman,Roman,alph,Alph }
\pagenumbering{arabic}
\setcounter{page}{1}

% select the wished distances using the general setlength order:
% S.34 { baselineskip| parskip | parindent }
% - general no indent for paragraphs
\setlength{\parindent}{0pt}
\setlength{\parskip}{1.2ex plus 0.2ex minus 0.2ex}


%%% (4) general package activation %%%
%\usepackage{utopia}
%\usepackage{courier}
%\usepackage{avant}
\usepackage[dvips]{epsfig}

% graphic
\usepackage{graphicx,color}
\usepackage{array}
\usepackage{shadow}
\usepackage{fancybox}

%- start(footnote-configuration)
%  flush the cite numbers out of the vertical line and let
%  the footnote text directly start in the left vertical line
\usepackage[marginal]{footmisc}
%- end(footnote-configuration)


\begin{document}

%% use all entries of the bliography

%%-- start(titlepage)
\titlehead{Literaturexzerpt}
\subject{Autor(en): Niels C. Taubert}
\title{Titel: Produktive Anarchie}
\subtitle{Jahr: 2006 }
\author{K. Reincke% Telekom osCompendium License Include Module
%
% (c) Karsten Reincke, Deutsche Telekom AG, Darmstadt 2011
%
% This LaTeX-File is licensed under the Creative Commons Attribution-ShareAlike
% 3.0 Germany License (http://creativecommons.org/licenses/by-sa/3.0/de/): Feel
% free 'to share (to copy, distribute and transmit)' or 'to remix (to adapt)'
% it, if you '... distribute the resulting work under the same or similar
% license to this one' and if you respect how 'you must attribute the work in
% the manner specified by the author ...':
%
% In an internet based reuse please link the reused parts to www.telekom.com and
% mention the original authors and Deutsche Telekom AG in a suitable manner. In
% a paper-like reuse please insert a short hint to www.telekom.com and to the
% original authors and Deutsche Telekom AG into your preface. For normal
% quotations please use the scientific standard to cite.
%
% [ File structure derived from 'mind your Scholar Research Framework' 
%   mycsrf (c) K. Reincke CC BY 3.0  http://mycsrf.fodina.de/ ]
%
\footnote{
This text is licensed under the Creative Commons Attribution-ShareAlike 3.0 Germany
License (http://creativecommons.org/licenses/by-sa/3.0/de/): Feel free \enquote{to
share (to copy, distribute and transmit)} or \enquote{to remix (to
adapt)} it, if you \enquote{[\ldots] distribute the resulting work under the
same or similar license to this one} and if you respect how \enquote{you
must attribute the work in the manner specified by the author(s)
[\ldots]}):
\newline
In an internet based reuse please mention the initial authors in a suitable
manner, name their sponsor \textit{Deutsche Telekom AG} and link it to
\texttt{http://www.telekom.com}. In a paper-like reuse please insert a short
hint to \texttt{http://www.telekom.com}, to the initial authors, and to their
sponsor \textit{Deutsche Telekom AG} into your preface. For normal citations
please use the scientific standard.
\newline
{ \tiny \itshape [based on myCsrf (= mind your Scholar Research Framework) 
\copyright K. Reincke CC BY 3.0  https://github.com/kreincke/mycsrf/)] }}

% Local Variables:
% mode: latex
% fill-column: 80
% End:
}
%\thanks{den Autoren von KOMA-Script und denen von Jurabib}
\maketitle
%%-- end(titlepage)
%\nocite{*}

\begin{abstract}
\noindent
\cite[(ist:)][]{Taubert2006a} \\
Die Arbeit/The work\footcite[][]{Taubert2006a} \\
\noindent \itshape
\ldots geht als soziologisches Werk der Frage nach, in welchem Sinne die Idee
von 'Open Source' resp. 'Freier Software' auch deren Erarbeitungsstil beeinflusst.
Grob gesagt muss dann Integration heterogener Aspekte eher über Argumentationen
und Kompromisse erfolgen, als durch normative Zielvorgaben.
\\
\noindent
\ldots This sociological book analyzes, in which sense the idea of 'Open Source'
or 'Free Software' determines also the style of a cooperating elaboration.
Roughly said this kind of collaboration must be characterized by argueing and
making compromises than being determined my external decisions.
\end{abstract}
\footnotesize
%\tableofcontents
\normalsize

\section{Allgemeines}

Tauber spricht von der \enquote{sozialen Konstruktion freier Software}, wenn
er die Frage beantworten will, \enquote{wie [\ldots] es den Entwicklern von freier
Software unter den Bedingungen der Abwesenheit von Hierarchie und einer klaren
Rollenzuordnung, einer Projekt-Mailingliste als Medium der Kommunikation sowie
einer hohen Fluktuation der Teilnehmerschaft (gelinge), ihr Handeln auf ein
gemeinsames Entwicklungsziel hin zu
koordinieren}\footcite[cf.][205]{Taubert2006a}. Unter diesen Umständen
sei für das Gelingen die herrschende \enquote{prinzipielle
Begründungspflicht} ein eher \enquote{schwacher Modus} der
Erfolgsabsicherung: endlose Diskussionen und \enquote{hartnäckigen Dissenz}
seien nicht ausgeschlossen. Zur Abfederung und Integration heterogener Ansätze
diene deshalb der \enquote{Einbau von Konfigurationsoptionen} und die
Modularisierung\footcite[cf.][206]{Taubert2006a}:

\begin{quote}
\enquote{Die Strukturierung des Programms in Module - als Eigenschaft des
Artefakts - resultiert im Fall von freier Software nicht nur aus der Komplexität
der zu lösenden Entwicklungsaufgabe, sondern auch aus den
Integrationserfordernissen, die sich aufgrund der motivationalen Gründe für eine
Beteiligung auf Seiten der Entwickler und der organisatorischen
Rahmenbedingungen im Fall von offenen Programmierprojekten
stellen.}\footcite[][206]{Taubert2006a}
\end{quote}

\section{Spezifisches}
\subsection{Zur Definition von 'Open Source Software' und 'Freier Software'}
Die konzeptuelle Klärung rekuriert hier zunächst auf den Begriff der
\enquote{freien Software}, durchaus im Sinne des GNU Projektes und seinem
Schwerpunkt der 'Freiheit'\footcite[cf.][17]{Taubert2006a}.  Dem gegenüber
gestellt wird der später entstandene Begriff 'Open Source' mit seinem
Schwerpunkt der \enquote{Zugänglichkeit des
Quellcodes}\footcite[cf.][18]{Taubert2006a}.

Allerdings, so Taubert, \enquote{[\ldots] (ließen sich) gegen sämtliche
Bezeichnungen Ein\-wände vorbringen
[\ldots]}\footcite[cf.][19]{Taubert2006a}: Bei der 'Freien Software'
unterstütze die vordergründige Assoziation 'kostenlos' die intendierte Bedeutung
der \enquote{emanzipatorischen Bewegung} nicht, zumal nicht jedes kostenlose
Programm zugleich auch 'freie Software' sei\footcite[cf.][19]{Taubert2006a}.
Und umgekehrt sei mittlerweile auch nicht mehr all das, dessen Quellcode
offengelegt werde, zugleich auch 'Open Source' im Sinne dessen, was die OSI
definiert habe: so könnte Code offengelegt werden, ohne dass das Recht der
Modifikation mit dem Recht der Einsicht verbunden werde. Das Wort 'Open Source'
werde so zum \enquote{Einfallstor} von missbräuchlichen
\enquote{Marketing-Initiativen}\footcite[cf.][19f]{Taubert2006a}.

\subsection{Zur Klassifikation von 'Software' versus 'Open Source Software'}

Bei der Typisierung von Software entscheidet Taubert formal 5 Formen, die
\enquote{proprietäre (zum Normalgebrauch)}, die
\enquote{proprietäre (Händerlizenz)}, \enquote{Shareware}, \enquote{Public
Domain} und \enquote{Freie Software}. Inhaltlich markiert er diese
Typen über die Eigenschaften \enquote{Anwendung},
\enquote{Vervielfältigung}, \enquote{Verbreitung} und
\enquote{Modifkation}. Überraschenderweise werden bei ihm sodann Public
Domain Software und Freie Software mit derselben Merkmalkombination spezifziert
(4 * erlaubt), was ihn zu dem Fazit führt, dass der \enquote{[\ldots] Typus Public
Domain-Software eine Unterkategorie von freier Software
(bilde)}\footcite[cf.][25]{Taubert2006a}.

Bedauerlicherweise erläutert Taubert nicht, wieso Public Domain-Software eine
Unterkategorie von freier Software sei und nicht umgekehrt. Aus den Kriterien
ist dies ebenso wenig abzuleiten, wie der Unterschied zwischen den beiden
Typen ansich. Dies deutet an, dass hier eine klassifikatorische Schwäche
vorliegt. So wird denn auch üblicherweise Public Domain-Software eher in die
Nähe von Shareware gerückt, als in die Nähe von Open Source Software [wird es
das wirklich?]

\small
\bibliography{../bibfiles/oscResourcesDe}

\end{document}
