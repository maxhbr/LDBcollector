% Telekom osCompendium cloak file English text
%
% (c) Karsten Reincke, Deutsche Telekom AG, Darmstadt 2011
%
% This LaTeX-File is licensed under the Creative Commons Attribution-ShareAlike
% 3.0 Germany License (http://creativecommons.org/licenses/by-sa/3.0/de/): Feel
% free 'to share (to copy, distribute and transmit)' or 'to remix (to adapt)'
% it, if you '... distribute the resulting work under the same or similar
% license to this one' and if you respect how 'you must attribute the work in
% the manner specified by the author ...':
%
% In an internet based reuse please link the reused parts to www.telekom.com and
% mention the original authors and Deutsche Telekom AG in a suitable manner. In
% a paper-like reuse please insert a short hint to www.telekom.com and to the
% original authors and Deutsche Telekom AG into your preface. For normal
% quotations please use the scientific standard to cite.
%
% [ File structure derived from 'mind your Scholar Research Framework' 
%   mycsrf (c) K. Reincke CC BY 3.0  http://mycsrf.fodina.de/ ]

%\documentclass[DIV=calc,BCOR=5mm,12pt,headings=small,oneside,toc=bib,draft]{scrbook}
\documentclass[DIV=calc,BCOR=5mm,12pt,headings=small,oneside,toc=bib]{scrbook}

%%% (1) general configurations %%%
\usepackage[utf8]{inputenc}

%%% (2) language specific configurations %%%
\usepackage[]{a4}
\usepackage[english]{babel}
\selectlanguage{english}

\usepackage{microtype}

\usepackage{to_oscad/oscad}

%language specific quoting signs
%default for language emglish is american style of quotes
%\usepackage[english=british]{csquotes}
\usepackage[english=american]{csquotes}

% jurabib configuration
\usepackage[see]{jurabib}
\bibliographystyle{jurabib}
% do not comment the litrature any longer
% Telekom osCompendium English Jurabib Configuration Include Module 
%
% (c) Karsten Reincke, Deutsche Telekom AG, Darmstadt 2011
%
% This LaTeX-File is licensed under the Creative Commons Attribution-ShareAlike
% 3.0 Germany License (http://creativecommons.org/licenses/by-sa/3.0/de/): Feel
% free 'to share (to copy, distribute and transmit)' or 'to remix (to adapt)'
% it, if you '... distribute the resulting work under the same or similar
% license to this one' and if you respect how 'you must attribute the work in
% the manner specified by the author ...':
%
% In an internet based reuse please link the reused parts to www.telekom.com and
% mention the original authors and Deutsche Telekom AG in a suitable manner. In
% a paper-like reuse please insert a short hint to www.telekom.com and to the
% original authors and Deutsche Telekom AG into your preface. For normal
% quotations please use the scientific standard to cite.
%
% [ File structure derived from 'mind your Scholar Research Framework' 
%   mycsrf (c) K. Reincke CC BY 3.0  http://mycsrf.fodina.de/ ]

% the first time cite with all data, later with shorttitle
\jurabibsetup{citefull=first}

%%% (1) author / editor list configuration
%\jurabibsetup{authorformat=and} % uses 'und' instead of 'u.'
% therefore define your own abbreviated conjunction: 
% an 'and before last author explicetly written conjunction

% for authors in citations
\renewcommand*{\jbbtasep}{ a.\ } % bta = between two authors sep
\renewcommand*{\jbbfsasep}{, } % bfsa = between first and second author sep
\renewcommand*{\jbbstasep}{, a.\ }% bsta = between second and third author sep
% for editors in citations
\renewcommand*{\jbbtesep}{ a.\ } % bta = between two authors sep
\renewcommand*{\jbbfsesep}{, } % bfsa = between first and second author sep
\renewcommand*{\jbbstesep}{, a.\ }% bsta = between second and third author sep

% for authors in literature list
\renewcommand*{\bibbtasep}{ a.\ } % bta = between two authors sep
\renewcommand*{\bibbfsasep}{, } % bfsa = between first and second author sep
\renewcommand*{\bibbstasep}{, a.\ }% bsta = between second and third author sep
% for editors  in literature list
\renewcommand*{\bibbtesep}{ a.\ } % bte = between two editors sep
\renewcommand*{\bibbfsesep}{, } % bfse = between first and second editor sep
\renewcommand*{\bibbstesep}{, a.\ }% bste = between second and third editor sep

% use: name, forname, forname lastname u. forname lastname
\jurabibsetup{authorformat=firstnotreversed}
\jurabibsetup{authorformat=italic}

%%% (2) title configuration
% in every case print the title, let it be seperated from the 
% author by a colon and use the slanted font
\jurabibsetup{titleformat={all,colonsep}}
%\renewcommand*{\jbtitlefont}{\textit}

%%% (3) seperators in bib data
% separate bibliographical hints and page hints by a comma
\jurabibsetup{commabeforerest}

%%% (4) specific configuration of bibdata in quotes / footnote
% use a.a.O if possible
\jurabibsetup{ibidem=strict}
% replace ugly a.a.O. by translation of ders., a.a.O.
\AddTo\bibsgerman{
  \renewcommand*{\ibidemname}{Id., l.c.}
  \renewcommand*{\ibidemmidname}{id., l.c.}
}
\renewcommand*{\samepageibidemname}{Id., ibid.}
\renewcommand*{\samepageibidemmidname}{id., ibid.}


%%% (5) specific configuration of bibdata in bibliography
% ever an in: before journal and collection/book-tiltes 
\renewcommand*{\bibbtsep}{in: }
\renewcommand*{\bibjtsep}{in: }
% ever a colon after author names 
\renewcommand*{\bibansep}{: }
% ever a semi colon after the title
% \AddTo\bibsgerman{\renewcommand*{\urldatecomment}{Referenzdownload: }}
\renewcommand*{\bibatsep}{; }
% ever a comma before date/year
\renewcommand*{\bibbdsep}{, }

% let jurabib insert the S. and p. information
% no S. necessary in bib-files and in cites/footcites
\jurabibsetup{pages=format}

% use a compressed literature-list using a small line indent
\jurabibsetup{bibformat=compress}
\setlength{\jbbibhang}{1em}

% which follows the design of the cites and offers comments
\jurabibsetup{biblikecite}

% configuruation for not commenting the literature
% print annotations into bibliography
% \jurabibsetup{annote}
% \renewcommand*{\jbannoteformat}[1]{{ \itshape #1 }}

%refine the prefix of url download
\AddTo\bibsgerman{\renewcommand*{\urldatecomment}{reference download: }}

% we want to have the year of articles in brackets
\renewcommand*{\bibaldelim}{(}
\renewcommand*{\bibardelim}{)}

% in english version Nr. must be replaced by No.
\renewcommand*{\artnumberformat}[1]{\unskip,\space No.~#1}
\renewcommand*{\pernumberformat}[1]{\unskip\space No.~#1}%
\renewcommand*{\revnumberformat}[1]{\unskip\space No.~#1}%


%Reformatierung Seriestitels and Seriesnumber
\DeclareRobustCommand{\numberandseries}[2]{%
\unskip\unskip%,
\space\bibsnfont{(=~#2}%
\ifthenelse{\equal{#1}{}}{)}{, [Vol./No.]~#1)}%
}%


% Local Variables:
% mode: latex
% fill-column: 80
% End:
 %no annotations

% language specific hyphenation
% Telekom osCompendium osHyphenation Include Module
%
% (c) Karsten Reincke, Deutsche Telekom AG, Darmstadt 2011
%
% This LaTeX-File is licensed under the Creative Commons Attribution-ShareAlike
% 3.0 Germany License (http://creativecommons.org/licenses/by-sa/3.0/de/): Feel
% free 'to share (to copy, distribute and transmit)' or 'to remix (to adapt)'
% it, if you '... distribute the resulting work under the same or similar
% license to this one' and if you respect how 'you must attribute the work in
% the manner specified by the author ...':
%
% In an internet based reuse please link the reused parts to www.telekom.com and
% mention the original authors and Deutsche Telekom AG in a suitable manner. In
% a paper-like reuse please insert a short hint to www.telekom.com and to the
% original authors and Deutsche Telekom AG into your preface. For normal
% quotations please use the scientific standard to cite.
%
% [ File structure derived from 'mind your Scholar Research Framework' 
%   mycsrf (c) K. Reincke CC BY 3.0  http://mycsrf.fodina.de/ ]
%


\hyphenation{rein-cke}
\hyphenation{Rein-cke}
\hyphenation{OS-LiC}
\hyphenation{ori-gi-nal}
\hyphenation{bi-na-ry}
\hyphenation{Li-cence}
\hyphenation{li-cence}

% Local Variables:
% mode: latex
% fill-column: 80
% End:


%%% (3) layout page configuration %%%

% select the visible parts of a page
% S.31: { plain|empty|headings|myheadings }
%\pagestyle{myheadings}
\pagestyle{headings}

% select the wished style of page-numbering
% S.32: { arabic,roman,Roman,alph,Alph }
\pagenumbering{arabic}
\setcounter{page}{1}

% select the wished distances using the general setlength order:
% S.34 { baselineskip| parskip | parindent }
% - general no indent for paragraphs
\setlength{\parindent}{0pt}
\setlength{\parskip}{1.2ex plus 0.2ex minus 0.2ex}

%%% (4) general package activation %%%
%\usepackage{utopia}
%\usepackage{courier}
%\usepackage{avant}

% graphic
\usepackage{graphicx,color}
\usepackage{array}
\usepackage{shadow}
\usepackage{fancybox}
\usepackage{alltt}

%- start(footnote-configuration)
%  flush the cite numbers out of the vertical line and let
%  the footnote text directly start in the left vertical line
% \usepackage[marginal,hang]{footmisc}
% \renewcommand\footnotemargin{1.5em}

% formatting the footnote with koma script tools
% \deffootnote[1em]{1.5em}{1em}{\textsuperscript{\thefootnotemark}}
\deffootnote[1.5em]{1.5em}{1.5em}{\textsuperscript{\thefootnotemark)\ }}


%\deffootnote[0em]{1.5em}{1em}{\textsuperscript{\thefootnotemark}}
%- end(footnote-configuration)

% %- start(endnote-configuration) uncomment to activate
% % Let all notes being marked with \endnote instead of \footnote
% % become endnotes. This set of endnotes replaces the next 
% % arising command \theendnotes - even if it is not located
% % at the end of the text.
% 
% \usepackage{endnotes}
% 
% % Format endnotes as Block with indention - Solution 1
% %\renewcommand\enoteformat{%
% %   \noindent\theenmark.) \ \hangindent .7\parindent%
% %}
% 
% % Format endnotes as Block with indention - Solution 2
% \makeatletter
% \def\enoteformat{\rightskip\z@ \leftskip0em \parindent=0em \parskip=0em
% \leavevmode\llap{\hbox{\@theenmark.~}}}
% \makeatother
% 
% \renewcommand\notesname{Annotations}
% % additionally we shall active a special jurabib option
% % if we want to get all jurabib footnotes as endnotes
% \jurabibsetup{citetoend=true}
% %- end(footnote-configuration)

% - additional packages

\usepackage{tikz}
\usetikzlibrary{arrows}
\usetikzlibrary{shapes,snakes}
\usetikzlibrary{positioning}
\usetikzlibrary{decorations.text}
\usetikzlibrary{trees}

\usepackage{multirow}

%RPD%%%\usepackage{blindtext}
\usepackage{caption}

\usetikzlibrary{matrix}

\usepackage{amsmath}
\usepackage{amsfonts}
\usepackage{amssymb}
\usepackage{wasysym}
\usepackage{chngcntr}
\usepackage{nameref}


\counterwithout{footnote}{chapter}

\usepackage[intoc]{nomencl}
\let\abbr\nomenclature
% Modify Section Title of nomenclature
\renewcommand{\nomname}{Periodicals, Shortcuts, and Abbreviations}
%\renewcommand{\nomname}{Periodika, ihre Kurzformen und generelle Abkürzungen}

% insert point between abbrewviation and explanation
\setlength{\nomlabelwidth}{.24\hsize}
\renewcommand{\nomlabel}[1]{#1 \dotfill}
% reduce the line distance
\setlength{\nomitemsep}{-\parsep}
\makenomenclature

% depth of contents
\setcounter{secnumdepth}{5}
\setcounter{tocdepth}{5}

% Hyperlinks
\usepackage{hyperref}
\hypersetup{bookmarks=true,breaklinks=true,colorlinks=true,citecolor=blue,draft=false}

% Compatibility command if hyperref cannot be used
%\newcommand{\texorpdfstring}[2]{#1}

% Abbreviations
\newcommand{\oslic}{OSLiC}

% Often Cited Sources
% --------------------
% first (optional) argument is text at beginning, defaults to "cf."
% second argument is location, like "§2"; must be "wp" if no paragraph is given 
\newcommand*{\citeAGPL}[2][cf.]{\footcite[#1][\nopage wp #2]{Agpl30OsiLicense2007a}}
\newcommand*{\citeAPL}[2][cf.]{\footcite[#1][\nopage wp #2]{Apl20OsiLicense2004a}}
\newcommand*{\citeBSDnew}[2][cf.]{\footcite[#1][\nopage wp #2]{BsdLicense3Clause}}
\newcommand*{\citeBSDsimple}[2][cf.]{\footcite[#1][\nopage wp #2]{BsdLicense2Clause}}
\newcommand*{\citeCDDL}[2][cf.]{\footcite[#1][\nopage wp #2]{Cddl10OsiLicense2004a}}
\newcommand*{\citeEPL}[2][cf.]{\footcite[#1][\nopage wp #2]{Epl10OsiLicense2005a}}
\newcommand*{\citeEUPL}[2][cf.]{\footcite[#1][\nopage wp #2]{Eupl11OsiLicense2007a}}
\newcommand*{\citeGPLthree}[2][cf.]{\footcite[#1][\nopage wp #2]{Gpl30OsiLicense2007a}}
\newcommand*{\citeGPLtwo}[2][cf.]{\footcite[#1][\nopage wp #2]{Gpl20OsiLicense1991a}}
\newcommand*{\citeLGPLthree}[2][cf.]{\footcite[#1][\nopage wp #2]{Lgpl30OsiLicense2007a}}
\newcommand*{\citeLGPLtwo}[2][cf.]{\footcite[#1][\nopage wp #2]{Lgpl21OsiLicense1999a}}
\newcommand*{\citeMIT}[2][cf.]{\footcite[#1][\nopage wp #2]{MitLicense2012a}}
\newcommand*{\citeMPL}[2][cf.]{\footcite[#1][\nopage wp #2]{Mpl20OsiLicense2013a}}
\newcommand*{\citeMSPL}[2][cf.]{\footcite[#1][\nopage wp #2]{MsplOsiLicense2013a}}
\newcommand*{\citePGL}[2][cf.]{\footcite[#1][\nopage wp #2]{PglOsiLicense2013a}}
\newcommand*{\citePHP}[2][cf.]{\footcite[#1][\nopage wp #2]{Php30OsiLicense2013a}}
%%%%%%%%%%%%%%

\begin{document}

%% use all entries of the bliography
\nocite{*}

%%-- start(titlepage)
\titlehead{Version 1.0.3
}

\subject{\small \itshape A Practical Guide for Developers, Managers, OS Experts, 
and Companies} 

\title{Open Source License Compendium}

\subtitle{How to Achieve Open Source License Compliance% Telekom osCompendium License Include Module
%
% (c) Karsten Reincke, Deutsche Telekom AG, Darmstadt 2011
%
% This LaTeX-File is licensed under the Creative Commons Attribution-ShareAlike
% 3.0 Germany License (http://creativecommons.org/licenses/by-sa/3.0/de/): Feel
% free 'to share (to copy, distribute and transmit)' or 'to remix (to adapt)'
% it, if you '... distribute the resulting work under the same or similar
% license to this one' and if you respect how 'you must attribute the work in
% the manner specified by the author ...':
%
% In an internet based reuse please link the reused parts to www.telekom.com and
% mention the original authors and Deutsche Telekom AG in a suitable manner. In
% a paper-like reuse please insert a short hint to www.telekom.com and to the
% original authors and Deutsche Telekom AG into your preface. For normal
% quotations please use the scientific standard to cite.
%
% [ File structure derived from 'mind your Scholar Research Framework' 
%   mycsrf (c) K. Reincke CC BY 3.0  http://mycsrf.fodina.de/ ]
%
\footnote{
This text is licensed under the Creative Commons Attribution-ShareAlike 3.0 Germany
License (http://creativecommons.org/licenses/by-sa/3.0/de/): Feel free \enquote{to
share (to copy, distribute and transmit)} or \enquote{to remix (to
adapt)} it, if you \enquote{[\ldots] distribute the resulting work under the
same or similar license to this one} and if you respect how \enquote{you
must attribute the work in the manner specified by the author(s)
[\ldots]}):
\newline
In an internet based reuse please mention the initial authors in a suitable
manner, name their sponsor \textit{Deutsche Telekom AG} and link it to
\texttt{http://www.telekom.com}. In a paper-like reuse please insert a short
hint to \texttt{http://www.telekom.com}, to the initial authors, and to their
sponsor \textit{Deutsche Telekom AG} into your preface. For normal citations
please use the scientific standard.
\newline
{ \tiny \itshape [based on myCsrf (= mind your Scholar Research Framework) 
\copyright K. Reincke CC BY 3.0  https://github.com/kreincke/mycsrf/)] }}

% Local Variables:
% mode: latex
% fill-column: 80
% End:
}
\author{
Karsten Reincke\thanks{Deutsche Telekom AG, Products \& Innovation, 
T-Online-Allee 1, 64295 Darmstadt}
\and
Greg Sharpe\thanks{Deutsche Telekom AG, Telekom Deutschland GmbH, 
Landgrabenweg, Bonn}
}
\date{2018-06-14}
\maketitle
%%-- end(titlepage)

\footnotesize
\begin{flushright} 

\parbox{100mm}{\itshape
The Open Source Community is a swarm: it is more powerful than a set of
arbritarily selected experts. We are proud to have its support. Gladly we thank
(in alphabetical order):
}

\parbox{80mm}{
\tiny
\begin{flushright}
Eitan Adler,\\
Stefan Altmeyer (Deutsche Telekom AG),\\
Ronald Dauster,\\
John Dobson, \\
Steffen Härtlein, \\
Ta'Id Holmes (Deutsche Telekom AG), \\
Michael Kern (Deutsche Telekom AG),\\
Michael Machado (Deutsche Telekom AG),\\
Thorsten Müller (Deutsche Telekom AG),\\
Tanja Neske (Deutsche Telekom AG),\\
Oliver Podebradt (Deutsche Telekom AG),\\
Thomas Quiehl (Deutsche Telekom AG),\\
Peter Schichl (Deutsche Telekom AG),\\
Michael Schierl,\\
Helene Tamer (T-Systems Internationl AG),\\
Bernhard Tsai (Deutsche Telekom AG),\\
Thomas Weißschuh (Amadeus Germany GmbH),\\
\ldots additionally all the feedback giving participants of the 
European Legal \& Licensing Workshop 2013 in Amstardam\\
and all the others\ldots
\end{flushright}
}
\end{flushright}
\normalsize
\newpage

\footnotesize
\tableofcontents
\newpage

% Telekom osCompendium 'for being included' snippet template
%
% (c) Karsten Reincke, Deutsche Telekom AG, Darmstadt 2011
%
% This LaTeX-File is licensed under the Creative Commons Attribution-ShareAlike
% 3.0 Germany License (http://creativecommons.org/licenses/by-sa/3.0/de/): Feel
% free 'to share (to copy, distribute and transmit)' or 'to remix (to adapt)'
% it, if you '... distribute the resulting work under the same or similar
% license to this one' and if you respect how 'you must attribute the work in
% the manner specified by the author ...':
%
% In an internet based reuse please link the reused parts to www.telekom.com and
% mention the original authors and Deutsche Telekom AG in a suitable manner. In
% a paper-like reuse please insert a short hint to www.telekom.com and to the
% original authors and Deutsche Telekom AG into your preface. For normal
% quotations please use the scientific standard to cite.
%
% [ File structure derived from 'mind your Scholar Research Framework' 
%   mycsrf (c) K. Reincke CC BY 3.0  http://mycsrf.fodina.de/ ]


\chapter*{Backlog} 

\begin{footnotesize}
\begin{itemize}
  \item Insert task lists for AL, AFL, CDDL, MPL-1.[0|1], MS-RL, OSL
  \item Complete the concept of being a derivative work in the context of
  software development
  \item Explain how to deal with modifications transforming a proapse into a
  snimoli and v.v.
  \item Discuss license compatibility
  \item Explain the relationship between open source and earning money
  \item Enrich the literature list
\end{itemize}
\end{footnotesize}

%\bibliography{../bibfiles/oscResourcesEn}

% Local Variables:
% mode: latex
% fill-column: 80
% End:

% Telekom osCompendium 'for being included' snippet template
%
% (c) Karsten Reincke, Deutsche Telekom AG, Darmstadt 2011
%
% This LaTeX-File is licensed under the Creative Commons Attribution-ShareAlike
% 3.0 Germany License (http://creativecommons.org/licenses/by-sa/3.0/de/): Feel
% free 'to share (to copy, distribute and transmit)' or 'to remix (to adapt)'
% it, if you '... distribute the resulting work under the same or similar
% license to this one' and if you respect how 'you must attribute the work in
% the manner specified by the author ...':
%
% In an internet based reuse please link the reused parts to www.telekom.com and
% mention the original authors and Deutsche Telekom AG in a suitable manner. In
% a paper-like reuse please insert a short hint to www.telekom.com and to the
% original authors and Deutsche Telekom AG into your preface. For normal
% quotations please use the scientific standard to cite.
%
% [ File structure derived from 'mind your Scholar Research Framework' 
%   mycsrf (c) K. Reincke CC BY 3.0  http://mycsrf.fodina.de/ ]

%


%% use all entries of the bibliography

%\chapter*{History}

\begin{table}
\footnotesize
\caption{History of the Open Source License Compendium}
\begin{center}
\begin{tabular}{|r|c|p{9.4cm}|}
\hline
    \texttt{2015-03-01}
  & \texttt{1.0.0}
  & Target Release\newline
  $\vartriangleright$ Form expanded by a 6th AGPL relevant question\newline
  $\vartriangleright$ Expanded OSUC-03 by AGPL subtypes L(ocal) \& I(nternet) \newline 
  $\vartriangleright$ Added AGPL specific finder and license fulfilling to-do lists\\
\hline
    \texttt{2015-01-21}
  & \texttt{0.99.9}
  & $\vartriangleright$ added solution for the reverse engineering challenge \\
\hline
    \texttt{2014-03-09}
  & \texttt{0.99.1}
  & $\vartriangleright$ Generate data file for use in OSCAd from the \LaTeX source\newline
  $\vartriangleright$ Fixed Bug in LGPL C9 Case \newline
  $\vartriangleright$ general copy-editing of chapter 6\\
\hline
    \texttt{2014-01-08}
  & \texttt{0.98.2}
  & $\vartriangleright$ New section about the patent clauses in the CDDL\newline
  $\vartriangleright$ hyperlinked PDF file (using hyperref and pdftex)\newline
  $\vartriangleright$ general copy-editing of chapter 1 to 5\\
\hline
    \texttt{2013-11-27}
  & \texttt{0.98.1}
  & Korean FLOSS conference release\\
\hline
    \texttt{2013-08-19}
  & \texttt{0.97.2}
  & $\vartriangleright$ incorpation of the typo fixes offered by M.Schierl\newline
  $\vartriangleright$ some improvements concerning the derivative work\newline
  $\vartriangleright$ enhancing that the OSLiC deals with prototypic cases\\
\hline
    \texttt{2013-07-28}
  & \texttt{0.97.1} 
  & $\vartriangleright$ indirectly used secondary literature added\newline
    $\vartriangleright$ LGPL specific finder improved\newline
    $\vartriangleright$ OSCAd aligned, interface improved\\
\hline
    \texttt{2013-05-20}
  & \texttt{0.96.1} 
  & Linux Days release\newline    
    $\vartriangleright$ open source use cases and licenses specific usecase renamed\newline
    $\vartriangleright$ version matches the content of OSCAd\\
\hline
    \texttt{2013-04-15}
  & \texttt{0.95.2} 
  & FSFE LLW post release\newline
    $\vartriangleright$ to-do lists for nearly all popular OSI licenses\newline
    $\vartriangleright$ improved finder for GPL and EUPL\newline
    $\vartriangleright$ simplified form and improved structure of the OSLiC finder\\
\hline
    \texttt{2013-04-05}
  & \texttt{0.95.1} 
  & FSFE LLW pre release\newline
    $\vartriangleright$ to-do lists for all permissive and all weak copyleft licenses\\
\hline
    \texttt{2013-03-15}
  & \texttt{0.94.1} 
  & Chemnitzer Linux Day release\newline
    $\vartriangleright$ to-do lists for all permissive and some weak copyleft licenses\newline
    $\vartriangleright$ branches merged and new master published\\
\hline
    \texttt{2013-03-08}
  & \texttt{0.90.1} 
  & CeBIT release\newline
    $\vartriangleright$ to-do lists for the some important licenses added\\
\hline
    \texttt{2013-02-16}
  & \texttt{0.8.90} 
  & $\vartriangleright$ new arguing structure focused on the topic license fulfillment\newline
    $\vartriangleright$ new classifying license review\newline   
    $\vartriangleright$ new top down introduction\\
\hline
    \texttt{2012-12-28}
  & \texttt{0.8.0} 
  & internal EOY release\newline
    $\vartriangleright$ many distributed improvements unified in branch kreinck\\
\hline
    \texttt{2012-08-25}
  & \texttt{0.5.2} 
  & $\vartriangleright$ MIT license fulfilling to-do lists\newline
    $\vartriangleright$ using integrated Eclipse spell checking methods\\
\hline
    \texttt{2012-07-06}
  & \texttt{0.4.0} 
  & break through release\newline
    $\vartriangleright$ open source use case definition and taxonomy\newline 
    $\vartriangleright$ open source use case based general finder\newline 
    $\vartriangleright$ BSD specific mini finder \& BSD fulfilling to-do lists\\
\hline
    \texttt{2012-03-22}
  & \texttt{0.2.1} 
  & $\vartriangleright$ framework published as first community edition\\
\hline
    \texttt{2012-01-31}
  & \texttt{0.1.8} 
  & $\vartriangleright$ renamed existing introduction as prolegomena\newline
    $\vartriangleright$ inserted a shorter top-down written introduction\newline
    $\vartriangleright$ added an OSLiC disclaimer \& many bibliographic data\\
\hline
    \texttt{2011-09-29}
  & \texttt{0.1.4} 
  & $\vartriangleright$ document history integrated\\
\hline
    \texttt{2011-09-12}
  & \texttt{0.1.0} 
  & $\vartriangleright$ introduction completed: purpose and methods \\
\hline
\hline 
\end{tabular}
\end{center}
\end{table}

%\bibliography{../bibfiles/oscResourcesEn}

% Local Variables:
% mode: latex
% fill-column: 80
% End:


\normalsize

\chapter*{Disclaimer}
% Telekom osCompendium 'for beeing included' snippet template
%
% (c) Karsten Reincke, Deutsche Telekom AG, Darmstadt 2011
%
% This LaTeX-File is licensed under the Creative Commons Attribution-ShareAlike
% 3.0 Germany License (http://creativecommons.org/licenses/by-sa/3.0/de/): Feel
% free 'to share (to copy, distribute and transmit)' or 'to remix (to adapt)'
% it, if you '... distribute the resulting work under the same or similar
% license to this one' and if you respect how 'you must attribute the work in
% the manner specified by the author ...':
%
% In an internet based reuse please link the reused parts to www.telekom.com and
% mention the original authors and Deutsche Telekom AG in a suitable manner. In
% a paper-like reuse please insert a short hint to www.telekom.com and to the
% original authors and Deutsche Telekom AG into your preface. For normal
% quotations please use the scientific standard to cite.
%
% [ File structure derived from 'mind your Scholar Research Framework' 
%   mycsrf (c) K. Reincke CC BY 3.0  http://mycsrf.fodina.de/ ]

%

%%% \chapter*{Disclaimer} %%%

This book shall be thoroughly developed---together with the open source
community. At the end it shall deliver reliable information. But nevertheless,
the \oslic{} can not offer more than the opinion(s) of its authors and
contributors. It is only one voice of the chorus discussing the open source
licenses. For protecting the authors and contributors from charges and claims of
indemnification we adopt the lightly modified GPL3 disclaimer:

{
\newcommand\gummi{\hspace{0pt plus 1pc}}
THERE IS NO WARRANTY FOR THE \oslic{}, TO THE EXTENT PERMITTED BY APPLICABLE LAW.
THE COPYRIGHT HOLDERS AND/OR OTHER PARTIES PROVIDE THE TEXT “AS IS” WITHOUT
WARRANTY OF ANY KIND, EITHER EXPRESSED OR IMPLIED, INCLUDING, BUT NOT LIMITED
TO, THE IMPLIED WARRANTIES OF MERCHANTABILITY AND FITNESS FOR A PARTICULAR
PURPOSE. THE ENTIRE RISK AS TO THE QUALITY AND PERFORMANCE OF THE \oslic{} IS
WITH YOU. \gummi SHOULD THE \oslic{} PROVE DEFECTIVE, \gummi YOU ASSUME THE COST OF ALL
NECESSARY SERVICING, REPAIR OR CORRECTION.

IN NO EVENT \gummi UNLESS REQUIRED BY APPLICABLE LAW \gummi OR AGREED TO IN WRITING WILL ANY
COPYRIGHT HOLDER, OR ANY OTHER PARTY WHO MODIFIES AND/OR CONVEYS THE \oslic{} AS
PERMITTED ABOVE, BE LIABLE TO YOU FOR DAMAGES, INCLUDING ANY GENERAL, \gummi SPECIAL, \gummi
INCIDENTAL \gummi OR CONSEQUENTIAL DAMAGES ARISING OUT OF THE USE OR INABILITY TO USE
THE \oslic{} (INCLUDING BUT NOT LIMITED TO LOSS OF DATA OR DATA BEING RENDERED
INACCURATE OR LOSSES SUSTAINED BY YOU OR THIRD PARTIES OR A FAILURE OF THE
\oslic{} TO COOPERATE WITH ANY OTHER TOOLS), EVEN IF SUCH HOLDER OR OTHER PARTY
HAS BEEN ADVISED OF THE POSSIBILITY OF SUCH DAMAGES.
}

\textit{Particularly, it must be highlighted that - referred to your solitary
case - the \oslic{} can not and shall not replace a legal review or a legal advice
by lawyers. The \oslic{} is only dealing with prototypic use cases. So, it may
inspire developers, managers, open source experts, and companies to find good
solutions which they finally should let be reviewed by legal
counselors.}\footnote{For German readers: The \oslic{} naturally respects the German
'Rechtsdienstleistungsgesetz'. It only contains legal reflections addressed to a
general public. The \oslic{} may only be read as an \enquote{nur an die
Allgemeinheit gerichtete Darstellung und Erörterung von Rechtsfragen.}}

% Local Variables:
% mode: latex
% fill-column: 80
% End:


%%%%%%%%%%%%
% Telekom osCompendium 'for being included' snippet template
%
% (c) Karsten Reincke, Deutsche Telekom AG, Darmstadt 2011
%
% This LaTeX-File is licensed under the Creative Commons Attribution-ShareAlike
% 3.0 Germany License (http://creativecommons.org/licenses/by-sa/3.0/de/): Feel
% free 'to share (to copy, distribute and transmit)' or 'to remix (to adapt)'
% it, if you '... distribute the resulting work under the same or similar
% license to this one' and if you respect how 'you must attribute the work in
% the manner specified by the author ...':
%
% In an internet based reuse please link the reused parts to www.telekom.com and
% mention the original authors and Deutsche Telekom AG in a suitable manner. In
% a paper-like reuse please insert a short hint to www.telekom.com and to the
% original authors and Deutsche Telekom AG into your preface. For normal
% quotations please use the scientific standard to cite.
%
% [ Framework derived from 'mind your Scholar Research Framework' 
%   mycsrf (c) K. Reincke 2012 CC BY 3.0  http://mycsrf.fodina.de/ ]
%


%% use all entries of the bibliography
%\nocite{*}


\chapter{Introduction}

% Abstract
\footnotesize \begin{quote}\itshape
This chapter briefly describes the idea behind the \oslic, the way it should be
used and the way it can be read---which is indeed not quite the same.
\end{quote}
\normalsize{}

% Content
This book focuses on just one issue: \emph{What needs to be done in order to act
in accordance with the licenses of those \emph{open source software} we use?}
The \emph{Open Source License Compendium} aims at reliably answering this
question---in a simple and easy to understand manner. However, it is not just
another book on \emph{open source} in ge\-ne\-ral.\footnote{Meanwhile, there are
tons of literature dealing with open source. Trying to expand your knowledge by
means of books and articles might let you get lost in literature: our list of
secondary literature may adumbrate this `danger of being overwhelmed'. But
nevertheless, our bibliography at the end of the \oslic{} is not complete.
Moreover, it is not intended to be complete. It is only an extract representing
the background information we did not directly cite in the \oslic. If we were
forced to indicate two books for attaining a good overview on the topic of
\emph{open source (licenses)} we would name (a) the `Rebel Code' (\cite[for a
German version cf.][\nopage passim]{Moody2001a}---\cite[for an English version
cf.][passim]{Moody2002a}) and (b) the `legal basic conditions'
(\cite[cf.][\nopage passim]{JaeMet2011a}). But fortunately, we are not forced to
do so.} The intention is, rather, for it to be a tool for simplifying the
activities for achieving license conformity.


This compendium was created out of necessity at \emph{Deutsche Telekom AG} to
counter a challenge some of its software developers and project managers were
facing: Of course, the company itself wants to behave as license compliantly as
its employees, but, unfortunately, they could not find a reference text which
simply lists what precisely must be done in order to comply with the license of
that piece of open source being used.

As some of these co-workers in Telekom projects, even we---the initial authors
of the \oslic---did not want to become open source license experts only for being
able to use open source software in accordance with their respective licenses. We
did not want to become lawyers. We just wanted to do more efficiently, what
in those days claimed much time and many resources. We were searching for clear
guidance instead of having to determine a correct way through the jungle of open
source licenses---over and over again, project for project. We loved using the
high-quality open source software to improve our performance. We liked using it
legally. But we did not like to laboriously discuss the legal constraints of the
many and different open source licenses.

What we needed was an easy-to-use handout which would lead us without any
detours to executable lists of work items. We wished to obtain to-do lists,
tailored to our usecases and our licenses. We needed reliable lists of tasks we
only had to execute for being sure that we were acting in accordance with the
open source license. When we started out, such a compendium did not exist.

For solving this problem our company took three decisions:

The first decision our company arrived at was to support a small group of
employees to act as \emph{a board of open source license experts}: They should
offer a service for the whole company. Projects, managers, and developers should
be able to ask this board what they have to do for complying with a specific
open Source License under specific circumstances. And this board should answer
with authoritative to-do lists whose executions would assure that the requestors
are acting according to the corresponding open source licenses. The idea behind
this decision was simple. It would save cost and increase quality if one had one
central group of experts instead of being obliged to select (and to train)
developers---over and over again, for every new project. So, the \emph{OSRB} 
(the \emph{Telekom Open Source Review Board}) was founded as an internal expert
group---as a self-organizing, bottom-up driven community.

The second decision our company took was to allow this \emph{Telekom OSRB} to
collect their results systematically---in the form of a reusable compendium.
The idea behind this decision was also simple: The more the internal service
became known, the more the workload would increase: the more work, the more
resources, the more costs. So, such a compendium should save costs and enable
the requestors to find answers by themselves without becoming license experts:
For all default cases, they should find an answer in the compendium instead of
having to request that their work is analyzed by the OSRB. Thus, the planned
\emph{Telekom Open Source License Compendium} will prevent the need
to increase the size of the OSRB in the future.

The third decision our company reached was to allow the \emph{Telekom OSRB} to
create the compendium in the same mode of cooperation that open source projects
usually use. Again, a simple reason evoked this ruling: If in the future---as a
rule---not a reviewing OSRB, but a simple manual should assure the open 
source license compliant behavior of projects, programmers, and managers, this
book had of course to be particularly reliable. There is a known feature of the
open source working model: the ongoing review by the cooperating community
increases the quality. Therefore, the decision not only to write an internal
`Telekom handout,' but to enable the whole community to use, modify, and 
redistribute a broader \emph{Open Source License Compendium} was a decision for
improving quality. Consequently, the \emph{OSRB} decided to publish the
\emph{\oslic} as a set of \LaTeX sources, publicly available via the open
repository GitHub.\footnote{Get the code by using the link
\texttt{https://github.com/dtag-dbu/oslic}; get project information by
\texttt{http://dtag-dbu.github.com/oslic/} or by
\texttt{http://www.oslic.org/}.}  And it licensed the \oslic{} under Creative
Commons Attribution-ShareAlike 3.0 Germany License.\footnote{This text is
licensed under the Creative Commons Attribution-ShareAlike 3.0 Germany License
(\texttt{http://creativecommons.org/licenses/by-sa/3.0/de/}): Feel free
\enquote{to share (to copy, distribute and transmit)} or \enquote{to remix (to
adapt)} it, if you \enquote{[\ldots] distribute the resulting work under the
same or similar license to this one} and if you respect how \enquote{you must
attribute the work in the manner specified by the author(s) [\ldots]}):
In an internet based reuse please mention the initial authors in a suitable
manner, name their sponsor \textit{Deutsche Telekom AG} and link it to
\texttt{http://www.telekom.com}. In a paper-like reuse please insert a short
hint to \texttt{http://www.telekom.com}, to the initial authors, and to their
sponsor \textit{Deutsche Telekom AG} into your preface. For normal citations
please use the scientific standard.}

But to publish the \emph{\oslic} as a free book has another important
connotation---at least for the \emph{Telekom OSRB}: It is also intended to be an
appreciative \emph{giving back} to the \emph{open source community} which has
enriched and simplified the life of so many employees and companies over so many
years. 

Altogether, the \oslic{} follows five principles:

\begin{description}
  \item[To-do lists as the core, discussions around them] Based on a simple
  form for gathering information concerning the use of a piece of open
  source software and its license, the \oslic{} shall offer an easy to use finder
  taking the requestor to the respective to-do list for ensuring license
  conformity. In addition, all these elements of the \oslic{} should comprehensibly
  be introduced and discussed without disturbing the usage itself.

  \item[Quotations with thoroughly specified sources]\label{QuotationPrinciple}
  The \oslic{} shall be revisable and reliable. It shall comprehensibly argue and
  explicitly specify why it adopts which information, from whom, in which
  version, and why.\footnote{For that purpose, we are using an `old-fashioned'
  bibliographic style with footnotes, instead of endnotes or inline-hints.
  We want to enable the users to review or to ignore our comments and hints just
  as they prefer---but on all accounts without being disturbed by large inline
  comments or frequent page turnings. We know that modern writer guides prefer
  less `noisy' styles (\cite[pars pro toto cf.][\nopage passim]{Mla2009a}). But
  for a reliable usage---challenged by the often modified internet
  sources---these methods are still a little imprecise (for details
  $\rightarrow$ \oslic, pp.\ \pageref{sec:QuotationAppendix}. For a short
  motivation of the style used in the \oslic{} \cite[cf.][\nopage
  passim]{Reincke2012a}. For a more elaborated legitimizing version
  \cite[cf.][\nopage passim]{Reincke2012b}).} 

  \item[Not clearing the forest, but cutting a swath] The \oslic{} has to deal with
  licenses and their legal aspects, no doubt. But it shall not discuss all
  details of every aspect. It shall focus on one possible way to act according
  to a license in a specific usecase---even if it is known that there might be
  alternatives.\footnote{The \oslic{} shall not counsel projects with respect to
  their specific needs. This must remain the task for lawyers and legal experts.
  The \oslic{} cannot and shall not replace a legal review or a legal advice by
  lawyers. It shall inspire developers, managers, open source experts, and
  companies to find good solutions, which they finally should have reviewed by
  legal counselors. For the German readers let us repeat again: The
  \oslic{} naturally respects the German \emph{Rechtsdienstleistungsgesetz}. It only
  contains legal reflections addressed to a general public. Its content may only
  be read as a \enquote{nur an die Allgemeinheit gerichtete Darstellung und
  Erörterung von Rechtsfragen}.}
  
  \item[Take the license text seriously!] The \oslic{} shall not give general
  lectures on legal discussions, much less shall it participate in them. It
  shall only find one dependable way for each license and each usecase to comply
  with the license. The main source for this analysis shall be the exact reading
  of the open source licenses themselves---based on and supported by the
  interpretation of benevolent lawyers and rationally arguing software
  developers. The \oslic{} shall respect that open source licenses are written for
  software developers (and sometimes by developers).
  
  \item[Trust the swarm!] The \oslic{} shall be open for improvements and
  adjustments encouraged and stimulated also by other people than employees of
  \emph{Deutsche Telekom AG}.
\end{description}

Based on these principles the \oslic{} offers two methods for being used:

First and foremost the readers expect to simply and quickly find those to-do
lists fitting their needs. Here is the respective process:%
  \footnote{For the well known `quick and dirty hackers'---as we tend to be, 
  too---we have integrated a shortcut: If you already know the license of the 
  open source package you want to use and if you are very familiar with the 
  meaning of the open source use cases we defined, then you might directly 
  jump to the corresponding license specific chapter, without `struggling' 
  with \textit{OSLiC 5 query form} ($\rightarrow$ \oslic{} p. 
  \pageref{OSLiCStandardFormForGatheringInformation}), the taxonomic
  \textit{Open Source Use Case Finder} ($\rightarrow$
  \pageref{OSLiCUseCaseFinder}) or the \textit{O}pen \textit{S}ource \textit{U}se
  \textit{C}ase page ($\rightarrow$ \pageref{OSUCList}ff.): Some of the chapters
  dedicated to specific open source licenses start with a license specific
  finder offering a set of license specific use cases---which, according to the
  complexity of the license, in some cases could be stripped down. But the
  disadvantage of this method is that you have to apply your knowledge about the
  use cases and their side effects by yourself without being systematically guided
  by the \oslic{} process.}

\tikzstyle{decision} = [diamond, draw, fill=gray!20, 
    text width=4.5em, text badly centered, node distance=4cm, inner sep=0pt]

\tikzstyle{preparation} = [rectangle, draw, fill=gray!30, 
    text width=11.5em, text centered, rounded corners, minimum height=3em]
 
\tikzstyle{lprocs} = [rectangle, draw, fill=gray!40, 
    text width=11.5em, text centered, rounded corners, minimum height=3em]
    
\tikzstyle{processing} = [rectangle, draw, fill=gray!40, node distance=2.4cm,
    text width=15em, text centered, rounded corners, minimum height=4em]
    
\tikzstyle{line} = [draw, -latex']

\tikzstyle{cloud} = [draw, ellipse, text centered, fill=gray!10]
 
    
\begin{tikzpicture}[node distance =1.5cm, auto]
\footnotesize
    % Place nodes
    
  \node [cloud, anchor=south, text width=7em] (start) at (1,10) 
    {$\forall$ open source \\ components};
  \node [preparation, below of=start] (select) 
    {select next open source component};     
  \node [preparation,  below of=select] (analyze) 
    {analyze its role as part of software system};
     
  \node [preparation,  below of=analyze] (determine) 
    {determine usage of final software product / service};     
    
  \node [preparation,  below of=determine] (detect) 
    {detect respective open source license};
    
  \node [lprocs,  below of=detect] (fillin)
    {\textbf{fill in the 5 query form} ($\rightarrow$ p.
    \pageref{OSLiCStandardFormForGatheringInformation})};
    
  \node [decision, right of=fillin] (success) {success?};
  
  \node [processing,  below of=success] (traverse)
    {\textbf{traverse} taxonomic \textbf{Open Source Use Case Finder}
    ($\rightarrow$ \pageref{OSLiCUseCaseFinder}) \& jump to indicated
    \textbf{O}pen \textbf{S}ource \textbf{U}se \textbf{C}ase page ($\rightarrow$
    \pageref{OSUCList}ff.)};
    
  \node [processing,  below of=traverse] (find)
    {\textbf{Determine} page of \textbf{license and use case specific to-do
    list} being presented in license specific chapter};
 
  \node [processing,  below of=find] (process)
    {Jump to indicated page \& \textbf{process license and use case specific
    to-do list} ($\rightarrow$ \pageref{OSUCToDoLists}ff.)};
    
  \node [decision, right of=process] (other) {more?};
  \node [cloud, right of=other, anchor=west] (stop) {stop};

  \path [line] (start) -- (select);  
     
  \path [line] (select) -- (analyze);      
  \path [line] (analyze) -- (determine);
  \path [line] (determine) -- (detect);       
  \path [line] (detect) -- (fillin);
  \path [line] (fillin) -- (success);
  
  \path [line] (success) |- node [near start] {no} (analyze);
  \path [line] (success) -- node [near start] {yes} (traverse);             
  
  \path [line] (traverse) -- (find);              
  \path [line] (find) -- (process);
  \path [line] (process) -- (other);

  \path [line] (other) |- node [near start] {yes} (select);
  \path [line] (other) -- node [near start] {no} (stop);                      

\end{tikzpicture}

Second, the readers might wish to comprehend the whole analysis. So, we briefly
discuss open source license taxonomies as the basis for a license compliant
behavior.\footnote{$\rightarrow$ OSLIC \enquote{\nameref{sec:LicenseTaxonomies}},
pp.\ \pageref{sec:LicenseTaxonomies}}  We consider some side effects of acting
according to the open source licenses.\footnote{$\rightarrow$ OSLiC
\enquote{\nameref{sec:SideEffects}}, pp.\ \pageref{sec:SideEffects}} Finally,
we study the structure of open source use cases.\footnote{$\rightarrow$ OSLiC
\enquote{\nameref{sec:OSUCdeduction}}, pp.\ \pageref{sec:OSUCdeduction}}

So, let us close our introduction by using, modifying, and (re)distributing a
well known wish of a well known man: Happy (Legally) Hacking.

%\bibliography{../../../bibfiles/oscResourcesEn}

% Local Variables:
% mode: latex
% fill-column: 80
% End:

%% Telekom osCompendium 'for beeing included' snippet template
%
% (c) Karsten Reincke, Deutsche Telekom AG, Darmstadt 2011
%
% This LaTeX-File is licensed under the Creative Commons Attribution-ShareAlike
% 3.0 Germany License (http://creativecommons.org/licenses/by-sa/3.0/de/): Feel
% free 'to share (to copy, distribute and transmit)' or 'to remix (to adapt)'
% it, if you '... distribute the resulting work under the same or similar
% license to this one' and if you respect how 'you must attribute the work in
% the manner specified by the author ...':
%
% In an internet based reuse please link the reused parts to www.telekom.com and
% mention the original authors and Deutsche Telekom AG in a suitable manner. In
% a paper-like reuse please insert a short hint to www.telekom.com and to the
% original authors and Deutsche Telekom AG into your preface. For normal
% quotations please use the scientific standard to cite.
%
% [ File structure derived from 'mind your Scholar Research Framework' 
%   mycsrf (c) K. Reincke CC BY 3.0  http://mycsrf.fodina.de/ ]

%
\newpage
\section{Form [only to demo our lib style. will be replaced]}
\begin{itemize}
  \item first initially quoted book\footnote{\cite[cf.][123ff]{Grassmuck2002a}
  (expected: complete bibl. data)} using LaTeX \texttt{$\backslash$footnote}
  \item second initially quoted book\footcite[cf.][120 (expected: complete bibl.
  data)]{Fogel2006a} using jurabib \texttt{$\backslash$footcite} (same
  appereance)
  \item initially mentioned collection /
  proceedings\footnote{\cite[cf.][123ff]{DjoGehGraKreSpi2008a} (expected: complete
  bibl. data)}
  \item first initially mentioned article in an initially mentioned collection /
  proceedings\footnote{\cite[cf.][123ff]{Spielkamp2008a} (expected: complete
  bibl. data of article, short title data of collection)} using LaTeX
  \texttt{$\backslash$footnote}
  \item second initially mentioned article in an already mentioned collection
  / proceedings\footcite[cf.][123ff (expected: complete
  bibl. data of article, short title data of collection)]{Kreutzer2008a} using
  jurabib \texttt{$\backslash$footcite} 
  \item rementioned book\footnote{\cite[cf.][120]{Fogel2006a} (expected: short
  title)}
  \item directly rementioned same book same
  page\footnote{\cite[cf.][120]{Fogel2006a} (expected: id., ibid, / ders.,
  ebda.,)}
  \item directly rementioned same book different
  page\footnote{\cite[cf.][121]{Fogel2006a} (expected: id., lc., / ders.,
  a.a.O. \& page)}
  \item rementioned collection article\footnote{\cite[cf.][120 ]{Kreutzer2008a} (expected: short
  title)}
  \item directly rementioned collection article same
  page\footnote{\cite[cf.][120]{Kreutzer2008a} (expected: id., ibid, / ders.,
  ebda.,)}
  \item directly rementioned collection article different
  page\footnote{\cite[cf.][121]{Kreutzer2008a} (expected: id., lc., / ders.,
  a.a.O. \& page)}
\end{itemize}

%\bibliography{../bibfiles/oscResourcesEn}

% Local Variables:
% mode: latex
% fill-column: 80
% End:


%%%%%%%%%%%%%%%
% Telekom osCompendium 'for being included' snippet template
%
% (c) Karsten Reincke, Deutsche Telekom AG, Darmstadt 2011
%
% This LaTeX-File is licensed under the Creative Commons Attribution-ShareAlike
% 3.0 Germany License (http://creativecommons.org/licenses/by-sa/3.0/de/): Feel
% free 'to share (to copy, distribute and transmit)' or 'to remix (to adapt)'
% it, if you '... distribute the resulting work under the same or similar
% license to this one' and if you respect how 'you must attribute the work in
% the manner specified by the author ...':
%
% In an internet based reuse please link the reused parts to www.telekom.com and
% mention the original authors and Deutsche Telekom AG in a suitable manner. In
% a paper-like reuse please insert a short hint to www.telekom.com and to the
% original authors and Deutsche Telekom AG into your preface. For normal
% quotations please use the scientific standard to cite.
%
% [ Framework derived from 'mind your Scholar Research Framework' 
%   mycsrf (c) K. Reincke 2012 CC BY 3.0  http://mycsrf.fodina.de/ ]
%

\chapter{Open Source: The Same Idea, Different Licenses}\label{sec:LicenseTaxonomies}

%% use all entries of the bibliography
%\nocite{*}
\footnotesize \begin{quote}\itshape This chapter describes different license
models which follow the common idea of free open source software. We want to
discuss existing ways of grouping licenses to underline the limits of building
such clusters: These groups are often used as `virtual prototypical licenses'
which are supposed to provide simplified conditions for acting according to
the respective real license instances. But one has to meet the requirements of a
specific license, not one's own generalized idea of a set of licenses.
Nonetheless, we, too,  offer a new way of structuring the world of the open
source licenses. We will use a novel set of grouping criteria by referring
to the common intended purpose of licenses: each license is designed to protect
something or someone against something or someone. Following this pattern, we
can indeed summarize all Open Source Licenses in a comparable way.
\end{quote}
\normalsize{}

Grouping open source licenses\footnote{Talking about licenses is sometimes a bit
tricky: Normally, they have a longer official name and a well known, often
abbreviating inofficial nickname. But that's not enough for talking about a
specific license adequately: one has additionally to refer to the version of the
license itself. The Linux Foundation offers a set of normalized names and
identifiers, to minimize the confusion how to denote a license correctly
(\cite[cf.][\nopage wp]{LinuxFounSpdxList2014a}). The OSLiC tries to use these
SPDX identifiers as far as possible. But sometimes the OSLiC wants to group
specific licenses by their authors without discriminating the release numbers.
Then, the OSLiC uses prefixes of the SPDX.} is commonly done.
Even the set of the \emph{open source li\-cen\-ses}\footcite[cf.][\nopage
wp]{OSI2012b} itself is already a cluster being established by a set of grouping
criteria: The \enquote{distribution terms} of each software license that intends
to become an open source license \enquote{[\ldots] must comply with the [\ldots]
criteria} of the \emph{Open Source De\-fi\-ni\-tion,}\footcite[cf.][\nopage
wp]{OSI2012a} maintained by the \emph{Open Source
Initiative}\footcite[cf.][\nopage wp]{OSI2012c} and often abbreviated as
\emph{OSD}. So, this \emph{OSD} demarcates `the group of [potential] open source
licenses' against `the group of not open sources licenses.'\footnote{More
precisely: meeting the OSD is only a necessary condition for becoming an
\emph{open source license}. The sufficient condition for becoming an \emph{open
source license} is the approval by the OSI, which offers a process for the
official approval of \emph{open source license} (\cite[cf.][\nopage
wp]{OSI2012d}).}

Another way to cluster the \emph{Free Software Licenses} is specified by the
\enquote{Free Software Definition.} This \emph{FSD} contains four conditions
which must be met by any free software license: any FSD compliant license must
grant \enquote{the freedom to run a program, for any purpose [\ldots]},
\enquote{the freedom to study how it works, and adapt it to (one's) needs
[\ldots]}, \enquote{the freedom to redistribute copies [\ldots]}, and finally
\enquote{the freedom to improve the program, and release your improvements
[\ldots]}\footcite[cf.][41]{Stallman1996a} Surprisingly this definition
implies that the requirement \emph{the sourcecode must be openly accessible}
is `only' a derived condition. If the \enquote{freedom to make changes and the
freedom to publish improved versions} shall be \enquote{meaningful}, then the
\enquote{access to the source code of the program} is a prerequisite.
\enquote{Therefore, accessibility of source code is a necessary condition for
free software.}\footcite[cf.][41]{Stallman1996a}

The difference between the OSD and the FSD has often been described as a
difference of emphasis:%
  \footnote{This is also the viewpoint of Richard M. Stallman: 
  On the one hand, he clearly states that the \enquote{Free Software
  movement} and the \enquote{open source movement} generally \enquote{[\ldots]
  disagree on the basic principles, but agree more or less on the practical
  recommendations} and that he \enquote{[\ldots] (does) not think of the open
  source movement as an enemy}.  On the other hand, he delineates the two
  movements by stating that \enquote{for the open source movement, the issue of
  whether software should be open source is a practical question, not an ethical
  one}, while \enquote{for the Free Software movement, non-free software is a
  social problem and free software is the solution}
  (\cite[cf.][55]{Stallman1998a}). \label{RmsFsPriority} Consequently, Richard
  M. Stallman summarizes the positions in a simple way: \enquote{[\ldots] `open
  source' was designed not to raise [\ldots] the point that users deserve
  freedom}. But he and his friends want \enquote{to spread the idea of freedom}
  and therefore \enquote{[\ldots] stick to the term `free software'}
  (\cite[][59]{Stallman1998a}). For a brush-up of this position, expressing
  again that \enquote{(o)pen source is a development methodology [and that] free
  software is a social movement} with an \enquote{ethical imparative}
  \cite[cf.][31]{Stallman2009a} }
Although both definitions \enquote{[\ldots]
(cover) almost exactly the same range of software}, the \emph{Free Software
Foundation}---as it is said---\enquote{prefers [\ldots] (to emphazise) the
idea of freedom [\ldots]} while the \emph{OSI} wants to underline the
philosophically indifferent \enquote{development methodology.}\footcite[pars pro
toto: cf.][232]{Fogel2006a}

A third method to group of free software and free software licenses is specified
by the \enquote{Debian Free Software Guideline}, which is embedded into the
\enquote{Debian Social Contract}. This \enquote{DFSG} contains nine defining
criteria, which---as Debian itself says---have been \enquote{[\ldots] adopted
by the free[sic!] software community as the basis of the Open Source
Definition.}\footcite[cf.][wp]{DFSG2013a}

A rough understanding of these methods might result in the conclusion that these
three definitions are extensionally equal and only differ intensionally.
But that is not true. To unveil the differences, let us compare the clusters
\emph{OSI approved licenses}, \emph{OSD compliant licenses}, \emph{DFSG
compliant licenses}, and \emph{FSD compliant licenses} extensionally, by asking
whether they \emph{could} establish different sets of licenses.\footnote{Indeed,
for analyzing the extensional power of the definition we have to regard all
potentially covered licenses, not only the already existing licenses, because
the subset of really existing licenses still could be expanded be developing new
licenses which fit the definition.}

First, the difference most easy to determine is that of an unidirectional
inclusion: By definition, the \emph{OSI approved licenses} and the \emph{OSD
compliant licenses} meet the requirements of the OSD.\footcite[cf.][\nopage
wp]{OSI2012a} But only the \emph{OSI approved licenses} have successfully
passed the OSI process\footcite[cf.][\nopage wp]{OSI2012a} and therefore are
officially listed as \emph{open source licenses.}\footcite[cf.][\nopage
wp]{OSI2012b}
% TODO: what does this mean? 
Hence, on the one hand, \emph{OSI approved licenses} are
\emph{open source licenses} and vice versa. On the other hand, both---the
\emph{OSI approved licenses} and the \emph{open source licenses}---are
\emph{OSD compliant licenses}, but not vice versa.

Second, a similar argumentation allows us to distinguish the \emph{DFSG compliant
licenses} from the \emph{OSI approved licenses}. As it is stated, the OSD
\enquote{[\ldots] is based on the Debian Free Software Guideline and any
license that meets one definition almost meets the
other.}\footcite[cf.][233]{Fogel2006a} But then again, meeting the definition is
not enough for being an official open source license: the license has to be
approved by the OSI.\footcite[cf.][\nopage wp]{OSI2012b} Thus, it follows that
all \emph{OSI approved licenses} are also \emph{DFSG compliant licenses}, but
not vice versa.

Third, by ignoring the \enquote{few exceptions} which have appeared
\enquote{over the years,}\footcite[cf.][233]{Fogel2006a} it can be said that,
because of their `kinsmanlike' relation, at least the \emph{OSD compliant
licenses} are also \emph{DFSG compliant licenses} and vice versa.

Last but not least, it must be stated that the (potential) set of free software
licenses must be greater than all the other three sets: On the one side, the FSD
requires that a license of free software must not only allow to read the
software, but must also permit to use, to modify, and to distribute
it.\footcite[cf.][41]{Stallman1996a} These conditions are covered by at least
the first three paragraphs of the OSD concerning the topics \enquote{Free
Redistribution,} \enquote{Source Code,} and \enquote{Derived
Works.}\footcite[cf.][\nopage wp]{OSI2012a} On the other side, the OSD contains
at least some requirements which are not mentioned by the FSD and which
nevertheless must be met by a license in order to be qualified as an OSD
compliant license.\footnote{For example, see the condition that \enquote{the
license must be technology-neutral} (\cite[cf.][\nopage wp]{OSI2012a}).} It
follows then that there may exist licenses which fulfill all conditions of the
FSD and nevertheless do not fulfill at least some conditions of the
OSD.\footnote{Again: we must consider the extensional potential of the
definitions, not the set of really existing licenses. In this context, it is
irrelevant that actually all existing Free Software Licenses like GPL, LGPL or
AGPL indeed are also classfied as open source licenses. We are referring to the
fact that there might be generated licenses which fulfill the FSD, but not the
OSD.} So, the set of all (potential) \emph{Free Software Licenses} must be
greater than the set of all (potential) \emph{open source licenses} and greater
than the set of \emph{OSD compliant licenses}.

All in all, we can visualize the situation as follows:

\begin{center}

\begin{tikzpicture}
\label{LICTAX}
\small

\node[ellipse,minimum height=5.8cm,minimum width=11.6cm,draw,fill=gray!10] (l0210) at (5,5)
{ };

\draw [-,dotted,line width=0pt,white,
    decoration={text along path,
              text align={center},
              text={|\itshape|All Software Licenses}},
              postaction={decorate}] (0,6.1) arc (120:60:10cm);

\node[ellipse,minimum height=4.4cm,minimum width=10cm,draw,fill=gray!20] (l0210) at (5,5)
{ };

\draw [-,dotted,line width=0pt,white,
    decoration={text along path,
              text align={center},
              text={|\itshape|FSD Compliant Licenses}},
              postaction={decorate}] (0,5.4) arc (120:60:10cm);
              

\node[ellipse,minimum height=3cm,minimum width=8.4cm,draw,fill=gray!30] (l0210) at (5,5)
{ };


             
\draw [-,dotted,line width=0pt,white,
    decoration={text along path,
              text align={center},
              text={|\itshape|OSD Compliant Licenses}},
              postaction={decorate}] (0,4.7) arc (120:60:10cm);
              
\draw [-,dotted,line width=0pt,white,
    decoration={text along path,
              text align={center},
              text={|\itshape|DFSG Compliant Licenses}},
              postaction={decorate}] (0,5) arc (240:300:10cm);
          

\node[ellipse,text width=4.4cm, text centered,minimum height=1.6cm,minimum width=6cm,draw,fill=gray!40] (l0210) at (5,5)
{ \textit{OSI approved licenses} = \\ \textit{\textbf{open source licenses}}
};

\end{tikzpicture}
\end{center}

It should be clear without longer explanations that these clusters don't allow
to extrapolate to the correct compliant behaviour according to the \emph{open source
licenses}: On the one hand, all larger clusters do not talk about the \emph{open
source licenses}. On the other hand, the \emph{open source license cluster}
itself only collects its elements on the basis of the OSD which does not stipulate
concrete license fulfilling actions for the licensee.

The next level of clustering \emph{open source licenses} concerns the inner
structure of these \emph{OSI approved licenses}. Even the OSI itself has recently
discussed whether a different way of grouping the listed licenses would better fit
the needs of the visitors of the OSI site.\footcite[cf.][\nopage wp]{OSI2013a}
And finally the OSI came up with the categories \enquote{popular and widely used
(licenses) or with strong communities,} \enquote{special purpose licenses,}
\enquote{other/miscellaneous licenses,} \enquote{licenses that are redundant
with more popular licenses,} \enquote{non-reusable licenses,} \enquote{superseded
licenses,} \enquote{licenses that have been voluntarily retired,} and \enquote{
uncategorized licenses.}\footcite[cf.][\nopage wp]{OSI2013b}

Another way to structure the field of open source licenses is to think in
\enquote{types of open source licenses} by grouping the \emph{academic
licenses}, \enquote{named as such because they were originally created by academic
institutions,}\footcite[cf.][69]{Rosen2005a} the \emph{reciprocal
licenses}, named as such because they \enquote{[\ldots] require the
distributors of derivative works to dis\-tri\-bu\-te those works under same
license including the requirement that the source code of those derivative works
be published,}\footcite[cf.][70]{Rosen2005a} the \emph{standard
licenses,} named as such because they refer to the reusability of
\enquote{industry standards,}\footcite[cf.][70]{Rosen2005a} and the
\emph{content licenses}, named as such because they refer to
\enquote{[\ldots] other than software, such as music art, film, literary works}
and so on.\footcite[cf.][71]{Rosen2005a}

Both kinds of taxonomies directly help to find the relevant licenses that should
be used for new (software) projects. But again: none of these categories 
allows us to infer license compliant behaviour, because the categories are
mostly defined based on license external criteria: whether a license is
published by a specific kind of organization or whether a license deals with
industry standards or other kind of works than software inherently does not
determine a license fulfilling behaviour.

Only the act of grouping into \emph{academic licenses} and 
\emph{reciprocal licenses} touches the idea of license fulfillment
tasks, if one---as it has been done---expands the definition of the  
\emph{academic licenses} by the specification that these licenses
\enquote{[\ldots] allow the software to be used for any purpose whatsoever with
no obligation on the part of the licensee to distribute the source code of
derivative works.}\footcite[cf.][71]{Rosen2005a} With respect to this additional
specification, the clusters \emph{academic licenses} and the
\emph{reciprocal licenses} indeed might be referred as the
\enquote{main categories} of (open source)
licenses:\footcite[cf.][179]{Rosen2005a} By definition, they are constituting
not only a contrary, but contradictory opposite. However, it must  be kept in
mind that they constitute an inherent antagonism, an antinomy inside of the set
of open source licenses.\footnote{Hence, it is at least a little confusing to say
that \enquote{the open source license (OSL) is a reciprocal license} and
\enquote{the Academic Free License (AFL) is the exact same license without the
reciprocity provisions} (\cite[cf.][180]{Rosen2005a}): If the BSD license is an
AFL and if an AFL is not an OSL and if the OSI approves only OSLs, then the BSD
license can not be an approved open source license. But in fact, it still is 
(\cite[cf.][\nopage wp]{OSI2012b}).}

Similiar in nature to the clustering into \emph{academic licenses} and 
\emph{reciprocal licenses} is the grouping into \emph{permissive licenses}, 
\emph{weak copyleft licenses}, and \emph{strong copyleft licenses}: 
Even Wikipedia uses the term \enquote{permissive free software licence} in the
meaning of \enquote{a class of free software licence[s] with minimal
requirements about how the software can be redistributed} and \enquote{contrasts}
them with the\enquote{copyleft licences} as those with \enquote{reciprocity%
\,/\,share-alike requirements.}\footcite[cf.][\nopage wp]{wpPermLic2013a}

Some other authors name the set of \emph{academic licenses} the
\enquote{permissive licenses} and specify the \emph{reciprocal licenses} as
\enquote{restrictive licenses}, because in this case---as a consequence of the
embedded \enquote{copyleft} effect---the source code must be published in case
of modifications. They also introduce the subset of \enquote{strong
restrictive licenses} which additionally require that an (overarching)
derivative work must be published under the same license.\footcite[pars pro toto
cf.][57]{Buchtala2007a} The next refinement of such clustering concepts
directly uses the categories \enquote{[open source] licenses with a strict
copyleft clause,}\footcite[Originally stated as \enquote{Lizenzen mit einer
strengen Copyleft-Klausel.} Cf.][24]{JaeMet2011a} \enquote{[open source]
licenses with a restricted copyleft clause,}\footcite[Originally stated as
\enquote{Lizenzen mit einer beschränkten Copyleft-Klausel.}
Cf.][71]{JaeMet2011a} and \enquote{[open source] licenses without any copyleft
clause.}\footcite[Originally stated as \enquote{Lizenzen ohne Copyleft-Klausel.}
Cf.][83]{JaeMet2011a} Finally, this viewpoint can directly be mapped to the
categories \emph{strong copyleft} and \emph{weak copyleft:} While on the one
hand, \enquote{only changes to the weak-copylefted software itself become
subject to the copyleft provisions of such a license, [and] not changes to the
software that links to it}, on the other hand, the \enquote{strong copyleft}
states \enquote{[\ldots] that the copyleft provisions can be efficiently imposed
on all kinds of derived works.}\footcite[cf.][\nopage wp]{wpCopyleft2013a}

Based on this approach to an adequate clustering and labeling,%
  \footnote{Finally, we should also mention that there exists still other
  classifications which might become important in other contexts. For example,
  the ifross license subsumes under the main category \enquote{Open Source
  Licenses} the subcategories \enquote{Licenses without Copyleft Effect,}
  \enquote{Licenses with Strong Copyleft,} \enquote{Licenses with Restricted
  Copyleft,} \enquote{Licenses with Restricted Choice,} or \enquote{Licenses
  with Privileges}---and lets finally denote these categories also licenses
  which are not listed by the OSI (\cite[cf.][\nopage wp]{ifross2011a}). This is
  reasonable if one refers to the meaning of the OSD (\cite[cf.][\nopage
  wp]{OSI2012a}). The \oslic{} wants to simplify its object of study by
  referring to the approved open source licenses (\cite[cf.][\nopage
    wp]{OSI2012d}) listed by the OSI (\cite[cf.][\nopage wp]{OSI2012b}).}
we can develop the following picture:

\begin{center}

\begin{tikzpicture}
\label{OSLICTAX}
\small

\node[ellipse,minimum height=8.5cm,minimum width=14cm,draw,fill=gray!10] (l0100) at (6.8,6.8)
{  };

\draw [-,dotted,line width=0pt,white,
    decoration={text along path,
              text align={center},
              text={|\itshape| OSI approved licenses}},
              postaction={decorate}] (-0.8,6.5) arc (142:38:9.5cm);

\draw [-,dotted,line width=0pt,white,
    decoration={text along path,
              text align={center},
              text={|\itshape|open source licenses}},
              postaction={decorate}] (-0.8,6.5) arc (218:322:9.5cm);
              
\node[ellipse,minimum height=6.2cm,minimum width=4cm,draw,fill=gray!20] (l0100) at (2.75,6.8)
{  };

\draw [-,dotted,line width=0pt,white,
    decoration={text along path,
              text align={center},
              text={|\itshape| permissive licenses}},
              postaction={decorate}] (0.9,7.4) arc (180:0:1.8cm);

\node[rectangle,draw,text width=1.3cm, text height=0.36cm, fill=gray!40, text
centered] (l0101) at (1.9,7.7) { \footnotesize  \textit{Apache-2.0}};
\node[rectangle,draw,text width=1.3cm, text height=0.36cm, fill=gray!40, text
centered] (l0102) at (3.6,7.7) { \footnotesize  \textit{BSD-X-Clause}};
\node[rectangle,draw,text width=1.3cm, text height=0.36cm, fill=gray!40, text
centered] (l0103) at (1.9,6.7) {  \footnotesize  \textit{MIT}};
\node[rectangle,draw,text width=1.3cm, text height=0.36cm, fill=gray!40, text
centered] (l0104) at (3.6,6.7) {  \footnotesize  \textit{MS-PL}};
\node[rectangle,draw,text width=1.3cm, text height=0.36cm, fill=gray!40, text
centered] (l0105) at (1.9,5.7) {  \footnotesize  \textit{Post-greSQL}};
\node[rectangle,draw,text width=1.3cm, text height=0.36cm, fill=gray!40, text
centered] (l0106) at (3.6,5.7) {  \footnotesize  \textit{PHP-3.X}};

\node[ellipse,minimum height=6cm,minimum width=8.5cm,draw,fill=gray!20] (l0200) at (9.2,6.5)
{  };

\draw [-,dotted,line width=0pt,white,
    decoration={text along path,
              text align={center},
              text={|\itshape| copyleft licenses}},
              postaction={decorate}] (7.5,8.5) arc (120:60:4cm);


\node[ellipse,minimum height=4.5cm,minimum width=4.2cm,draw,fill=gray!30] (l0210) at (7.45,6.5)
{  };

\draw [-,dotted,line width=0pt,white,
    decoration={text along path,
              text align={center},
              text={|\itshape| weak copyleft licenses}},
              postaction={decorate}] (5.4,6.2) arc (180:0:2cm);

\node[rectangle,draw,text width=1.2cm, text height=0.36cm, fill=gray!40, text
centered] (l0211) at (6.7,6.9) {  \footnotesize  \textit{EPL-1.X}};
\node[rectangle,draw,text width=1.2cm, text height=0.36cm, fill=gray!40, text
centered] (l0212) at (8.2,6.9) {  \footnotesize  \textit{EUPL-1.X}};
\node[rectangle,draw,text width=1.2cm, text height=0.36cm, fill=gray!40, text
centered] (l0213) at (6.7,5.7) {  \footnotesize  \textit{LGPL-Y.Y}};
\node[rectangle,draw,text width=1.2cm, text height=0.36cm, fill=gray!40, text
centered] (l0214) at (8.2,5.7) {  \footnotesize  \textit{MPL-X.Y}};

\node[ellipse,minimum height=4.5cm,minimum width=3cm,draw,fill=gray!30] (l0220) at (11.4,6.5)
{  };
 
% line width=0pt,white,
\draw [-,dotted,line width=0pt,white,
    decoration={text along path,
              text align={center},
              text={|\itshape| strong copyleft}},
              postaction={decorate}] (10.4,7) arc (180:0:1cm);

\draw [-,dotted,line width=0pt,white,
    decoration={text along path,
              text align={center},
              text={|\itshape| licenses}},
              postaction={decorate}] (10.4,5.4) arc (180:360:1cm);        

\node[rectangle,draw,text width=1.2cm, text height=0.36cm, fill=gray!40, text
centered] (l0221) at (11.4,6.8) {  \footnotesize  \textit{GPL-X.Y}};
\node[rectangle,draw,text width=1.2cm, text height=0.36cm, fill=gray!40, text
centered] (l0222) at (11.4,5.6) {  \footnotesize \textit{AGPL-3.X}};


\end{tikzpicture}
\end{center}

This extensionally based clarification of a possible open source license
taxonomy is probably well-known and often---more or less explicitly---%
referred to.\footnote{Even the FSF itself uses the term `permissive non-copyleft
free software license' (\cite[pars pro toto: cf.][\nopage wp/section `Original BSD
license']{FsfLicenseList2013a}) and contrasts it with the terms `weak copyleft'
and `strong copyleft' (\cite[pars pro toto: cf.][\nopage wp/section `European
Union Public License']{FsfLicenseList2013a})} Unfortunately, this taxonomy
still contains some misleading underlying messages:

\emph{Permissive} has a very positive connotation. So, the antinomy of
\emph{permissive licenses} versus \emph{copyleft licenses} implicitly signals,
that the \emph{permissive licenses} are in some sense better than the
\emph{copyleft licenses}. Naturally, this `conclusion' is evoked by
confusing the extensional definition and the intensional power of the labels.
But that is the way we---the human beings---like to think. 

Anyway, this underlying message is not necessarily `wrong.' It might be
convenient for those people or companies who only want to use open source
software without being restricted by the \emph{obligation to give something
back} as it has been introduced by the `copyleft.'\footnote{De facto,
\emph{copyleft} is not \emph{copyleft}. Apart from the definition, its effect
depends on the par\-ti\-cu\-ar licenses which determine the conditions for
applying the copyleft `method.' For example, in the GPL, the copyleft effect is
bound to the criteria of `being distributed.' Later on, we will collect these
conditions systematically (see chapter \emph{\nameref{sec:OSUCdeduction}}, pp.\
\pageref{sec:OSUCdeduction}). Therefore, here we still permit ourselves to use a
somewhat `generalizing' mode of speaking.} But there might be other people and
companies who emphasize the protecting effect of the copyleft licenses. And,
indeed, at least the open source license\footnote{Although RMS naturally prefers
to call it a \emph{Free Software License} (s. p.\ \pageref{RmsFsPriority})
} \emph{GPL}\footnote{As the original source \cite[cf.][\nopage
wp]{Gpl20FsfLicense1991a}. Inside of the \oslic, we constantly refer to the
license versions which are published by the OSI, because we are dealing with
officially approved open source licenses. For the `OSI-GPL' \cite[cf.][\nopage
wp]{Gpl20OsiLicense1991a}} has initially been developed to protect the freedom,
to enable the developers to help their \enquote{neighbours}, and to get the
modifications back:%
  \footnote{The history of the GNU project is multiply told. For
  the GNU project and its initiator \cite[cf.\ pars pro toto][\nopage
  passim]{Williams2002a}. For a broader survey \cite[cf.\ pars pro toto][\nopage
  passim]{Moody2001a}. A very short version is delivered by Richard M. Stallman
  himself where he states that---in the years when the early free community was
  destroyed---he saw the \enquote{nondisclosure agreement} which must be signed ,
  \enquote{[\ldots] even to get an executable copy} as a clear \enquote{[\ldots]
  promise not to help your neighbour}: \enquote{A cooperating community was
  forbidden.} (\cite[cf.][16]{Stallman1999a}).}
So, \enquote{Copyleft} is defined
as a \enquote{[\ldots] method for making a program free software and requiring
all modified and extended versions of the program to be free software as
well.}\footcite[cf.][89]{Stallman1996c} It is a method\footnote{Based on the
American legal copyright system, this method uses two steps: first one states,
\enquote{[\ldots] that it is copyrighted [\ldots]} and second one adds those
\enquote{[\ldots] distribution terms, which are a legal instrument that gives
everyone the rights to use, modify, and redistribute the program's code or any
program derived from it but only if the distribution terms are unchanged}
(\cite[cf.][89]{Stallman1996c}).} by which \enquote{[\ldots] the code and the
freedoms become legally inseparable}.\footcite[cf.][89]{Stallman1996c} Because
of these disparate interests of hoping not to be restricted and hoping to be
protected, it could be helpful to find a better label---an impartial name for
the cluster of \emph{permissive licenses}. But until that time, we should at
least know that this taxonomy still contains an underlying declassing message.

The other misleading interpretation is---counter-intuitively---prompted by using
the concept of `copyleft licenses.' By referring to a cluster of \emph{copyleft
licenses} as the opposite of the \emph{permissive licenses}, one implicitly also
sends two messages: First, that republishing one's own modifications is
sufficient to comply with the \emph{copyleft licenses}. And, second, that the
\emph{permissive licenses} do not require anything to be done for obtaining the
right to use the software. Even if one does not wish to evoke such an
interpretation, we---the human beings---tend to take the things as simple as
possible.\footnote{And indeed, in the experience of the authors sometimes
such simplifications gain their independent existence and determine decisions at
the management level. But that is not the fault of the managers. It is their
job to aggregate, generalize and simplify information. It is the job of the
experts to offer better viewpoints without overwhelming the others with
details.} But because of several aspects, this understanding of the antinomy of
\emph{copyleft licenses} and \emph{permissive licenses} is too misleading for
taking it as a serious generalization:

On the one hand, even the `strongly copylefted' GPL imposes other obligations
in addtion to republishing derivative works. For example, it also requires
giving \enquote{[\ldots] any other recipients of the [GPL licensed] Program a
copy of this License along with the Program.}\footcite[cf.][\nopage wp.\
§1]{Gpl20OsiLicense1991a} Furthermore, the `weakly copylefted' licenses require
also more and different criteria to be fulfilled for acting in accordance with
these licenses. For example, the EUPL requires that the licensor, who does not
directly deliver the binaries together with the sourcecode, must offer a
sourcecode version of his work free of charge,\footnote{The German version of the
EUPL uses the phrase \enquote{problemlos und unentgeltlich(sic!) auf den
Quellcode (zugreifen können)} (\cite[cf.][3, section 3]{EuplLicense2007de})
while the English version contains the specification \enquote{the Source Code is
easily and freely accessible} (\cite[cf.][2, section 3]{EuplLicense2007en})}
while the MPL requires that under the same circumstances a recipient
\enquote{[\ldots] can obtain a copy of such Source Code Form [\ldots] at a
charge no more than the cost of distribution to the recipient
[\ldots]}\footcite[cf.][\nopage section 3.2.a]{Mpl20OsiLicense2013a}
And last but not least, also the \emph{permissive licenses} require tasks 
to be fulfilled for a license compliant usage---moreover, they also require
different things. For example, the BSD license demands that \enquote{the
(re)distributions [\ldots] must (retain [and/or]) reproduce the above copyright
notice [\ldots]}. Because of the structure of the \enquote{copyright notice},
this compulsory notice implies that the authors / copyright holders of the
software must be publicly named.\footcite[cf.][\nopage wp]{BsdLicense2Clause} As
opposed to this, the Apache License requires that \enquote{if the Work includes
a \enquote{NOTICE} text file as part of its distribution, then any Derivative Works that
You distribute must include a readable copy of the attribution notices contained
within such NOTICE file} which often means that you have to present central
parts of such files publicly\footcite[cf.][\nopage wp.\ section
4.4]{Apl20OsiLicense2004a}---parts which can contain much more information than
only the names of the authors or copyright holders.

So, no doubt---and contrary to the intuitive interpretation of this taxonomy---%
each \emph{open source license} must be fulfilled by some actions, even the most
permissive one. And for ascertaining these tasks, one has to look into these
licenses themselves, not the generalized concepts of licenses taxonomies. Hence
again, we have to state that even this well known type of grouping of \emph{open
source licenses} does not allow to derive a specific license compliant behavior:
The taxonomy might be appropriate, if one wants to live with the implicit
messages and generalizations of some of its concepts. But the taxonomy is not an
adequate tool to determine, what one has to do for fulfilling an \emph{open
source license}. A license compliant behaviour for obtaining the right to use a
specific piece of \emph{open source software} must be based on the concrete
\emph{open source license} by which the licensor has licensed the software.
There is no shortcut.

Nevertheless, human beings need generalizing and structuring viewpoints for
enabling themselves to talk about a domain---even if they finally have to
regard the single objects of the domain for specific purposes. We think that
there is a subtler method to regard and to structure the domain of \emph{open
source licenses}. So, we want to offer this other possibility to cluster the
\emph{open source licenses}:\footnote{even if we also have to concede that,
ultimately, one has to always look into the license itself}

We think that, in general, licenses have a common purpose: they should protect
someone or something against something. The structure of this task is based on
the nature of the word `protect' which is a trivalent verb: it links someone or
something who protects, to someone or something who is protected and both
combined to something against which the protector protects and against the other
one is protected. Licenses in general do that. Moreover, to \enquote{protect}
the \enquote{rights} of the licensees is explicitly mentioned in the
GPL-2.0,\footcite[cf.][\nopage wp. Preamble]{Gpl20OsiLicense1991a} in the
LGPL-2.1,\footcite[cf.][\nopage wp. Preamble]{Lgpl21OsiLicense1999a} and the
GPL-3.0\footcite[cf.][\nopage wp. Preamble]{Gpl30OsiLicense2007a}---by which the
LGPL-3.0 inherits this purpose.\footcite[cf.][\nopage wp.
prefix]{Lgpl30OsiLicense2007a} Following this viewpoint, we want to generally
assume that open source licenses are designed to protect: They can protect
the user (recipient) of the software, its contributor resp.\ developer and/or
distributor, and the software itself. And they can protect them against
different threats:

\begin{itemize}
  \item First, we assume, that---in the context of open source software---the
  user can be protected against the loss of the right to use it, to modify it,
  and to redistribute it. Additionally, he can be protected against patent
  disputes.
  \item Second, we assume, that open source contributors and distributors can be
  protected against the loss of feedback in the form of code improvements and
  derivatives, against warranty claims, and against patent disputes.
  \item Third, we assume, that the open source programs and their specific forms%
  ---may they be distributed or not, may they be modified or not, may they be
  distributed as binaries or as sources---can be protected against the
  re-closing resp. against the re-privatization of their further development.
  \item Fourth, we want to assume that new on-top developments being based on
  open source components can be protected against the privatization for enlarging
  the world of freely usable software.\footnote{In a more rigid version, this
  capability of a license could also be identified as the power to protect the
  community against a stagnation of the set of open source software---but this
  description is at least a little to long to be used by the following pages}
\end{itemize}

With respect to these viewpoints, one gets a subtler picture of the license
specific protecting power. Thus, we are going to describe and deduce the
protecting power of each of the open source licenses on the following pages.
Table \ref{tab:powerOfLicenses} summarizes the results as a quick
reference.\footnote{$\rightarrow$ table \ref{tab:powerOfLicenses} on p.\
\pageref{tab:powerOfLicenses}. In February 2014, the Black Duck list of the
\enquote{Top 20 Open Source Licenses} additionally mentions the Artistic License
(AL), the Code Open Project License, the Common Public License, the zlib/png
License, the Academic Free License (AFL), the Microsoft Reciprocal License
(MS-RL) and the Open Software License (OSL) (\cite[cf.][\nopage
wp.]{wpBlackDuck2014a}). The Code Open Project License and Common Public License
are still not OSI approved open source licenses. (\cite[cf.][\nopage
wp.]{OSI2012b}). Thus, finally the OSLiC should additionally analyze not only
the AGPL and the CDDL, but also the AL, the AFL, the MS-RL, the OSL and the
zlib/png License for being able to justiufiably say, that the OSLiC covers the
most important open source licenses.
}

\begin{table}
\begin{minipage}{\textwidth}
\centering
\footnotesize
\caption{Open Source Licenses as Protectors}
\label{tab:powerOfLicenses}

\begin{tabular}{|c|c||c|c|c|c|c|c|c|c|c|c|c|c|c|c|c|}
\hline
  \multicolumn{2}{|c|}{\textit{Open}} &
  \multicolumn{13}{c|}{\textit{are protecting}}\\
\cline{3-15}
  \multicolumn{2}{|c|}{\textit{Source}} &
  \multicolumn{4}{c|}{ \textbf{Users}} &
  \multicolumn{3}{c|}{\textbf{Contributors}} &
  \multicolumn{5}{c|}{\textbf{Open Source Software}} &
  \multirow{4}{*}{\rotatebox{270}{\scriptsize{\textbf{On-Top Develop.\ }}}} 
  \\
\cline{10-14}
  \multicolumn{2}{|c|}{\textit{Licenses\footnote{'\checkmark' indicates that the
  license protects with respect to the meaning of the column, `$\neg$' indicates
  that the license does not protect with regard to the meaning of the column,
  and `--' indicates, that the corresponding statement must still be evaluated.
  \textit{Slanted names of licenses} indicate that these licenses are only
  listed in this table while the corresponding mindmap ($\rightarrow$ p.\
  \pageref{OSCLICMM}) does not cover them }}} &
  \multicolumn{4}{c|}{} &
  \multicolumn{3}{c|}{\tiny{(Distributors)}} &  
  not &
  \multicolumn{4}{c|}{distributed as} 
  & \\
\cline{3-9}\cline{11-14}
  \multicolumn{2}{|c|}{} &
  \multicolumn{4}{c|}{\scriptsize{\textit{who have already got}}} &
  \multicolumn{3}{c|}{\scriptsize{\textit{who spread open}}} & 
  dis- &
  \multicolumn{2}{c|}{unmodified} &
  \multicolumn{2}{c|}{modified} 
  & \\
  \cline{11-14}
  \multicolumn{2}{|c|}{} &
  \multicolumn{4}{c|}{\scriptsize{\textit{sources or binaries}}} &
  \multicolumn{3}{c|}{\scriptsize{\textit{source software}}} & 
  \parbox[t]{1cm}{tri\-bu\-ted} & 
 \rotatebox{270}{\footnotesize{sources\ }} &
 \rotatebox{270}{\footnotesize{binaries\ }} &
 \rotatebox{270}{\footnotesize{sources\ }} &
 \rotatebox{270}{\footnotesize{binaries\ }} 
 & \\
\cline{3-15}
  \multicolumn{2}{|c|}{} &
  \multicolumn{13}{c|}{\textit{against}}\\
\cline{3-15}
  \multicolumn{2}{|c|}{} &
  \multicolumn{3}{c|}{the loss of} & 
  \multirow{3}{*}{\rotatebox{270}{Patent Disputes}} &
  \multirow{3}{*}{\rotatebox{270}{Loss of Feedback}} & 
  \multirow{3}{*}{\rotatebox{270}{Warranty Claims}} & 
  \multirow{3}{*}{\rotatebox{270}{Patent Disputes}} & 
  \multicolumn{5}{c|}{}
  & \\
% no seperator line 
  \multicolumn{2}{|c|}{} &
  \multicolumn{3}{c|}{the right to} &
  & & & &
  \multicolumn{5}{c|}{\footnotesize{Re-Closings / Re-Privatization}} &
  \multirow{3}{*}{\rotatebox{270}{Privatization}}
   \\
\cline{3-5}
  \multicolumn{2}{|c|}{} & 
  \rotatebox{270}{use it} & 
  \rotatebox{270}{modify it} & 
  \rotatebox{270}{redistribute it\ } &
  &  &  &  &
  \multicolumn{5}{c|}{of already opened software}
  & \\
\hline
\hline
  Apache & 2.0 & \checkmark  & \checkmark  & \checkmark  &
  \checkmark & $\neg$ & \checkmark & \checkmark & $\neg$ &
   \checkmark  & $\neg$ & \checkmark & $\neg$ & $\neg$ \\
\hline
  \multirow{2}{*}{BSD} & 3-Cl & \checkmark & \checkmark  & \checkmark  & 
    $\neg$ & $\neg$ & \checkmark & $\neg$  &
    $\neg$ & \checkmark  & $\neg$ & \checkmark & $\neg$ & $\neg$ \\
\cline{2-15}
   & 2-Cl & \checkmark  & \checkmark  & \checkmark  & 
    $\neg$ & $\neg$ & \checkmark & $\neg$  &
    $\neg$ & \checkmark  & $\neg$ & \checkmark & $\neg$ & $\neg$ \\
\hline
  MIT & ~ & \checkmark  & \checkmark  & \checkmark  &
  $\neg$ & $\neg$ & \checkmark & $\neg$ & $\neg$ &
   \checkmark  & $\neg$ & \checkmark & $\neg$ & $\neg$ \\
\hline
  MS-PL & ~ & \checkmark  & \checkmark  & \checkmark  &
  \checkmark & $\neg$ & \checkmark & \checkmark & $\neg$ &
   \checkmark  & $\neg$ & \checkmark & $\neg$ & $\neg$ \\
\hline
  PostgreSQL & ~ & \checkmark  & \checkmark  & \checkmark  &
  $\neg$ & $\neg$ & \checkmark & $\neg$ & $\neg$ &
   \checkmark  & $\neg$ & \checkmark & $\neg$ & $\neg$ \\
\hline
  PHP & 3.0 & \checkmark  & \checkmark  & \checkmark  &
  $\neg$ & $\neg$ & \checkmark & $\neg$ & $\neg$ &
   \checkmark  & $\neg$ & \checkmark & $\neg$ & $\neg$ \\
\hline
\hline
  \textit{CDDL} & 1.0 & \checkmark & \checkmark & \checkmark &
  -- & -- & -- & -- & -- & -- & -- & -- & -- & -- \\
\hline
  EPL & 1.0 & \checkmark  & \checkmark  & \checkmark  &
  \checkmark  & \checkmark  & \checkmark & \checkmark & $\neg$ &
   \checkmark  & \checkmark & \checkmark & \checkmark & $\neg$ \\
\hline
  EUPL & 1.1 & \checkmark  & \checkmark  & \checkmark  &
  \checkmark  & \checkmark  & \checkmark & \checkmark & $\neg$ &
   \checkmark  & \checkmark & \checkmark & \checkmark & $\neg$ \\
\hline
  \multirow{2}{*}{LGPL} & 2.1 & \checkmark  & \checkmark  & \checkmark  &
   $\neg$ & \checkmark  & \checkmark & $\neg$ & $\neg$ &
   \checkmark  & \checkmark & \checkmark & \checkmark & $\neg$ \\
\cline{2-15}
   & 3.0 & \checkmark  & \checkmark  & \checkmark  &
   \checkmark & \checkmark  & \checkmark & \checkmark & $\neg$ &
   \checkmark  & \checkmark & \checkmark & \checkmark & $\neg$ \\
\hline
  \multirow{3}{*}{MPL} & 1.0 & --  & --  & --  &
   -- & --  & -- & -- & -- &
   --  & -- & -- & -- & -- \\
\cline{2-15}
   & 1.1 & --  & --  & --  &
   -- & --  & -- & -- & -- &
   --  & -- & -- & -- & -- \\
\cline{2-15}
   & 2.0 & \checkmark  & \checkmark  & \checkmark  &
  \checkmark  & \checkmark  & \checkmark & \checkmark & $\neg$ &
   \checkmark  & \checkmark & \checkmark & \checkmark & $\neg$ \\
\hline
  \textit{MS-RL} & ~ & \checkmark & \checkmark & \checkmark &
  -- & -- & -- & -- & -- & -- & -- & -- & -- & -- \\
\hline
\hline
  AGPL & 3.0 & \checkmark  & \checkmark  & \checkmark  &
   \checkmark & \checkmark  & \checkmark & \checkmark & \checkmark &
   \checkmark  & \checkmark & \checkmark & \checkmark & \checkmark \\
\hline
  \multirow{2}{*}{GPL} & 2.1 & \checkmark  & \checkmark  & \checkmark  &
   $\neg$ & \checkmark  & \checkmark & $\neg$ & $\neg$ &
   \checkmark  & \checkmark & \checkmark & \checkmark & \checkmark \\
\cline{2-15}
  & 3.0 & \checkmark  & \checkmark  & \checkmark  &
   \checkmark & \checkmark  & \checkmark & \checkmark & $\neg$ &
   \checkmark  & \checkmark & \checkmark & \checkmark & \checkmark \\
\hline
\hline

\end{tabular}

\end{minipage}
\end{table}

\section{\texorpdfstring{The protecting power of the}{The} GNU Affero General Public License (AGPL)}
\protectionlabel{AGPL}

[TODO...]

\section{\texorpdfstring{The protecting power of the}{The} Apache License
(Apache-2.0)}
\protectionlabel{APL}

As an approved \emph{open source license,}\footcite[cf.][\nopage wp]{OSI2012b}
the Apache License%
  \footnote{The Apache License, version 2.0 is maintained by the
  Apache Software Foundation (\cite[cf.][\nopage wp]{AsfApacheLicense20a}).  Of
  course, the OSI is hosting a duplicate of the Apache license
  (\cite[cf.][\nopage wp]{Apl20OsiLicense2004a}) and is listing it as an
  officially approved open source license (\cite[cf.][\nopage wp]{OSI2012b}). The
  Apache license 1.1 is classified by the OSI as \enquote{superseded
  license}(\cite[cf.][\nopage wp]{OSI2013b}). In the same spirit, the Apache
  Software Foundation itself classifies the releases 1.0 and 1.1 as
  \enquote{historic} (\cite[cf.][\nopage wp]{AsfLicenses2013a}). Thus, the \oslic{}
  only focuses on the most recent license Apache-2.0 version. For those who have
  to fulfill these earlier Apache licenses it could be helpful to read them as siblings of
  the BSD-2-Clause and BSD-3-Clause licenses.}
protects the user against the loss of the
right to use, to modify and/or to distribute the received copy of the source
code or the binaries.\footcite[cf.][\nopage wp. §2]{Apl20OsiLicense2004a}
Furthermore, based on its patent clause,\footnote{$\rightarrow$ \oslic{} pp.\
\patentpageref{APL}} the Apache-2.0 protects the users against patent
disputes.\footcite[cf.][\nopage wp. §3]{Apl20OsiLicense2004a} Because of this
patent clause and the \enquote{disclaimer of warranty} together with the
\enquote{limitation of liability,} the Apache license also protects the
contributors and distributors against patent disputes and warranty
claims.\footcite[cf.][\nopage wp. §3, §7, §8]{Apl20OsiLicense2004a} Finally, the
Apache-2.0 protects the distributed sources themselves \emph{against} a change of the
license which would \emph{convert} the work \emph{to closed software}, because,
first, one \enquote{[\ldots] must give any other recipients of the Work or
Derivative Works a copy of (the Apache) license,} second, \enquote{in the Source
form of any Derivative Works that (one) distributes}, one has \enquote{[\ldots]
to retain [\ldots] all copyright, patent, trademark, and attribution notices
[\ldots],} and third, one must \enquote{[\ldots] include a readable copy [\ldots
of the] NOTICE file} being supplied by the original package one has
received.\footcite[cf.][\nopage wp. §4]{Apl20OsiLicense2004a}

But the Apache License does not protect the contributors against the loss of
feedback because it does not `copyleft' the software: the Apache license does
not contain any sentence requiring that one has also to publish the source code.
In the same spirit, the Apache-2.0 does not protect the undistributed software or the
distributed binaries against re-closing (neither in unmodified nor in
modified form) because the Apache License allows to (re)distribute the
binaries without also supplying the sources---even if the binaries rest upon
sources modified by the distributor. Finally, the Apache-2.0 does not protect the
on-top developments against privatization.


\section{\texorpdfstring{The protecting power of the}{The} BSD licenses}
\protectionlabel{BSD2}
\protectionlabel{BSD3}

As approved \emph{open source licenses,}\footcite[cf.][\nopage wp]{OSI2012b} the
BSD Licenses%
  \footnote{BSD has to be resolved as \emph{Berkely Software Distribution}. 
  For details of the BSD license release and namings
  \cite[cf.][\nopage wp.\ editorial]{BsdLicense3Clause}} 
protect the user against
the loss of the right to use, to modify and/or to distribute the received copy
of the source code or the binaries.\footcite[cf.][\nopage wp. §1ff]{OSI2012a}
Additionally, they protect the contributors and/or distributors against warranty
claims of the software users, because these licenses contain a `No Warranty
Clause.'\footcite[one for all version cf.][\nopage wp]{BsdLicense2Clause} And
finally they protect the distributed sources against a change of the license
which closes the sources, because each modification and \enquote{redistributions
of [the] source code must retain the [\ldots] copyright notice, this list of
conditions and the [\ldots] disclaimer}:\footcite[cf.][\nopage
wp]{BsdLicense2Clause} Therefore it is incorrect to distribute BSD licensed
code under another license---regardless of whether it closes the sources or
not.%
  \footnote{In common sense based discussions you may have heard that BSD
  licenses allow to republish the work under another, an own license. Taking the
  words of the BSD License seriously that is not valid under all circumstances:
  Yes, it is true, you are not required to redistribute the sourcecode of a
  modified (derivative) work. You are allowed to modify a received version and to
  distribute the results only as binary code and to keep your improvements closed.
  But if you distribute the source code of your modifications, you have retain the
  licensing, because \enquote{Redistribution [\ldots] in source [\ldots], with or
  without modification, are permitted provided that [\ldots] (the) redistributions
  of source code [\ldots] retain the above copyright notice, this list of
  conditions and the following disclaimer} (\cite[cf.][\nopage
  wp]{BsdLicense2Clause})}

But the BSD Licenses protect neither the users nor the contributors
and/or distributors against patent disputes (because they do not contain any
patent clause). They do not protect the contributors against the loss of
feedback (because they do not `copyleft' the software). Moreover, they do not
protect the undistributed software or the distributed binaries against
re-closing---neither in unmodified nor in modified form---because they
allow to redistribute only the binaries without also supplying the source
code.\footnote{see both, the BSD-2-Clause License (\cite[cf.][\nopage
wp]{BsdLicense2Clause}), and the BSD-3Clause License (\cite[cf.][\nopage
wp]{BsdLicense3Clause})} Finally, the BSD licenses do not protect the on-top
developments against privatization.

\section{\texorpdfstring{The protecting power of the}{The} CDDL [tbd]}
\protectionlabel{CDDL}

As an approved \emph{open source license,}\footcite[cf.][\nopage wp]{OSI2012b}
the Common Develop and Distribution License protects the user
against the loss of the right to use, to modify and/or to distribute the
received copy of the source code or the binaries\footcite[cf.][\nopage wp. 
§?]{Cddl10OsiLicense2004a}

[\ldots]

\section{\texorpdfstring{The protecting power of the}{The} Eclipse Public License (EPL)}
\protectionlabel{EPL}

As an approved \emph{open source license,}\footcite[cf.][\nopage wp]{OSI2012b}
the Eclipse Public License%
  \footnote{The Eclipse Public License, version 1.0 is maintained by the Eclipse
  Software Foundation (\cite[cf.][\nopage wp]{Epl10EclipseFoundation2005a}).
  Of course, also the OSI is hosting a duplicate (\cite[cf.][\nopage
  wp]{Epl10OsiLicense2005a}).} 
protects the user against the loss of the right to use, to modify and/or to
distribute the received copy of the source code or the binaries\citeEPL{§2a}. 
Furthermore, based on its patent clause,\footnote{$\rightarrow$ \oslic{} 
pp.\ \patentpageref{EPL}} the EPL protects the users also against
patent disputes.\citeEPL{§2b \& §2c} Besides this patent clause, the EPL contains the
sections \enquote{no warranty} and \enquote{disclaimer of
liability.}\citeEPL{§5 \& §6} These three elements together protect the
contributors\,/\,distributors against patents disputes and warranty
claims. Finally, the EPL protects the distributed sources themselves
\emph{against} a change of the license which would \emph{reset} the work
\emph{as closed software}: First, the Eclipse Public Licenses requires that 
if a work---released under the EPL---\enquote{[\ldots] is made available in
source code form [\ldots] (then) it must be made available under this (EPL)
agreement, too} while this act of `making avalaible' \enquote{must} incorporate
a \enquote{copy} of the EPL into \enquote{each copy of the [distributed]
program} or program package.\citeEPL{§3} But in opposite to the permissive
licenses, the EPL does not only protect the distributed source code---regardless
whether it is modified or not. The EPL also protects the distributed modified or
unmodified binaries: The EPL allows each modifying \enquote{contributor} and
distributor \enquote{[\ldots] to distribute the Program in object code form
under (one's) own license agreement [\ldots]} provided this license clearly
states that the \enquote{source code for the Program is available} and where the
\enquote{licensees} can \enquote{[\ldots] obtain it in a reasonable manner on or
through a medium customarily used for software exchange.}\citeEPL{§3, esp. §3.b.iv}
Thus, one has to conclude that the EPL is a copyleft license.

But the Eclipse Public License is not a license with strong copyleft; the EPL
uses `only' a weak copyleft effect:%
  \footnote{Even if one can find contrary specifications in the
  internet. \cite[Pars pro toto cf.][\nopage wp]{ifross2011a}: This page is
  listing the EPL in the section \enquote{Other Licenses with strong Copyleft
  Effect}}
Indeed, the EPL says that for each EPL
licensed \enquote{program}---distributed in object form---a place must be made
known where one can get the corresponding source code.\citeEPL{§3, esp. §3.b.iv}
The term `Program' is defined as any \enquote{Contribution distributed in
accordance with [\ldots] (the EPL)} while the term `Contribution'
refers---besides other elements---to \enquote{changes to the Program, and
additions to the Program.}\citeEPL{§1} Unfortunately, this is a circular definition:
`Program' is defined by `Contribution'; and `Contribution' is defined by
`Program.' Nevertheless, one has to read the license benevolently.
Uncontroversial should be this: If one distributes any modified EPL licensed
program, library, module, or plugin, then one has to publish the modified source
code, too. If one \enquote{adds} some own plugins or additional libraries which
are used by an EPL licensed program (which on behalf of this use must have been
modified by adding [sic!] procedure calls) then one has to publish the code of
both parts: that of the program and that of the added elements. In this sense,
the EPL clearly protects the binaries against re-closings like other weak
copyleft using licenses. But if one distributes only an EPL licensed library
which is used as a component by another not EPL licensed on-top program, then
this library does not depend on the top development---provided that the library
itself does not call any (program) functions or procedures delivered by the
overarching on-top development. Hence, nothing is added to the library; and
hence, no other code than that of the library must be published. Therefore, the
EPL does not use the strong copyleft effect in the meaning of---for example --
the GPL.
 
\section{\texorpdfstring{The protecting power of the}{The} European Union Public License (EUPL)}
\protectionlabel{EUPL}

As an approved \emph{open source license,}\footcite[cf.][\nopage wp]{OSI2012b}
the European Union Public License%
  \footnote{The European Union Public License, version 1.1 is maintained by the
  European Union and hosted under the label \enquote{Joinup} 
  (\cite[cf.][\nopage wp]{EuplLicense2007en}).  This EUPL has officially been
  translated into many languages, among others into German 
  (\cite[cf.][\nopage wp]{EuplLicense2007de}). Because of this multi lingual
  instances, the OSI does not offer its own version, but just a landing page
  linked to the lading page of the European host \enquote{Joinup} 
  (\cite[cf.][\nopage wp]{Eupl11OsiLicense2007a}).} 
protects the user against the loss of the right to use, to modify and/or to
distribute the received copy of the source code or the binaries.\citeEUPL{§2}
Furthermore, based on its patent clause\footnote{$\rightarrow$ \oslic{}
pp.\ \patentpageref{EUPL}}, the EUPL protects the users against
patent disputes.\citeEUPL{§2, at its end} Besides this patent clause, the EUPL
additionally contains a \enquote{Disclaimer of Warranty} and a
\enquote{Disclaimer of Liability.}\citeEUPL{§7 \& §8} These three elements
together protect the contributors\,/\,distributors against patents disputes and
warranty claims. Finally, the EUPL also protects the distributed sources against
a re-closing\,/\,re-privatization and the contributors against the loss of
feedback. This protection is based on two steps: First, the European Public
License contains a particular paragraph titled \enquote{Copyleft clause} which
stipulates that \enquote{copies of the Original Work or Derivative Works based
upon the Original Work} must be distributed \enquote{under the terms of (the
European Union Public) License.}\citeEUPL{§5} Second, the EUPL requires that
each licensee---as long as he \enquote{[\ldots] continues to distribute and/or
communicate the Work}---has also to \enquote{[\ldots] provide [\ldots] the
Source Code}, either directly or by \enquote{[\ldots] (indicating) 
a repository where this Source will be easily and freely available
[\ldots]}\citeEUPL{§5} This condition seems to be so important for the EUPL that
the license repeats its message: in another paragraph the EUPL requires again
that \enquote{if the Work is provided as Executable Code, the Licensor provides
in addition a machine-readable copy of the Source Code of the Work along with
each copy of the Work [\ldots] or indicates, in a notice [\ldots], a repository
where the Source Code is easily and freely accessible for as long as the
Licensor continues to distribute [\ldots] the Work.}\citeEUPL{§3} Based on 
the meaning of \enquote{Work} which is defined by the EUPL as \enquote{the
Original Work and/or its Derivative Works}\citeEUPL{§1} it must be concluded
that the EUPL is a copyleft license. 

But nevertheless, the European Union Public License is not a license with strong
copyleft: On the one hand, if one takes the core of the EUPL then the license
seems to protect not only the modifications of the original work against
re-closings and (re-)privatization, but also the on-top developments because
normally you have to publish the source code in both cases. Understood in this
way, the EUPL would be a `strong copyleft license.' But on the other hand, the
EUPL additionally contains a \enquote{Compatibility clause} stating that
\enquote{if the Licensee Distributes [\ldots] Derivative Works or copies thereof
based upon both the Original Work and another work licensed under a Compatible
Licence, this Distribution [\ldots] can be done under the terms of this
Compatible Licence}\citeEUPL{§5}---while the term \enquote{Compatible Licence}
is explicitly defined by a list of compatible licenses, for example the Eclipse
Public License.\citeEUPL{Appendix}. Based on this compatibility clause the
obligation to publish the code of an on-top development can be subverted: As
% RPD: futile???
first step, you could release a little, more or less futile on-top application
licensed under the Eclipse Public License%
  \footnote{Taking the license text very seriously, it is not even necessary
  that this little futile application must depend on the EUPL library by calling
  functions of EUPL library. The license text only says that \enquote{another
  [any other] work licensed under a Compatible Licence} can be distributed
  together with \enquote{derivative works}. By this wording, the license itself
  is establishing a contrast between the derivative work and the other
  work---what indicates that the other work has not necessarily also to be a
  derivative work.} 
which uses a library licensed under the EUPL. As second step, you add this `EUPL
library' which you now may also distribute under the EPL instead of retaining
the EUPL licensing. So, finally you obtain the same work under the Eclipse
Public License which is a weak copyleft license\footnote{$\rightarrow$ \oslic,
p.\ \protectionpageref{EPL}}. Hence the protection of the EUPL-1.1 is
not as comprehensive as one might assume on the basis of the license text
itself,%
  \footnote{This kind of specifiying the protective power of the EUPL is
  initially presented by the FSF (\cite[cf.][wp.\ section `European Union
  Public License']{FsfLicenseList2013a}). The EU answers that publishing such a
  trick will comprise its user in the eyes of the open source community
  (\cite[cf.][wp]{FsfEuplRecomment2013}). That is undoubtely true. But
  unfortunately, this argument does not close the hole in the protecting shield
  put up by the EUPL.}
it can at most be a weak copyleft license---even if the reader might get the
impression that the authors of the EUPL wished to write a strong copyleft
license. Howsoever, the EUPL license does not protect the on-top developments
against a privatization. 

\section{\texorpdfstring{The protecting power of the}{The} GNU General Public License (GPL)}
\protectionlabel{GPL}

The GNU General Public License---also known as GPL---is maintained and offered
by the Free Software Foundation and hosted as part of the well known
\enquote{GNU operating system homepage.}\footcite[cf.][\nopage
wp]{FsfGnuOsLicenses2011a} Currently, there are two versions of the GPL which
are classified as OSI approved open source licenses\footcite[cf.][\nopage
wp]{OSI2012b}, the GPL-2.0%
  \footnote{For the original version, offered by the FSF 
  \cite[cf.][\nopage wp]{Gpl20FsfLicense1991a}. For the version, offered by the 
  OSI \cite[cf.][\nopage wp]{Gpl20OsiLicense1991a}.}
and the GPL-3.0.
  \footnote{For the original version, offered by the FSF 
  \cite[cf.][\nopage wp]{Gpl30FsfLicense2007a}. For the version, offered by the
  OSI \cite[cf.][\nopage wp]{Gpl30OsiLicense2007a}.}
Although both versions of the GPL aim for the same results and the same spirit,
they differ with respect to textual and arguing structure. Therefore, it is
helpful to treat these two licenses separately.  

\subsection{GPL-2.0}
\protectionlabel{GPL2}

The protecting power of the GPL-2.0 can easily be determined: First, the license
allows the users of a received software to \enquote{copy and distribute}
unmodified \enquote{copies of the [\ldots] source code}\citeGPLtwo{§1} as well
as to \enquote{[\ldots] modify [\ldots] copies [\ldots] or any portion of it,
[\ldots] and (to) distribute such modifications [\ldots]}\citeGPLtwo{§2}---%
not only in the form of source code, but also in the form of
binaries.\citeGPLtwo{§3} Thus---and in accordance of being an approved
\emph{open source license}\footcite[cf.][\nopage wp]{OSI2012b}---the GPL-2.0
protects the user against the loss of the right to use, to modify and/or to
distribute the received copy of the source code or the binaries. Second, it
protects the contributors against warranty claims\citeGPLtwo{§11, §12}
and---based on its copyleft effect\citeGPLtwo{§3}---also against the
loss of feedback. Third, the GPL-2.0 protects the source code itself in a nearly
complete mode against privatization: even if one initially distributes only the
binary version of a modification which one has generated (as a \enquote{work
based on the} original) by \enquote{copying} any {portion} of the original work
into this new derivative work,\citeGPLtwo{§2} then one has nevertheless to offer
a possibility to get the source code\citeGPLtwo{§4}---namely for \enquote{the
modified work as whole.}\citeGPLtwo{§3} This modified \enquote{work based on the
[original] Program} has to be read in a very broad sense; it \enquote{[\ldots]
means either the Program or any derivative work under copyright law: that is to
say, a work containing the Program or a portion of it, either verbatim or with
modifications and/or translated into another language.}\citeGPLtwo{§0} Hence, in
the context of software distribution, the GPL-2.0 does not only protect the
software against re-privatization, but also possible on-top developments against
privatization. 

But the GPL-2.0 does not protect against patent disputes\footnote{$\rightarrow$
\oslic, p. \patentpageref{GPL2}}---neither the users, nor the
contributors or distributors---and it does not protect the (modified) software
which is not distributed against (re-)privatization.%
  \footnote{This is a `lack' in the GPL which the AGPL wants to close: you are
  indeed allowed to modify and install a GPL-2.0 licensed server software on
  your own machine for offering a service based on this modified software
  without being obliged to give your improvements back to the
  community. But---at least in Germany---this viewpoint seems to have to respect
  rigorous limits. Sometimes, it is said that even distributing software over
  the parts of a holding is already a distribution which---in the case of 
  GPL-2.0 licensed software---would evoke the obligation to distribute the
  source code, too. [IMPORTANT: citation still needed!]}

\subsection{GPL-3.0}
\protectionlabel{GPL3}

An important modification of the GPL-3.0 is evoked by the use of the new wording
to \enquote{propagate} or to \enquote{convey} a \enquote{covered work}: On the
one hand a \enquote{covered work} denotes \enquote{either the unmodified Program
or a work based on the Program}. This \enquote{work based on the Program} is
defined as a \enquote{modified version} of an \enquote{earlier} instance of the
program which has been derived from this earlier instance by \enquote{(copying
it) from or (adapting) all or part of it} in way other than exactly copying the
earlier instance.\citeGPLthree{§0} On the other hand, \enquote{to propagate a
work} denotes \enquote{copying, distribution (with or without modification),
making available to the public} and any other kind of treating the work
\enquote{[\ldots] except executing it on a computer or modifying a private
copy.}%
  \footnote{\cite[cf.][\nopage wp. §0]{Gpl30OsiLicense2007a}. The GPL 3.0 wants
  to cover the copyright systems of all countries of the world without dealing
  with their particular constraints directly. Therefore it generally states,
  that the meaning of the phrase \enquote{to propagate a work}---in the spirit
  of the FSF---is whatever the specific copyright system wants to be covered by
  these words, \enquote{[\ldots] except executing it on a computer or modifying
  a private copy}.}
Third, the GPL 3.0 specifies that to \enquote{convey} a work \enquote{[\ldots]
means any kind of propagation that enables other parties to make or receive
copies.}\citeGPLthree{§0} This specification shall later on help to clarify that
it is an act of distribution if the recipient himself actively copies or fetches
a program. 

Referring to this new wording, the GPL-3.0 allows as a \enquote{basic
permission} to \enquote{[\ldots] make, run and propagate covered works
[\ldots] without conditions so long as your license otherwise remains in
force.}\citeGPLthree{§2} This might be read as \emph{anything is allowed without
any restrictions---provided there does not exist any rule which must be
respected}. Based on these specifications, the use and the modification of a
GPL-3.0 program only for yourself is not restricted.%
  \footnote{In general, you have to infer that you do not have to fulfill any
  tasks if you are using a piece of open source software only for
  yourself---namely based of the fact that the particular license rules focus 
  only on the distribution of the software, not on the private use. But in the
  GPL-3.0, this assertion concerning the private use becomes more explicit: It is
  one of your \enquote{basic permissions} to \enquote{[\ldots] make, run and
  propagate covered works that you do not convey, without conditions so long as
  your license otherwise remains in force}. And \enquote{to propagate a work}
  refers to anything \enquote{[\ldots] except executing it on a computer or
  modifying a private copy} 
  (\cite[cf.][\nopage wp. §2 and §0]{Gpl30OsiLicense2007a}). 
  Thus, the GPL-3.0 supports your total freedom on your own machine: Do whatever
  you want to do; anything goes---as long as you do not hand the result over to
  any third party in any sense.}  

So, in general---like all the other open source licenses and in accordance to
the OSD\footcite[cf.][\nopage wp]{OSI2012a}---also the GPL protects the user
against the loss of the right to use, to modify and/or to distribute the
received copy of the source code or the binaries.\citeGPLthree{§3, §4, §5, and~§6} 
Furthermore, based on its patent clauses, the GPL-3.0 protects the users and the
contributors of a software against patent disputes.%
  \footnote{$\rightarrow$ \oslic, p.\ \patentpageref{GPL3}}
Additionally, the GPL-3.0 tries to protect the contributors or distributors
against warranty claims by its well known \enquote{Disclaimer of
Warranty}\citeGPLthree{§15} and \enquote{Limitation of
Liability}\citeGPLthree{§16} which must explicitly made been known at least in  
each case of source code distribution.\citeGPLthree{§4} Finally, the most forceful
protection of the GPL-3.0 concerns the protection against the loss of feedback
and against the privatization: Whenever you distribute a GPL-3.0 licensed
program in the form of binaries, you have to make the source accessible,
too.\citeGPLthree{§6} Moreover, this obligation concerns every covered 
work, hence not only the unmodified original, but also any modification or
adaption derived by any other kind of copying parts of the original into the
\enquote{resulting work}:\citeGPLthree{§0} \enquote{You may convey a covered
work in object code form under the terms of sections 4 and 5, provided that you
also convey the machine-readable Corresponding Source under the terms of this
License.}\citeGPLthree{§6} So, no doubt: the GPL wants also the source code of
all on-top developments to be published, not only the modified programs and
libraries used as base of these on-top developments. The single mode of use, the
GPL does not protect against privatization, is the mode of using the software
only for yourself.%
  \footnote{Quite the contrary: The GPL-3.0 explicitly allows to
  delegate the modification to third parties and allows to distribute the source
  code as working base \enquote{[\ldots] to others for the sole purpose of having
  them make modifications exclusively for you [\ldots]} 
  (\cite[cf.][\nopage wp. §2]{Gpl30OsiLicense2007a}).}

\section{\texorpdfstring{The protecting power of the}{The} GNU Lesser General Public License (LGPL)}
\protectionlabel{LGPL}

The LGPL is maintained and offered by the Free Software Foundation and hosted as
part of the well known \enquote{GNU operating system
homepage.}\footcite[cf.][\nopage wp]{FsfGnuOsLicenses2011a} The meaning of the
name \emph{LGPL} was changed in the course of time. First, in 1991, it should be
resolved as \enquote{GNU Library General Public License} and should denote the
\enquote{first released version of the library GPL} which was \enquote{[\ldots]
numbered~2 because it goes with version~2 of the ordinary GPL.} Today, this
license is marked as \enquote{superseded by the GNU Lesser General Public
License}\footcite[cf.][\nopage wp]{Lgpl20FsfLicense1991a}. This newer
\emph{LGPL} version from 1999 was released as \enquote{the successor of the GNU
Library Public License, version 2, hence [as] the version
number~2.1.}\footcite[cf.][\nopage wp]{Lgpl21FsfLicense1999a} Finally, in June
2007, the---for now---last version of the \emph{LGPL} was released---namely with a
new structure: While GPL-2.0 and LGPL-2.1 are similar, but independent licenses,
the LGPL-3.0 has to be read as an addendum to GPL-3.0. At the beginning of the
LGPL-3.0 license, the content of the corresponding GPL-3.0 was included into
the LGPL by the sentence that \enquote{this version of the GNU Lesser General
Public License incorporates the terms and conditions of version~3 of the GNU
General Public License, supplemented by the additional permissions listed
below.}\footcite[cf.][\nopage wp]{Lgpl30FsfLicense2007a} Based on these
differences, it seems to be suitable to treat the different LGPLs separately.

\subsection{LGPL-2.1}
\protectionlabel{LGPL2}

Like the other versions of the GPL or LGPL, the LGPL-2.1 also explicitly
describes its purpose as the task to \enquote{protect} the \enquote{rights} of
the software users: it states that generally all \enquote{[\ldots] the GNU
General Public Licenses are intended to guarantee your freedom to share and
change free software [\ldots]}\citeLGPLtwo{Preamble} Of course, the LGPL-2.1 is
an approved \emph{open source license}\footcite[cf.][\nopage wp]{OSI2012b} which
protects the user against the loss of the right to use, to modify and/or to
distribute the received copy of the source code or the binaries.\citeLGPLtwo{§1, §2, §4} 
But the LGPL-2.1 does not offer any sentences to infer that it grants any patent
rights to the software user.\footnote{$\rightarrow$ \oslic,
p.\ \patentpageref{LGPL2}} So, it does not protect anyone against
patent disputes, neither the users, nor the
contributors\,/\,distributors. Instead of this, the LGPL-2.1 contains a special 
section \enquote{No Warranty} offering two paragraphs which together establish
the protection of the contributors and distributors against warranty
claims.\citeLGPLtwo{§15, §16} Finally, the LGPL-2.1 also protects the
distributed sources against a re-closing\,/\,re-privatization and the
contributors against the loss of feedback. For that purpose, the LGPL-2.1 on the
one hand states that the recipient \enquote{[\ldots] may modify (his) copy or
copies of the Library or any portion of it [\ldots] and copy and distribute such
modifications [\ldots]} provided that the results of these modifications are
\enquote{[\ldots] licensed at no charge to all third parties under the terms of
(the LGPL-2.1).}\citeLGPLtwo{§2} On the other hand, this LGPL version allows to
distribute such modifications \enquote{in object code or executable form}
provided that one accompanies these entities \enquote{[\ldots] with the complete
corresponding machine-readable source code} which itself must be distributed
under the terms of the LGPL-2.1.\citeLGPLtwo{§4}

But contrary to the GPL, the LGPL does not require to publish the code of an
overarching program or any on-top development: It distinguishes the
\enquote{work that \emph{uses} the Library} from the \enquote{work \emph{based
on} the Library}: First, it defines the \enquote{Library} as any
\enquote{software library or work} licensed under the LGPL-2.1 and adds that
\enquote{a `work \emph{based on} the Library' means either the Library or any
derivative work under copyright law.}\citeLGPLtwo{§0, emphasis ours} 
Second, it defines the \enquote{work that \emph{uses} the Library} as any
\enquote{[\ldots] program that contains no derivative of any portion of the
Library, but is designed to work with the Library by being compiled or linked
with it} whereas this \enquote{work that \emph{uses} the Library}---taken
\enquote{in isolation}---clearly \enquote{[\ldots] is not a derivative work of
the Library [\ldots]}% 
  \footnote{\cite[cf.][\nopage wp. §5, emphasis ours]{Lgpl21OsiLicense1999a}. To
  be exact: the LGPL states also, that this work can nevertheless become a
  derivative work under the particular circumstances of being linked to the
  library. But even then, the LGPL allows to treat this `derivative work' as a
  work which is not a derivative work, provided one fulfills some additional
  conditions. With respect to this viewpoint, the hint of the LGPL that the
  non-derivative work becomes a derivate work by linking it, seems not to be as
  crucial as one might expect.}
Third---and explictily \enquote{as an exception to the Sections above}---the
LGPL-2.1 allows to \enquote{[\ldots] combine or link a \enquote{work that uses
the Library} with the Library to produce a work containing portions of the
Library, and distribute that work under terms of (one's own) choice} provided
one \enquote{(accompanies) the work with the complete corresponding
machine-readable source code for the Library}. Together, these three
specifications clearly require that one must publish\,/\,distribute the source
code of the library itself---regardless, whether it is modified or not, and
regardless, whether one distributes the code directly or makes `only' written
offer for receiving the source code of the library separately.\citeLGPLtwo{§6}
But these  specifications do not require that one also must publish\,/\,distribute
the source code of the \emph{work that uses the library} or---as the \oslic{} is
using to say---the \emph{the on-top developments}.

Thus---no surprise---it has to be inferred that the LGPL does not protect the
on-top developments against a privatization. And of course, that is the reason why
it is called the \emph{GNU \emph{Lesser} General Public License}.


\subsection{LGPL-3.0}
\protectionlabel{LGPL3}

The LGPL-3.0 wants to be read as an extension of the GPL-3.0. For that purpose,
it explicitly \enquote{[\ldots] incorporates the terms and conditions of
version~3 of the GNU General Public License, supplemented by (some) additional 
permissions [\ldots]}\citeLGPLthree{just before §0} Thus, the LGPL-3.0 inherits
the most parts of the protecting power of the GPL-3.0---except those parts which
deal with the overarching on-top development: In opposite of the GPL-3.0, the
LGPL allows to embed LGPL-3.0 licensed libraries into libraries of higher
complexity\citeLGPLthree{§3}, into on-top applications\citeLGPLthree{§4}
and into sets of reorganized library systems.\citeLGPLthree{§5} Moreover, the
LGPL-3.0 allows to \enquote{convey} these overarching units \enquote{under terms
of (one's own) choice.}\citeLGPLthree{§4}  Therefore, one is not necessarily obliged to
publish the source code of these on-top developments, too%
  \footnote{To be exact:  The LGPL-3.0 wants to assure that \enquote{combined
  works} can be re-combined on the base of newer versions of the embedded
  library. For that purpose, one has either to use \enquote{a suitable shared
  libary mechanism} which allows to replace the embedded library without
  relinking the larger unit, or one has to publish at least \enquote{the minimal
  corresponding source [code]} and a set of binaries by which the user himself
  can relink the overarching unit on the base of a newer version ob the embedded
  library (\cite[cf.][\nopage wp. §4]{Lgpl30FsfLicense2007a})}%
---but, of course,  one is obliged to publish the source code of the (modified)
embedded libraries themselves. 

Based on the already described protecting power of the GPL-3.0%
\footnote{$\rightarrow$ \oslic, p.\ \protectionpageref{GPL3}}  
and on these additional specifications of the LGPL-3.0, one can summarize the
protecting power of the LGPL-3.0 this way:

First, the LGPL protects the users against the loss of the right to use, to
modify and/or to distribute the received software. Additionally, it protects
them against patent disputes. Second, it protects the contributors and
distributors against the loss of feedback, against warranty claims and against
patent disputes. Finally, it protects the distributed software itself against
re-privatization.

But the LGPL-3.0 does not protect the undistributed source code and does not
protect the on-top developments against privatization.

\section{\texorpdfstring{The protecting power of the}{The} MIT license}
\protectionlabel{MIT}

As an approved \emph{open source license,}\footcite[cf.][\nopage wp]{OSI2012b}
the MIT License%
  \footnote{`MIT' has to be resolved as \enquote{Massachusetts
  Institute of Technology} (\cite[cf.][\nopage wp]{wpMitLic2011a}).} 
protects the user against the loss of the right to use, to modify and/or to
distribute the received copy of the source code or the binaries.%
\footcite[cf.][\nopage wp 1ff]{OSI2012a} 
Additionally, it protects the contributors and/or distributors
against warranty claims of the software users, because it contains a `No
Warranty Clause.'\footcite[cf.][\nopage wp]{MitLicense2012a} And finally it
protects the distributed sources against a change of the license which would
close the sources, because the \enquote{permission [\ldots] to use, copy,
modify, [\ldots] distribute, [\ldots] (is granted) subject to the [\ldots]
conditions, [that] the [\ldots] copyright notice and this permission notice
shall be included in all copies or substantial portions of the
Software.}%
  \footnote{\cite[cf.][\nopage wp]{MitLicense2012a}. The argumentation
  why the source code is protected, but not the binary form follows that of the 
  BSD licenses: By these requirements, one is not obliged to redistribute the
  sourcecode of a modified (derivative) work. One is allowed to modify a received
  version and to distribute the results only in binary form and to keep one's
  improvements closed. But if one distribute the source code of the modifications,
  the licensing is retained, simply because the MIT \enquote{[\ldots] permission
  note shall be included in all copies or substantial portions of the software}.}

But the MIT License does not protect the users or the contributors and/or
distributors against patent disputes (because it does not contain any patent
clause). Additionally, it does not protect the contributors against the loss of
feedback (because it does not `copyleft' the software). Moreover, the MIT
license does not protect the undistributed software or the distributed binaries
against re-closings---neither in unmodified nor in modified form---because it
allows to redistribute only the binaries without also supplying the source
code.\footcite[cf.][\nopage wp]{MitLicense2012a} Finally, the MIT license does
not protect the on-top developments against a privatization.

\section{\texorpdfstring{The protecting power of the}{The} Mozilla Public License (MPL)}
\protectionlabel{MPL}
 
As an approved \emph{open source license,}\footcite[cf.][\nopage wp]{OSI2012b}
the Mozilla Public License%
  \footnote{In 2012, the Mozilla Public License 2.0 
  (\cite[cf.][\nopage wp]{Mpl20MozFoundation2012a}) has been released as a
  result of a longer \enquote{Revision Process}(\cite[cf.][\nopage
  wp]{Mpl11To20MozFoundation2013a}) by which the  Mozilla Public License 1.1 
  (\cite[cf.][\nopage wp]{Mpl11MozFoundation2013a}) has been ousted. The OSI is 
  also hosting its version of the MPL-2.0 (\cite[cf.][\nopage
  wp]{Mpl20OsiLicense2013a}) and is listing it as an OSI approved license 
  (\cite[cf.][\nopage wp]{OSI2012b}) while it classifies the MPL-1.1 as a
  \enquote{superseded license}(\cite[cf.][\nopage wp]{OSI2013b}). The Mozilla
  Foundation itself says concerning the difference between the two licenses that 
  \enquote{the most important part of the license---the file-level copyleft---is
  essentially the same in MPL~2.0 and MPL~1.1} (\cite[cf.][\nopage
  wp]{Mpl11To20MozFoundation2013a}). By reading the MPL-1.1, one could get the
  impression that fulfilling all conditions of the MPL-2.0 would imply also to act
  in accordance to the MPL-1.1. Thus the \oslic{} focuses on the MPL-2.0, at least
  for the moment. Nevertheless, in this section we want to use the general label
  `MPL' without any release number for indicating that with respect to its
  protecting power the MPL-2.0 and the MPL-1.1 can be taken as equipollent.}
protects the user against the loss of the right to use, to modify and/or to
distribute the received copy of the source code or the
binaries.\citeMPL{§2.1.a}  Furthermore, based on its split and distributed patent
clause,\footnote{$\rightarrow$ \oslic{} pp.\ \patentpageref{MPL}} the
MPL protects the users against patent disputes.\citeMPL{§2.1.b, §2.3, §5.2}
Besides this patent sections, the MPL additionally contains a
\enquote{Disclaimer of Warranty} and a \enquote{Limitation of
Liability.}\citeMPL{§6 \& §7} These three elements together protect the
contributors\,/\,distributors against patents disputes and warranty claims.
Finally, the MPL also protects the distributed sources against a
re-closing\,/\,re-privatization and the contributors against the loss of
feedback: The MPL clearly says that, on the one hand, \enquote{all distribution
of Covered Software in Source Code Form, including any Modifications [\ldots]
must be under the terms of this License}\citeMPL{§3.1} and that, on the other
hand, MPL licensed software \enquote{[\ldots] (distributed) in Executable
Form [\ldots] must also be made available in Source Code Form
[\ldots]}\citeMPL{§3.2} So, it must be inferred that the MPL is a copyleft
license. 

But nevertheless, the Mozilla Public License is not a license with strong
copyleft. It does not protect on-top developments against privatization: First,
the MPL does not use the term \emph{derivative work.}%
  \footnote{\cite[cf.][\nopage wp]{Mpl20OsiLicense2013a}. The MPL-1.1 uses the
  term \emph{derivative work} only in the context of writing new
  \enquote{versions of the license}, not in the context of licensing software
  (\cite[cf.][\nopage wp. §6.3]{Mpl11MozFoundation2013a}).}
Instead of this, the MPL denotes the
\enquote{[\ldots] (initial) Source Code Form [\ldots] and Modifications of such
Source Code Form} by the label \enquote{Covered Software}\citeMPL{§1.4}---while
the term \enquote{Modifications} refers to \enquote{any file in Source Code Form
that results from an addition to, deletion from, or modification of the contents
of Covered Software or any file in Source Code Form that results from an
addition to, deletion from, or modification of the contents of Covered
Software.}%
  \footnote{\cite[cf.][\nopage wp.\ §1.10]{Mpl20OsiLicense2013a}. The Mozilla
  Foundation denotes this reading by the term \enquote{file-level copyleft}
  (\cite[cf.][\nopage wp]{Mpl11To20MozFoundation2013a}).}
Second, the MPL contrasts the source code
form and its modifications with the \enquote{Larger Work} by specifying that the
larger work is \enquote{[\ldots] material, in a seperate file or files, that is
not covered software.}\citeMPL{§1.7}
Finally, the MPL states, that \enquote{you may create and distribute a Larger
Work under terms of Your choice, provided that You also comply with the
requirements of this License for the Covered Software.}\citeMPL{§3.3} Based on
these specifications, one has to reason that an on-top development which depends
on MPL licensed libraries by calling some of their functions, is undoubtably a
derivative work,%
  \footnote{This follows from the general meaning of a \emph{derivative work} as
  a benevolent software developer would read this term ($\rightarrow$ \oslic, pp.\
  \pageref{sec:BenevolentDerivativeWorkUnderstanding}). But again: The MPL does
  not focus on this general aspect; it uses its own concept of a \emph{larger
  work}.}
but also only a larger work in the meaning of the MPL so that code of
this on-top application needs not to be published---provided, that the library
and the on-top development are distributed as different files.%
  \footnote{It might be discussed whether integrating a declaration of a
  function, class, or method into the on-top development by including the
  corresponding header files indeed means that one is \enquote{including
  portions (of the Source Code Form)} into a file which therefore has to be
  taken as \enquote{Modification} (\cite[cf.][\nopage
  wp.\ §1.4]{Mpl11MozFoundation2013a}). From the viewpoint of a benevolent
  developer it should be difficult to argue that the including of declaring
  (header) files alone can evoke a derivative work. It is the call of the
  function in one's code which establishes the dependency. But that is not the 
  point, the MPL focuses. The MPL aims on the textual reuse of (defining) code
  snippets. Hence, one could ignore the textual integration of parts of the
  declaring header files: it should not trigger that one's own work becomes a
  modification in the eyes of the Mozilla Findation. But of course, one would
  circumvent the idea of the MPL if one hides defining code in header files and
  reuses that code by one's own compilation. This would undoubtably be an
  incorporation of portions and therefore would make the incorporating file
  becoming a modification of the MPL licensed initial work.} 
Hence, the MPL is license with a weak copyleft effect and does not protect the
on-top developments against privatization.

\section{\texorpdfstring{The protecting power of the}{The} Microsoft Public License (MS-PL)}
\protectionlabel{MSPL}

As an approved \emph{open source license,}\footcite[cf.][\nopage wp]{OSI2012b}
the Microsoft Public License protects the user against the loss of the right to
use, to modify and/or to distribute the received copy of the source code or the
binaries.\citeMSPL{§2} Furthermore, based on its patent
clause,\footnote{$\rightarrow$ \oslic{} pp.\ \patentpageref{MSPL}} the
MS-PL protects the users against patent disputes.\citeMSPL{§2.B and §3.B}
Because of this patent clause and of its 
concise \emph{disclaimer of warranty}, the MS-PL also protects the
contributors\,/\,distributors against patents disputes and warranty 
claims.\citeMSPL{§2B, §3B, §3D}
Finally, the Microsoft Public License protects the distributed sources
themselves---and even \enquote{portions of these sources}---\emph{against} a
change of the license which would \emph{reset} the work \emph{as closed
software}, because first, one \enquote{[\ldots] must retain all copyright,
patent, trademark, and attribution notices that are part of the
software,}\citeMSPL{§3C} and because, second, one must also incorporate
\enquote{a complete copy of this license} into one's own distribution premised
one distributes the source code.\citeMSPL{§3D}

But the Microsoft Public License does not protect the contributors against the
loss of feedback because it does not `copyleft' the software: The license does
not contain any sentence which requires that one has to publish the sources,
too.%
  \footnote{There seems to be some misunderstandings on the internet: The
  English wikipedia specifies the MS-PL as a permissive license and the MS-RL as
  a license with copyleft effect (\cite[cf.][\nopage wp]{wpMsSharedSources2013a}).
  The German wikipedia says that the MS-PL is a license with a \enquote{schwachen
  [weak] copyleft} (\cite[cf.][\nopage wp]{wpMspl2013a}). And it says also that
  the \enquote{Microsoft Reciprocal License} (MS-RL) is a license with weak
  copyleft, too (\cite[cf.][\nopage wp]{wpMsrl2013a}). But for the very
  thoroughly working \enquote{ifross license center}, the MS-RL is a license
  with restricted (weak) copyleft, while the MS-PL is a permissive license with
  some selectable options (\cite[cf.][\nopage wp]{ifross2011a}). Based on the
  license text itself and these other readings, we decided to take the MS-PL as
  a permissive license in accordance to the English wikipedia page and the
  ifross page.}
In the same spirit, the MS-PL does not protect the undistributed
software or the distributed binaries against re-closings---neither in
unmodified nor in modified form---because the MS-PL License allows to
(re)distribute the binaries without also supplying the sources---even if the
binaries rest upon sources modified by the distributor. Finally, also the MS-PL
does not protect the on-top developments against a privatization.


\section{\texorpdfstring{The protecting power of the}{The} Postgres License
(PostgreSQL)}
\protectionlabel{PGL}

As an approved \emph{open source license,}\footcite[cf.][\nopage wp]{OSI2012b}
the PostgreSQL License protects the user against the loss of the right to use,
to modify and/or to distribute the received copy of the source code or the
binaries.\citePGL{}
Because of its \emph{disclaimer of warranty}, the PostgreSQL also protects the
contributors\,/\,distributors against warranty claims.\citePGL{} Finally, the
PostgreSQL protects the distributed sources themselves \emph{against} a change of the
license which would \emph{reset} the work \emph{as closed software}, because the
\enquote{copyright notice} and the whole license must \enquote{[\ldots] appear
in all copies.}\citePGL{}

But the PostgreSQL License does not protect the contributors against the loss of
feedback because it does not `copyleft' the software: The license does not
contain any sentence which requires that one has to publish the sources, too. 
In the same spirit, the PostgreSQL does not protect the undistributed software or the
distributed binaries against re-closings---neither in unmodified nor in
modified form---because the PostgreSQL allows to (re)distribute the binaries without
also supplying the sources---even if the binaries rest upon sources modified by
the distributor. Finally, the PostgreSQL does not protect the on-top developments
against a privatization.


\section{\texorpdfstring{The protecting power of the}{The} PHP License}
\protectionlabel{PHP}

As an approved \emph{open source license,}\footcite[cf.][\nopage wp]{OSI2012b}
the PHP-3.0 License protects the user against the loss of the right to use, to
modify and/or to distribute the received copy of the source code or the
binaries.\citePHP{} Because of its \emph{disclaimer of warranty,} the PHP
license also protects the contributors\,/\,distributors against warranty
claims.\citePHP{}  Finally, the PHP license protects the distributed
sources themselves \emph{against} a change of the license which would
\emph{reset} the work \emph{as closed software,} because
\enquote{redistributions of source code must retain the [\ldots] copyright
notice, this list of conditions and the [\ldots] disclaimer.}\citePHP{}

But the PHP-3.0 License does not protect the contributors against the loss of
feedback because it does not `copyleft' the software: The license does not
contain any sentence which requires that one has to publish the sources, too. 
In the same spirit, the PHP license does not protect the undistributed software
or the distributed binaries against re-closings---neither in unmodified nor in
modified form---because the PHP license allows to (re)distribute the binaries
without also supplying the sources---even if the binaries rest upon sources
modified by the distributor.
  
\section{Summary}

All these specifications can not only be summarized by a
table,\footnote{$\rightarrow$ \oslic{}, p. \pageref{tab:powerOfLicenses}} but also
by a mindmap as it is shown at the end of this chapter. Moreover, based on these
specifications, one could generate new groups of open source licenses, new
classes, like `user protecting licenses,'\footnote{all of them because all of
them have to fulfill the OSD} `patent disputes fending licenses' up to more
sophisticated taxonomies.

However, one must keep in mind that all of these grouping viewpoints do not
legitimate the conclusion that all members of a group can be respected by
fulfilling the same requirements. This would only be possible if the grouping
criteria would directly refer to the fulfilling tasks. Indeed, nearly all open
source licenses do differ with respect to these criteria, and even if the
differences are very small, they can't be neglected.%
  \footnote{Pars pro toto:
  Both, the BSD license and the Apache license require that you provide an
  indication to the developers of the application. But in case of the BSD license
  you have to publish the copyright notice\,/\,line, while in case of the Apache
  license you have exactly to present the content of the notice file distributed
  together with the application.} 
So: reflecting on possible classes of open
source licenses is a good method to become familiar with the area of open source
licenses. But it is not a method to determine, what needs to be done to
obtain the right to use the software. For that purpose every license must be
considered individually.


\begin{tikzpicture}
\label{OSCLICMM}
\footnotesize

% (1.A) list of all licenses and their release numbers Level 5/6
\node[rectangle,draw,text width=1.4cm] (l0100) at (9,4)
{ \textit{BSD License} };
\node[text width=1.4cm] (l0101) at (8.25,3)
{ \scriptsize{3-Clauses} };
\node[text width=1.4cm] (l0102) at (10,3)
{ \scriptsize{2-Clauses} };
  
\node[rectangle,draw,text width=1.4cm] (l0200) at (10.2,5)
{ \textit{MIT License} }; 
  
\node[rectangle,draw,text width=1.4cm] (l0300) at (12,5.5)
{ \textit{\textbf{Apache} License}};
\node[text width=0.4cm] (l0301) at (12,4.5) {\scriptsize{2.0}};

\node[rectangle,draw,text width=1.4cm] (l0400) at (13,6.8)
{ \scriptsize{\textit{\textbf{M}icro\textbf{S}oft} \textbf{P}ublic
\textbf{L}icense} };
  
\node[rectangle,draw,text width=2.0cm] (l0500) at (13,8)
{\textit{\textbf{PostgreS[Q]} \textbf{L}icense}};
  
\node[rectangle,draw,text width=1.4cm] (l0600) at (13,9)
{\textit{\textbf{PHP} License}};
\node[text width=0.4cm] (l0601) at (14.5,9){\scriptsize{3.0}};
  

\node[rectangle,draw,text width=1.4cm] (l0800) at (13,10.7)
{ \textit{\textbf{M}ozilla \textbf{P}ublic \textbf{L}icense}};
\node[text width=0.4cm] (l0801) at (14.5,10.2){\scriptsize{1.1}};
\node[text width=0.4cm] (l0802) at (14.5,11.2){\scriptsize{2.0}};

\node[rectangle,draw,text width=1.4cm] (l0900) at (13,12.25)
{\textit{\textbf{E}clipse \textbf{P}ublic \textbf{L}icense}};
\node[text width=0.4cm] (l0901) at (14.5,12.25) {\scriptsize{1.0}};
 
\node[rectangle,draw,text width=1.5cm] (l1000) at (13,13.8)
{\textit{\textbf{E}uropean \textbf{P}ublic \textbf{L}icense}}; 
\node[text width=0.4cm] (l1001) at (14.5,13.3){\scriptsize{1.1}};
\node[text width=0.4cm,style=dotted] (l1002) at (14.5,14.3){\scriptsize{\textit{1.2}}};
  
\node[rectangle,draw,text width=1.4cm] (l1100) at (13,15.5)
{\textit{\textbf{L}esser \textbf{G}NU \textbf{P}ublic \textbf{L}icense}};

\node[text width=0.4cm] (l1101) at (14.5,15){\scriptsize{2.1}};
\node[text width=0.4cm] (l1102) at (14.5,16){\scriptsize{3.0} };

\node[rectangle,draw,text width=1.4cm] (l1200) at (13,17.5)
{\textit{\textbf{G}NU \textbf{P}ublic \textbf{L}icense}};

\node[text width=0.4cm] (l1201) at (14.5,17){\scriptsize{2.1}};
\node[text width=0.4cm] (l1202) at (14.5,18){\scriptsize{3.0} };

\node[rectangle,draw,text width=1.4cm] (l1300) at (13,19.5)
{ \textit{\textbf{A}ffero \textbf{G}NU \textbf{P}ublic \textbf{L}icense}};
\node[text width=0.4cm] (l1302) at (14.5,19.5){\scriptsize{3.0}};

% 2. the clustering concepts of licenses (level 4)
\node[rectangle,draw,text width=2.3cm] (n0100) at (10,8)
 { \textit{protecting the user, the con\-tri\-butor \& the initial code}\\
   \tiny{Permissive Licenses}      
 };

\node[rectangle,draw,text width=2.3cm] (n0200) at (10,12.5)
{ \textit{protecting the user, the con\-tri\-butor, the
  initial code, \& all di\-rect de\-ri\-va\-tions}\\
  \tiny{Weak Copyleft}        
};

\node[rectangle,draw,text width=2.3cm] (n0300) at (10,16.5)
{ \textit{protecting the user, the con\-tri\-bu\-tor, the 
  initial code, all di\-rect de\-ri\-va\-tions \& the 
  (in\-di\-rect\-ly de\-ri\-ved) on-top-deve\-lop\-ments}\\ 
  \tiny{Strong Copyleft}    
 };

% 3. the threats (level 3)
\node[ellipse,draw,text width=1.6cm] (c110000) at (4.5,0)
{ \textbf{\textit{Patent Disputes}}};

\node[ellipse,draw,text width=1.6cm] (c120000) at (4.5,2)
{ \textbf{\textit{Loss of Rights}} };

\node[ellipse,draw,text width=1.6cm] (c210000) at (4.5,4)
{ \textbf{\textit{Warranty Claims}} };
 
\node[ellipse,draw,text width=1.6cm] (c220000) at (4.5,6)
{ \textbf{\textit{Loss of Feeback}}};

\node[ellipse,draw,text width=0.6cm] (c311000) at (6.2,8)
{ \tiny{\textit{\textbf{reclos\-ings}}}};

\node[ellipse,,draw,text width=0.6cm] (c321000) at (6.2,10)
{ \tiny{\textit{\textbf{reclos\-ings}}} };

\node[ellipse,,draw,text width=0.6cm] (c331000) at (6.2,12)
{ \tiny{\textit{\textbf{reclos\-ings}}} };

\node[ellipse,,draw,text width=0.6cm] (c341000) at (6.2,14)
{ \tiny{\textit{\textbf{reclos\-ings}}} };

\node[ellipse,,draw,text width=0.6cm] (c351000) at (6.8,16.2)
{ \tiny{\textit{\textbf{reclos\-ings}}} };

\node[ellipse,,draw,text width=0.7cm] (c361000) at (7.5,17.5)
{ \tiny{\textit{\textbf{privati\-zings}}} };

\node[ellipse,,draw,text width=1.6cm] (c411000) at (6.5,19)
{ \textit{\textbf{clos\-ings}} };


% 4. the subtypes of protected entities (level 2)
\node[ellipse,draw,text width=1.5cm] (c310000) at (3,8)
 { \scriptsize{un\-modified} \textbf{Sources}};

\node[ellipse,draw,text width=1.5cm] (c320000) at (3.25,10)
 { \scriptsize{un\-modified} \textbf{Binaries}};

\node[ellipse,draw,text width=1.2cm] (c330000) at (3.5,12)
 { \scriptsize{modified} \textbf{Sources}};

\node[ellipse,draw,text width=1.4cm] (c340000) at (3.25,14)
 { \scriptsize{modified} \textbf{Binaries}};

\node[ellipse,draw,text width=2cm] (c350000) at (3.6,16)
 { \tiny{\textbf{part of} On-Top-Developments}};

\node[ellipse,draw,text width=2.9cm] (c360000) at (3.4,17.5)
 { \tiny{\textbf{On-Top-Developments}}};


% 5. the protected entities (level 1)
\node[ellipse,draw,text width=1cm] (c100000) at (1,1)
 { \textbf{Users} };

\node[ellipse,draw,text width=0.8cm] (c200000) at (1,5)
 { \textbf{Con\-tribu\-tors}};

\node[ellipse,draw,text width=0.8cm] (c300000) at (1,12)
 { distri\-buted \textbf{Soft\-ware}};
 
\node[ellipse,draw,text width=2.2cm] (c400000) at (1,19)
 { un\-distri\-buted \textbf{Soft\-ware}}; 

% 6. main node (leve 0)
\node[ellipse,draw,text width=1.3cm] (c000000) at (0,8)
{ \textbf{open source license}};

% a linking Licenses to their release numbers (Linking level 5 to 6)
\foreach \father/\daughter in {
  l0100/l0101/,
  l0100/l0102/,
  l0300/l0301/,
  l0600/l0601/,
  l0800/l0801/,
  l0800/l0802/,
  l0900/l0901/,
  l1000/l1001/,
  l1000/l1002/,
  l1100/l1101/,
  l1100/l1102/,
  l1200/l1201/,
  l1200/l1202/,
  l1300/l1302/
  }
  \draw[dashed] (\father) to  (\daughter) ;

% b) linking Licenses to license concepts (Linking level 5 to 4)
\foreach \father/\daughter/\outangle/\inangle in {
  n0100/l0100/270/150,       
  n0100/l0200/280/155,
  n0100/l0300/290/160,
  n0100/l0400/300/165,
  n0100/l0500/310/150,
  n0100/l0600/340/160,
  n0200/l0800/300/160,
  n0200/l0900/340/170,
  n0200/l1000/20/190,
  n0200/l1100/60/200,
  n0300/l1200/40/180,
  n0300/l1300/80/180 
  }
  %\draw[dashed] (\father) to [out=\outangle,in=\inangle] (\daughter) ;
  \draw[dashed] (\father) to  (\daughter) ;

% c) linking license concepts to the threats against they protect
% c.1) strong copyleft licenses
\foreach \father/\daughter/\outangle/\inangle in {
  c361000/n0300/0/180,
  c351000/n0300/0/180,
  c341000/n0300/45/190,
  c331000/n0300/50/200,
  c321000/n0300/55/210,
  c311000/n0300/60/220,
  c220000/n0300/25/225,
  c210000/n0300/25/230,
  c120000/n0300/25/235
  }
  \draw[<-,color=blue] (\father) to [out=\outangle,in=\inangle] (\daughter) ;
% c.2) weak copyleft licenses
\foreach \father/\daughter/\outangle/\inangle in {
  c341000/n0200/330/170,
  c331000/n0200/0/180,
  c321000/n0200/0/180,
  c311000/n0200/20/190,
  c220000/n0200/15/220,
  c210000/n0200/15/230,
  c120000/n0200/15/235
  }
  \draw[<-,color=cyan] (\father) to [out=\outangle,in=\inangle] (\daughter) ;
% c.3) permissive licenses
\foreach \father/\daughter/\outangle/\inangle in {
  c331000/n0100/355/150,
  c311000/n0100/0/180,
  c210000/n0100/5/210,
  c120000/n0100/10/230
  }
  \draw[<-,color=red] (\father) to [out=\outangle,in=\inangle] (\daughter) ;
%c.4 agpl license
\foreach \father/\daughter/\outangle/\inangle in {
  c411000/l1300/0/180    
}
  \draw[<-,color=green] (\father) to [out=\outangle,in=\inangle] (\daughter) ;


%d linking protected entities, their subtypes and the the relations
\foreach \father/\daughter/\edgetext/\outangle/\inangle in {
  c000000/c100000/protecting/260/120,
  c100000/c110000/against/360/180,
  c100000/c120000/against/360/180,
  c000000/c200000/protecting/270/180,
  c200000/c110000/against/340/150,
  c200000/c210000/against/0/180,
  c200000/c220000/against/0/180,
  c000000/c300000/protecting/90/230,
  c300000/c310000/as/300/180,
  c300000/c320000/as/330/180,
  c300000/c330000/as/0/180,
  c300000/c340000/as/30/180,
  c300000/c350000/as/60/180,
  c300000/c360000/as/70/180,
  c000000/c400000/protecting/100/240,
  c400000/c411000/against/0/180        
}
  \draw[->,dotted,
    decoration={text along path,
              text align={center},
              text={|\itshape|\edgetext}},
              postaction={decorate},] (\father) to [out=\outangle,in=\inangle] (\daughter) ;

\foreach \father/\daughter/\edgetext/\outangle/\inangle in {
  c310000/c311000/against/0/180,
  c320000/c321000/against/0/180,
  c330000/c331000/against/0/180,
  c340000/c341000/against/00/180,
  c350000/c351000/against/0/180,
  c360000/c361000/against/0/180      
}
  \draw[->,dotted,
    decoration={text along path,
              text align={center},
              text={|\itshape \tiny|\edgetext}},
              postaction={decorate},] (\father) to [out=\outangle,in=\inangle] (\daughter) ;

%f linking the patent clauses
\foreach \father/\daughter/\outangle/\inangle in {
  c110000/l1302/0/305,
  c110000/l1202/0/303,
  c110000/l1201/0/301,
  c110000/l1102/0/299,
  c110000/l1101/0/297,
  c110000/l1002/0/295,
  c110000/l0901/0/290,
  c110000/l0802/0/285,
  c110000/l0400/0/275,
  c110000/l0301/0/270   
}
  \draw[<-,color=gray] (\father) to [out=\outangle,in=\inangle] (\daughter) ;

\end{tikzpicture}




%\bibliography{../../../bibfiles/oscResourcesEn}

% Local Variables:
% mode: latex
% fill-column: 80
% End:



%%%%%%%%%%%%%%%

% Telekom osCompendium 'for being included' snippet template
%
% (c) Karsten Reincke, Deutsche Telekom AG, Darmstadt 2011
%
% This LaTeX-File is licensed under the Creative Commons Attribution-ShareAlike
% 3.0 Germany License (http://creativecommons.org/licenses/by-sa/3.0/de/): Feel
% free 'to share (to copy, distribute and transmit)' or 'to remix (to adapt)'
% it, if you '... distribute the resulting work under the same or similar
% license to this one' and if you respect how 'you must attribute the work in
% the manner specified by the author ...':
%
% In an internet based reuse please link the reused parts to www.telekom.com and
% mention the original authors and Deutsche Telekom AG in a suitable manner. In
% a paper-like reuse please insert a short hint to www.telekom.com and to the
% original authors and Deutsche Telekom AG into your preface. For normal
% quotations please use the scientific standard to cite.
%
% [ File structure derived from 'mind your Scholar Research Framework' 
%   mycsrf (c) K. Reincke CC BY 3.0  http://mycsrf.fodina.de/ ]
%

% Chapter Abstract
% ----------------
\chapter{Open Source: About Some Side Effects}\label{sec:SideEffects}

% Local Variables:
% mode: latex
% fill-column: 80
% End:

% Telekom osCompendium 'for being included' snippet template
%
% (c) Karsten Reincke, Deutsche Telekom AG, Darmstadt 2011
%
% This LaTeX-File is licensed under the Creative Commons Attribution-ShareAlike
% 3.0 Germany License (http://creativecommons.org/licenses/by-sa/3.0/de/): Feel
% free 'to share (to copy, distribute and transmit)' or 'to remix (to adapt)'
% it, if you '... distribute the resulting work under the same or similar
% license to this one' and if you respect how 'you must attribute the work in
% the manner specified by the author ...':
%
% In an internet based reuse please link the reused parts to www.telekom.com and
% mention the original authors and Deutsche Telekom AG in a suitable manner. In
% a paper-like reuse please insert a short hint to www.telekom.com and to the
% original authors and Deutsche Telekom AG into your preface. For normal
% quotations please use the scientific standard to cite.
%
% [ Framework derived from 'mind your Scholar Research Framework' 
%   mycsrf (c) K. Reincke 2012 CC BY 3.0  http://mycsrf.fodina.de/ ]
%

%% use all entries of the bibliography
%\nocite{*}

\section{The problem of implicitly releasing patents}
\footnotesize \begin{quote}\itshape In this chapter, we briefly analyze
the effects of patent clauses in open source licenses---not in general, but with
respect to the license fulfilling tasks they require, also known as the
`implicit acceptance of a patent use' by distributing open source software.
\end{quote}
\normalsize

At least the free software movement frowns on the existence of software
patents.%
  \footnote{For an early and elaborate description on the effects of
  software patents based on the viewpoint of the free software movement
  \cite[see][\nopage wp]{Stallman2001a}. This lecture seems to have been given
  more than once and printed later on (\cite[cf.][\nopage wp]{Stallman2002a}).
  Within the first decade of 2000, the focus switched to a more political fight
  against software patents (\cite[cf.][\nopage wp]{Stallman2004a}). But recently
  there seems to have appeared another turn in dealing with software patents:
  Not fighting against the patents, but mitigating their effects. The proposal is
  `[...] (to legislate) that developing, distributing, or running a program on
  generally used computing hardware does not constitute patent infringement'
  (\cite[cf.][\nopage wp]{Stallman2012a})}
One of the best known witnesses for that attitude is the GPL itself. Its
preamble purports that \enquote{[\ldots] any free program is threatened
constantly by software patents.}\citeGPLtwo{} One can read that the open
source community fears three risks: First, they are apprehensive of people who
hijack the idea of a piece of open source software they do not have developed,
register a corresponding patent, and finally try to earn money by preventing the
use of the software or by involving its users in patent
ligitations.\footcite[cf.][234]{JaeMet2011a} Second, they fear a bramble of
general software patents which practically prohibits them to develop open source
software legally.\footcite[cf.][234]{JaeMet2011a} Third, they anticipate the
possibility that (not quite benevolent) open source developers could try to
register patents with the intention of undermining the open source
principles.\footcite[cf.][235]{JaeMet2011a}

Howsoever, regardless whether one tries to fight against software patents or not,
software patents have become a reality. To abide by the law requires managing the
constraints of patents properly. Open source licenses know and respect this
necessity. Moreover, at least some of them try to manage the effect of software
patents by specific patent clauses\citeAPL[pars pro toto cf.]{§3} or by several
sentences distributed in the license text.\citeEPL[pars pro toto cf.]{wp} But why
does the \oslic{} have to deal with this topic, if the \oslic{} does not want to
participate in general discussions?

Opposite to the other conditions of the open source licenses, their patent
clauses or propositions in general do not directly refer to a specific set of
actions which have to be executed for acting in accordance with the licenses. Open
source patent clauses normally do not join in the game `paying by doing.' So,
actually, it does not seem to be necessary to mention the patent clauses here.

Unfortunately, although the patent clauses do not directly say \emph{`do this or
that in these or those circumstances,'} some of them nevertheless have side
effects which imply that the distributors of open source software already
have something done if they actually distribute a piece of open
source software. This implicit effect makes it necessary to deal with the patent
clauses even in an only pragmatic \oslic.

Patent clauses in open source licenses can have two different directions of
impact. They use two methods to protect the users of the open source software---%
and sometimes these methods are combined:

\begin{itemize}
  \item First, an open source license can assure that all contributors to and
  distributors of a piece of open source software grant to all users/%
  recipients not only the right to use the open source software itself, but
  automatically and implicitly also the right to use all those patents 
  belonging to the contributors/distributors which as patents are necessary
  to use the software legally.%
    \footnote{There might arise a legal discussion
    whether even a distributor who does not contribute to the software development
    has to grant the necessary rights of his patent
    portfolio. The \oslic{} does not want to participate in this discussion. We take a
    simple and pragmatic position: to be sure that you are acting according to
    an open source license with such a patent clause you should simply assume that
    you have to do so. If this default position is not reasonable for you it might
    be a good idea to consult legal experts which---perhaps---may find another
    way for you to use the software legally.} 
  So, let us---a little simplifying and therefore only on the following few
  pages---name such licenses the \emph{granting licenses}.
  \item Second, an open source license can try to automatically terminate the
  right to use, to modify, and to distribute the software if its user initiates
  litigations against any of the contributors/distributors with respect to an
  infringement of patent. That can be seen as a revocation of rights granted 
  earlier. So, let us name these license the \emph{revoking licenses.}
\end{itemize}

Later on, we will summarize the concrete patent clauses of all the licenses
discussed in the \oslic{} as a proof for the following classification:

\begin{small}

\begin{center}
\begin{tikzpicture}
\label{PATTAX}


\node[ellipse,minimum height=8.5cm,minimum width=14.2cm,draw,fill=gray!10] (l0100) at (6.7,6.8)
{  };

\draw [-,dotted,line width=0pt,white,
    decoration={text along path,
              text align={center},
              text={|\itshape|open source licenses}},
              postaction={decorate}] (-0.8,6.5) arc (218:322:9.5cm);
              
\node[ellipse,minimum height=6.2cm,minimum width=5cm,draw,fill=gray!20] (l0100)
at (2.5,6.8) {  };

\draw [-,dotted,line width=0pt,white,
    decoration={text along path,
              text align={center},
              text={|\itshape| without granting patent clauses}},
              postaction={decorate}] (0.75,7.5) arc (180:0:1.8cm);

\node[rectangle,draw,text width=1.2cm, text height=0.36cm, fill=gray!40, text
centered] (l0101) at (2.5,8.6) {\footnotesize \textit{MIT}};
\node[rectangle,draw,text width=1.2cm, text height=0.36cm, fill=gray!40, text
centered] (l0102) at (1.7,7.6) {\footnotesize \textit{BSD-X-Clause}};
\node[rectangle,draw,text width=1.2cm, text height=0.36cm, fill=gray!40, text
centered] (l0103) at (3.4,6.4) {\footnotesize \textit{LGPL-2.1}};
\node[rectangle,draw,text width=1.2cm, text height=0.36cm, fill=gray!40, text
centered] (l0104) at (3.4,7.6) {\footnotesize \textit{GPL-2.0}};
\node[rectangle,draw,text width=1.2cm, text height=0.36cm, fill=gray!40, text
centered] (l0105) at (1.7,6.4) {\footnotesize \textit{PHP-3.X}};
\node[rectangle,draw,text width=1.4cm, text height=0.36cm, fill=gray!40, text
centered] (l0106) at (2.5,5.2) {\footnotesize \textit{Post-greSQL}};

\node[ellipse,minimum height=6cm,minimum width=8.5cm,draw,fill=gray!20] (l0200)
at (9.4,6.5) {  };

\draw [-,dotted,line width=0pt,white,
    decoration={text along path,
              text align={center},
              text={|\itshape| with granting patent clauses}},
              postaction={decorate}] (2.2,2) arc (180:0:7cm);


\node[ellipse,minimum height=4.5cm,minimum width=5.6cm,draw,fill=gray!30]
(l0210) at (8.4,6.1) {  };

\draw [-,dotted,line width=0pt,white,
    decoration={text along path,
              text align={center},
              text={|\itshape| granting + revoking}},
              postaction={decorate}] (4.4,3.8) arc (180:0:4cm);

\node[rectangle,draw, text width=2cm, text height=0.34cm, fill=gray!40, text
centered] (l0212) at (7.2,6.9) {  \footnotesize \textit{Apache-2.0}};

\node[rectangle,draw, text width=2cm, text height=0.34cm, fill=gray!40, text
centered] (l0211) at (9.6,6.9) {  \footnotesize  \textit{EPL-1.X}};

\node[rectangle,draw, text width=2cm, text height=0.34cm, fill=gray!40, text
centered] (l0213) at (7.2,6.1) {  \footnotesize  \textit{MPL-X.Y}};

\node[rectangle,draw, text width=2cm, text height=0.34cm, fill=gray!40, text
centered] (l0214) at (9.6,6.1) {  \footnotesize  \textit{MS-PL}};

\node[rectangle,draw, text width=2cm, text height=0.34cm, fill=gray!40, text
centered] (l0213) at (7.2,5.3) {  \footnotesize  \textit{LGPL-3.X}};

\node[rectangle,draw, text width=2cm, text height=0.34cm, fill=gray!40,
text centered] (l0213) at (9.6,5.3) {  \footnotesize  \textit{GPL-3.0}};

\node[rectangle,draw,text width=1.6cm, text height=0.34cm, fill=gray!40, text
centered] (l0214) at (8.4,4.5) {  \footnotesize  \textit{AGPL-3.0}};
 
 
\node[rectangle,draw, text width=1.8cm, text height=0.34cm, fill=gray!40, text
centered] (l0221) at (11.6,8) {  \footnotesize  \textit{EUPL-1.X}};

\end{tikzpicture}
\end{center}

\end{small}


But regardless of the final textual form a license uses to express its
granting or revoking positions, in any case one has to consider some aspects: 

\begin{itemize}
  
  \item Overall, one has to keep in mind that of course no licensor, contributor
  and/or distributor can release the right to use any patents he does not own---%
  not even if he \emph{tries} to release them by an open source patent
  clause.%
    \footnote{The EPL is one of the licenses which insists on this aspect:
    It the second half of its patent clause, the EPL underlines that
    \enquote{[\ldots] no assurances are provided by any Contributor that the
    Program does not infringe the patent or other intellectual property rights of
    any other entity.} Moreover, it explicitly adds that \enquote{[\ldots] if a
    third party patent license is required to allow Recipient to distribute the
    Program, it is Recipient's responsibility to acquire that license before
    distributing the Program} (\cite[cf.][\nopage wp §2c]{Epl10OsiLicense2005a}).}
  Implictly touched patents of third parties not having contributed to the
  development and/or participated in the distribution can never be implicitly
  and automatically released on the base of such an (open source) patent clause:
  no rights, no right to release.% 
    \footnote{This is an important aspect which is sometimes not considered by
    programmers. Inside of DTAG we had a fruitful discussion evoked by Mr. Stephan
    Altmeyer who---as patent lawyer---patiently explained this constraint to us.} 
  Hence: even for those open source licenses which try to protect the users,
  finally the users themselves must nevertheless ensure that they do not violate
  the patents of third parties being unwillingly touched by the way the code
  works or the processes in which the software is used.%
    \footnote{Sometimes, this problem of willingly or
    unwillingly violated third party patents is seen as a weakness of open source
    software. But that is not true. It is a weakness of every software. Even a
    commercial licensor (developer) has only the right to license the use of those
    patents he really owns or he has `bought' for relicensing. Moreover, even
    commercial licensors can willingly or unwillingly violate patents of other
    persons.}
  
  \item In the context of a granting license, one has also to consider that
  contributing to and distributing a piece of software implicitly evokes that
  all patents of the contributor and/or distributor are `given free' which are
  necessary to use the software as whole---including the more or less deeply
  embedded libraries. So, if one wants to check whether some of the core patents
  of one's patent portfolio are afflicted by a patent clause (and whether one
  therefore better should not use/distribute the corresponding piece of open
  source software), one should not forget to check the embedded libraries, too.
  
  \item Finally, one has to consider in the context of a granting license that
  its patent clause only releases the use of the patents in the meaning of
  `allowed to be used for enabling the use of the distributed software.' The
  patent clause does not release the patents generally. Thus, the threat of
  (unwillingly) releasing patents by open source software is not as large as
  sometimes feared: the use of the patent is only granted in combination with
  the software. On the one hand, you may not use the open source software
  without having the right to use the patent because the use of the patent is
  inherently necessary for using the software---regardless, whether the open
  source software is embedded into a larger process or not. On the other hand,
  you are not allowed to use patents---released by the patent clause of an open
  source license---without exactly that open source software which has been
  licensed under this open source license, because the patent clause only refers
  to the use of just that open source software.
  % TODO: this is not completely accurat. OSS license grant downstream patent
  % licenses even for modified software.  The extent to which the software may
  % be modified varies between diffenrent licenses. (RPD)

  \item Summarized, one has to consider that the granting open source licenses
  automatically and implicitly force you to grant all the rights which are
  necessary to use the software legally. Open source contributors and
  distributors should know that.\footnote{Again: It might be debatable whether
  this is also valid for the distributors which do not contribute anything to
  the development. That's a legal discussion the \oslic{} does not wish to participate
  in. From the viewpoint of an open source user who only wants to have one
  reliable and secure way to use open source software compliantly, one should
  perhaps assume that there is no difference.}

  \item With respect to the revoking licenses, one has to consider that their
  patent clauses contain negative conditions which may be read as interdictions.
  The \oslic{} will integrate these conditions into specific `prohibits'-sections
  of its to-do lists.
  
  \item Finally one should mention that in some cases, the form of the
  revocation used by the revoking license refers to the use of the software, in
  other cases to the use of the patents. But nevertheless, one can reason that%
  ---from the pragmatic viewpoint of a benevolent open source software user---%
  this second case of patent revocation also implicitly terminates the right to
  use the software: If the use of a patent is necessary to use a piece of
  software legally, one is not allowed to use the software without having the
  right to use the patent, too; and if the use of the patent is not necessary
  for using the software, then the patent is not covered by the patent clause.
  So, in any case, this kind of patent clauses seems to terminate the right to
  use, distribute or modify the software. Hence, single users as well
  as companies or organizations should also respect such patent clauses if they
  want to be sure to use open source software compliantly.
\end{itemize}

The \oslic{} wants to support its readers not only to act according to the licenses
in general, but also according to its patent clause. Thus, we now briefly cite
and summarize the meaning of particular patent clauses:

\subsection{AGPL statements concerning patents}
\patentlabel{AGPL}

(prelimiary text)

The AGPL-3.0 is a license derived from the GPL-3.0: apart from the preamble and
the paragraphs §11 and §13, they contain nearly the same text.%
  \footnote{compare \cite[][\nopage]{Agpl30OsiLicense2007a} and
  \cite[][\nopage]{Gpl30OsiLicense2007a} in both §1 \ldots §11}
In §13, the AGPL explictly refers to the focus on a \enquote{remote network
interaction} which shall also be able to trigger the delivery of the
corresponding source code; and in §11, the AGPL establishes its specific patent
clause \cite[cf.][\nopage §11 and §13]{Agpl30OsiLicense2007a}.

Like the GPL-3.0, the AGPL-3.0 tries to protect all licensees against patent
claims. This kind of protection is then established by three steps:

First, the AGPL-3.0 assures that \enquote{each contributor grants a non
exclusive, worldwide, royalty free patent license under the contributor’s
essential patent claims, to make, use, sell offer for sale, import and
otherwise run, modify and propagate the contents of its contributor
version.}\citeAGPL{§11} Furthermore, the patent license defines that this patent
license granted by the contributor is automatically extended to all downstream
recipients who later on receive any version of the work even if they indirectly
receive them by third parties and even if they receive a covered work or work
based on the program.\citeAGPL{§11}

Second, the AGPL enforces not only the grant of patent licenses by the
\enquote{contributors,} the license even requires the same from licensees who
distributes the program unchanged: \enquote{If, pursuant to or in connection
with a single transaction or arrangement, you convey, or propagate by procuring
conveyance of, a covered work, and grant a patent license to some of the parties
receiving the covered work authorizing them to use, propagate, modify or convey
a specific copy of the covered work, then the patent license you grant is
automatically extended to all recipients of the covered work and works based on
it.}\citeAGPL{§11}

Finally, the AGPL-3.0 introduces an revoking clause by stating that a licensee
\enquote{[\ldots] may not initiate litigation (including a cross-claim or
counterclaim in a lawsuit) alleging that any patent claim is infringed by
making, using, selling, offering for sale, or importing the Program or any
portion of it}\citeAGPL{§10} and that this licensee \enquote{automatically}
loses the rights granted by the AGPL-3.0 \enquote{including any patent
licenses} if he tries to propagate or modify a covered work against the
regulations of the AGPL-3.0.\citeAGPL{§8} 

According to that, the AGPL-3.0 is like the GPL-3.0 a granting and a revoking
license: At first, one is granted the right to use all patents of all
contributors which are necessary to use the software legally. But if one
installs any litigation regarding an infringement of patents, then the rights
granted to him are revoked.


\subsection{Apache-2.0 statements concerning patents}
\patentlabel{APL}

Titled by the headline \enquote{Grant of Patent License}, the Apache License~2.0
contains a specific patent clause being comprised of two very long and condensed
sentences.\citeAPL{§3} Outside of this patent clause, the word \emph{patent} is
only used once again---for requiring that one \enquote{[\ldots] must retain, in
the (sources) [\ldots] all [\ldots] patent [\ldots] notices [\ldots]}\citeAPL{§4.3}

The one core message of the Apache-2.0 patent clause is that
\enquote{[\ldots] each Contributor hereby grants to You a perpetual, worldwide,
non-exclusive, no-charge, royalty-free, irrevocable [\ldots] patent license to
make, have made, use, offer to sell, sell, import, and otherwise transfer the
Work [\ldots]}%
  \footnote{\cite[cf.][\nopage wp §3]{Apl20OsiLicense2004a}. The
  \enquote{Contributor,} \enquote{Work,} and \enquote{You} are defined in §1:
  \emph{Contributor} refers to the original licensor and to all others whose
  contributions have been incorporated into the Work. The \emph{Work} denotes
  the result of the development process regardless of its form. \emph{You}
  denotes the licensees.}

The second core message of the Apache-2.0 patent clause is the statement that
\enquote{if You institute patent litigation against any entity [\ldots] alleging
that the Work [\ldots] constitutes [\ldots] patent infringement, then any patent
licenses granted to You [\ldots] shall terminate [\ldots]}\citeAPL{§3}

The third message of the Apache-2.0 patent clause is the statement, that the
\enquote{[\ldots] license applies only to those patent claims licensable by such
Contributor that are necessarily infringed by their Contribution(s) alone or by
combination of their Contribution(s) with the Work to which such Contribution(s)
was submitted}.\citeAPL{§3}

Thus, the Apache-2.0 is---as we use to say in this chapter---a granting and a
revoking license: At first you are granted to use all patents of all
contributors which are necessary to use the software legally. But if you---with
respect to the software---install any litigation concerning the infringement of
patents, then the rights granted to you are revoked.

\subsection{CDDL statements concerning patents}
\patentlabel{CDDL}

The patent clauses of the CDDL are similiar in spirit to the Apache License: 
The license grants rights to each contributors patents that are neccessarily
infringed by distributing or using the software. The license also revokes all
rights granted to someone who files a patent litigation with respect to the
software against any contributor.  The CDDL differs from other licenses in that
the litigant does not lose his rights automatically and immediately but gets a
grace period of 60 days. If he withdraws his claims during this period, the
license granted to him will not be terminated.

The actual wording used in the CDDL is complicated by the fact that the CDDL
distinguished between the \enquote{Initial Developer} and other
\enquote{Contributors.}  A \enquote{Contributor} receives a version of the
software to which he then adds some \enquote{Modifications} thus creating the
\enquote{Contributor Version.} For all practical purposes we can treat the
\enquote{Initial Developer} as another contributor who happens to not receive
any software and whose \enquote{Contributor Version} (officially called
\enquote{Original Software}) equals his \enquote{Modifications.}

The patent licenses are granted in the clause (b) of the sections titled
\enquote{The Initial Developer Grant}\citeCDDL{§2.1(b)} and \enquote{Contributor
  Grant.}\citeCDDL{§2.2(b)} Each contributor grants the licensee \enquote{a
  world-wide, royalty-free, non-exclusive license under Patent Claims infringed
  by the making, using, or selling of Modifications made by that Contributor
  either alone and/or in combination with its Contributor Version [\ldots], to
  make, use, sell, offer for sale, have made, and/or otherwise dispose of: (1)
  Modifications made by that Contributor [\ldots]; and (2) the combination of
  Modifications made by that Contributor with its Contributor Version [\ldots]} 
This limits the patent license to patents infringed by code present in the
contributor version. And clause (d) limits the grant even further to exclude
\enquote{infringements caused by[\ldots]third party modifications of Contributor
Version}\citeCDDL{§2.2(d)} or {Covered Software in the absence of Modifications
made by that Contributor.}\citeCDDL{§2.2(d)}
This ensures that no contributor is required to tolerate an infringement of his
patents caused by code modified after he made his contribution and, in
particular, it is not possible to remove the contributors modifications completely
without also removing all other causes of infringement of the patent claims
because the patent license does not carry over to such a use of the software.

The section titled \enquote{TERMINATION} contains the usual defense
against patent infringement claims by declaring that any such claim
against a \enquote{Participant%
  \footnote{The \enquote{Contributor} or \enquote{Initial Developer} against
  whom the claim is made}
[\ldots] alleging that the Participant Software [\ldots] directly or indirectly
infringes any patent, then any and all rights granted directly or indirectly to 
You\footnote{The party making the patent infringement claim}
[\ldots] under Sections 2.1 and/or 2.2 of this
License shall, upon 60 days notice from Participant terminate prospectively and
automatically at the expiration of such 60 day notice period, unless [\ldots] 
You withdraw Your claim [\ldots] against such Participant either unilaterally or
pursuant to a written agreement with Participant.}

Thus, not only has the Participant to actively initiate the termination of the
licenses, the licensee also has 60 days to either settle the case by an
agreement with the Participant or to withdraw his claims.


\subsection{EPL statements concerning patents}
\patentlabel{EPL}

The Eclipse Public License treats the patents necessary to use the program
in the same section and under the same headline \enquote{Grant of Rights} like
all the other rights: First, the EPL clearly states that \enquote{[\ldots] each
Contributor [\ldots] grants (the recipient) a non-exclusive, worldwide,
royalty-free patent license under Licensed Patents to make, use, sell, offer to
sell, import and otherwise transfer the Contribution of such Contributor, if
any, in source code and object code form.}\citeEPL{§2.b} Then the EPL delimits
the extend of this act of granting: Neither hardware patents of the contributors
are covered by this releasing patent clause, nor patents that concern aspects
out of the area of the initially intended software combination.\citeEPL{§2.b}
Finally, the EPL hints to the general fact that 3$^{rd}$ party patents not
belonging to the contributors can never be implicity be released by such a
patent clause. Moreover, it gives the example that \enquote{[\ldots] if a third
party patent license is required to allow Recipient to distribute the Program,
it is Recipient's responsibility to acquire that license before distributing the
Program.}\citeEPL{§2.c}

Like other open source licenses, the EPL announces at its end that
\enquote{if (a) Recipient institutes patent litigation against any entity
[\ldots] alleging that the Program [\ldots] infringes such Recipient's
patent(s), then such (granted) Recipient's rights [\ldots] shall terminate
[\ldots]}\citeEPL{§7}

Thus, the EPL, too, is a granting and a revoking license: 
At first you are granted the use of all patents of all
contributors which are necessary to use the software legally. But if you---with
respect to the software---install any litigation concerning an infringement of
patents, then the rights granted to you are revoked.

\subsection{EUPL statements concerning patents}
\patentlabel{EUPL}

The European Union Public License contains a very brief patent clause. It only
states, that \enquote{the Licensor grants to the Licensee royalty-free, non
exclusive usage rights to any patents held by the Licensor, to the extent
necessary to make use of the rights granted on the Work under this
Licence.}\citeEUPL{end of §2}
Furthermore the EUPL does not contain any patent specific revoking clause, but
only an abstract clause requiring that all \enquote{[\ldots] the rights granted
hereunder will terminate automatically upon any breach by the Licensee of the
terms of the Licence}\citeEUPL{§12}. Thus, the EUPL is---as we are using to say
in this chapter---a granting license but not a revoking license.

\subsection{GPL statements concerning patents}

Although the GPL versions 2.0 and 3.0 are aiming for the same results, they
differ heavily with respect to textual and arguing structure. Therefore, it
should be helpful to treat these two licenses separately.

\subsubsection{GPL-2.0}
\patentlabel{GPL2}

The GPL-2.0 does not contain any specific patent clause by which it would grant
(and revoke) the rights to use those patents belonging to the contributors and 
being necessary to use the software in accordance with the legal patent system.

Instead of this, the preamble of the GPL-2.0 alleges that \enquote{[\ldots] any
free program is threatened constantly by software patents} and that the authors
of the GPL---for tackling this threat---\enquote{[\ldots] had made it clear
that any patent must be licensed for everyone's free use or not licensed at
all}\citeGPLtwo{Preamble}. Unfortunately, this specification is only an indirect
claim which needs a lot of arguing for establishing a protective effect against
patent disputes. Howsoever, this paragraph of the GPL-2.0 does not directly
grant any rights to the software users to use necessary patents, too.

With respect to the patent problem, the GPL-2.0 also states that a licensee has
to fulfill the conditions of the GPL-2.0 completely, even if an existing patent
infringement---being \enquote{imposed} on the GPL licensee---\enquote{[\ldots]
contradicts the conditions of this license} so, that a waiver of the use of the
software is the only way to fulfill both constraints.\citeGPLtwo{§11} And
finally the GPL-2.0 allows the original copyright holder to \enquote{add an
explicit geographical distribution limitation excluding [\ldots] countries}
provided that these countries \enquote{[\ldots] (restict) the distribution
and/or use of the library [\ldots] by patents [\ldots]}\citeGPLtwo{§12}
Based on these statements, one cannot infer that the GPL-2.0 grants any patent
rights to the software user, neither directly, nor indirectly.

Thus, the GPL-2.0 is neither a granting nor a revoking license.

\subsubsection{GPL-3.0}
\patentlabel{GPL3}

Initially, the GPL-3.0 regrets that \enquote{[\ldots] every program is
threatened constantly by software patents} what should be seen as the
\enquote{[\ldots] danger that patents applied to a free program could make it
effectively proprietary}. And therefore---as the GPL-3.0 itself summarizes its
patent rules---\enquote{[\ldots] the GPL assures that patents cannot be used to
render the program non-free.}\citeGPLthree{Preamble}. This kind of protection is
then established by three steps. First, the GPL-3.0 stipulates that
\enquote{each contributor grants [\ldots the licensees] a non-exclusive,
worldwide, royalty-free patent license under the contributor's essential patent
claims, to make, use, sell, offer for sale, import and otherwise run, modify and
propagate the contents of its contributor version.}\citeGPLthree{§11}
Second, the GPL-3.0 defines that this patent license granted by the contributor
\enquote{[\ldots] is automatically extended to all recipients} who later on
receive any version of the work, even if they indirectly receive them by third
parties and even if they receive a \enquote{covered work} or \enquote{works
based on it.}\citeGPLthree{§11} Moreover, the GPL-3.0 also specifies that those
distributors of a \enquote{covered work} who have the right to use a patent
necessary for the use of the distributed software but who are not allowed to
relicense this patent to third parties must solve this problem by making the
source code available nevertheless, by \enquote{depriving} themselves or by
\enquote{extending the patent license to downstream recipients.}\citeGPLthree{§11} 
And finally, the GPL-3.0 also introduces a revoking clause by stating that a
licensee \enquote{[\ldots] may not initiate litigation [\ldots] alleging that
any patent claim is infringed by making, using, selling, offering for sale, or
importing the Program or any portion of it}\citeGPLthree{§10} and that this
licensee \enquote{automatically} loses the rights granted by the GPL-3.0
\enquote{including any patent licenses} if he tries to propagate or modify a
covered work against the rules of the GPL-3.0.\citeGPLthree{§8}

Thus, GPL-3.0 is a granting and a revoking license: At first, one is granted the
right to use all patents of all contributors which are necessary to use the
software legally. But if you---with respect to the software---install any
litigation concerning an infringement of patents, then the rights granted to you
are revoked. 


\subsection{LGPL statements concerning patents}

As already mentioned above, the LGPL versions 2.1 and~3.0 differ heavily with
respect to textual and arguing structure. Therefore, they should be treated
separately.

\subsubsection{LGPL-2.1}
\patentlabel{LGPL2}

Like the GPL-2.0, the LGPL-2.1 does not contain any specific patent clause by
which it would grant (and revoke) the rights to use those patents belonging to
the contributors and being necessary to use the software in accordance with the
legal patent system.

Instead of this, the preamble of the LGPL-2.1 says that \enquote{[\ldots]
software patents pose a constant threat to the existence of any free program}
and that the authors of the LGPL---for tackling this threat---%
\enquote{[\ldots] insist that any patent license obtained for a version of the
library must be consistent with the full freedom of use specified in this
license.}\citeLGPLtwo{Preamble}
Unfortunately, this specification is again only an indirect claim which needs a
lot of arguing to establish a protective effect against patent disputes.
Howsoever, this paragraph of the LGPL-2.1 does not directly grant any rights to
the software users to use necessary patents.

With respect to the patent problem, the LGPL-2.1 also states that a licensee has
to fulfill the conditions of the LGPL-2.1 completely, even if an existing patent
infringement---being \enquote{imposed} on the LGPL licensee---%
\enquote{[\ldots] contradicts the conditions of this license} so that a waiving
of the use of the software is the only way to fulfill both
constraints.\citeLGPLtwo{§11} And finally the LGPL-2.1 allows the original
copyright holder to \enquote{add an explicit geographical distribution limitation
excluding [\ldots] countries} provided that these countries \enquote{[\ldots]
(restict) the distribution and/or use of the library [\ldots] by patents
[\ldots]}\citeLGPLtwo{§12} Based on these statements, one cannot infer that 
the LGPL grants any patent rights to the software user, neither directly, nor
indirectly.

Thus, the LGPL-2.1 is neither a granting nor revoking license.

\subsubsection{LGPL-3.0}
\patentlabel{LGPL3}

The LGPL-3.0 is an extension of the GPL-3.0. Before starting with a section
\enquote{Additional Definitions}, the LGPL-3.0 states that it \enquote{[\ldots]
incorporates the terms and conditions of version~3 of the GNU General Public
License} and then \enquote{supplements} this GPL-3.0 content by some
\enquote{additional permissions.}\citeLGPLthree{wp} The LGPL-3.0 itself does not
contain the word `patent,' but the GPL-3.0 does.\citeGPLthree{§11}
So, the LGPL-3.0 inherits its patent clause from the GPL-3.0 which is---as we
already described\footnote{$\rightarrow$ \oslic{}, p.\
\patentpageref{GPL3}}---a granting and a revoking license.
 
\subsection{MPL statements concerning patents}
\patentlabel{MPL}

The MPL distributes its statements concerning the tolerated use of the patents
over three paragraphs: First, it clearly says that \enquote{each Contributor
[\ldots] grants [\ldots the licensee] a world-wide, royalty-free,
non-exclusive license [\ldots] under Patent Claims of such Contributor to
make, use, sell, offer for sale, have made, import, and otherwise transfer
either its Contributions or its Contributor Version}\citeMPL{§2.1,
esp. §2.1.b} Second, it hihlights some \enquote{limitations.}\citeMPL{§2.3}
And finally, the MPL introduces a revoking clause which signifies that the
rights, granted to the licensee \enquote{[\ldots] by any and all Contributors
[\ldots] shall terminate} if the licensee \enquote{initiates litigation
against any entity by asserting a patent infringement claim [\ldots] alleging
that a Contributor Version directly or indirectly infringes any patent
[\ldots]}\citeMPL{§5.2}

Thus, the MPL is a granting license and a revoking license.

\subsection{MS-PL statements concerning patents}
\patentlabel{MSPL}

First, the MS-PL contains a statement, by which \enquote{[\ldots] each 
contributor grants (the software users) a non-exclusive, worldwide, royalty-free 
license under its licensed patents to make, have made, use, sell, offer for 
sale, import, and/or otherwise dispose of its contribution in the software or 
derivative works of the contribution in the software.}\citeMSPL{§2.B} Second,
the MS-PL says that \enquote{if you bring a patent claim against any
contributor[\ldots] your patent license from such contributor to the software
ends automatically.}\citeMSPL{§3.B} 

Thus, the MS-PL is a granting and a revoking license: At first you are granted
to use all patents of all contributors which are necessary to use the software
legally. But if you install any litigation concerning an infringement of
patents with respect to the software, then the rights granted to you are revoked. 

% \bibliography{../../../bibfiles/oscResourcesEn}

% Local Variables:
% mode: latex
% fill-column: 80
% End:

% Telekom osCompendium 'for being included' snippet template
%
% (c) Karsten Reincke, Deutsche Telekom AG, Darmstadt 2011
%
% This LaTeX-File is licensed under the Creative Commons Attribution-ShareAlike
% 3.0 Germany License (http://creativecommons.org/licenses/by-sa/3.0/de/): Feel
% free 'to share (to copy, distribute and transmit)' or 'to remix (to adapt)'
% it, if you '... distribute the resulting work under the same or similar
% license to this one' and if you respect how 'you must attribute the work in
% the manner specified by the author ...':
%
% In an internet based reuse please link the reused parts to www.telekom.com and
% mention the original authors and Deutsche Telekom AG in a suitable manner. In
% a paper-like reuse please insert a short hint to www.telekom.com and to the
% original authors and Deutsche Telekom AG into your preface. For normal
% quotations please use the scientific standard to cite.
%
% [ Framework derived from 'mind your Scholar Research Framework' 
%   mycsrf (c) K. Reincke 2012 CC BY 3.0  http://mycsrf.fodina.de/ ]
%


%% use all entries of the bibliography
%\nocite{*}

\section{Excursion: Why linking is a secondary criterium}
\label{sec:LinkingSecondary}
\footnotesize
\begin{quote}\itshape
Distributing statically or dynamically linked software is often discussed as a
problem (and sometimes as a solution) for acting compliantly. In this chapter,
we briefly discuss why this aspect can mostly be ignored and why it does not
help to determine the existence of a derivative work.
\end{quote}
\normalsize

In some earlier versions of the \oslic{}, its finder subclassified some use cases
with respect to the way an application was `composed' as a larger unit: In the
previous form for gathering the necessary information, the \oslic{} user had to
answer whether he \emph{was going to combine the received open source software
with other software components by linking them together statically, by linking
them dynamically, or by textually including (parts of) the open source software
into a larger unit}. Today, this question has totally been erased. The
authors could convince themselves that it is not necessary to consider this
aspect.

Of course, we know that being linked statically or dynamically is often and
deeply discussed by license experts.\footnote{Even on the \emph{European Legal
and Licensing Workshop, 2013} in Amsterdam, there was given an excellent lecture
concerning the nature and concequences of linking elf files.} It seems to be an
important aspect:

[TBD: Discussion of the literature]
%TODO Discus statically dynamicall discussion.

So, let us start with some undeniable facts: The \oslic{} deals with 
the Apache-2.0 license,\citeAPL{}
the BSD 2-Clause license,\citeBSDsimple{} 
the BSD 3-Clause license,\citeBSDnew{} 
the MIT license,\citeMIT{} 
the MS-PL,\citeMSPL{} 
the PostgreSQL,\citePGL{}
and the PHP license\citePHP{} 
as instances of permissive licenses.
Additionally, the \oslic{} treats 
the EPL,\citeEPL{} 
the EUPL,\citeEUPL{} 
the LGPL,%
  \footnote{For LGPL-2.1 see \cite[cf.][\nopage wp]{Lgpl21OsiLicense1999a}. 
  For LGPL-3.0 see \cite [cf.][\nopage wp]{Lgpl30OsiLicense2007a}.} 
and the MPL\citeMPL{}
as licenses with weak copyleft. Finally, the \oslic{} thoroughly discusses 
the GPL%
  \footnote{For GPL-2.0 see \cite [cf.][\nopage wp]{Gpl20OsiLicense1991a} 
  For GPL-3.0 see \cite [cf.][\nopage wp]{Gpl30OsiLicense2007a} } 
and the AGPL\citeAGPL{}
as licenses with strong copyleft.%
  \footnote{You can find html based instances of these licenses in the
    \oslic{} directory `licenses.' They have been downloaded from the
    OSI pages. All of the following statements refer to these files.}

% Backslash in typewriter type for OT1/T2; 
% \textbackslash will substitute a non-tt backslash
\newcommand{\ttbs}{\symbol{'134}}

Only three of these licenses mention the word \emph{linking} (or variants of
it): Using the command \texttt{grep -i link * | grep -v
"<link\ttbs|links\ttbs|skip-link"} in a shell---executed
as an operation on a set of html formatted license files---directly shows that
only the AGPL-3.0, the Apache-2.0, the GPL-2.0, the GPL-3.0, the LGPL-2.1 and
the LGPL-3.0 are using mutations of the word \emph{linking}. Additionally, the
results of the command \texttt{grep -i statical *} show that only the LGPL-2.1
uses the word `statical,' while using the command \texttt{grep -i dynamical *}
only hints to the AGPL-3.0 and the GPL-3.0. Finally, the command \texttt{grep -i
"shared" *}---executed on the same set of files---shows that the term
\emph{shared libary} is also only used by these licenses.

This analysis already indicates that being statically or dynamically linked
might not be as important for acting compliantly as it is often suggested.
% 
If one reads the concrete statements, then one can see, that acting compliantly
depends only slightly and only rarely on the kind of being `combined':

\begin{description}

  \item[Apache-2.0:] This version of the Apache license uses the word
  \emph{link} only once for stating that \enquote{[\ldots] Derivative Works shall not
  include works that remain separable from, or merely link [\ldots] to the interfaces of, 
  the Work and Derivative Works thereof.}\footcite [cf.][\nopage wp.\
  §0]{Apl20OsiLicense2004a} Thus, the Apache-2.0 does not use the criteria \emph{being
  linked} for determining a derivative work, neither \emph{being linked} in
  general, nor \emph{being statically linked}, nor being \emph{dynamically
  linked}. Hence, for acting in accordance to the Apache-2.0, this class of attributes
  can completely be ignored.

  \item[GPL-3.0:] The GPL-3.0 uses the word \emph{link} three times: First, it
  defines the \enquote{\enquote{Corresponding Source} for a work in object code
  form [\ldots as] all the source code needed to generate, install, and [\ldots]
  run the object code and to modify the work [\ldots]}. Additionally the GPL-3.0
  also explains in this context that this definition shall include
  \enquote{[\ldots] the source code for shared libraries and dynamically linked
  subprograms that the work is specifically designed to
  require}\footcite[cf.][\nopage wp.\ §0]{Gpl30OsiLicense2007a}. Second, the
  GPL-3.0 allows \enquote{[\ldots] to link or combine any covered work with a
  work licensed under version~3 of the GNU Affero General Public License into a
  single combined work, and to convey the resulting work.}\footcite[cf.][\nopage
  wp.\ §13]{Gpl30OsiLicense2007a} Finally, the GPL-3.0 explains that
  \enquote{the GNU General Public License [itself] does not permit incorporating
  your program into proprietary programs} and that the LGPL might be a better
  license for those licensors who have written a \enquote{subroutine library
  [\ldots] and may consider it more useful to permit linking proprietary
  applications with the library [\ldots]}\footcite[cf.][\nopage wp.\ last
  parapgraph]{Gpl30OsiLicense2007a}.
  
  So, also in this text, the features \emph{statically linked} or
  \emph{dynamically linked} are not used to trigger any license fulfilling
  actions. The conditions for \enquote{Conveying Modified [\ldots] Versions}
  refer to the \enquote{work based on the Program}\footcite[cf.][\nopage wp.\
  §5]{Gpl30OsiLicense2007a} which itself denotes a \enquote{\enquote{modified
  version} of the earlier work}\footcite[cf.][\nopage wp.\
  §0]{Gpl30OsiLicense2007a}. Moreover, the licensee---as modifier, distributor,
  and subsequent licensor---is required by the GPL-3.0 \enquote{[\ldots] to
  license the entire work [which has been developed on the base of a GPL-3.0
  component], as a whole, under this License to anyone who comes into possession
  of a copy}\footcite[cf.][\nopage wp.\ §5]{Gpl30OsiLicense2007a}. The GPL-3.0
  does not limit this claim---especially not by referring to a mode of being
  linked. Hence, also with respect to the GPL-3.0, one can completely ignore
  these features of the software, its use and its distribution for determining
  how to use the software compliantly.

  \item[AGPL-3.0:] Concerning the use and the meaning of the words
  \emph{dynamically} and \emph{linking}, the AGPL-3.0 exactly follows the
  structure of the GPL-3.0: first the terms arise in the context of defining the
  \enquote{Corresponding Source};\footcite[cf.][\nopage wp.\
  §0]{Agpl30OsiLicense2007a} then the word \emph{link} helps to say that AGPL
  and GPL are compatible licenses;\footcite[cf.][\nopage wp.\
  §13]{Agpl30OsiLicense2007a} and finally the word \emph{link} is used to hint
  to the LGPL.\footcite[cf.][\nopage wp.\ §5]{Agpl30OsiLicense2007a} So, again,
  one can ignore the feature of being statically or dynamically linked if one
  wants to determine how to use the software compliantly.

  \item[GPL-2.0:] In the GPL-2.0, the word \emph{link} only arises in the context
  of hinting to the LGPL.\footcite [cf.][\nopage wp.\ last
  paragraph]{Gpl20OsiLicense1991a} Moreover, the words \emph{statical} and
  \emph{dynamical} are not used in this text---not at all and in no sense: the
  copy left feature of the GPL depends `only' on a specification which refers to
  a \enquote{work based on the Program [\ldots] that in whole or in part
  contains or is derived from the Program or any part thereof [\ldots]}\footcite
  [cf.][\nopage wp.\ §2]{Gpl20OsiLicense1991a} Thus, even in this old version
  of the GPL, the criteria of being linked---in which way ever---does not
  trigger any task for using the software compliantly.

  \item[LGPL-3.0:] In this license, variants of the word \emph{link} are used to
  define the concept of a \enquote{Combined Work} which shall be the name for a
  \enquote{[\ldots] work produced by combining or linking an Application with
  the Library.}\footcite [cf.][\nopage wp.\ §0]{Lgpl30OsiLicense2007a} In the
  end the LGPL-3.0 allows to \enquote{[\ldots] convey a Combined Work under
  terms of your choice [\ldots]}, provided that one distributes also all
  material (including the object files of the overarching on-top developments)
  necessary for enabling the receiver to relink the whole product with a
  later version of the library or that one presupposes the use of
  a \enquote{suitable shared library mechanism} so that the receiver can update
  the library simply by replacing the binary library file\footcite[cf.][\nopage
  wp.\ §4]{Lgpl30OsiLicense2007a}. For fulfilling these conditions it is
  sufficient to require that a distributor shall \emph{either distribute the
  on-top development and the library in the form of dynamically linkable parts
  or distribute the statically linked application together with a written offer,
  valid for at least three years, to give the user all object-files of the
  on-top development and the library, so that he can relink the application on
  its own behalf}.

  \item[LGPL-2.1:] Even if the LGPL-2.1 argues more sophistically than all
  the other licenses, in its preamble this license clearly states what it wants
  to evoke: \enquote{If you link other code with the library, you must provide
  complete object files to the recipients, so that they can relink them with the
  library after making changes to the library and recompiling it.
  [\ldots]}\footcite[cf.][\nopage wp.\ preamble]{Lgpl21OsiLicense1999a} For
  that purpose, the LGPL-2.1 defines at the beginning that if \enquote{a program
  is linked with a library, whether statically or using a shared library, [then]
  the combination of the two is legally speaking a combined work, a derivative
  of the original library}:\footcite[cf.][\nopage wp.\
  preamble]{Lgpl21OsiLicense1999a} On the one hand a \enquote{work that uses
  the Libary}---which is only \enquote{[\ldots] designed to work with the
  Library by being compiled or linked with it [\ldots]}---\enquote{[\ldots] in
  isolation, is not a derivative work of the library [\ldots]}. On the other
  hand, it is no question for the LGPL-2.1, that \enquote{linking a
  \enquote{work that uses the Library} with the Library creates an executable
  that is a derivative of the Library (because it contains portions of the
  Library).}\footcite[cf.][\nopage wp.\ §5]{Lgpl21OsiLicense1999a} But then---%
  \enquote{as an exeption}---the LGPL-2.1 allows to \enquote{[\ldots] combine
  or link a \enquote{work that uses the Library} with the Library to produce a work
  containing portions of the Library, and distribute that work under terms of
  your choice}. The right to do this is granted provided that the distributor
  either presupposes the use of a \enquote{suitable shared library mechanism} or
  that he distributes also the complete material (including the object files of
  the overarching on-top developments) which is necessary to enable the receiver
  to relink the whole product with a later incoming newer version of the
  library\footcite[cf.][\nopage wp.\ §6, §6b and §6c together with
  §6c]{Lgpl21OsiLicense1999a}. Again, for fulfilling all these conditions it is
  sufficient to require that a distributor shall \emph{either distribute the
  on-top development and the library in the form of dynamically linkable parts
  or distribute the statically linked application together with a written offer,
  valid for at least three years, to give the user all object-files of the
  on-top development and the library, so that he can relink the application on
  its own behalf}.

\end{description}

Thus, with respect to this analysis, we can conclude that---in general---there
is no need to investigate whether one wants to distribute software in the form 
of statically or dynamically linked binaries for deriving the necessary tasks 
to distribute this software compliantly. 
%% RPD FIXME: what are the following two sentences supposed to mean?
Instead of this, we can directly incorporate those doings into the task lists 
of the LGPL what has been discovered as sufficient doings. 
Moreover, it is also sufficient to insert this statement only in the task list 
of the LGPL. 
There is no need to generalize this discussion. 
So, we could simplify our form offered to gather the information to find the 
adequate license fulfilling task list.


%\bibliography{../../../bibfiles/oscResourcesEn}

% Local Variables:
% mode: latex
% fill-column: 80
% End:

% Telekom osCompendium 'for being included' snippet template
%
% (c) Karsten Reincke, Deutsche Telekom AG, Darmstadt 2011
%
% This LaTeX-File is licensed under the Creative Commons Attribution-ShareAlike
% 3.0 Germany License (http://creativecommons.org/licenses/by-sa/3.0/de/): Feel
% free 'to share (to copy, distribute and transmit)' or 'to remix (to adapt)'
% it, if you '... distribute the resulting work under the same or similar
% license to this one' and if you respect how 'you must attribute the work in
% the manner specified by the author ...':
%
% In an internet based reuse please link the reused parts to www.telekom.com and
% mention the original authors and Deutsche Telekom AG in a suitable manner. In
% a paper-like reuse please insert a short hint to www.telekom.com and to the
% original authors and Deutsche Telekom AG into your preface. For normal
% quotations please use the scientific standard to cite.
%
% [ Framework derived from 'mind your Scholar Research Framework' 
%   mycsrf (c) K. Reincke 2012 CC BY 3.0  http://mycsrf.fodina.de/ ]
%

% Apache 
% ----------------
% "Derivative Works" shall mean any work, whether in Source or Object
% form, that is based on (or derived from) the Work and for which the
% editorial revisions, annotations, elaborations, or other
% modifications represent, as a whole, an original work of
% authorship. For the purposes of this License, Derivative Works shall
% not include works that remain separable from, or merely link (or
% bind by name) to the interfaces of, the Work and Derivative Works
% thereof. [Preamble]
%
% BSD
% ----------------
% with or without modification
%
% MPL 2.0
% ----------------
% 1.10. “Modifications”
%
%   means any of the following:
%
%   a.  any file in Source Code Form that results from an addition to,
%       deletion from, or modification of the contents of Covered
%       Software; or
%
%   b.  any new file in Source Code Form that contains any Covered
%       Software.
%
% CDDL
% ----------------
% 1.6. Larger Work means a work which combines Covered Software or
%      portions thereof with code not governed by the terms of this
%      License. 
%
%  1.9. Modifications means the Source Code and Executable form of any of the following:
%
%   A. Any file that results from an addition to, deletion from or
%      modification of the contents of a file containing Original
%      Software or previous Modifications; 
%
%   B. Any new file that contains any part of the Original Software or
%      previous Modification; or 
%
%   C. Any new file that is contributed or otherwise made available
%      under the terms of this License. 
%
% EPL
% ----------------
% Contributions do not include additions to the Program which: (i) are
% separate modules of software distributed in conjunction with the
% Program under their own license agreement, and (ii) are not
% derivative works of the Program.  
%
% GPL
% ----------------
% §0. The "Program", below, refers to any such program or work, and a
% "work based on the Program" means either the Program or any
% derivative work under copyright law: that is to say, a work
% containing the Program or a portion of it, either verbatim or with
% modifications and/or translated into another language.
%
% §2. Thus, it is not the intent of this section to claim rights or
% contest your rights to work written entirely by you; rather, the
% intent is to exercise the right to control the distribution of
% derivative or collective works based on the Program. 
%
% GPL 3
% ----------------
% §0. To “modify” a work means to copy from or adapt all or part of
% the work in a fashion requiring copyright permission, other than the 
% making of an exact copy. The resulting work is called a “modified
% version” of the earlier work or a work “based on” the earlier work. 
%
% §5c. You must license the entire work, as a whole, under this
% License to anyone who comes into possession of a copy. This License
% will therefore apply, along with any applicable section 7 additional
% terms, to the whole of the work, and all its parts, regardless of how
% they are packaged.
%
% §5 at the end:
% A compilation of a covered work with other separate and independent
% works, which are not by their nature extensions of the covered work,
% and which are not combined with it such as to form a larger program,
% in or on a volume of a storage or distribution medium, is called an
% “aggregate” if the compilation and its resulting copyright are not
% used to limit the access or legal rights of the compilation's users
% beyond what the individual works permit. Inclusion of a covered work
% in an aggregate does not cause this License to apply to the other
% parts of the aggregate.
%
% http://www.gnu.org/licenses/gpl-faq.html#MereAggregation

%% use all entries of the bibliography
%\nocite{*}

{
\newcommand{\softbreak}{\hspace{0pt plus 1cm}}
\newcommand{\sourceNeeded}{\footnote{cite the sources}}

\section{Excursion: What is a 'Derivative Work' - the basic idea of open source}
\footnotesize \begin{quote}\itshape This chapter briefly discusses aspects of
being a derivated pieces of software which have to be known for using open
source software compliantly. As usually, the \oslic{}
only tries to find one safe interpretation. The authors know that there
exist many other ways to consider this topic. So, if you feel, that the
viewpoint of the \oslic{} does not fit the specific circumstances of your
particular case, do not hesitate to ask your own lawyer. But if you agree with
the \oslic{}, be aware that you dealing with this topic from the viewpoint of a
benevolent user.
\end{quote}
\normalsize
Let us outline the argumentation:

\begin{description}
  \item[The meaning `derivative work' must be known!]\softbreak
    Many open source licenses use the term `derivative work,'\sourceNeeded
    either directly or indirectly in form of the word `modification.'\sourceNeeded
    [Write a table as survey] 
    And nearly all licenses that are using the term `derivative work' etc., 
    are linking tasks that must be executed to comply with the corresponding
    license, to the precondition that something is a derivative work. 
    [table survey] 
    \textbf{Hence, for acting in accordance with such a license, it has to be
    known what a derivate work is.}  

  \item[Unfortunately the meaning is not clearly fixed.]\softbreak
    There exist different readings of the term `derivative work.' 
    [specify the differences and cite the sources] 
    \textbf{Hence, it is not as clear what a derivative work is as one could wish}

  \item[So, let us argue from the viewpoint of a benevolent developer:]\softbreak
    Open source licenses are written for software developers, mostly to preserve
    their freedom to develop software. And sometimes these licenses are also
    written by software developers---or at least with their assistance. So, one
    should be able to answer the question under which circumstances a piece of
    software is a `derivative work' of another piece of software based on two
    principles: 
  \begin{itemize}
    \item Let us argue from the viewpoint of a benevolent neutral software
      developer without hidden interests or a hidden agenda.
    \item In case of doubts let us preferably assume that the two pieces
      interrelate as source and derivative work---so that the \oslic{} rather
      recommends to perform the required tasks.
  \end{itemize}
\end{description}

We generalize a specific viewpoint of the LGPL. It uses three terms:

\begin{description}
  \item[\enquote{library}] is defined as \enquote{a collection of software
  functions and/or data prepared so as to be conveniently linked with
  application programs.}\footcite[cf.][\nopage wp §0]{Lgpl21OsiLicense1999a}
  \item[\enquote{work based on the library}] is defined as \enquote{either the
  library or any derivative work.}\footcite[cf.][\nopage wp
  §0]{Lgpl21OsiLicense1999a}
  \item[\enquote{work that uses the library}] is defined as something which
  initially \enquote{[\ldots] is not a derivative work of the library [\ldots]}
  but can become a derivative work by being combined / linked to the library it
  uses.\footcite[cf.][\nopage wp §5]{Lgpl21OsiLicense1999a}
\end{description}

Following these specifications, one has to conclude that 
\emph{derivative works} of the library can be drieved in two different ways: 
First, the library itself can be enhanced without changing the character of 
being a library. Then, of course, the resulting library is a derivative work 
of the initial library.  Second, an overaching program can use the library by 
calling functions, methods, or data offered by the library. In this case, the 
overarching program functionally depends on the library and is a derivative work 
(as soon as it is linked to the library).

This viewpoint can be generalized: snippets, modules, plugins can be
enhanced and used by overarching programs or even by more complex libraries.
Based on this viewpoint---which should finally be formulated as the viewpoint of
a benevolent impartial developer---the \oslic{} uses the following rules by which
the \oslic{} decides to take something as derivative Work:
\label{sec:BenevolentDerivativeWorkUnderstanding}

\begin{description}
  \item[Copy-Case] Copying a piece of code from a source file and pasting it
  into a target file makes the target file a derivatve work of the source
  file.\footnote{Be careful: this case must still be distinguished from the case
  of an automatic inclusion (header files, script libraries) during the
  compilation / execution: Inclusion of header files alone should not create a derivative
  work.}
  \item[Modify-Case] Inserting any new content or deleting any existing content
  of a source file makes the resulting target file a derivate work of the
  source file.
  %% RPD: what is the rationale for the call-case?
  \item[Call-Case] Inserting the call of function which is defined inside of and 
  delivered by a sourcefile into a target file makes that target file
  depending on the source file and therefore a derivative work of the delivering
  source file.
\end{description}

And here are some applications of these rules:

\begin{itemize}
  \item \textbf{Enlarging an existing source file by an external text creates a
  derivative work!} Why? \emph{Because you are going to reuse the
  external code for simplifying our life.} [see 'Copy Case']
  \item \textbf{Reducing a source file creates a derivative work!} Why?
  \emph{Because you are going to prepare the given file(s) for a better reuse.}
  [see 'Modify-Case']
  \item \textbf{Replacing something in a source file creates a derivative work!}
  Why? \emph{Because you are going to reuse parts of the existing code for
  simplifying your life.} [see 'Modify Case']
  \item \textbf{Integrating a foreign snippet into an existing source code
  creates a derivative work!} Why? \emph{Because you are going to simplify
  your life by reusing both, the foreign snippet and the original file.} [see
  'Copy Case' and 'Modify-Case']
  %% RPD The following example claims that if you split a file A into a
  %% library part L (offering a function that was part of A)  and a client 
  %% part C (containing everything in A that is not in L) then
  %% - C and L are derived from A (obviuosly)
  %% - C is derived from L (call-case; but this makes it clear, that there 
  %%   is no such thing as a call case.)
  \item \textbf{Refactoring a given work by extracting a function / method into
  an autonomous file creates a derivative work in two respects!} Why?
  \emph{Because, first, all modified / generated files depend
  on the original file and, second, because those function calls in the files
  introduce a dependecy on the file defining the function itself.}
  [see 'Modify-Case' and 'Call-Case]
  \item \textbf{Calling a function - served by a defining module - lets the
  calling file become a derivative work of the serving module!} Why?
  \emph{Because you are going to simplify your life by reusing an already
  prepared work (often offered by other developers).} [see 'Call-Case']
  \item \textbf{By calling elements - served by a defining library - the
  calling file becomes a derivative work of the serving library!} Why? 
  \emph{Because you are going to simplify your life by reusing an already
  prepared work (often offered by other developers).} [see
  'Call-Case']\footnote{In this context, you may sometimes read that one has to
  differentiate the defining file (for example the C-code) and the declaring
  file (for example the C-Header). But in our view, it is not so important to
  make such a difference: The using file, which includes the declaring header
  file depends on the defining source code file ('Call-Case'). So, one can
  ignore the formal dependance on the declaring header file ('Copy-Case').}
\end{itemize}

And now some additional 'ideas' which might invite to be discussed:

\begin{itemize}
  \item \textbf{Does a plugin depend on its framework? No.} Why? \emph{
  Because it is like a module: it offers a function (normally without using
  a function, offered by the framework itself).}
  \item \textbf{Does a framework depend on its plugin? Let us try to answer:
  Sometimes yes, sometimes no.} Why? \emph{If the framwork crashes when it is
  missing its plugin, then it clearly depends on the plugin. No doubt. It is
  simply not autonomous. But if it does not crash, then it perfectly does for
  which it has been designed: it is offering a place which might be filled by
  the plugin, but not necessarily. This kind of a framework is like an
  application listing to a port for getting data which it shall process and
  which are served by another application.}
  \item \textbf{Does a program using inter process communication depend on its
  IO-partners? Definitely not!} Why? \emph{Because, otherwise, we need not discuss
  all these cases, every thing would depend on everything---in each running system.}
\end{itemize}

[\ldots TBD \ldots]

}
%\bibliography{../../../bibfiles/oscResourcesEn}

% Local Variables:
% mode: latex
% fill-column: 80
% End:


\section{Excursion: Reverse Engineering and Open Source}
% Telekom osCompendium 'for being included' snippet template
%
% (c) Karsten Reincke, Deutsche Telekom AG, Darmstadt 2011
%
% This LaTeX-File is licensed under the Creative Commons Attribution-ShareAlike
% 3.0 Germany License (http://creativecommons.org/licenses/by-sa/3.0/de/): Feel
% free 'to share (to copy, distribute and transmit)' or 'to remix (to adapt)'
% it, if you '... distribute the resulting work under the same or similar
% license to this one' and if you respect how 'you must attribute the work in
% the manner specified by the author ...':
%
% In an internet based reuse please link the reused parts to www.telekom.com and
% mention the original authors and Deutsche Telekom AG in a suitable manner. In
% a paper-like reuse please insert a short hint to www.telekom.com and to the
% original authors and Deutsche Telekom AG into your preface. For normal
% quotations please use the scientific standard to cite.
%
% [ Framework derived from 'mind your Scholar Research Framework' 
%   mycsrf (c) K. Reincke 2012 CC BY 3.0  http://mycsrf.fodina.de/ ]
%


%% use all entries of the bibliography
%\nocite{*}

Beyond any doubt, the LGPL mentions \enquote{reverse engineering}
literally\footnote{For the LGPL-v2 \cite[cf.][\nopage wp., 
§6]{Lgpl21OsiLicense1999a}; for the LGPL-v3 \cite[cf.][\nopage wp., 
§4]{Lgpl30OsiLicense2007a} } for indicating that reverse engineering in any
respect must be allowed to use and distribute LGPL software compliantly:

\begin{quote}\noindent\emph{\enquote{[\ldots] you may [\ldots] distribute a work
(containing portions of the Library) under terms of your choice, provided that
the terms permit [\ldots] \emph{reverse engineering} [\ldots]}
\footnote{\cite[cf.][\nopage wp, §6]{Lgpl21OsiLicense1999a}. The LGPL-v2 uses
the capitalized word \enquote{Library} for denoting a library which
\enquote{[\ldots] has been distributed under (the) terms} of the LGPL-v2.
(\cite[cf.][\nopage wp, §0]{Lgpl21OsiLicense1999a}). Inside of our LGPL
chapter(s) we follow this interpretation. } }
\end{quote}

There are three strategies for dealing with such provisions: one can try to
fully honor its meaning, one can mitigate its meaning, or one can avoid to
discuss this requirement altogether:

A first group of well known open source experts take the sentence of the LGPL-v2
as a strict rule which requires that one has to allow reverse engineering of the
whole software product if one embeds any LGPL licensed component into that
product\footnote{For example, a very trustworthy German expert states that the
LGPL-2.1 generally requires that a distributor of a program which accesses a
LGPL-2.1 licensed library, must grant his customer also the right to modify the
accessing program and hence also the right to execute reverse engineering.
Literally the German text says:
\begin{quote}\enquote{Ziffer 6 LGPLv2.1 knüpft die Erlaubnis, das zugreifende
Programm unter beliebigen Lizenzbestimmungen verbreiten zu drüfen, an eine Reihe
von Verpflichtungen, die in der Praxis oft übersehen werden: Zunächst muss dem
Kunden, dem die Software geliefert wird, die Veränderung des zugreifenden
Programms gestattet werden und zu diesem Zweck auch ein Reverse Engineering zur
Fehlerbehebung. \emph{Dies dürfte alle Formen des Debugging und das
Dekompilieren des zugreifenden Programms umfassen}.} (\cite[cf.][81; emphasis
KR]{JaeMet2011a}).\end{quote} At first glance, also \enquote{copyleft.org} --
the \enquote{[...] collaborative project to create and disseminate useful
information, tutorial material, and new policy ideas regarding all forms of
copyleft licensing} (\cite[cf.][\nopage wp.]{CopyLeftOrg2014a}) -- could be
taken as another witness for such an attitude of strict reading: Some of its
contributors elucidate in a chapter dealing with \enquote{special topics in
compliance} that \enquote{the license of the whole work must [sic!] permit
\enquote{reverse engineering for debugging such modifications} to the library}
and that one therefore \enquote{ should take care that the EULA used for the
Application does not contradict this
permission}(\cite[cf.][86]{KuhSebGin2014a}}.

A second group of well known and knowledgeable open source experts signify that
the LGPL-v2 indeed literally contains a strict rule, but that this rule actually
is not meant as it sounds: For example, two of these experts explain that
\enquote{these requirements on the licensed combination require that the license
chosen not prohibit the customer’s modification and reverse engineering for
debugging these modifications in the work as a whole}. But then they directly
add the limitation, that \enquote{in practice, enforcement history suggests, it
means that the license terms chosen may not prohibit modification and reverse
engineering for debugging of modification in the LGPL’d code included in the
combination}\footnote{\cite[cf.][\nopage wp., chapter LGPLv2.1, section
6]{MogCho2014a}. Such a mitigation can also be found in the tutorial of
copyleft.org: After they have summarized the LGPL-v2 sentence as a strict rule,
they directly continue, that one \enquote{[\ldots] must refrain from license
terms on works based on the licensed work that prohibit replacement of the
licensed components of the larger non-LGPL'd work, or prohibit decompilation or
reverse engineering in order to enhance or fix bugs in the LGPL'd components}
(\cite[cf.][86]{KuhSebGin2014a}). This added specification indicates, that one
only has to facilitate the modification of the library and that reverse
engineering can be ignored as long as there are other ways to improve the
embedded library.}.

Finally, a third group of experts prefers not to discuss the problem of reverse
engineering, although this technique is literally mentioned in the license and
although they want explain how to use GPL/LGPL licensed software
compliantly\footnote{An article of Terry J. Ilardi might be taken as a first
witness of this third strategy: he profoundly explains the essence of the LGPL,
he especially discusses §6, and he delivers applicable rules like \enquote{DO
NOT statically link to LGPL [\ldots] code if you wish to keep your program
proprietary}. But he does not discuss \emph{reverse engineering}
(\cite[cf.][5f]{Ilardi2010a}). Similarily argues Rosen
(\cite[cf.][121ff]{Rosen2005a}). And -- despite their comments on reverse
engineering in the specific chapter \emph{special topics in compliance} -- the
copyleft.org document can also be taken as an instance of this attitude:
Although its authors recommend to \enquote{study chapter 10 carefully} for
establishing an adequate \enquote{compliance with LGPLv2.1}
(\cite[cf.][86]{KuhSebGin2014a}), this chapter 10 -- dedicated to the meaning of
the \enquote{Lesser GPL} -- does not deal with reverse engineering, although it
discusses the §6 of the LGPLv2.1 in depth (\cite[cf.][56ff, esp.
60f]{KuhSebGin2014a}).}.

This situation must bother companies and people who want to use open source
software compliantly and who therefore are looking for guidance. Particularly it
disturbs those who want to protect their business relevant software. At the end,
they might consider that this sentence is not consistently understood by the
open source community itself. And -- as far as we know -- at least some of these
companies preventively prohibit their developers to embed LGPL licensed
components into programs which contain business relevant techniques.
Unfortunately, this consequence does not only obstruct the access to a large set
of well written free software, but it is scarcely possible to obey such an
interdiction consequently: The glibc, which enables the programs to talk with
the kernel of the GNU/Linux system\footnote{cf.
http://www.gnu.org/software/libc/}, is licensed under the LGPL\footnote{cf.
http://en.wikipedia.org/wiki/GNU\_C\_Library}. And hence, this library is
indirectly linked to or combined with any program running on the GNU/Linux
system. So, if the LGPL-v2 indeed required, that reverse engineering of every
program must be allowed, which contains portions of any LGPL Library, then every
GNU/Linux user would be allowed to examine every program running on GNU/Linux by
\emph{reverse engineering}, simply, because finally every 'GNU/Linux program' is
linked to or combined with the glibc\footnote{This conclusion might surprise.
But it is inferred with exactly the same arguments as the conclusion, that
without a licence offering a weaker copyleft every program would have been
licensed under the GPL. The copyleft.org document explains this argumentation in
great detail (\cite[cf.][56f]{KuhSebGin2014a}).}. In other words: if the LGPL
indeed required the permission of reverse engineering, then
every program executed on GNU/Linux may be reverse engineered.

But an exhaustive reading of the LGPL-v2 strongly indicates, that there must be
another valid, more 'liquid' understanding of the LGPL: The preamble explains
the reason for offering another weaker license beside the GPL. It says that
\enquote{[\ldots] on rare occasions, there may be a special need to encourage
the widest possible use of a certain library, so that it becomes a de-facto
standard} and that therefore it could be strategicly necessary to \enquote{allow
[\ldots] non-free programs [\ldots] to use the library} without enforcing that
these programs become free software too\footcite[cf.][\nopage wp,
§preamble]{Lgpl21OsiLicense1999a}.

So, if the LGPL had indeed determined that every program linked to or combined
with any LGPL library may be reverse engineered, then the LGPL would have an
effect contrary to its own intention. It would have introduced something like
\emph{'security by obscurity'}: First, the LGPL would allow to protect the
internals of your own work against investigation by allowing to keep the code of
the non-free program using the library a scecret\footnote{The weak copyleft has
been introduced for encouring the widest possible use of the library.}. But then
-- in the end -- the LGPL would also allow the user to reverse engineer the
received binary and hence would enable him to nevertheless discover all
internals\footnote{It would only cost a little more effort - as security by
obscurity indicates.}. Hence, finally the LGPL-v2 would undermine its own
raison d'$\grave{e}$tre introduced by its inventors: under such circumstances
there probably would have been less hope that any LGPL library could have become
a defacto standard.

We know that the inventors of the GNU licenses and GNU software are very
sophisticated experts. They never would have published such an inconsistent
document. So, this dissent read in(to) the document is a strong indicator for
the fact, that there must be a better way to understand the license. And thus,
it is up to us, the followers, to explicate a more adequate interpretation. Of
course, such an interpretation must be grounded on the written text. No doubt:
we, the scholars, are not allowed to add our own wishes. We must read the
license very strictly. We have to deduce 'understandings' only by matching the
interpretations explicitly and reasonably back to the license text itself.

\label{RevEngOslicOsLisences}
Encouraged by the indication that a better understanding of the LGPL may exist
and contrary to the other strategies, we are going to prove that there is a
valid way to compliantly distribute any open source based software without
permitting reverse engineering: We want to show that none of the main open
source licenses\footnote{Just as the OSLiC, also this part focuses only on the
most important open source licenses (cf.
https://www.blackducksoftware.com/resources/data/top-20-open-source-licenses
wp.): the Apache license (\cite[cf.][\nopage wp.]{Apl20OsiLicense2004a}), the
BSD licenses (\cite[cf.][\nopage wp.]{BsdLicense3Clause} and \cite[cf.][\nopage
wp.]{BsdLicense2Clause})), the MIT license (\cite[cf.][\nopage
wp.]{MitLicense2012a}), the MS-PL (\cite[cf.][\nopage
wp.]{MsplOsiLicense2013a}), the PostgreSXQL (\cite[cf.][\nopage
wp.]{PglOsiLicense2013a}), the PHP license (\cite[cf.][\nopage
wp.]{Php30OsiLicense2013a}), the EPL (\cite[cf.][\nopage
wp.]{Epl10OsiLicense2005a}), the EUPL (\cite[cf.][\nopage
wp.]{Eupl11OsiLicense2007a}), the MPL (\cite[cf.][\nopage
wp.]{Mpl20OsiLicense2013a}), the LGPLs (\cite[cf.][\nopage
wp.]{Lgpl21OsiLicense1999a} and \cite[cf.][\nopage wp.]{Lgpl30OsiLicense2007a}),
the GPLs (\cite[cf.][\nopage wp.]{Gpl20OsiLicense1991a} and \cite[cf.][\nopage
wp.]{Gpl30OsiLicense2007a}) and the AGPL (\cite[cf.][\nopage
wp.]{Agpl30OsiLicense2007a})} requires reverse engineering if the work using
the open source Library is distributed in form of dynamically linkable files. In
particular, we are going to prove that one even has not to allow reverse
engineering of the work using an LGPL Library, if one distributes that work as
dynamically linkable files. And we want to show that in all other cases one has
at least to fear that one has implicitly allowed the reverse engineering of the
work using the open source Library if one distribute that work. In particular,
we want to prove that one has to fear this implicitly given permission even if
one distributes a work using a library licensed under any permissive
license\footnote{By the way, our analysis should also provide proof that the
LGPL is not something like a 'poisoned' license containing \enquote{an
imprenetrable maze of technology babble} which \enquote{[\ldots] should not be
in a general-purpose software license} (\cite[cf.][124]{Rosen2005a}). The
challenge of the today's descendants is to understand the former inventors of
the GNU licenses and their way to think about computing - including all the
hassle the computing language C might provoke.}.

In general, we hope that our analysis, grounded on the license text itself, will
support companies and people to compliantly use open source software more often
and with less hesitation due to the fear that they have to deliver themselves
unclear textual aspects.

But -- with respect to the discussion about this text in the OSI and Free
Sofwtare Mailing lists -- we have to add a disclaimer here: The license text
alone is not all. In the concrete situation of using free and open source
software, it is the intention of the licensor which has to be respected. Or in
the words of Eben Moglen:

\begin{quote}\noindent\emph{A license is, by definition, a unilateral permission
to make use of the property or intangible rights of another. The measure of the
permission is the intention of the party giving it.}\footnote{Eben
Moglen, eMail to ftf-legal-bounces@fsfeurope.org, 2015-03-06}\end{quote}

Nevertheless, we believe that each text firstly has its own inherent independent
meaning and message. But -- of course, in the specific situation of legally
contending about the practical consequences of a license, one has indeed to
consider what the specfic licensor really had had in his mind, when he released
his work. One has to consider his intention.

So, the pure textual meaning of the license might be overloaded or overwritten
by some external facts, traditions or understandings, not founded on the license
text itself. The problem with this legal fact is, that in a concrete legal case,
one has to prove what the licensor really had in his mind. As long as we do not
have direct insights into the brain of our fellow human beings, this can again
only be done by referring to other orally uttered or written words and texts.
Therefore, we indeed believe, that it is firstly important to know what the text
itself says and means.

Hence, let us prove our position 'bottom up'. Let us firstly show that it is
true for the LGPL-v2 -- by explicating the license text lingually, then
logically, and finally empirically, before we infer the correct understanding.
Then let us show that it is also true for the LGPL-v3. And in the end let us
show that it is true for all other licenses\footnote{analysed by the OSLiC:
$\rightarrow$ p. \pageref{RevEngOslicOsLisences}.}.


\subsection{Reverse Engineering in the LGPL-v2}
% Telekom osCompendium 'for being included' snippet template
%
% (c) Karsten Reincke, Deutsche Telekom AG, Darmstadt 2011
%
% This LaTeX-File is licensed under the Creative Commons Attribution-ShareAlike
% 3.0 Germany License (http://creativecommons.org/licenses/by-sa/3.0/de/): Feel
% free 'to share (to copy, distribute and transmit)' or 'to remix (to adapt)'
% it, if you '... distribute the resulting work under the same or similar
% license to this one' and if you respect how 'you must attribute the work in
% the manner specified by the author ...':
%
% In an internet based reuse please link the reused parts to www.telekom.com and
% mention the original authors and Deutsche Telekom AG in a suitable manner. In
% a paper-like reuse please insert a short hint to www.telekom.com and to the
% original authors and Deutsche Telekom AG into your preface. For normal
% quotations please use the scientific standard to cite.
%
% [ Framework derived from 'mind your Scholar Research Framework' 
%   mycsrf (c) K. Reincke 2012 CC BY 3.0  http://mycsrf.fodina.de/ ]
%


%% use all entries of the bibliography
%\nocite{*}

The LGPL-v2.1 contains one sentence which literally refers to the issues of
\emph{reverse engineering}:

\begin{quote}\noindent\emph{\enquote{[\ldots] you may [\ldots] combine or link a
\enquote{work that uses the Library} with the Library to produce a work
containing portions of the Library, and distribute that work under terms of your
choice, provided that the terms permit modification of the work for the
customer's own use and \emph{reverse engineering} for debugging such
modifications.}\footnote{\cite[cf.][\nopage wp., §6. ]{Lgpl21OsiLicense1999a}.
The first ellipse in this citation -- notated by the string '[\ldots]' -- refers
to the phrase \enquote{As an exception to the Sections above,}, the second to
the phrase \enquote{also}. These words together want to indicate, that the LGPL
offers its §6-way-of-distribution as an exception to the intended default way of
distributing such a Library. So, the nature of the extraordinary way itself is
not affected by this hint. Thus, we feel free to erase this contextual
link.}}\end{quote}

Hereinafter, we will sometimes denote these lines by
the term \emph{LGPL2-RefEng-Sentence}.

\subsubsection{Linguistical Clarification}
% Telekom osCompendium 'for being included' snippet template
%
% (c) Karsten Reincke, Deutsche Telekom AG, Darmstadt 2011
%
% This LaTeX-File is licensed under the Creative Commons Attribution-ShareAlike
% 3.0 Germany License (http://creativecommons.org/licenses/by-sa/3.0/de/): Feel
% free 'to share (to copy, distribute and transmit)' or 'to remix (to adapt)'
% it, if you '... distribute the resulting work under the same or similar
% license to this one' and if you respect how 'you must attribute the work in
% the manner specified by the author ...':
%
% In an internet based reuse please link the reused parts to www.telekom.com and
% mention the original authors and Deutsche Telekom AG in a suitable manner. In
% a paper-like reuse please insert a short hint to www.telekom.com and to the
% original authors and Deutsche Telekom AG into your preface. For normal
% quotations please use the scientific standard to cite.
%
% [ Framework derived from 'mind your Scholar Research Framework' 
%   mycsrf (c) K. Reincke 2012 CC BY 3.0  http://mycsrf.fodina.de/ ]
%

For fulfilling our rule, to read the text strictly and deduce our
interpretations reasonably, let us firstly only highlight the syntactical
conjunctions for simplifying the understanding:

\begin{quote}\noindent\emph{\enquote{[\ldots] you may [\ldots] combine
\textbf{or} link a \enquote{work that uses the Library} with the Library to
produce a work containing portions of the Library \textbf{and} distribute that
work under terms of your choice, \textbf{provided that} the terms permit
modification of the work for the customer's own use \textbf{and} \emph{reverse
engineering} for debugging such modifications.}\footnote{\cite[cf.][\nopage wp.,
§6, emphasis KR.]{Lgpl21OsiLicense1999a}.}}
\end{quote}

It is evident that the conjunction \emph{'provided that'} is splitting the
sentence into two parts: you are allowed to do something \emph{provided that} a
condition is fulfilled. Additionally, both parts of the sentence -- the one
before the conjunction \emph{'provided that'} and the part after it -- are
syntactically condensed embedded phrases which also contain subordinated
conjunctions and elliptical constructions\footnote{cf.
http://en.wikipedia.org/wiki/Ellipsis\_\%28linguistics\%29, wp.
}. These syntactical interconnections must be disbanded:

Let us firstly \textbf{dissolve the syntactical compression \underline{before}
the conjunction \emph{'provided that'}}: It is established by using the two
other conjunctions \emph{and} and \emph{or} and introduced by the subordinating
phrase \emph{you may [\ldots]}. Unfortunately, from a formal point of view, one
can read the phrase \emph{you may (X or Y and Z)} as two different groupings:
either as \emph{you may ((X or Y) and Z)} or as \emph{you may (X or (Y and Z))}.

But, fortunately, we know from the semantic point of view that speaking about
\emph{\enquote{[\ldots] combining \textbf{or} linking [\ldots something] to
produce a work containing portions of the Library}} denotes two different
methods which both can \emph{join} the components \emph{\enquote{[\ldots] to
produce a work containing portions of the Library}}. So, let us -- only for a
moment\footnote{Later on we will re-insert th original phrase!} -- simply
replace the string \emph{\enquote{combine or link}} by the string
\emph{\enquote{*join}}\footnote{When the LGPL and the GPL were initially
defined, the C programming language was the predominant model of software
development. Knowing this method eases the understanding of these licenses.
Thus, it is not totally wrong to take this token *join also as a curtsey to the
C programming language.}. This reduces the syntactical structure of the sentence
back to the simple phrase \emph{you may (W and Z)} in which \emph{W} stands for
\emph{(X or Y)}.

Now, we can directly state that the phrase \emph{you may (W and Z)} itself is a
condensed version of the explicit phrase \emph{ (you may W) and (you may Z)}.

Finally we have to note, that the phrase before the conjunction \emph{'provided
that'} contains also a linguistic ellipsis\footnote{cf.
http://en.wikipedia.org/wiki/Ellipsis\_\%28linguistics\%29, wp.
}: It says that you may *join the components \enquote{to produce \textbf{a work
containing portions of the Library} \textbf{and} distribute \textbf{that work}
under terms of your choice}. With respect to the English grammar, we may
conclude that the second term \emph{that work} refers back to the previously
introduced specification of \emph{a work containing portions of the Library}: if
a complete phrase has just been introduced explicitly, then the English language
allows to reduce its next occurence syntactically while its complete meaning
is retained. Hence, conversely, we are allowed to unfold the reduced form to
restore the complete phrase.

So -- overall -- we may understand the phrase before the conjunction
\emph{'provided that'} as a phrase with the structure \emph{(you may W) and (you
may Z')}:

\begin{quote}\noindent\emph{\textbf{((}you may [\ldots] \emph{*join} a
\enquote{work that uses the Library} with the Library to produce a work
containing portions of the Library\textbf{) and (}you may [\ldots] distribute
that work containing portions of the Library under terms of your
choice\textbf{))}} \textbf{provided that} [\ldots]\end{quote}

Theoretically, a reader could reject our first dissolution of the
LGPL2-RefEng-Sentence. But for reasonably denying our interpretation he has to
deliver other resolutions of the lingustic elliptical subphrases or other
dissolvations of the conjunctions. Fortunately, it seems to be evident that such
attempts must violate the English grammar.

Let us secondly \textbf{dissolve the part \underline{after} the conjunction
\emph{'provided that'}}: With respect to the subordinated conjunction
\emph{'and'}, the subphrase \emph{the terms permit} syntactically refers to
both, the \emph{modification} and the \emph{reverse engineering}: An embedded
conjunction \emph{'and'} allows to use a more stylish grammatical compaction.
So, it should be clear, that saying

\begin{quote}\noindent\textbf{provided that} \emph{the terms permit modification
of the work for the customer's own use \emph{\textbf{and}} reverse engineering
for debugging such modifications}\end{quote}

means

\begin{quote}\noindent\textbf{provided that} \emph{the terms permit
\textbf{(} modification of the work for the customer's own use \emph{\textbf{and}}
reverse engineering for debugging such modifications\textbf{)}}\end{quote}

and is totally equivalent to the sentence 

\begin{quote}\noindent[\ldots] \textbf{provided that} \emph{\textbf{((}the terms
permit modification of the work for the customer's own use\textbf{)}
\emph{\textbf{and}} \textbf{(}the terms permit reverse engineering for debugging
such modifications\textbf{))}}.
\end{quote}

We believe that there is no other possibility to understand this part of the
LGPL2-RefEng-Sentence with respect to the rules of the English language.
Nevertheless, this is a next point where our reader may formally disagree with
us. If he really wants to object our dissolution, he must deliver another valid
interpreation of the scope of the conjunction \emph{and} or he must deliver
another resolutions of the linguistic ellipsis. But we reckon, that one can not
reasonably argue for such alternatives.

Finally, there are other deeply embedded ellipses, which need to be resolved
as well:

\begin{enumerate}
  \item  In the part before the splitting conjunction \emph{'provided that'} we
  already had to expand the abridging \emph{'that work'} by its intended
  explicated version \emph{'that work containing portions of the Library'}.  In
  the part after the splitting conjunction the first subphrase also contains the
  term \emph{'the work'}. Formally, this term can either refer to \emph{'the
  work that uses the library'} as one of the components which are joined, or it
  can refer to \emph{'the work containing portions of the Library'} as the
  result of joining the components. We decide to constantly dissolve the
  elliptic abridgement by the phrase \emph{'the work containing portions of the
  Library'}.
  \item The first clause of the part after the splitting conjunction
  \emph{'provided that'} talks about the purpose of \enquote{permitting
  modification of the work} which we just had to unfold to the phrase
  \emph{'permitting modification of the work containing portions of the
  Library'}. The second clause talks about the purpose of \enquote{permitting
  reverse engineering}: it shall support the \enquote{debugging [of] such
  modifications}. The pronoun \emph{'such'} indicates that the word
  \emph{'modifications'} refers back to the just unfolded phrase
  \emph{modification of the work containing portions of the Library}. So, even
  the second sentence has to be expanded to that explicit phrase.
  \item Finally and only for being complete, we also have to unfold the clause
  \enquote{the terms} to the form which is predetermined by the first referred
  instance \enquote{the terms of your choice}
\end{enumerate}

So -- overall -- we are allowed to rewrite the LGPL2-RevEng-Sentence 
in the following form, namely without having changed its
meaning\footnote{Recollect that '*join' still stands for 'combine or link'.}:

\begin{verbatim}
( ( you may 
       *join a work that uses the Library with the Library
        to produce a work containing portions of the Library )
  AND 
  ( you may 
        distribute that work containing portions of the Library
        under terms of your choice 
) )
PROVIDED THAT
( ( the terms of your choice permit 
        modification of the work containing portions of 
        the Library for the customer's own use )
  AND
  ( the terms of your choice permit
        reverse engineering for debugging modifications 
        of the work containing portions of the Library   
) )
\end{verbatim}

At this point we must recommend all our readers to verify that this
'structurally explicated presentation' does exactly mean the same as the
initially quoted LGPL2-RefEng-Sentence. We are now going to discuss some of its
logical aspects by some formal transformations. For accepting these operations
and linking the results back to the original LGPL2-RefEng-Sentence, it is very
helpful to know that one already has accepted the equivalence of this explicated
form and the more condensed original version. For reviewing the equivalence the
reader could -- for example -- ask himself which of our rewritings are wrong,
why they are wrong and which alternatives can reasonably be offered for solving
the syntactical issues which disposed us to chose our solutions. Again, we
ourselves -- of course -- are profoundly convinced that both versions are
completely equivalent.

%% use all entries of the bibliography
%\nocite{*}


\subsubsection{Logical Clarification}
% Telekom osCompendium 'for being included' snippet template
%
% (c) Karsten Reincke, Deutsche Telekom AG, Darmstadt 2011
%
% This LaTeX-File is licensed under the Creative Commons Attribution-ShareAlike
% 3.0 Germany License (http://creativecommons.org/licenses/by-sa/3.0/de/): Feel
% free 'to share (to copy, distribute and transmit)' or 'to remix (to adapt)'
% it, if you '... distribute the resulting work under the same or similar
% license to this one' and if you respect how 'you must attribute the work in
% the manner specified by the author ...':
%
% In an internet based reuse please link the reused parts to www.telekom.com and
% mention the original authors and Deutsche Telekom AG in a suitable manner. In
% a paper-like reuse please insert a short hint to www.telekom.com and to the
% original authors and Deutsche Telekom AG into your preface. For normal
% quotations please use the scientific standard to cite.
%
% [ Framework derived from 'mind your Scholar Research Framework' 
%   mycsrf (c) K. Reincke 2012 CC BY 3.0  http://mycsrf.fodina.de/ ]
%

For simplifying our discussion let us now replace the meaningful terminal
phrases of our form by some logical variables:

\begin{description}
  \item[$\Gamma$] :- (you may *join a work that uses the Library with the
  Library to produce a work containing portions of the Library) 
  \item[$\Delta$] :- (you may distribute that work containing portions of the
  Library under terms of your choice)
  \item[$\Phi$] :- (the terms of your choice permit modification of the work 
  containing portions of the Library for the customer's own use)
  \item[$\Sigma$] :- (the terms of your choice permit reverse engineering for
  debugging modifications of the work containing portions of the Library)
  \item[$\Theta$] :- \emph{$\Gamma$ and $\Delta$}
  \item[$\Omega$] :- \emph{$\Phi$ and $\Sigma$}
\end{description}

Based on these definitions, we can syntactically reduce the
LGPL2-RefEng-Sentence to the formula \emph{$(\Gamma$ and $\Delta)$ provided that
$(\Phi$ and $\Sigma)$} or -- even shorter -- to \emph{$(\Theta$ provided that
$\Omega)$}.

Now, we have to clarify the meaning of the conjunction \emph{'provided that'}:

Obviously, \emph{provided that} means something like \emph{under the condition
that}. So, one might try to take this conjunction as another more stylish
version of the common \emph{if(\ldots)then(\ldots)}-formula, sometimes also
identified as a (logical) implication\footnote{Actually the logical implication
and the computational if-then-construct are not equivalent. Fortunately, we
later on can show, that in the context of this discussion the difference can be
ignored.}. Thus, we have to consider the process of sequencing the linguistic
form into a logical formula: if we indeed take the conjunction \emph{provided
that} as another form of the logical implication, it is not evident, which part
of the linguistic sentence must become the premise, and which the conclusion:
Does \emph{($\Theta$ provided that $\Omega$)} mean \emph{(if $\Theta$ then
$\Omega$)} or \emph{(if $\Omega$ then $\Theta$)}?

Apparently, \emph{provided that} wants to establish something like a
precondition. So, one might conclude that \emph{$(\Theta$ provided that
$\Omega)$} means \emph{(if $\Omega$ then $\Theta)$} or -- more logically notated
-- \emph{$((\Phi$ $\wedge$ $\Sigma)$ $\rightarrow$ $(\Gamma$ $\wedge$
$\Delta))$}. If this interpretation is adequate, it must of course fulfill the
intended purpose of the corresponding LGPL-v2-section, which wants to regulate
the distribution of works containing portions of LGPL libraries.

For facilitating the understanding of our argumentation, let us first check
whether this logical interpretation of the linguistic conjunction fits the
purpose of the LGPL -- by unfolding the slightly reduced version \emph{$(\Sigma$
$\rightarrow$ $\Delta)$} back to the corresponding verbal form:

\begin{quote}\noindent\emph{\textbf{if (} [\ldots] the terms permit reverse
engineering for debugging modifications of the work containing portions of the
Library, \textbf{) then (} [\ldots] you may distribute that work containing
portions of the Library under the terms of your choice.\textbf{)}}\end{quote}

Now we can better see the problem: An implication as a whole is false only if
the premise is true and the conclusion is false. In all other cases it is true.
Especially, it is true, if the premise is false: If the premise is false, then
the truth value of the conclusion does not matter in any sense. Thus, if we take
this implication as a rule, which shall determine our behaviour, then this
implication only supports us, if we already have decided to permit reverse
engineering. In this case the rule successfully tells us that we are allowed to
distribute the work containing portions of the Library. But from the converse
decision that we will not permit reverse engineering, follows nothing - because
a false premise does not influence the truth value of the conclusion.
Especially, the rule does not tell us that we may not distribute the work
containing portions of the Library. So -- from the viewpoint of the formal logic
-- this translation of the original conjunction \emph{'provided that'} says,
that if the terms of your own license do not permit reverse engineering for
debugging modifications of the work containing portions of the
Library\footnote{The premise is false.}, then \textbf{you may or may not}
distribute that work containing portions of the Library under the terms of your
choice\footnote{The truth value of the conlusion is undetermined by the rule.}.
Hence, we must state that this interpretation does not fulfill the purpose of
the LGPL-V2: if reverse engineering is not allowed, the distribution of the work
containing portions of the Library is not regulated. We have to conclude, that
this sequencing the LGPL2-RefEng-Sentence as a logical implication is wrong.

But we deduced this consequence from a slighty reduced form of the
LGPL2-RefEng-Sentence. Thus, we still have to ask, whether we have to derive
this conclusion also on the base of the completely unfolded formula
\emph{$((\Phi$ $\wedge$ $\Sigma)$ $\rightarrow$ $(\Gamma$ $\wedge$ $\Delta))$}?
The answer is yes: the premise \emph{$((\Phi$ $\wedge$ $\Sigma)$} contains a
logical conjunction. So the truth value of the whole premise depends on the
truth value of each of its terminal statements, particularly on that of the
statement $\Sigma$: If we decide not to permit reverse engineering, then the
premise as whole is false, regardless we forbid or allow modifications.
Consequently, the premise does not influence the truth value of the conclusion.
So, there is no way, to conclude that we have to allow or that we do not have to
allow reverse engineering. Hence we can transfer our result, deduced for the
slightly reduced formula to the unfolded complete formula: assuming that
\emph{$(\Theta$ provided that $\Omega)$} means \emph{(if $\Omega$ then
$\Theta)$} is wrong.

So, let us test the other combination. Let us ask, whether \emph{($\Theta$
provided that $\Omega$)} means \emph{(if $\Theta$ then $\Omega$)} or -- more
logically notated -- \emph{$((\Gamma$ $\wedge$ $\Delta)$ $\rightarrow$ $(\Phi$
$\wedge$ $\Sigma))$}. If we again for a moment focus on the reduced version
\emph{$(\Delta$ $\rightarrow$ $\Sigma)$} and dissolve our replacements, then we
get back the rule:

\begin{quote}\noindent\emph{\textbf{if (} [\ldots] you may distribute that work
containing portions of the Library under the terms of your choice, \textbf{)
then (} [\ldots] the terms permit reverse engineering for debugging
modifications of the work containing portions of the Library.
\textbf{)}}\end{quote}

Now we can see, that this version perfectly regulates the distribution of works
containing portions of LGPL libraries: If we are allowed to do so or -- in other
words: if we are compliantly distributing works containing portions of LGPL
libraries\footnote{The premise is true.}, then we have to permit reverse
engineering\footnote{The conclusion must be true, too!}. This follows from
applying \emph{Modus Ponens} to the implication\footnote{A true premise evokes a
true conclusion based on the given truth of the implication / rule itself.}. And
if we do not permit reverse engineering\footnote{The conclusion is false.}, then
we are not allowed to distribute works containing portions of LGPL
libraries\footnote{The premise must be false, too!}. This follows from applying
\emph{Modus Tollens} to the implication\footnote{A false conclusion evokes a
false premise based on the given truth of the implication / rule itself.}

But -- again -- we have to consider that we have deduced this consequence from a
slighty reduced version of our LGPL2-RefEng-Sentence. Thus, we still have to
show that our result also holds for the completely unfolded formula
\emph{$((\Gamma$ $\wedge$ $\Delta)$ $\rightarrow$ $(\Phi$ $\wedge$ $\Sigma))$}:
If we want to distribute works containing portions of the Library which have
been produced by joining the Library and the work using the
Library\footnote{Premise is true.}, then our terms must permit the modification
\emph{and} reverse engineering of the distributed product\footnote{Conclusion
must become true by Modus Ponens.}. And if we do not allow its modification
\emph{or} reverse engineering\footnote{Conclusion is false.}, then we do not
compliantly distribute works containing portions of the Library which have been
produced by joining the Library and the work using the Library\footnote{Premise
must become false by Modus Tollens.} Thus, we may generally state, that the
logical explication \emph{$((\Gamma$ $\wedge$ $\Delta)$ $\rightarrow$ $(\Phi$
$\wedge$ $\Sigma))$} perfectly regulates the distribution of works containing
portions of LGPL libraries.

Based on this clarification, we can reasonably replace the more stylish
conjunction \emph{'provided that'} by its more known equivalent
\emph{'implication'}\footnote{Here we can also see, that the difference between
the if-then-command as part of a procedural computer language and the logical
implication does not influence our results: In the context of a procedural
if-then-command the truth of the premise triggers the execution of the
conclusion. In our discussion, this aspect is totally covered by the Modus
Ponens derivation of the logical interpretation. And the Modus Tollens
derivation of the logical interpretation on the other side does not play any
role in a procedural if-then-command. So, it was the right decision to
understand the LGPL2-RefEng-Sentence logically and not as procedual command.},
which we indicate by the commonly used character for a logical implication, the
sign '\emph{$\rightarrow$}':

\begin{description}
  \item[\#]  $\Theta$ provided that $\Omega$
  \item[$\equiv$] $\Theta$ $\rightarrow$ $\Omega$
  \item[$\equiv$] ($\Phi$ $\wedge$ $\Sigma$) $\rightarrow$ ($\Gamma$ $\wedge$
  $\Delta$)
  \item[$\equiv$]
\begin{alltt}   
  ( ( [\(\Phi\)] you may 
       *join a work that uses the Library with the Library
       to produce a work containing portions of the Library )
  \(\wedge\)
  ( [\(\Sigma\)] you may 
        distribute that work containing portions of the 
        Library under terms of your choice 
) )
\(\rightarrow\)
( ( [\(\Gamma\)] the terms of your choice permit 
        modification of the work containing portions of 
        the Library for the customer's own use )
  \(\wedge\)
  ( [\(\Delta\)] the terms of your choice permit
        reverse engineering for debugging modifications 
        of the work containing portions of the Library   
) )
\end{alltt}
\end{description}

%% use all entries of the bibliography
%\nocite{*}


\subsubsection{Empirical Clarification}
% Telekom osCompendium 'for being included' snippet template
%
% (c) Karsten Reincke, Deutsche Telekom AG, Darmstadt 2011
%
% This LaTeX-File is licensed under the Creative Commons Attribution-ShareAlike
% 3.0 Germany License (http://creativecommons.org/licenses/by-sa/3.0/de/): Feel
% free 'to share (to copy, distribute and transmit)' or 'to remix (to adapt)'
% it, if you '... distribute the resulting work under the same or similar
% license to this one' and if you respect how 'you must attribute the work in
% the manner specified by the author ...':
%
% In an internet based reuse please link the reused parts to www.telekom.com and
% mention the original authors and Deutsche Telekom AG in a suitable manner. In
% a paper-like reuse please insert a short hint to www.telekom.com and to the
% original authors and Deutsche Telekom AG into your preface. For normal
% quotations please use the scientific standard to cite.
%
% [ Framework derived from 'mind your Scholar Research Framework' 
%   mycsrf (c) K. Reincke 2012 CC BY 3.0  http://mycsrf.fodina.de/ ]
%

We can now simplify this formula once more by considering some empirical facts
and explicating some underlying understandings:

The first sentence $\Phi$ explains that the \emph{work that uses the Library}
and the used \emph{Library} itself together are joined and thereby transformed
into a \emph{work containing portions of the Library}. So, formally, one might
ask, whether this newly generated \emph{work containing portions of the Library}
also still \emph{uses the Library}?

Unfortunately, it is empirically possible, that such a process for combining the
two components could (a) copy all original portions of the library into a
something like a 'dead end section' of the program where they are never excuted,
and could (b) replace all original portions of the library by functionally
equivalent portions of any other library. Thus, the resulting \emph{work
containing portions of the Library} would indeed still contain portions of the
Library, although it would not use it any longer. And because of this
possibility, we are not allowed to say, that every work containing portions of a
library also uses the library\footnote{\ldots even if we think that this is a
really silly way to organize the joining process!}.

But, fortunately, the normal computational process of \emph{combining and
linking a work that uses the Library with the Library to produce a work
containing portions of the Library} inherently preserves the utilization of the
joined library: It is the general purpose of a software library to offer
functions and/or data (structures) for really being used by applications. And
vice versa, software developers refer to a specific library because they prefer
its service: They use readily prepared libraries (or classes or anything else)
because they want to simplify their own work while they conserve the quality
level of their work. Thus, they chose a library based on the assertion, that the
standard compiling and linking process guarantees, that indeed the chosen
library is used (and not secretly substituted by a mysterious 'equivalent').
With respect to this praxis of programming we are allowed to say that a
\emph{work containing portions of the Library} which has been \textbf{built by
the normal development processes} of combining, compiling, and linking source
and object files, indeed also uses the intended library.

Now, we are able to consider an empirical correlation between the first sentence
$\Phi$ and the second sentence $\Sigma$:

It seems to be evident, that we must already have done $\Phi$, in other words:
that we must already have \emph{*joined -- respectively: combined or linked -- a
work that uses the Library with the Library to produce a work containing
portions of the Library}, if we are going to compliantly \emph{distribute that
work containing portions of the Library under terms of your choice}. Or briefly
spoken: It seems to be conclusive that $\Sigma$ \emph{\textbf{empirically}
implies} $\Phi$\footnote{but not vice versa.}.

But is this conclusion correct? Let us check this statement by assuming the
opposite: If the contrary was true, there had to exist a \emph{work containing
portions of the Library} which had been gained without having linked or combined
the work and the Library in any sense. But from the inference above we already
know that \emph{works containing portions of the Library}, which have been
produced by the standard computational processes of \emph{combining and linking
a work that uses the Library with the Library}, indeed also \emph{use the
Library}. Thus, it would be self-contradictory to talk about a \emph{work
containing portions of the Library}, which was produced by the standard
combining and linking processes, and similarily to state, that exactly this work
is not combined with the library in any sense. And from a proof by contradiction
we may infer the truth of the logical opposite:

So, with respect to the meaning of \emph{being standardly combined or linked
with}, we may now say, that
\begin{itemize}
  \item it is necessarily true that a computional work, which is standardly
  produced on the base of \emph{a work that uses the Library} and \emph{the
  Library} and which therefore literally contains more or less
  \emph{portions of a library}, indeed uses the \emph{the Library} and \emph{is}
  therefore \emph{combined with the library}.
  \item  $\Sigma$\footnote{distributing \emph{a work that uses the Library and
  contains portions of a library}} empirically implies $\Phi$\footnote{A work
  that uses the Library has been *joined with the Library to produce a work
  containing portions of the Library} (in the standardized world of software
  development), because $\Phi$ must ever have been executed when $\Sigma$ is
  going to be realized.
\end{itemize}

Thus, we can now reduce the LGPL2-RefEng-Sentence to its real core, the
LGPL2-RefEng-Rule:

\label{RevEngEssentialLgplSection6Meaning}
\begin{quote}
\begin{alltt}   
(   [\(\Sigma\)] you may
        distribute (a) work containing portions of the 
        Library\(\footnote{which previously has been prepared for being distributed by standardly combining and
linking the work that uses the Library with the Library in a way that this prepared work indeed
also uses the Library}\) under terms of your choice )   
\(\rightarrow\)
( ( [\(\Gamma\)] the terms of your choice permit 
        modification of the work containing portions of 
        the Library for the customer's own use )
  \(\wedge\)
  ( [\(\Delta\)] the terms of your choice permit
        reverse engineering for debugging modifications 
        of the work containing portions of the Library   
) )
\end{alltt}
\end{quote}

This is indeed the essence of the LGPL2-RefEng-Sentence. It logically explains
us that we have to \emph{allow reverse engineering} and modification of a
\emph{work containing portions of the Library} if we distribute it (Modus
Ponens) and that we are \emph{not allowed to distribute a work containing
portions of the Library}, if we do \emph{not allow} its modification or
\emph{reverse engineering} (Modus Tollens).

Thus, for applying this rule correctly, we now only must know whether a work
indeed contains portions of the Library or not.

%% use all entries of the bibliography
%\nocite{*}


\subsubsection{Final Conclusion}
% Telekom osCompendium 'for being included' snippet template
%
% (c) Karsten Reincke, Deutsche Telekom AG, Darmstadt 2011
%
% This LaTeX-File is licensed under the Creative Commons Attribution-ShareAlike
% 3.0 Germany License (http://creativecommons.org/licenses/by-sa/3.0/de/): Feel
% free 'to share (to copy, distribute and transmit)' or 'to remix (to adapt)'
% it, if you '... distribute the resulting work under the same or similar
% license to this one' and if you respect how 'you must attribute the work in
% the manner specified by the author ...':
%
% In an internet based reuse please link the reused parts to www.telekom.com and
% mention the original authors and Deutsche Telekom AG in a suitable manner. In
% a paper-like reuse please insert a short hint to www.telekom.com and to the
% original authors and Deutsche Telekom AG into your preface. For normal
% quotations please use the scientific standard to cite.
%
% [ Framework derived from 'mind your Scholar Research Framework' 
%   mycsrf (c) K. Reincke 2012 CC BY 3.0  http://mycsrf.fodina.de/ ]
%

Unfortunately, there are more than one software developing scenarios, which must
be considered for answering this question in detail. We see three general types
of developing computer software:

\begin{enumerate}
  \item You can produce software by using script languages. Source files which
  contain script language commands are distributed and executed by an
  interpreter without priorly being transformed into another 'more' machine
  specific language.
  \item You can develop software by using languages which are designed for being
  compiled into a machine independent bytecode. Later on, this independent
  bytecode is executed by a machine specific virtual machine.
  \item You can write traditional software files. Sometimes, these files are
  remastered by a preprocessor before the real process starts. The traditional
  sources themselves or the output files of the preprocessor are then compiled
  and linked as machine specific binary file(s).
\end{enumerate}
  
You may take 'php' is an example for the first environment, 'Java' an example
for the second, and 'C/C++' an example for the third.

Fortunately, the nature of these environments simplifies the answer to the
question under which conditions the work using the Library contains portions of
the Library:

\paragraph{Distributing works with manually copied portions of the Library
evokes the copyleft effect:}
\label{RevEngCopyCodeManually}
Manually copying code from the sources of the Library into the overarching
work that uses the Library, is not the standard way of combining both
components, neither in the world of script programming, nor in the world of
bytecode programming, nor in the world of programming machine specific code:

Normally, the work which uses the Library is joined to the intended Library by
an include statement, an input statement, an import statement, a package
statement, or anything else. These *join-statements are inserted into the code
of the work. They denote the file(s) which deliver(s) the used functions,
methods, classes, or data. It is an integrated feature of the normal development
tools that inserting such *join-statements does not directly augment the work
using the library by some code of the Library: The development processes are
designed to offer an automatic augmention as part of the standard compilation
which is started after the actual development loop has been terminated.

Nevertheless, developers can circumvent these standard methods for using a
Library. Technically, they can directly copy code from the Library into their
own work. Consequently, these manually copied extensions of the code will be
compiled and/or executed together with the 'own' code of the work. Thus, it is
clear that in this case the work that uses the Library already contains
portions of the Library, particularly before the normal *join-processes of the
environment are executed.

Hence, if you are going to distribute works that contain literal copies of the
Library source code, then you have to allow reverse engineering, even if they
have already been compiled (but still not linked) on the base of such augmented
files\footnote{This directly follows from the LGPL2-RefEng-Rule by Modus Ponens.
But nevertheless, one might reply here, that even the result of manually copying
code from the Library to the work using the Library is covered by the limits of
tolerance, introduced by the LGPL-v2-§5. Formally, this argument seems to be
appropriate. And indeed, also we have to consider these limits of
tolerance later on. But in the context of copying code from the Library into the
work manually, a closer look reduces its impact. You have to discriminate three
cases:
\begin{enumerate}
  \item Developers can  \textbf{manually copy / transfer some or at most all
  elements of the Library header files into the code of the work} which the
  preprocessor itself would copy / transfer into that code automatically. But
  developers will not do that. Some simple include commands would cause the same
  effect. And developers want to save resources, especially their own working
  time. So, why should they manually do what they can delegate to the standard
  development process. Thus, it is reasonable to assume that developers, who
  nevertheless copy portions from Library into their work, do not want to repeat
  the service of the preprocessor manually, but to transfer more than only these
  elements. Hence, it is reasonable to assume, that their work is covered by the
  LGPL2-RefEng-Rule.
  \item Developers can \textbf{manually copy / transfer more than only the
  elements of the Library header files from the Library sources into the code of
  the work using the Library and they can nevertheless let the work being linked
  to the Library}. But again developers will not do that, because -- again --
  some simple include and linking commands would cause the same effect. So it is
  reasonable to start from the premise that copying developers in fact do more
  than this. Thus, it is reasonable to assume that their work is covered by the
  LGPL2-RefEng-Rule.
  \item Developers can \textbf{manually copy / transfer more than only the
  elements of the Library header files from the Library sources into the code of
  the work using the Library without linking it to the Library}. This is a
  reasonable step of work, because it spares the developers to link their work
  to the Library. But -- by definition -- such an augmented work contains more elements
  of the Library than LGPL-v2-§5 tolerates. Thus it is -- again -- reasonable to
  assume, that such a work is covered by the LGPL2-RefEng-Rule.
\end{enumerate}
Hence -- overall and from a practical point of view -- we can indeed say that
manually copying code from the Library into the work using the Library
requires to allow reverse engineering.}.

But, if we manually copy code from the Library to our work using the Library, we
also have to consider that the LGPL-v2 directly regulates this kind of using
the Library: It says, that \enquote{you may modify your copy or copies of the
Library or any portion of it [\ldots] provided that you also [\ldots] cause the
whole of the work to be licensed [\ldots] under the terms of this
License}\footcite[cf.][\nopage wp., §2, escpcially §2c]{Lgpl21OsiLicense1999a}.
Thus, there are strong arguments for the proposition, that the LGPL causes the
copyleft effect in case of literally copying code from the Library into the work
using the Library: The code of the work using the library has to be made
accessible, as well.

So, overall, we might say, that 'manually' copying code from an LGPL-v2 Library
into a work using that Library as a bypass of the standard software combining
processes and distributing the result indeed requires to additionally permit its
reverse engineering -- even if this permission is probably not very important
for the recipient, because he probably must have a direct access to the code.

\paragraph{Distributing scripts does not need reverse engineering:}
\label{RevEngDistributeScripts}
Computer programs written in a script language are distributed as they have been
developed. They are not transformed into another kind of code\footnote{Java
script is often offered as compressed code. Roughly spoken, this means that at
first all white space signs have been replaced by blanks and then all rows of
blanks have been reduced to at most one single blank. So, even then, the code
itself is directly readable and comprehensible -- even if only for very
sophisticated experts.}. The interpreter takes the script file as it is and
directly executes it. Thus, there is no special technique of reverse engineering
for understanding these kind of software: you can directly read it if you know
the script language.

So, again, we might conclude, that a script using a script Library perhaps
requires to permit its reverse engineering -- but probably this permission is
not very important for the recipient, because he can directly read the code.

\paragraph{Distributing statically combined bytecode requires the
permission of reverse engineering:}
\label{RevEngDistributeStaticallyCombinedByteCode}
In Java -- the prototype for languages which are compiled to machine independent
portable bytecode -- each class is compiled as a separate class file. These
class files have to be stored somewhere in the classpath. Aside from that,
classes can also be collected and distributed in form of packages which then can
be used like 'traditional' Libraries. These packages must also be stored
somewhere in the classpath. A single class is made known to the work that wants
to use it by an import statement which contains the class name; a whole Java
library is made usable by integrating a package statement into the code.

The code which follows such import- or package statements, can then use the
definitions offered by the classes. It denotes the elements of the classes by
the (qualified) names of its public or protected member variables or methods.
Thus, -- from a strict viewpoint -- the code of such a Java work using a Library
indeed contains portions of that library, even if these portions are only
identifying names or data structures containing identifying names. The Java
compilation process which generates the bytecode, preserves these denoting
names. It does not replace the referring names by the referred code of the
methods and so on. Only just at the end, when the java virtual machine itself
tries to execute the work using the Library, it collects all necessary commands
of all 'joined' classes.

So, one might tend to argue that answering the question whether a distributed
java bytecode already contains portions of the used Library depends on the
interpretation whether a denoting identifier of a Library indeed is a portion
of the Library. We will discuss this case together with the corresponding
C/C++-Case. 

But there is another Java specific aspect, which has to be considered as well.
As already mentioned, in Java you can also join your work containing the
denoting identifiers and the denoted Library by building a new package, which
then contains both, the work using the Library and the used Library. Hence, one
can say, that this package is quasi statically linked: if you distribute such an
integrated package, then you are distributing both components together. Thus, if
you distribute a complete package, in other words: a quasi statically linked
work containing the work using the Library and (all portions of) the Library,
then you have to permit reverse engineering\footnote{This directly follows from
the LGPL2-RefEng-Rule by Modus Ponens}.

So, preliminarily we conclude that, with respect to Java programming you (a)
have to permit reverse engineering, if you distribute your work using the
Library and the Library itself as a (statically linked) integrated
package\footnote{This follows from the LGPL2-RevEng-Rule by Modus Ponens.} and
that (b) in all other cases your obligation to permit reverse engineering
depends on the interpretation whether the identifiers declared by a Library are
indeed portions of the Library.

Fortunately, we can reasonably decide the issue of case (b) soon.

\paragraph{Distributing statically combined binaries require the
permission of reverse engineering:}
\label{RevEngDistributeStaticallyLinkedBinaries}
Similar to Java, in C/C++ -- the prototype of those languages, which are
compiled as machine specific code -- a C/C++ Library is also explicitly made
known to the work that wants to use it, namely by some include statements. These
include statements denote the header files offered by and distributed with the
Library. They contain the declarations of those elements which the Library wants
to publish. Or briefly worded: the Library contains the definitions in form of
code, the header files the corresponding declarations.

The C/C++ code following such include statements can refer to the definitions
offered by the Library by using the declarations anounced by the header files.
So, again, -- from a strict viewpoint -- the code of such a C/C++ work using the
Library indeed contains portions of the library, even if these portions are only
identifying names or data structures published by the header files.

Beyond that conceptual relation, the C/C++ development process finally compiles
the work using the library as an object file containing machine specific code.
Just as the Java compilation, this process does not replace the referring
names by the referred code of the Library; it still preserves the denoting
names. The resulting file, which has been compiled into machine specific code,
but still contains the denoting identifiers, is also known as 'object code
file'.

The C/C++ compilation process is (mostly) managed by a make file, which is
executed by the make command\footnote{Sometimes there additionally exists a
complete meta environment which generates such make files. The GNU build system
for example offers a complex set of configure scripts and make file templates
(cf. http://en.wikipedia.org/wiki/GNU\_build\_system, wp.).}. This development
tool calls the compiler for each source file, makes known the directories which
contain the compiled target object files, and finally calls the linker.
The linker recursively scans the compiled object files and replaces each
embedded identifier by a truly executable jump command into that set of Library
commands which are denoted by the identifier and which shall be executed as part
of the work using the Library. So, only at the end, the linker collects all
necessary commands of all 'joined' object files and Libraries and produces the
really executable work.

But -- notwithstanding the above -- the linker can either be called as
integrated step of the development process itself, or the linker can
be called separatedly, especially on another machine. In the first case, the
development process generates a \emph{statically linked executable} which
already contains all necessary portions of all used Libraries. In the second
case, the development process generates a \emph{dynamically linkable program} by
collecting the (set of) still unlinked object code file(s) as a distributable
package. Thus, if you distribute a statically linked executable, it definitely
contains 'portions' of the library; if you distribute a dynamically linkable
program you have to decide whether the embedded identifying names of a Library
have indeed to be interpreted as portions of the Library.

Unfortunately, we still have to consider a little complication, based on the
nature of the a C/C++ development process: contrary to the Java development
environment, a C/C++ development process inherently uses a preprocessor engine.
This engine takes the header files delivered by the Library, verifies the
syntactically correct use of the Library and can indeed replace some tokens of
the work using the Library by commands and/or lines from the Library. This
technique is known as \emph{inline functions} or \emph{macros}. They have been
invented for those cases where expanding the stack of commands during the
compilation by a real function call is more expensive than writing the embedded
commands of the function more than one time into the whole code. Hence, in the
C/C++ development process the compiled object files can indeed contain more than
only the referring names which denote portions of the Library: beside the
denoting identifiers, they can also already contain real, functionally relevant
portions of the Library.
 
Thus, -- again and similar to Java compilation -- we may conclude, that with
respect to C/C++ programming you (a) have to permit reverse engineering, if you
distribute your work using the Library together with the Library as a statically
linked program\footnote{This follows from the LGPL2-RevEng-Rule by Modus
Ponens.} and that (b) in all other cases your obligation to permit reverse
engineering depends on the interpretation whether the used identifiers or
dissolved inline functions and macros, which have been declared by the Library
and which therefore have automatically and standard conformably been embedded
into an object file, are indeed portions of the Library.

Obviously, it is time to answer this crucial question:
\label{RevEngLgplSection5Derivation}
\paragraph{Distributing dynamically combinable bytecode and linkable object code
does not require the permission of reverse engineering:} 
\label{RevEngDistributeDynamicallyLinkedCode}
Of course, there is only one instance that can answer the question whether
identifiers and dissolved inline-functions or macros, which are -- according to
the development standard -- embedded into a work using the Library, indeed are
portions of the Library. This instance is the LGPL-v2 itself. And -- fortunately
-- this license supports us in a very clear way to answer this question, even if
not by its §6 which deals with the reverse engineering, but by its §5:

The LGPL simply specifies that \enquote{linking a \enquote{work that uses the
Library} with the Library creates an executable that is a derivative of the
Library (because it contains portions of the Library) [\ldots]} and that
\enquote{the executable is therefore covered by this
License}\footcite[cf.][\nopage wp. §5]{Lgpl21OsiLicense1999a}. Additionally, it
talks about compiled, but still unlinked \enquote{object files}, which therefore
are not executables. Such an unexecutable \enquote{object file} -- for example
that of the \enquote{work using the Library} --, which \enquote{[\ldots] uses
only numerical parameters, data structure layouts and accessors, and small
macros and small inline functions (ten lines or less in length)} shall
practically not be covered by the license of the Library, because
\enquote{[\ldots] the use of the object file is unrestricted regardless of
whether it is legally a derivative work}\footcite[cf.][\nopage wp.
§5]{Lgpl21OsiLicense1999a} - as long as it does not exceed the given limits.

Obviously, the answer of the LGPL to our question is this: (a) yes, such
object files containing names and snippets offered by the used Library, could
contain portions of the Library. But it is not necessary to clarify the details,
because (b) -- up to a specific limit of sizes -- these kind of 'little'
portions being embedded into the object file by the standard compilation
processes do not evoke any requirements: they especially do not evoke the
obligation to allow reverse engineering. In other words: these little portions
of a Library which are embedded by the standard development process and which do
not contain more than the specified size of code may be regarded as another type
of portions compared to the normal, real portions which indeed evoke the
obligation to allow reverse engineering. From the viewpoint of the LGPL, they
are \emph{pseudo portions} of the Library, because they do not restrict the
containg object file in any respect.

So, from the LGPL-RevEng-Rule we can now indirectly conclude, that distributing
dynamically linkable or combinable bytecode or object code files which contain
\enquote{only numerical parameters, data structure layouts and accessors, and
small macros and small inline functions (ten lines or less in length)} being
delivered by a Library does not require to allow reverse
engineering\footnote{From the decision not to allow reverse engineering follows
by Modus Tollens applied to the LGPL2-RevEng-Rule, that the distribution of the
work using the Library must not contain real portions of the Library. From
LGPL-v2-§5 and the limit of the standard proccesses follows that here the work
using the Library does not contain normal, real portions. So, we know, that this
case is not covered by the LGPL2-RevEng-Rule and thus we are allowed to
distribute a work using the Library without allowing its reverse engineering.}.

Unfortunately, there might be a practical objection which seems to disturb our
simple result: For applying this rule correctly, we apparently have to assure
that a compiled work that uses the Library but is still not *joined to it,
indeed has only been expanded by \enquote{small macros or small inline functions
(ten lines or less in length)}. Thus, seemingly, we have to study all header
files of all used Libraries in detail, if we want to compliantly distribute a
work using a Library without permitting reverse engineering. This could be a lot
of work -- up to a bulk which practically can not be managed.

Fortunately, there is a simple solution for this challenge, a rule of thumb,
based on the principle \enquote{trust the upstream}\footnote{On the ELLW 2013,
we were told about this principle for the first time. We do not know, whether
Armijn Hemel invented it. But we can respectfully affirm that he has
persuasively explained the spirit and purpose of the principle \enquote{trust
the upstream}.}:

The Library developers of course publish the header files or the public members
and functions of the classes in exactly that form they want these elements to be
used. And they want their Library to be used as an LGPL library, otherwise they
would have chosen another License. So, they wish that improvements of the
Libraries shall be made accessible as well, but that the works using the Library
shall not necessarily be published in form of source code\footnote{The meaning
of the weak copyleft.}. Thus, as long as we use a Library exactly in that form,
the original authors have published, as long as we download the Library from
the official repository, and as long as we do not modify the intended interfaces
defined and published by the original header and class files, we may justifiably
assume that we are using the Libraries just as their copyright owners want them
to be used. And thus, -- in other words: as long as we trust the upstream -- we
might assume that the header and class files of our Libraries fit the
restrictions of the LGPL-v2.

\paragraph{LGPL-v2 compliance with or without permitting reverse engineering:} 
\label{RevEngLgpl2ComplianceByRenverseEngine}

Now, we have reached our target. Our last clarification can directly be applied
to the both open cases: to the case of distributing Java bytecode as well, as to
the case of distribution C/C++ object code. We now know, that the LGPL-v2
wishes, that not all portions of a Library covered by a work using the Library,
trigger the permission of reverse engineering. And we now know that the limits
-- given by the LGPL-v2-§5 -- up to which such pseudo portions indeed do not
trigger the obligation to permit reverse engineering, are respected, if we use
\emph{'upstream approved'} C/C++ and Java libraries in standard development
environments. Thus, we indeed finally may conclude, that the
LGPL-RevEng-Sentence

\begin{quote}\noindent\emph{\enquote{[\ldots] you may [\ldots] combine
\textbf{or} link a \enquote{work that uses the Library} with the Library to
produce a work containing portions of the Library \textbf{and} distribute that
work under terms of your choice, \textbf{provided that} the terms permit
modification of the work for the customer's own use \textbf{and} \emph{reverse
engineering} for debugging such modifications.}\footcite[cf.][\nopage wp., §6, 
emphasis KR.]{Lgpl21OsiLicense1999a}}
\end{quote}

means 'nothing else' than

\begin{itemize}
  \item \emph{With respect to a LGPL-v2 licensed Library, you are not required
  to allow reverse engineering, if you [A] develop your work using the Library,
  on the base of a standard version of the Library containing the interfaces as
  the original developers have designed it, if you [B] compile your work using
  this Library, as a discrete (set of) dynamically linkable or combinable
  file(s), if you [C] use only the standard compilation methods which preserve
  the upstream approved interfaces\footnote{and which therefore do not to exceed
  the LGPL-v2 limits!}, and if you [D] distribute the produced unlinked object
  code or bytecode files before they are linked as an executable.}
  \item \emph{In all other cases of distributing a work using such a Library,
  you are required to allow reverse engineering of the work using this Library
  -- especially, \ldots}
  \begin{itemize}
    \item \emph{if you distribute the work using the Library and the Library
    together as a statically linked program or as an integrated package
    containing both parts, the work using the library and the Library
    itself\footnote{This holds also if you distribute a script language based
    program or package, notwithstanding the fact, that one does not need the
    permission of reverse engineering to understand script language based
    applications.}.}
    \item \emph{if you distribute a work containing manually copied portions of
    the Library.}
  \end{itemize}
\end{itemize}

%% use all entries of the bibliography
%\nocite{*}


\subsubsection{Final Securing}
% Telekom osCompendium 'for being included' snippet template
%
% (c) Karsten Reincke, Deutsche Telekom AG, Darmstadt 2011
%
% This LaTeX-File is licensed under the Creative Commons Attribution-ShareAlike
% 3.0 Germany License (http://creativecommons.org/licenses/by-sa/3.0/de/): Feel
% free 'to share (to copy, distribute and transmit)' or 'to remix (to adapt)'
% it, if you '... distribute the resulting work under the same or similar
% license to this one' and if you respect how 'you must attribute the work in
% the manner specified by the author ...':
%
% In an internet based reuse please link the reused parts to www.telekom.com and
% mention the original authors and Deutsche Telekom AG in a suitable manner. In
% a paper-like reuse please insert a short hint to www.telekom.com and to the
% original authors and Deutsche Telekom AG into your preface. For normal
% quotations please use the scientific standard to cite.
%
% [ Framework derived from 'mind your Scholar Research Framework' 
%   mycsrf (c) K. Reincke 2012 CC BY 3.0  http://mycsrf.fodina.de/ ]
%

So far, we have done a lot of work: At first, we unfolded and dissolved some
stylisch condensed formulations of the original LGPL2-RevEng-Sentence by their
linguistically explicit version. At second, we explicated the logical structure
of the sentence. At third, we empirically carved out the real meaning of the
sentence. And finally we mapped the triggering part of that rule to some
verifiable facts. Indeed, a lot of work for understanding only one sentence
correctly\footnote{Here, some readers might ask why the original authors have
encapsulated their clear ideas in such a sophisticate sentence. Here are two
answers: First, this question is practically irrelevant: The authors of the
LGPL-v2 did, what they have done. And many developers have already licensed
their works under the terms of the LGPL-v2. Thus, we simply have to live with
the results -- just until the last software being published under the terms of
the LGPL-v2 is relicensed by a better version. Probably this won't happen during
our life time. Secondly, we appreciate the foresight of the LGPL-v2 authors.
They wrote a license which have successfully worked for more than twenty years.
They chosed a formulation which had also to cover 'uninvented' techniques. So,
it is not so surprizing, that we -- today -- have to do a lot of work to
understand all details the original authors want to be understood.}. So, it is a
good securing to verify that the derived result fits the spirit and the goals of
the LGPL-v2 perfectly. 

For that purpose, let us fist discuss a little (semi-) official article --
written by David Turner and published by the FSF\footcite[cf.][\nopage
wp.]{Turner2004a} -- which deals with the compliant use of LGPL licensed Java
libraries. Turner refers to the \enquote{FSF's position} which - as he says -
\enquote{[\ldots] has remained constant throughout}:

\begin{quote}\noindent\emph{\enquote{[\ldots] the LGPL works as intended with
all known programming languages, including Java. Applications which link to LGPL
libraries need not be released under the LGPL. Applications need only follow the
requirements in section 6 of the LGPL: allow new versions of the library to be
linked with the application; and allow reverse engineering to debug
this.}\footcite[cf.][\nopage wp]{Turner2004a}}\end{quote}

Then he describes, that Java libraries are \enquote{[\ldots] distributed as a
separate JAR (Java Archive) file} and that applications \enquote{[\ldots] use
Java's \enquote{import} functionality to access classes from these libraries}.
Moreover, he also explains, that the process of compilation \enquote{creates}
and integrates \enquote{links} into the compiled application which let become
the application a \enquote{derivative work} of the library.
Finally he states, that not only the LGPL permits to distribute such derivative
works, but that \enquote{[\ldots] it is easy to comply with the LGPL} if one
indeed wants to \enquote{[\ldots] distribute a Java application that imports
LGPL libraries}: \enquote{Your application's license needs to allow users to
modify the library, and reverse engineer your code to debug these
modifications.}\footcite[cf.][\nopage wp.]{Turner2004a}

So, we might state, that even this semi-official article argues very similarly
to us. There is only one little phrase in this text which differs a little:
Summarizing the \enquote{section 6 of the LGPL} by the statement
\enquote{\emph{[\ldots] allow new versions of the library to be linked with the
application; and allow reverse engineering to debug this}} does not consider
that the first sentence of the section 6 of the LGPL contains a complex
condition. The \emph{LGPL2-RefEng-Sentence} means -- as we could prove -- that
\emph{one may distribute (a) \textbf{work containing portions of the Library}
only if one's license permit reverse engineering for debugging
modifications}\footnote{$\rightarrow$ p.
\pageref{RevEngEssentialLgplSection6Meaning}}. But -- as we could also show --
for determining wether an application really contains portions of the Library,
one has additionally to consider the limits defined by section 5 of the
LGPL\footnote{$\rightarrow$ p. \pageref{RevEngLgplSection5Derivation}}: the 
application's license needs to allow to reverse engineer the application only if
it contains more elements of the Library than §5 of the LGPL-v2 has specified as
limit.

That our analysis fits the spirit of the LGPL, can also be shown by considering
the LGPL directly:

The LGPL-v2 clearly describes its goals. It wants to enable the community to let
an LGPL Library \enquote{[\ldots] become a de-facto standard}. And the LGPL
knows, that \enquote{to achieve this [goal], non-free programs must be allowed
to use the library}, because the \enquote{[\ldots] permission to use a
particular library in non-free programs enables a greater number of people to
use a large body of free software}. But the LGPL also asserts in this context,
that \enquote{although the Lesser General Public License is Less protective of
the users' freedom, it does ensure that the user of a program that is linked
with the Library has the freedom and the wherewithal to run that program using a
modified version of the Library}\footcite[cf.][\nopage wp., preamble, emphasis
KR]{Lgpl21OsiLicense1999a}.

So -- as a last check of our derivation -- we can analyze, whether our derived
result violates this goal. If it does, then we probably made a tremendous fault;
if not, then we are allowed to trust in the consistence our analysis:

If you receive a work using the Library in form of a discrete (set of)
dynamically linkable or combinable file(s) and if -- hence -- your provider
assumed that the files he delivers will be linked on your target machine which
-- therefore -- has to provide a linker and the the necessary dynamically
linkable Libraries, than you systematically have the freedom to replace the
dynamically linked Libraries by their updated versions\footnote{In GNU/Linux --
for example -- you must (only) copy the dynamically linkable new version of the
Library into the lib/-directory and replace the existing link by a version
pointing to the newer version. Sometimes you should additionally verify the
ld.so.conf files and call ldconfig tool.}. And as long as the newer versions of
the Libraries preserve the defined and declared interfaces, you can do that
successfully. That's, what the LGPL-v2 wants to ensure.

In all other cases, you must have the permission of reverse engineering or you
have a direct access to the source code. So, you can use the corresponding tools
and techniques to replace the embedded version of the Library by a newer
version; especially if you have received a statically linked package. Hence,
also the second part of our interpretation respects the spirit of the LGPL-v2.

So, finally we can say, everything is fine: The LGPL2-RevEng-Rule -- together
with the meaning of being a portion of a Library -- does not only verifiably
exeplicate the meaning of the LGPL2-RevEng-Sentence, but also fits the spirit
and the purpose of the LGPL-v2 as it has been announced by its preamble.


%% use all entries of the bibliography
%\nocite{*}


\subsection{Reverse Engineering in the LGPL-v3}
% Telekom osCompendium 'for being included' snippet template
%
% (c) Karsten Reincke, Deutsche Telekom AG, Darmstadt 2011
%
% This LaTeX-File is licensed under the Creative Commons Attribution-ShareAlike
% 3.0 Germany License (http://creativecommons.org/licenses/by-sa/3.0/de/): Feel
% free 'to share (to copy, distribute and transmit)' or 'to remix (to adapt)'
% it, if you '... distribute the resulting work under the same or similar
% license to this one' and if you respect how 'you must attribute the work in
% the manner specified by the author ...':
%
% In an internet based reuse please link the reused parts to www.telekom.com and
% mention the original authors and Deutsche Telekom AG in a suitable manner. In
% a paper-like reuse please insert a short hint to www.telekom.com and to the
% original authors and Deutsche Telekom AG into your preface. For normal
% quotations please use the scientific standard to cite.
%
% [ Framework derived from 'mind your Scholar Research Framework' 
%   mycsrf (c) K. Reincke 2012 CC BY 3.0  http://mycsrf.fodina.de/ ]
%

Based on our experiences how to successfully carve out the meaning of license
text, we can shorten the way to understand the one LGPL3-RevEng-Sentence
referring to \emph{reverse engineering}:

\begin{quote}\emph{\enquote{You may convey a Combined Work under terms of your
choice that, taken together, effectively do not restrict modification of the
portions of the Library contained in the Combined Work and reverse engineering
for debugging such modifications, if you also do each of the following
[\ldots]}\footnote{\cite[cf.][\nopage wp., §4]{Lgpl30OsiLicense2007a}. The
ellipsis at the end of the sentence denotes a set of tasks which we do not
listen here for saving resources, but which have to be considered as an
integrated part of this sentence.}}
\end{quote}

Reusing our method of disambiguation, we first can exemplify the meaning of the
LGPL3-RevEng-Sentence by the following text:

\begin{alltt}   
( \([\Theta:]\)
  ( You \(\emph{compliantly distribute}\) a Combined Work 
    under terms of your choice 
    (   (that together effectively, do not restrict modification of 
        the portions of the Library contained in the Combined Work)
    \(\textbf{AND}\) 
        (that together effectively, do not restrict reverse
        engineering for debugging modifications of the portions
        of the Library contained in the Combined Work)
  )  )
  \(\textbf{IF}\)
  \([\Omega:]\) 
  ( you also do each of the following \([\ldots]\))
)
\end{alltt}

But now, a simply executed logical serialization let us running into a problem:

If we serialized \emph{($\Theta$ IF $\Omega$)} as \emph{($\Omega$ $\rightarrow$
$\Theta$)}, then from not respecting $\Theta$ would follow by Modus Tollens,
that we are not allowed to realize $\Omega$ -- in other words:
that we may not do even one of the single tasks covered by the ellipsis -- which
is a silly result. 

If we serialized \emph{($\Theta$ IF $\Omega$)} as \emph{($\Theta$ $\rightarrow$
$\Omega$)} then from doing $\Theta$ would successfully follow by Modus Ponens
that we also have to do $\Omega$. And from not respecting $\Omega$ would
successfully follow by Modus Tollens, that we must not do $\Theta$. But
unfortunately, we can respect this second interdiction also \emph{by
distributing a Combined Work under terms} that restrict modifications and/or
reverse engineering (instead of not restricting these techniques) -- which,
again, is a silly result.

Obviously, a simple serialization based on a intutively unclear reading fails.
In fact, the LGPL3-RevEng-Sentence must have a more sophisticated underlying
structure. It must be logically serialized in a form, that integrates the
requirements, not to restrict modifications and reverse enigneering, as really
triggable conditions. Thus, the meaning of the sentence can logically be
explicated as the \emph{LGPL3-RevEng-Rule}:

\begin{alltt}
( \([\Sigma:]\)
  ( You \(\emph{compliantly distribute}\) a Combined Work 
    under terms of your choice 
  ) 
  \(\rightarrow\)  
  (    \([\Gamma:]\)
     ( the terms of your choice together effectively do 
       not restrict modification of the portions of the 
       Library contained in the Combined Work) 
     \(\wedge\) \([\Delta:]\)
     ( the terms of your choice together effectively, do 
       not restrict reverse engineering for debugging 
       modifications of the portions of the Library 
       contained in the Combined Work)
     \(\wedge\) \([\Omega:]\) 
     ( you also do each of the following \([\ldots]\))
) )
\end{alltt}  

This LGPL3-RevEng-Rule indeed successfully regulates how to compliantly
distribute a Combined Work by telling us,

\begin{itemize}
  \item that we have to respect $\Gamma$, $\Delta$ \textbf{and} all single parts
  of $\Omega$, if we distribute a Combined Work compliantly\footnote{follows by
  Modus Ponens. Thus, in this case especially our terms \enquote{[\ldots]
  together effectively \textbf{[must] not restrict reverse engineering} for
  debugging modifications of the portions of the Library contained in the
  Combined Work}.}.
  \item that we do not distribute a Combined Work compliantly, if we do not
  respect one of the requirements $\Gamma$, $\Delta$ or one of the single parts
  of $\Omega$\footnote{follows by Modus Tollens. Thus, especially we are not
  distributing a Combined Work compliantly, if our terms \enquote{[\ldots]
  together effectively \textbf{do restrict reverse engineering} for debugging
  modifications of the portions of the Library contained in the Combined
  Work}.}.
\end{itemize}

Now, we can directly see, that the LGPLv3 does not enforce us, not to obstruct
reverse engineering in all respects! The required reverse engineering is limited
to the purpose of supporting the debugging of modifications and focused to the
Combined Work containing portions of the Library. In other words: our terms may
obstruct other purposes of reverse engineering or may restrict reverse
engineering of other forms of our work which which can not be specified as
Combined Work or do not contain portions of the Library. Thus, the first crucial
question is, what the LGPL-v3 means if it talks about a \enquote{Combined Work}.
The second question is, what the LGPL-v3 specifies as a portion of the Library.

Again, fortunately, the LGPL-v3 answers clearly: \enquote{A \enquote{Combined
Work} is a work produced by combining or linking an Application with the
Library}\footcite[cf.][\nopage wp., §0]{Lgpl30OsiLicense2007a}. From our LGPL-v2
analysis we know the ways how works that uses a Library can technically be
linked or combined with the Library:

\begin{itemize}
  \item Copying code from the Library into the work using the
  Library\footnote{The LGPL-v3 designates the work using the Library as
  \enquote{Application} and defines that it \enquote{[\ldots] makes use of an
  interface provided by the Library [\ldots]} (\cite[cf.][\nopage wp.,
  §0]{Lgpl30OsiLicense2007a}).} causes that the application respectively the
  work using the Library indeed contains portions of the
  Library\footnote{$\rightarrow$ p. \pageref{RevEngCopyCodeManually}}.
  \item Combining script language based applications and Libraries may evoke
  that the resulting application contains portions of the Library. But the
  details can be neglected with respect to the reverse engineering, because
  script code is distributed as it has been developed and can therefore directly
  be understood\footnote{$\rightarrow$ p. \pageref{RevEngDistributeScripts}}.
  \item Combining java classes and libraries as integrated quasi statically
  linked packages causes, that the resulting package already contains all
  functionally necessary code of the Library\footnote{$\rightarrow$ p.
  \pageref{RevEngDistributeStaticallyCombinedByteCode}}.
  \item Compiling java classes without combining them with the referred Library
  classes causes, that the compiled classes at least contain identifiers having
  been declared by the Library\footnote{$\rightarrow$ p.
  \pageref{RevEngDistributeDynamicallyLinkedCode}}.
  \item Combiling C/C++ files or classes and linking them with the referred
  Libaries statically causes, that the resulting executable indeed contains all
  functional relevant code of all used Libraries\footnote{$\rightarrow$ p.
  \pageref{RevEngDistributeStaticallyLinkedBinaries}}.
  \item Combiling C/C++ files or classes without linking them to the referred
  Libaries causes, that the resulting object file can dynamically be linked on
  another machine and contains identifiers offered by the Library and sometimes
  some functional code injected by dissolving some inline functions or
  macros\footnote{$\rightarrow$ p.
  \pageref{RevEngDistributeDynamicallyLinkedCode}}.
\end{itemize}

So -- overall -- the situation is this: The LGPL3-RevEng-Rule tells us that we
have to allow reverse engineering of the portions of the Library
contained in the Combined Work. The LGPL3 additionally specifies, that a Combined Work
is simply the result of technically combining the work using the Library (the
application) and the Library. Finally the praxis tells us, that (a) combining
both components statically indeed causes that the resulting Combined Work contains
portions of the Library\footnote{So, it is triggering the LGPL3-RevEng-Rule.},
and that (b) we -- in case of preparing the both parts as dynamically
combinable components -- still have to clarify whether the resulting work
already contains portions of the Library.

Just as the LGPL-v2, the LGPL-v3 supports us to answer this question by its §3
whose linguistic conjunctions we thoroughly have to consider:

\begin{quote}\emph{The object code form of an Application may incorporate
material from a header file that is part of the Library. \textbf{You may convey}
such object code under terms of your choice, \emph{provided that}, \textbf{[}
\textbf{if} the incorporated material is \textbf{not} limited to numerical
parameters, data structure layouts and accessors, or small macros, inline
functions and templates (ten or fewer lines in length) \textbf{]}, \textbf{you
do both} of the following: \textbf{a)} Give prominent notice with each copy of
the object code that the Library is used in it and that the Library and its use
are covered by this License. \textbf{b)} Accompany the object code with a copy
of the GNU GPL and this license document\textbf{]}\footcite[cf.][\nopage wp.,
§3; emphasis and additional braces KR.]{Lgpl30OsiLicense2007a}.}
\end{quote}

The first sentence of this paragraph tells us that he is dedicated to object
files which are compiled and not linked to the used Library, but which
nevertheless can contain portions of the Library. The second sentence regulates
the distribution of such object files and can be logically serialized:

\begin{alltt}
( \([\Lambda:]\)
  ( You \(\emph{compliantly distribute}\) object code [incorporating 
    material from the Library] under terms of your choice ) 
  \(\rightarrow\)  
  \([\Xi:]\)
  ( \([\omega:]\)
    ( the incorporated material is not limited to numerical
      parameters, data structure layouts and accessors, or 
      small macros, inline functions and templates 
      [ten or fewer lines in length] ) 
    \(\rightarrow\) 
    ( \([\alpha:]\) ( you do [a] \(\ldots]\) )
    \(\wedge\) \([\beta:]\) ( you do [b] \(\ldots]\) )
) ) )
\end{alltt}  

We see, that this LGPL3-sentence concerning the distribution of object files
contains a main rule (\emph{($\Lambda$ $\rightarrow$ $\Xi$)}) and that the
conclusion $\Xi$ itself has the form of an embedded sub rule (\emph{($\omega$
$\rightarrow$ ( $\alpha$ $\wedge$ $\beta$)}).

Firstly, the main rule enforces us to respect the sub rule if we want to
distribute the object code compliantly\footnote{follows by Modus Ponens to
\emph{($\Lambda$ $\rightarrow$ $\Xi$)}.}. Secondly, the main rule tells us that
we do not distribute the object code compliantly if we do not respect the sub
rule \footnote{follows by Modus Ponens to \emph{($\Lambda$ $\rightarrow$
$\Xi$)}.}.

We have two ways to respect the sub rule, and one way not to respect it:
\begin{itemize}
  \item If the object code contains more and/or larger elements of the Library
  than the limit specifies, then \textbf{we do respect the sub rule}, if we do
  $\alpha$ and $\beta$\footnote{follows by Modus Ponens to \emph{($\omega$
  $\rightarrow$ ($\alpha$ $\wedge$ $\beta$))}.}.
  \item If the object code contains elements of the Library at most up to
  specified limits, then \textbf{we do respect the sub rule} without having to
  do some additionally tasks\footnote{follows by definition of an implication:
  if the premise of this sub rule is false, the sub rule is as whole is true and
  hence respected.}
  \item But if the object code contains more and/or larger elements of the
  Library than the limit specifies \textbf{and} if we do not do $\alpha$
  \emph{or} $\beta$, then \textbf{we do not respect the sub
  rule}\footnote{follows from definition of an implication: if the premise is
  true and the conclusion is false, the the implication as whole is false, as
  well.}.
\end{itemize}

Thus, -- at the end and based on the additional object code specification and
the known empirical background knowledge concerning the software programming --
the LGPL3-RevEng-Rule delivers the same result as the
LGPL2-RevEng-Rule\footnote{$\rightarrow$
\pageref{RevEngLgpl2ComplianceByRenverseEngine}}:

\begin{itemize}  
  \item \emph{With respect to a LGPL-v3 licensed Library, you are not required
  to allow reverse engineering, if you [A] develop your work using the Library,
  on the base of a standard version of the Library containing the interfaces as
  the original developers have designed it, if you [B] compile your work using
  this Library, as a discrete (set of) dynamically linkable or combinable
  file(s), if you [C] use only the standard compilation methods which preserve
  the upstream approved interfaces\footnote{and which therefore do not to exceed
  the LGPL-v3 limits}, and if you [D] distribute the produced unlinked object
  code or bytecode files before they are linked as an executable.}
  \item \emph{In all other cases of distributing a work using such a Library,
  you are required to allow reverse engineering of the work using this Library
  -- especially, \ldots}
  \begin{itemize}
    \item \emph{if you distribute the work using the Library and the Library
    together as a statically linked program or as an integrated package
    containing both parts, the work using the library and the Library
    itself\footnote{This holds also if you distribute a script language based
    program or package, notwithstanding the fact, that one does not need the
    permission of reverse engineering to understand script language based
    applications}.}
    \item \emph{if you distribute a work containing manually copied portions of
    the Library.}
  \end{itemize}
\end{itemize}


%% use all entries of the bibliography
%\nocite{*}


\subsection{Reverse Engineering in the other Open Source Licenses}
% Telekom osCompendium 'for being included' snippet template
%
% (c) Karsten Reincke, Deutsche Telekom AG, Darmstadt 2011
%
% This LaTeX-File is licensed under the Creative Commons Attribution-ShareAlike
% 3.0 Germany License (http://creativecommons.org/licenses/by-sa/3.0/de/): Feel
% free 'to share (to copy, distribute and transmit)' or 'to remix (to adapt)'
% it, if you '... distribute the resulting work under the same or similar
% license to this one' and if you respect how 'you must attribute the work in
% the manner specified by the author ...':
%
% In an internet based reuse please link the reused parts to www.telekom.com and
% mention the original authors and Deutsche Telekom AG in a suitable manner. In
% a paper-like reuse please insert a short hint to www.telekom.com and to the
% original authors and Deutsche Telekom AG into your preface. For normal
% quotations please use the scientific standard to cite.
%
% [ Framework derived from 'mind your Scholar Research Framework' 
%   mycsrf (c) K. Reincke 2012 CC BY 3.0  http://mycsrf.fodina.de/ ]
%

The rest of our way is simple: First, we can ascertain, that none of the other
open source licenses we consider\footnote{$\rightarrow$ p.
\pageref{RevEngOslicOsLisences} }, contain the phrase 'reverse engineering'.
Moreover, they even do not contain one of the single words\footnote{One can
verify this negative statement by (a) loading down all licenses from the OSI
homepage (http://opensource.org/licenses/alphabetical) and by (b) executing the
command \texttt{grep -i "engineering" *} respectively \texttt{grep -i "reverse"
*} in the directory into which the license files have been stored: grep will
find the words \emph{reverse} and \emph{engineering} only in the texts of the
LGPLs.}. So, we may infer, that these most important other open source licenses
could at most indirectly require the permission of reverse engineering. Second,
we know already that distributing script code let the allowance to reverse
engineer, become irrelevant: script code can directly be read and understood, if
one knows the script language\footnote{$\rightarrow$ p.
\pageref{RevEngDistributeScripts}}.
Third, from the definition of strong copleft we may derive, that distributing
software licensed under a strong copyleft license let the permission of reverse
engineering become unimportant, because the source code of the work using the
libraries licensed under a copleft license, must also be made
accessible\footcite[cf.][\nopage wp]{Stallman1996c}.

So -- overally -- we may conclude, that we have only to consider those cases,
where a piece of software is distributed in form of binaries or bytecode, which
uses libraries licensed under permissive open source licenses or under weak
copyleft licenses.

From the definition of being a permissive license or a weak copyleft license we
know already that the licenses of the open source components do not directly
influence the permission or interdiction to use the overarching work which uses
the open source software components\footcite[cf.][20ff.]{Reincke2015a}.

So, if we distribute such a work in form of dynamically linkable, but still not
linked binaries or bytecode files, then there is no way to reasonably derive
that the work using the components, may be reverse engineered: The permissive or
weak copyleft open source licenses mainly concern the open source components,
not the work using the components. On the one side, these licenses indeed
require that we add the license texts and the copyright lines of all the open
source components our work wants to use, to the distributed package containing
our work. And the lisenses prohibit to modify the licensing assertions being
integrated into the open source components our work wants to use\footnote{These
requirements are part of all the open source licenses we consider here. For
details \cite[cf.][chapter 6.]{Reincke2015a}}. But -- on the other side and in
accordance to the permissive or weak-copyleft licenses -- the freedom to use, to
study, to modify, or to distribute the software, which is established by these
open source licenses, concerns only the open source components themselves, not
the work using the open source components. So, as long as these components still
are not linked to or combined with the using work in accordance to the standard
compilation and computation methods, they can indeed be studied or modified
without the need to study or modify the work which uses these
components\footnote{The only way to infer that the licenses of the components
operates also on the using work, is to argue that the using work must at least
contain elements (identifiers etc.) of the interfaces declared (but not defined)
by the libraries and that therefore at least these elements may be investigated
or modified. This challenge is explicitly addressed by the
LGPL\footnote{$\rightarrow$ p.
\pageref{RevEngDistributeDynamicallyLinkedCode}}. Fortunately, it is a general
feature of software libraries that they must and shall be used in accordance to
the interfaces, the developers of the libraries have designed to make their
libraries practically usable. So, if the licenses -- in contrary to the LGPLs --
do not explicitly address the issue of implicitly included portions of the
library in case of unlinked binaries or bytecode files which have been compiled
in accordance to the standard methods and which therefore use open source
software by reffering to their standard interfaces, then one has to infer from
the nature of computation, that the developers have implictly allowed without
any requirements such an integration of declared, but not defined interface
elements, because they have designed the interface as they did and because they
have licensed their work as they did. If they had not wished to use these
elements without any requirements, hey had designed another interface. And if
they had wished to incorporate any copyleft effect or permission of reverse
engineering, then they would have selected another license. But again: this
conclusion holds only for the standard methods to use a software library.}.

On the other side, if we compliantly distribute the work using the components,
as a statically linked binary or bytecode file -- which therefore already
contains all the necessary components\footnote{instead of only the declared
interface elements!} and can directly be executed --, then we are also obliged
to add all the open source license texts and all the copyright lines to our
package, and we are not allowed to modify one of the licensing assertions
integrated into the original open source components\footcite[cf.][chapter
6.]{Reincke2015a}. Thus, one might conclude, that the freedom to use and to
modify the open source components themselves, survive if we distribute software
statically linked to or combined with the open source components. So, the
receiver of the statically linked work probably is allowed to modify the
embedded open source components - even if he had to edit the binary or bytecode
files. Methods to develop binary files reversely, are known as reverse
engineering. Hence, if we distribute a statically linked work using open source
licensed components, we have at least to fear that our receivers indirectly have
also got the permission to reverse engineer our complete product. And we have to
fear so even if the statically linked libraries are licensed under any
permissive or weak copleft license.

So, again, we can summarize the result in the following form:

\begin{itemize}
  \item \emph{With respect to a Library licensed under any permissive or weak
  copyleft license, you are not required to allow reverse engineering, if you
  [A] develop your work using the Library, on the base of a standard version of
  the Library containing the interfaces as the original developers have designed it,
  if you [B] compile your work using this Library, as a discrete (set of)
  dynamically linkable or combinable file(s), if you [C] use only the standard
  compilation methods which preserve the upstream approved interfaces, and if
  you [D] distribute the produced unlinked object code or bytecode files before
  they are linked as an executable.}
  \item \emph{In all other cases of distributing a work using such a Library,
  you have at least to fear that you are implictly allowing reverse engineering
  of the work using this Library -- especially, \ldots}
  \begin{itemize}
    \item \emph{if you distribute the work using the Library and the Library
    together as a statically linked program or as an integrated package
    containing both parts, the work using the library and the Library
    itself\footnote{This holds also if you distribute a script language based
    program or package, notwithstanding the fact, that one does not need the
    permission of reverse engineering to understand script language based
    applications}.}
    \item \emph{if you distribute a work containing manually copied portions of
    the Library.}
  \end{itemize}
\end{itemize}


%% use all entries of the bibliography
%\nocite{*}


\subsection{Reverse Engineering in Open Source Licenses: Summary}
% Telekom osCompendium 'for being included' snippet template
%
% (c) Karsten Reincke, Deutsche Telekom AG, Darmstadt 2011
%
% This LaTeX-File is licensed under the Creative Commons Attribution-ShareAlike
% 3.0 Germany License (http://creativecommons.org/licenses/by-sa/3.0/de/): Feel
% free 'to share (to copy, distribute and transmit)' or 'to remix (to adapt)'
% it, if you '... distribute the resulting work under the same or similar
% license to this one' and if you respect how 'you must attribute the work in
% the manner specified by the author ...':
%
% In an internet based reuse please link the reused parts to www.telekom.com and
% mention the original authors and Deutsche Telekom AG in a suitable manner. In
% a paper-like reuse please insert a short hint to www.telekom.com and to the
% original authors and Deutsche Telekom AG into your preface. For normal
% quotations please use the scientific standard to cite.
%
% [ Framework derived from 'mind your Scholar Research Framework' 
%   mycsrf (c) K. Reincke 2012 CC BY 3.0  http://mycsrf.fodina.de/ ]
%

So, finally we can compile all our results into one single result:

\begin{itemize}
  \item \emph{With respect to any open source Library\footnote{$\rightarrow$ p.
  \pageref{RevEngOslicOsLisences}}, you are not required to allow reverse
  engineering, if you [A] develop your work using the Library, on the base of a
  standard version of the Library containing the interfaces as the original
  developers have designed it, if you [B] compile your work using this Library,
  as a discrete (set of) dynamically linkable or combinable file(s), if you [C]
  use only the standard compilation methods which preserve the upstream approved
  interfaces\footnote{and which therefore do not to exceed limits, prescribed by
  the owners of the Library}, and if you [D] distribute the produced unlinked
  object code or bytecode files before they are linked as an executable.}
  \item \emph{In all other cases of distributing your work using such a Library,
  you are probably required to allow reverse engineering of your work. By all
  means, you have at least to fear that you are implictly allowing reverse
  engineering of your work using such a Library -- especially, \ldots}
  \begin{itemize}
    \item \emph{if you distribute the work using the Library and the Library
    together as a statically linked program or as an integrated package
    containing both parts, the work using the library and the Library
    itself\footnote{This holds also if you distribute a script language based
    program or package, notwithstanding the fact, that one does not need the
    permission of reverse engineering to understand script language based
    applications}.}
    \item \emph{if you distribute a work containing manually copied portions of
    the Library.}
  \end{itemize}
\end{itemize}
 
And, so, we can reformulate our result as a slightly modified \enquote{rule of
thumbs} originally offered by an open source expert who analyzed the problem of
protecting your own work from an other viewport:

\begin{itemize}
  \item \enquote{DO NOT statically link [or combine] [open source] code if you
  wish to keep your program proprietary [and if you want to protect it against reverse
  engineering]}\footcite[cf.][6; bracketed text KR.]{Ilardi2010a}.
  \item \enquote{DO dynamically link to [any open source code, not only to] LGPL
  code}\footcite[cf.][6; bracketed text KR.]{Ilardi2010a}.
\end{itemize}

\textbf{\textsf{q.e.d}}

%% use all entries of the bibliography
%\nocite{*}




% Telekom osCompendium 'for beeing included' snippet template
%
% (c) Karsten Reincke, Deutsche Telekom AG, Darmstadt 2011
%
% This LaTeX-File is licensed under the Creative Commons Attribution-ShareAlike
% 3.0 Germany License (http://creativecommons.org/licenses/by-sa/3.0/de/): Feel
% free 'to share (to copy, distribute and transmit)' or 'to remix (to adapt)'
% it, if you '... distribute the resulting work under the same or similar
% license to this one' and if you respect how 'you must attribute the work in
% the manner specified by the author ...':
%
% In an internet based reuse please link the reused parts to www.telekom.com and
% mention the original authors and Deutsche Telekom AG in a suitable manner. In
% a paper-like reuse please insert a short hint to www.telekom.com and to the
% original authors and Deutsche Telekom AG into your preface. For normal
% quotations please use the scientific standard to cite.
%
% [ Framework derived from 'mind your Scholar Research Framework' 
%   mycsrf (c) K. Reincke 2012 CC BY 3.0  http://mycsrf.fodina.de/ ]
%


%% use all entries of the bibliography
%\nocite{*}

\section{Excursion: The problem of license compatibility [tbd]}
\footnotesize
\begin{quote}\itshape
Here we discuss the often neglected or only superficially treated problem of combining
differently licensed software. We will hint to the Exclusion-List of the Free
software foundation; we will hint to the Eclipse / GPL-plugin problem; we will
mention the recent discussion whether the kernel requires to license the
complete Android as GPL; and finally we will discuss the just now published, short
analysis of Jaeger and Metzger presenting a combining matrix which seems to fall
into their lap. % RPD FIXME What is this supposed to mean?
We will argue that the question can simply be answered:
Only if you embed two libraries which both are licensed under an
on-top-development protecting license and if both these licenses require the
licensing of the derivated work by different licenses then you have a problem.
In all other cases which we will describe, there is no problem.
\end{quote}
\normalsize
\ldots

%\bibliography{../../../bibfiles/oscResourcesEn}

% Local Variables:
% mode: latex
% fill-column: 80
% End:

% Telekom osCompendium 'for being included' snippet template
%
% (c) Karsten Reincke, Deutsche Telekom AG, Darmstadt 2011
%
% This LaTeX-File is licensed under the Creative Commons Attribution-ShareAlike
% 3.0 Germany License (http://creativecommons.org/licenses/by-sa/3.0/de/): Feel
% free 'to share (to copy, distribute and transmit)' or 'to remix (to adapt)'
% it, if you '... distribute the resulting work under the same or similar
% license to this one' and if you respect how 'you must attribute the work in
% the manner specified by the author ...':
%
% In an internet based reuse please link the reused parts to www.telekom.com and
% mention the original authors and Deutsche Telekom AG in a suitable manner. In
% a paper-like reuse please insert a short hint to www.telekom.com and to the
% original authors and Deutsche Telekom AG into your preface. For normal
% quotations please use the scientific standard to cite.
%
% [ Framework derived from 'mind your Scholar Research Framework' 
%   mycsrf (c) K. Reincke 2012 CC BY 3.0  http://mycsrf.fodina.de/ ]
%


%% use all entries of the bibliography
%\nocite{*}

\section{Excursion: open source software and money [tbd]}
\footnotesize
\begin{quote}\itshape
Here we will shortly discuss ways in which money and Open Source is no problem.
\end{quote}
\normalsize
\ldots


%\bibliography{../../../bibfiles/oscResourcesEn}

% Local Variables:
% mode: latex
% fill-column: 80
% End:



%%%%%%%%%%%%%%%
% Telekom osCompendium 'for being included' snippet template
%
% (c) Karsten Reincke, Deutsche Telekom AG, Darmstadt 2011
%
% This LaTeX-File is licensed under the Creative Commons Attribution-ShareAlike
% 3.0 Germany License (http://creativecommons.org/licenses/by-sa/3.0/de/): Feel
% free 'to share (to copy, distribute and transmit)' or 'to remix (to adapt)'
% it, if you '... distribute the resulting work under the same or similar
% license to this one' and if you respect how 'you must attribute the work in
% the manner specified by the author ...':
%
% In an internet based reuse please link the reused parts to www.telekom.com and
% mention the original authors and Deutsche Telekom AG in a suitable manner. In
% a paper-like reuse please insert a short hint to www.telekom.com and to the
% original authors and Deutsche Telekom AG into your preface. For normal
% quotations please use the scientific standard to cite.
%
% [ Framework derived from 'mind your Scholar Research Framework' 
%   mycsrf (c) K. Reincke 2012 CC BY 3.0  http://mycsrf.fodina.de/ ]
%


%% use all entries of the bibliography
%\nocite{*}

\chapter{Open Source Use Cases: Concept and Taxonomy}\label{sec:OSUCdeduction}

\footnotesize \begin{quote}\itshape This chapter establishes our concept of
\emph{open source use cases} as a classification system for to-do lists. The
conditions of a specific license, in the context of a par\-ti\-cu\-lar
\emph{open source use case}, shall be satisfiable by following the corresponding
to-do list. Additionally this chapter introduces a taxonomy for these \emph{open
source use cases}. Later on, this taxonomy will organize the \emph{Open Source
Use Case Finder}.
\end{quote}
\normalsize{}

After all these introductory remarks, we can summarize our idea. We know that
the right to use open source software depends on the tasks required by the open
source licenses. As opposed to commercial licenses, you can not buy the right to
use a piece of open source software by paying money. It is embedded into the
\emph{Open Source Definition} that the right to use the software may not be
sold. The OSD states first that an open source license may \enquote{[\ldots]
not restrict any party from selling or giving away the software as a component
of (any) aggregate software distribution}, and adds second in the same context
that an open source license \enquote{[\ldots] shall not require a royalty or
other fee for such sale}\footcite[cf.][\nopage wp §1]{OSI2012a}.

However, it would be wrong to conclude that you are automatically allowed to use
open source software without any service in return: generally you have to do
something to gain the right to use the software. In other words: open source
software is covered by the idea of ’paying by doing’. Accordingly, open source
li\-cen\-ses describe specific circumstances under which the user must execute
some tasks in order to be compliant with the licenses. So, if we want to offer
to-do lists for fulfilling license conditions, we must consider these tasks and
circumstances.

In practice, such circumstances are not linear and simple. They contain
combinations of (sometimes context sensitive) conditions which can be grouped
into classes of tokens. Such a class of tokens might denote a feature of the
software itself---such as being an application or a library. Or it can refer to
the circumstances of using the software, such as 'using the software only for
yourself' or 'distributing the software also to third parties'.

At the end, we want to determine a set of specific OSUCs---the \emph{open source
use cases}. And we want to deliver for each of these OSUCs and for each of the
considered open source licenses one list of actions which fulfills the license
in that context\footnote{Fortunately, sometimes one task list fulfills the
conditions of more than one use case---a welcome reduction of complexity}.

Such an \emph{open source use case} shall be considered as a set of tokens
describing the circumstances of a specific usage. Hence, to begin, we must
specify the relevant classes of tokens, before we can determine the valid
combinations of these tokens---our \emph{open source use cases}. Finally, based
on the tokens, we generate a taxonomy in the form of a tree. This tree will
become the base of the \emph{Open Source Use Case Finder} which will be offered
in the next chapter, and which leads you to your specific OSUC by evaluating
just a few questions and answers.

There are only a handful of tokens which are relevant to the circumstances of
open source software licenses:

\label{OsucTokens}
\begin{itemize}
  \item The \textbf{\underline{type} of the open source software}: On the one
  hand, we regard code snippets, modules, libraries and plugins, and on the
  other hand, autonomous applications, programs and servers. We will take the
  word ’snimolis’ for the first set, and ’proapses’ for the second. This is
  necessary, as we are not only talking about libraries and applications in the
  everyday sense, but rather in the broadest sense\footnote{Of course, our newly
  introduced concepts of 'snimoli' and 'proapse' are not absolutely one of the
  most elegant words. So, initially we tried to talk about 'applications' and
  'libraries', although in our context these words should denote more, than they
  traditionally do. But we couldn't minimize the irritations of our
  interlocutors. Too often we had to remind them that we were not talking about
  applications and libraries in the strict sense of the words. Finally we
  decided to find our own words---and to stay open for better proposals ;-) }.
  More specifically, we will ask you, whether the open source software you want
  to use, is an includable code snippet, a linkable module or library, or a
  loadable plugin, or whether it is an autonomous application or server which
  can be executed or processed. In the first case, the answer should be 'it is a
  \underline{snimoli}', in the second 'it is a \underline{proapse}'.

  \item The \textbf{\underline{state} of the open source software}: It might be
  used exactly as one has received it. Or it can be modified, before being used.
  More specifically, we will ask you, whether you want to leave the open source
  software as you have received it, or whether you want to modify it before
  using and/or distributing it to 3rd parties. In the first case, the answer
  should be '\underline{unmodified}', in the second '\underline{modified}'.
  
  \item The \textbf{usage \underline{context} of the open source
  software}: On the one hand you might use the received open source software as a
  readily prepared application. On the other hand you might embed the received
  open source into a larger application as one of its components. More
  specifically, we will ask you, whether you are using the open source
  software as an autonomous piece of software, or whether you are using it as an
  embedded part of a larger, more complex piece of software. In the first case,
  the answer should be '\underline{independent}', in the second
  '\underline{embedded}'.
  
  \item The \textbf{\underline{recipient} of the open source software}:
  Sometimes you might wish to use the received open source software only for
  yourself. In other cases you might intend to hand over the software (also) to
  other people. More specifically, we will ask you, whether you are going to use
  the open source software only for yourself, or whether you plan to
  (re)distribute it (also) to third parties. In the first case, the answer
  should be '\underline{4yourself}', in the second '\underline{2others}'.
 
  \item The \textbf{\underline{form} of the distributed files}: Many licenses
  also draw a distinction between distributing the software as sources and
  distributing the files as binaries. In this case, we will ask you, whether you
  want to distribute the software in the form of binaries or as source code. In
  the first case, the answer should be '\underline{binaries}', in the second
  '\underline{sources}'
  
  \item The kind of the \textbf{\underline{ioAccess} of the executed program}:
  At least one license draws a distinction between an open source based work
  offering only local access to its io data and an open source based work
  distributing its io data via internet. In the first case, the answer should be
  '\underline{onlyLocally}', in the second '\underline{viaInternet}'
\end{itemize}

From a more programmatic point-of-view, we can summarize these tokens as
follows:

\begin{itemize}
  \item \texttt{type::snimoli} \emph{or} \texttt{type::proapse}
  \item \texttt{state::unmodified} \emph{or} \texttt{state::modified}
  \item \texttt{context::independent} \emph{or} \texttt{context::embedded}
  \item \texttt{recipient::4yourself} \emph{or} \texttt{recipient::2others}
  \item \texttt{form::binaries} \emph{or} \texttt{form::sources}
  \item \texttt{ioAccess::onlyLocally} \emph{or} \texttt{ioAccess::viaInternet}
\end{itemize}

We have already defined the \emph{open source use case} as the combination of
these tokens. If we simply combine all these tokens of all these classes with
all the tokens of the other classes\footnote{in the sense of the cross product
TYPE $\times$ STATE $\times$ CONTEXT $\times$ RECIPIENT $\times$ FORM $\times$
IOACCESS. In some earlier versions of the \oslic{}, we also asked whether you
are going to combine or to embed the open source software with other software
components by linking them statically or dynamically, or by textually including
(parts of) the open source software into your larger product. Meanwhile, we
clearly discovered that it is unnecessary to increase the complexity by the
results of this question. For Details $\rightarrow$ \oslic{} p.\
\pageref{sec:LinkingSecondary}}, we get $2 \cdot 2 \cdot 2 \cdot 2 \cdot 2 \cdot
2 = 62$ sets of tokens---or 62 \emph{open source use cases}. Fortunately, some
of the generated sets are invalid from an empirical or logical view, and some of
these sets are context sensitive:

\begin{enumerate}
  \label{InvalidFinderTokenCombinations}
  \item If you already have specified that the used open source software is a
  \emph{proapse}---an autonomous program, an application, or a server---then
  your answer implies that the software is used independently and is not
  embedded with other components into a larger unit. But if you have specified
  that the used open source software is a \emph{snimoli}---a snippet of
  code, a module, a plugin, or a library---then it can indeed be used as an
  embedded component of a constructed larger application or server, or it can be
  used independently in case you 'only' re-distribute it to 3rd. parties.
  
  \item If you already have specified that the used open source software is a
  \emph{snimoli}---a snippet of code, a module, a plugin, or a library---and
  that this \emph{snimoli} shall be used only by yourself (not distributed to
  other 3rd.\ parties) then your answer must also imply that this \emph{snimoli}
  is used in combination, as an embedded part of a larger unit. A library can
  not be used autonomously, without using it as a component of another
  application. In this case, it would simply sit on the disk and would do
  nothing more than occupying space.
  
  \item To enquire the \emph{form} of the distributed files is only relevant if
  you have decided to distribute the software to other recipients
  \emph{2others}.
  
  \item With respect to the one license using the type of ioAccess as a
  discriminator, it is only relevant to enquire the type of the ioAccess if you
  either locally execute a \emph{modified} open source program \emph{4yourself}
  or if you locally execute a program \emph{4yourself}, which uses an
  \emph{embedded} open source component, regardless whether it has been modified
  or not.
  
\end{enumerate}

Does this sound complex? We thought so, too. We spent much time explaining these
constraints to ourselves, and only when we had transposed all the combinations
and rules into a tree, the situation became clearer. The following diagram
summarizes the main results of our investigation\footnote{ Each of the invalid
use cases (= sets of tokens) [for details s. p.\
\pageref{InvalidFinderTokenCombinations}] is marked by an \lightning{} and leads
to an empty set (= $\varnothing$). We are using the word 'invalid' a little
ambigiuosly: A combination of values is invalid, if it is empirically
impossible, to combine the features or if it is irrelavant to subclassify a
concept by the added features. Particularly:
\begin{itemize}
  \item A proapse can not be embedded into another software unit, also
  containing a main-function.
  \item Using a software library only for yourself and independently (not in
  combination with larger software unit), is like having an unused heap of bytes
  on your disc.
  \item To discriminate between sources and binaries is only valid in case of
  distributing software.
  \item To discriminate between an executed program with an only locally based
  io access and that with an internet based io access is only relevant, if you
  are using the software for yourself what implies to execute it.
\end{itemize}
 . 

}::

\tikzstyle{StartNode} = [font=\tiny, ellipse, draw, fill=gray!5,
text width=1em, text centered, minimum height=1em]

\tikzstyle{TypeNode} = [font=\tiny, rectangle, draw, fill=gray!10,
text width=1cm, text centered, rounded corners, minimum height=1em]

\tikzstyle{VLabelNode} = [font=\tiny, draw, rectangle,  text
width=1.4cm, fill=white]

\tikzstyle{VNibelNode} = [font=\tiny, draw, rectangle,  text
width=1cm, ]

\tikzstyle{VdTupelNode} = [font=\tiny, draw, rectangle, dotted, text
width=1.4cm, ]

\tikzstyle{VmTupelNode} = [font=\tiny, draw, rectangle, dotted, text
width=2.3cm, ]

\tikzstyle{VQuadrupelNode} = [font=\tiny, draw, rectangle, dotted, text
width=3cm, ]

\tikzstyle{VsQuadrupelNode} = [font=\tiny, draw, rectangle, dotted, text
width=2.7cm, ]


\tikzstyle{VlTupelNode} = [font=\tiny, draw, rectangle, dotted, text
width=3.8cm, ]


\tikzstyle{VmTupelLeaf} = [font=\tiny, draw, rectangle, dashed, text
width=2.7cm, ]

\tikzstyle{VlTupelLeaf} = [font=\tiny, draw, rectangle, dashed, text
width=3.8cm, ]

\tikzstyle{OsucNode} = [font=\tiny, rectangle, draw, fill=gray!20,
text width=1.4cm, text centered, rounded corners, minimum height=1em]



\tikzstyle{arrow} = [draw, -latex']
\tikzstyle{edge} = [draw]


\begin{tikzpicture}

% drwa lines at first:




% classification types and their values

\node[TypeNode] (tForm) at (2,7.4) {\textit{form?}};

\node[VNibelNode] (vSources) at (3,6.7) {sources};
\node[VNibelNode] (vBinaries) at (2.5,6.2) {binaries};

\node[TypeNode] (tType) at (0,15.6) {\textit{type?}};
\node[VNibelNode] (vProapse) at (2.8,15.9) {proapse};
\node[VNibelNode] (vSnimoli) at (2.8,15.4) {snimoli};

\node[TypeNode] (tState) at (0.8,12.6) {\textit{state?}};
\node[VLabelNode] (vUnmodified) at (2.6,12.9) {unmodified};
\node[VLabelNode] (vModified) at (2.6,12.4) {modified};

\node[TypeNode] (tContext) at (1.0,10.6) {\textit{context?}};
\node[VLabelNode] (vIndependent) at (2.7,10.9) {independent};
\node[VLabelNode] (vEmbedded) at (2.7,10.4) {embedded};

\node[TypeNode] (tRecipient) at (0.8,8.6) {\textit{recipient?}};
\node[VNibelNode] (v4yourself) at (2.3,8.9) {4yourself};
\node[VNibelNode] (v2others) at (2.3,8.4) {2others};

\node[TypeNode] (tIoAccess) at (0,1.6) {\textit{ioAccess?}};
\node[VLabelNode] (vViaInternet) at (2,1.9) {viaInternet};
\node[VLabelNode] (vOnlyLocal) at (2,1.4) {onlyLocal};

\node[StartNode] (vStart) at (0,10) {\textbf{\#}};

% value collictions defining the osucs

\node[OsucNode] (o1) at (12,19.8) {OSUC-01};
\node[VlTupelLeaf, color=blue] (vo1) at (15,19.8)
{\{proapse, independent,\\4yourself, unmodified\}};

\node[OsucNode] (o2s) at (12,19) {OSUC-02S};
\node[VlTupelLeaf, color=blue] (vo2s) at (15,19)
{\{proapse, independent,\\2others, unmodified, sources\}};

\node[OsucNode] (o2b) at (12,18.2) {OSUC-02B};
\node[VlTupelLeaf, color=blue] (vo2b) at (15,18.2)
{\{proapse, independent,\\2others, unmodified, binaries\}};

\node[OsucNode] (o3l) at (12,17.4) {OSUC-03L};
\node[VlTupelLeaf, color=blue] (vo3l) at (15,17.4)
{\{proapse, independent,\\4yourself, modified, onlyLocal\}};

\node[OsucNode] (o3n) at (12,16.6) {OSUC-03N};
\node[VlTupelLeaf, color=blue] (vo3n) at (15,16.6)
{\{proapse, independent,\\4yourself, modified, viaInternet\}};

\node[OsucNode] (o4s) at (12,15.8) {OSUC-04S};
\node[VlTupelLeaf, color=blue] (vo4s) at (15,15.8)
{\{proapse, independent,\\2others, modified, sources\}};

\node[OsucNode] (o4b) at (12,15) {OSUC-04B};
\node[VlTupelLeaf, color=blue] (vo4b) at (15,15)
{\{proapse, independent,\\2others, modified, binaries\}};

\node[OsucNode] (o5s) at (13,12.2) {OSUC-05S};
\node[VmTupelLeaf, color=blue] (vo5s) at (15.5,12.2)
{\{snimoli, independent,\\2others, unmodified,\\sources\}};

\node[OsucNode] (o5b) at (13,11.1) {OSUC-05B};
\node[VmTupelLeaf, color=blue] (vo5b) at (15.5,11.1)
{\{snimoli, independent,\\2others, unmodified,\\binaries\}};

\node[OsucNode] (o6l) at (13,10) {OSUC-06L};
\node[VmTupelLeaf, color=blue] (vo6l) at (15.5,10)
{\{snimoli, embedded,\\4yourself, unmodified,\\onlyLocal\}};

\node[OsucNode] (o6n) at (13,8.9) {OSUC-06N};
\node[VmTupelLeaf, color=blue] (vo6l) at (15.5,8.9)
{\{snimoli, embedded,\\4yourself, unmodified,\\viaInternet\}};

\node[OsucNode] (o7s) at (13,7.8) {OSUC-07S};
\node[VmTupelLeaf, color=blue] (vo7s) at (15.5,7.8)
{\{snimoli, embedded,\\2others, unmodified,\\sources\}};

\node[OsucNode] (o7b) at (13,6.7) {OSUC-07B};
\node[VmTupelLeaf, color=blue] (vo7b) at (15.5,6.7)
{\{snimoli, embedded,\\2others, unmodified,\\binaries\}};

\node[OsucNode] (o8s) at (13,5.6) {OSUC-08S};
\node[VmTupelLeaf, color=blue] (vo8s) at (15.5,5.6)
{\{snimoli, independent,\\2others, modified,\\sources\}};

\node[OsucNode] (o8b) at (13,4.4) {OSUC-08B};
\node[VmTupelLeaf, color=blue] (vo8b) at (15.5,4.4)
{\{snimoli, independent,\\2others, modified,\\binaries\}};

\node[OsucNode] (o9l) at (13,3.3) {OSUC-09L};
\node[VmTupelLeaf, color=blue] (vo9l) at (15.5,3.3)
{\{snimoli, embedded,\\4yourself, modified,\\onlyLocal\}};

\node[OsucNode] (o9n) at (13,2.2) {OSUC-09N};
\node[VmTupelLeaf, color=blue] (vo9n) at (15.5,2.2)
{\{snimoli, embedded,\\4yourself, modified,\\viaInternet\}};

\node[OsucNode] (o10s) at (13,1.1) {OSUC-010S};
\node[VmTupelLeaf, color=blue] (vo10s) at (15.5,1.1)
{\{snimoli, embedded,\\2others, modified,\\sources\}};

\node[OsucNode] (o10b) at (13,0) {OSUC-010B};
\node[VmTupelLeaf, color=blue] (vo10b) at (15.5,0)
{\{snimoli, embedded,\\2others, modified,\\binaries\}};

% concepts directly referring osucs

\node[VQuadrupelNode] (v4osuc1) at (5.4,19.8)
{\{proapse, 4yourself,\\ independent, unmodified\}};

\node[VQuadrupelNode] (v4osuc2) at (7,18.7)
{\{proapse, 2others,\\ independent, unmodified\}};

\node[VsQuadrupelNode] (v4osuc3) at (8.3,17.6)
{\{proapse, 4yourself,\\ independent, modified\}};

\node[VsQuadrupelNode] (v4osuc4) at (9.6,16.5)
{\{proapse, 2others,\\ independent, modified\}};

\node[VlTupelLeaf, color=red] (vLightning1) at (10.4,14.3)
{\{proapse, 4yourself, embedded,\\ \{unmodified, modified\}\} \lightning{}};

\node[VlTupelLeaf, color=red] (vLightning2) at (10.4,13.6)
{\{proapse, 2others, embedded,\\ \{unmodified, modified\}\} \lightning{}};

\node[VlTupelLeaf, color=red] (vLightning3) at (10.4,12.9)
{\{snimoli, 4yourself, independent,\\ \{unmodified, modified\}\} \lightning{}};

\node[VQuadrupelNode] (v4osuc5) at (10.2,11.8)
{\{snimoli, 2others,\\ independent, unmodified\}};

\node[VsQuadrupelNode] (v4osuc6) at (9.5,10.8)
{\{snimoli, 4yourself,\\ embedded, unmodified\}};

\node[VsQuadrupelNode] (v4osuc7) at (9.2,9.8)
{\{snimoli, 2others,\\ embedded, unmodified\}};

\node[VsQuadrupelNode] (v4osuc8) at (8.2,8.8)
{\{snimoli, 4yourself,\\ embedded, modified\}};

\node[VsQuadrupelNode] (v4osuc9) at (7.3,7.6)
{\{snimoli, 2others,\\ independent, modified\}};

\node[VsQuadrupelNode] (v4osuc10) at (6.2,6.4)
{\{snimoli, 2others,\\ embedded, modified\}};

%meta nodes reffering concepts reffering osucs


\node[VmTupelNode] (v2otherSources) at (6,5.4) {\{2others, sources\}};
\node[VmTupelNode] (v2otherBinaries) at (6,4.8) {\{2others, binaries\}};

\node[VlTupelNode] (v4proapseViaInternet) at (7.2,4.0)
  {\{proapse, 4yourself,\\ modified, viaInternet\}};
\node[VlTupelNode] (v4proapseOnlyLocal) at (7.2,3.2)
  {\{proapse, 4yourself,\\ modified, onlyLocal\}};

\node[VlTupelLeaf, color=red] (vLightning4) at (7.2,2.4)
  {\{proapse, 4yourself, unmodified,\\ \{viaInternet, onlyLocal\} \} \lightning{}};

\node[VlTupelNode] (v4snimoliViaInternet) at (7.2,1.6)
  {\{snimoli, 4yourself,\\ embedded, viaInternet\}};
\node[VlTupelNode] (v4snimoliOnlyLocal) at (7.2,0.8)
  {\{snimoli, 4yourself,\\ embedded, onlyLocal\}};

\node[VlTupelLeaf, color=red] (vLightning5) at (7.2,0)
  {\{snimoli, 4yourself, independent,\\ \{viaInternet, onlyLocal\} \} \lightning{}};

% entry paths

\path [edge] (vStart) -- (tType);
\path [edge] (tType) -- (vProapse);
\path [edge] (tType) -- (vSnimoli);

\path [edge] (vStart) -- (tState);
\path [edge] (tState) -- (vUnmodified);
\path [edge] (tState) -- (vModified);

\path [edge] (vStart) -- (tContext);
\path [edge] (tContext) -- (vIndependent);
\path [edge] (tContext) -- (vEmbedded);

\path [edge] (vStart) -- (tRecipient);
\path [edge] (tRecipient) -- (v4yourself);
\path [edge] (tRecipient) -- (v2others);

\path [edge] (v2others) -- (tForm);
\path [edge] (tForm) -- (vSources);
\path [edge] (tForm) -- (vBinaries);

\path [edge] (vStart) -- (tIoAccess);
\path [edge] (tIoAccess) -- (vViaInternet);
\path [edge] (tIoAccess) -- (vOnlyLocal);


% middle paths
\path [edge] (vProapse) -- (v4osuc1);
\path [edge] (vProapse) -- (v4osuc2);
\path [edge] (vProapse) -- (v4osuc3);
\path [edge] (vProapse) -- (v4osuc4);
\path [edge] (vProapse) -- (vLightning1);
\path [edge] (vProapse) -- (vLightning2);
\path [edge] (vSnimoli) -- (vLightning3);
\path [edge] (vSnimoli) -- (v4osuc5);
\path [edge] (vSnimoli) -- (v4osuc6);
\path [edge] (vSnimoli) -- (v4osuc7);
\path [edge] (vSnimoli) -- (v4osuc8);
\path [edge] (vSnimoli) -- (v4osuc9);
\path [edge] (vSnimoli) -- (v4osuc10);

\path [edge] (vUnmodified) -- (v4osuc1);
\path [edge] (vUnmodified) -- (v4osuc2);
\path [edge] (vModified) -- (v4osuc3);
\path [edge] (vModified) -- (v4osuc4);
\path [edge] (vUnmodified) -- (vLightning1);
\path [edge] (vUnmodified) -- (vLightning2);
\path [edge] (vUnmodified) -- (vLightning3);
\path [edge] (vUnmodified) -- (v4osuc5);
\path [edge] (vUnmodified) -- (v4osuc6);
\path [edge] (vUnmodified) -- (v4osuc7);
\path [edge] (vModified) -- (v4osuc8);
\path [edge] (vModified) -- (v4osuc9);
\path [edge] (vModified) -- (v4osuc10);

\path [edge] (vIndependent) -- (v4osuc1);
\path [edge] (vIndependent) -- (v4osuc2);
\path [edge] (vIndependent) -- (v4osuc3);
\path [edge] (vIndependent) -- (v4osuc4);
\path [edge] (vIndependent) -- (vLightning3);
\path [edge] (vEmbedded) -- (vLightning1);
\path [edge] (vEmbedded) -- (vLightning2);
\path [edge] (vIndependent) -- (v4osuc5);
\path [edge] (vEmbedded) -- (v4osuc6);
\path [edge] (vEmbedded) -- (v4osuc7);
\path [edge] (vEmbedded) -- (v4osuc8);
\path [edge] (vIndependent) -- (v4osuc9);
\path [edge] (vEmbedded) -- (v4osuc10);

\path [edge] (v4yourself) -- (v4osuc1);
\path [edge] (v2others) -- (v4osuc2);
\path [edge] (v4yourself) -- (v4osuc3);
\path [edge] (v2others) -- (v4osuc4);
\path [edge] (v2others) -- (v4osuc5);
\path [edge] (v4yourself) -- (v4osuc6);
\path [edge] (v2others) -- (v4osuc7);
\path [edge] (v4yourself) -- (v4osuc8);
\path [edge] (v2others) -- (v4osuc9);
\path [edge] (v4yourself) -- (v4osuc10);


\draw[->] (vSources) to [out=0,in=180] (v2otherSources);
\draw[->] (vBinaries) to [out=0,in=180] (v2otherBinaries);

\draw[->] (vViaInternet) to [out=0,in=180] (v4proapseViaInternet);
\draw[->] (vViaInternet) to [out=0,in=180] (vLightning4);
\draw[->] (vViaInternet) to [out=0,in=180] (v4snimoliViaInternet);
\draw[->] (vViaInternet) to [out=0,in=180] (vLightning5);

\draw[->] (vOnlyLocal) to [out=0,in=180] (v4proapseOnlyLocal);
\draw[->] (vOnlyLocal) to [out=0,in=180] (vLightning4);
\draw[->] (vOnlyLocal) to [out=0,in=180] (v4snimoliOnlyLocal);
\draw[->] (vOnlyLocal) to [out=0,in=180] (vLightning5);

\draw[->] (v4yourself) to [out=0,in=180] (v4proapseViaInternet);
\draw[->] (v4yourself) to [out=0,in=180] (v4proapseOnlyLocal);
\draw[->] (v4yourself) to [out=0,in=180] (vLightning4);
\draw[->] (v4yourself) to [out=0,in=180] (v4snimoliViaInternet);
\draw[->] (v4yourself) to [out=0,in=180] (v4snimoliOnlyLocal);
\draw[->] (v4yourself) to [out=0,in=180] (vLightning5);

\draw[->] (vModified) to [out=0,in=180] (v4proapseViaInternet);
\draw[->] (vModified) to [out=0,in=180] (v4proapseOnlyLocal);
\draw[->] (vUnmodified) to [out=0,in=180] (vLightning4);
\draw[->] (vEmbedded) to [out=0,in=180] (v4snimoliViaInternet);
\draw[->] (vEmbedded) to [out=0,in=180] (v4snimoliOnlyLocal);
\draw[->] (vIndependent) to [out=0,in=180] (vLightning5);





\path [arrow, color=magenta] (v4osuc1) -- (o1);
\path [arrow, color=magenta] (v4osuc2) -- (o2s);
\path [arrow, color=magenta] (v4osuc2) -- (o2b);
\path [arrow, color=magenta] (v4osuc3) -- (o3l);
\path [arrow, color=magenta] (v4osuc3) -- (o3n);
\path [arrow, color=magenta] (v4osuc4) -- (o4s);
\path [arrow, color=magenta] (v4osuc4) -- (o4b);
\path [arrow, color=magenta] (v4osuc5) -- (o5s);
\path [arrow, color=magenta] (v4osuc5) -- (o5b);
\path [arrow, color=magenta] (v4osuc6) -- (o6l);
\path [arrow, color=magenta] (v4osuc6) -- (o6n);
\path [arrow, color=magenta] (v4osuc7) -- (o7s);
\path [arrow, color=magenta] (v4osuc7) -- (o7b);
\path [arrow, color=magenta] (v4osuc8) -- (o8s);
\path [arrow, color=magenta] (v4osuc8) -- (o8b);
\path [arrow, color=magenta] (v4osuc9) -- (o9l);
\path [arrow, color=magenta] (v4osuc9) -- (o9n);
\path [arrow, color=magenta] (v4osuc10) -- (o10s);
\path [arrow, color=magenta] (v4osuc10) -- (o10b);

\draw[->, color=violet] (v4snimoliOnlyLocal) to [out=0,in=180] (o9l);
\draw[->, color=violet] (v4snimoliViaInternet) to [out=0,in=180] (o9n);
\draw[->, color=violet] (v4snimoliOnlyLocal) to [out=0,in=180] (o6l);
\draw[->, color=violet] (v4snimoliViaInternet) to [out=0,in=180] (o6n);

\draw[->, color=violet] (v4proapseOnlyLocal) to [out=0,in=180] (o3l);
\draw[->, color=violet] (v4proapseViaInternet) to [out=0,in=180] (o3n);


   \draw[->, color=blue] (v2otherSources) to [out=0,in=224] (o2s);
   \draw[->, color=blue] (v2otherBinaries) to [out=0,in=220] (o2b);
   \draw[->, color=blue] (v2otherSources) to [out=0,in=216] (o4s);
   \draw[->, color=blue] (v2otherBinaries) to [out=0,in=212] (o4b);
   \draw[->, color=blue] (v2otherSources) to [out=0,in=208] (o5s);
   \draw[->, color=blue] (v2otherBinaries) to [out=0,in=204] (o5b);
   \draw[->, color=blue] (v2otherSources) to [out=0,in=200] (o7s);
   \draw[->, color=blue] (v2otherBinaries) to [out=0,in=196] (o7b);
   \draw[->, color=blue] (v2otherSources) to [out=0,in=192] (o8s);
   \draw[->, color=blue] (v2otherBinaries) to [out=0,in=188] (o8b);
   \draw[->, color=blue] (v2otherSources) to [out=0,in=184] (o10s);
   \draw[->, color=blue] (v2otherBinaries) to [out=0,in=180] (o10b);

% 


\end{tikzpicture}





%%%%%%%%%%%%%%%
% Telekom osCompendium 'for being included' snippet template
%
% (c) Karsten Reincke, Deutsche Telekom AG, Darmstadt 2011
%
% This LaTeX-File is licensed under the Creative Commons Attribution-ShareAlike
% 3.0 Germany License (http://creativecommons.org/licenses/by-sa/3.0/de/): Feel
% free 'to share (to copy, distribute and transmit)' or 'to remix (to adapt)'
% it, if you '... distribute the resulting work under the same or similar
% license to this one' and if you respect how 'you must attribute the work in
% the manner specified by the author ...':
%
% In an internet based reuse please link the reused parts to www.telekom.com and
% mention the original authors and Deutsche Telekom AG in a suitable manner. In
% a paper-like reuse please insert a short hint to www.telekom.com and to the
% original authors and Deutsche Telekom AG into your preface. For normal
% quotations please use the scientific standard to cite.
%
% [ Framework derived from 'mind your Scholar Research Framework' 
%   mycsrf (c) K. Reincke 2012 CC BY 3.0  http://mycsrf.fodina.de/ ]
%


%% use all entries of the bibliography
%\nocite{*}

\chapter{Open Source Use Cases: Find the License Fulfilling To-do Lists}\label{sec:OSUCfinder}

\footnotesize
\begin{quote}\itshape
This chapter offers the \emph{Open Source Use Case Finder}: Based on the
information gathered by a form, it allows to traverse a tree whose leaves are
linked to the \emph{open source use cases} which finally refer to the respective
to-do lists.
\end{quote}
\normalsize{}

\section{A standard form for gathering the relevant information}
\label{OSLiCStandardFormForGatheringInformation}
 
{
% The 8.5cm below were found by trial and error - 8.6cm will generate an
% overfull hbox. 
\newcommand{\question[1]}{\parbox[c][#1][c]{8.5cm}}
\newcommand{\checkboxes}[2]{%
    \parbox{7.5em}{
        $\square$\hspace{1em}#1\\
        $\square$\hspace{1em}#2}}

\begin{small}
\begin{tabular}[h]{|l|l|l|l|}
\hline 
  \ & \textit{Which open source software do you want to use?} & \ \\
\hline 
  \ & \textit{Under which open source license is it released?} & \ \\
\hline
\hline 
\textbf{Focus} & \textbf{Questions} & \textbf{Answers}\\
\hline 
\hline 
  Type
  & \question[2.9cm]{
    \textit{Is the open source software you want to use a library in the
    broadest sense (an includable code \textbf{\underline{sni}}ppet, a linkable
    \textbf{\underline{mo}}dule or \textbf{\underline{li}}brary, or a loadable
    plugin), or is it an autonomous \textbf{\underline{pro}}gram,
    \textbf{\underline{ap}}plication, or \textbf{\underline{se}}rver which can be
    executed?}} 
  & \checkboxes{snimoli}{proapse}
  \\
\hline 
  State 
  & \question[1.7cm]{
    \textit{Do you want to leave the open source software
    \textbf{\underline{unmodified}} as you have received it, or are you going to
    create a \textbf{\underline{modified}} version of it?}} 
  & \checkboxes{unmodified}{modified}
  \\
\hline 
  Context 
  & \question[2.15cm]{ 
    \textit{Are you going to use / distribute the open source software as an
    \textbf{\underline{independent}} unit, or do you plan to integrate it as an
    \textbf{\underline{embedded}} component into a complexer piece of software?}}
  & \checkboxes{independent}{embedded}
  \\
\hline 
  Recipient 
  & \question[1.7cm]{ 
    \textit{Are you going to use the open source
    software only \textbf{\underline{for}} \textbf{\underline{yourself}}, or do
    you plan to (re)distribute it (also) \textbf{\underline{to}}
    \textbf{\underline{other}} third parties?}}
  & \checkboxes{4yourself}{2others}
  \\
\hline 
\hline
  Form 
  & \question[1.7cm]{
    \textit{Given you want to (re)distribute an open source based work [2others],
    do you focus on distributing the \textbf{\underline{binaries}} or the
    \textbf{\underline{sources}}?}}
  & \checkboxes{binaries}{sources}
  \\
\hline
  IoAccess 
  & \question[2.6cm]{
    \textit{Given you are using open source software [4yourself] by executing a
    modified os program [modified] or by creating \& executing a program using
    an os library [embedded], does this program distribute its IO data
    \textbf{\underline{only locally}} or \textbf{\underline{via internet}}?}} &
    \checkboxes{onlyLocally}{viaInternet}
  \\
\hline 
\hline
\end{tabular}
\end{small}
}

As discussed earlier, there are of course some invalid or irrelevant
combinations.\footnote{type::proapse excludes state::embedded;
recipient::4yourself excludes the combination with state::independent and
type::snimoli; any value of class 'mode' implies state::embedded; form is only
relevant if recipient::2others; ioAccess is only relevant if
recipient::4yourself[for details see page
\pageref{InvalidFinderTokenCombinations}]. If you have encountered one of these
invalid combinations, please check the corresponding explanations.} Here are
some extra explanations concerning the classes resp. the focuses:

\begin{description}
\item[Type:] A piece of (open source) software is a program, an application, or
a server, only if you can start its binary form with your normal program
launcher, or (in case of a text file which still must be interpreted by an
interpreter like php, perl, bash etc.) if you can start an interpreter which
takes the file as one of its arguments and executes the commands.
\item[State:] You are modifying a piece of (open source) software if you expand,
reduce or modify at least one of the received software files, and---in case of
dealing with binary object code---if you (re)compile and (re)link the modified
software to a new binary file. But if you only modify some of the configuration
files, you are not modifying the open source software itself.
\item[Context:] You are using a piece of open source software as an embedded
component of a larger unit \ldots
  \begin{itemize}
  \item  if one of your files of the larger unit contains a verbatim or a
  modified copy (i.e.\ a snippet) of the received open source software, or
  \item if your larger unit contains an include statement referring to a
  functionally defining file of the received open source software, or
  \item if your larger unit calls a function defined in the received open source
  software, or
  \item if your development environment contains a compiler or linker directive
  referring to the received open source software (binaries) and if your larger
  unit can't be executed without resolving this linker directive.
  \end{itemize}
\item[Recipient:] You are using the received open source software only for
yourself, if you as a person do not pass it to other entities like persons,
organizations, companies etc., or if you---as a member of a specific
development group---pass it only to the other members of your development
group. But if you store open source software on any device such as a mobile
phone, an USB stick, etc.\ or if you attach it to any transport medium like
email etc.\ and if you then sell, give away, or simply send this device or
transport medium to anyone (other than a direct member of your development
group) then you indeed hand the open source software over to third
parties.\footnote{Please remember that---at least in Germany---there are
opinions that even handing over software to another legal entity or department
of the same company is also a kind of distribution. It is always safest to take
the broadest possible meaning.}
\item[Form:] Open source software knows two ways to distribute the software: in
the form of binaries and in the form of sources. Mostly it is up to you to
decide whether you want to distribute only the binaries or whether you are
intentionally going to distribute the sources (too). At a first glance, the
concepts 'sources' and 'binaries' seems to be clearly distinguished.
On the one hand, compiled sources should be taken as binaries. On the other
hand, editable pieces of software are denoted by the concept 'sources'. But
sometimes the difference is not as clear as wished: For example, you can modify
even already compiled object files by using an hex-editor. Or it is very
difficult to modify the minimized versions of javascript files even if they are
indeed text files. Therefore, the OSLiC 'reuses' a famous \textbf{rule of
thumb}: \enquote{The source code for a work means the preferred form of the work
for making modifications to it}.\citeGPLtwo{§3} All other forms are denoted by
the concept of 'binaries'. Based on this specification, you can respect some
special conditions if you want to distribute the sources and/or the binaries.
\item[ioAccess:] If you execute an open source program or an own program using
an open source library, then (normally) you do not distribute that software.
Under these circumstances, the most open source licenses do not require anything
for executing the program compliantly - even if it is the base of a globally
used internet service. For closing this 'gap', the AGPL has been invented: Like
the GPL, the AGPL let the obligation to fulfill the well known set of GPL tasks
be triggered by distributing the software. But, it let these tasks also be
triggered by an established remote network interaction: whoever interacts with
the locally executed program remotely through a computer network gets all the
rights which normally the receiver of a distribution gets. Nevertheless, the
AGPL does not wish to cause an overhead of tasks: Only \emph{locally excuted
open source programs which have been modfied} or \emph{locally executed own
programs using an AGPL licensed library} shall indeed trigger the fulfillment of
the requirements. Thus, we introduced the features \emph{ioAccess:onlyLocally}
and \emph{ioAccess:viaInternet}: They are only relevant if you uses a program
only for yourself (4yourself) \textbf{and} [ (if that AGPL licensed program has
been modified \{proapse and modified\}) \textbf{or} (if that program uses an
embedded AGPL licensed library \{snimoli and embedded\}) ].

\end{description}

\section{The taxonomic Open Source Use Case Finder}

Now, after having gathered the necessary information, determine your 
open source use case by traversing the following tree and its corresponding
branches:

{
\newcommand{\choicetext}[2]{\tiny #1:\\ \textbf{\textit{#2}}}

\newcommand{\cunmodified}{\choicetext{state}{unmodified}}
\newcommand{\cmodified}{\choicetext{state}{modified}}
\newcommand{\cindependent}{\choicetext{context}{independent}}
\newcommand{\cembedded}{\choicetext{context}{embedded}}
\newcommand{\cyourself}{\choicetext{recipient}{4yourself}}
\newcommand{\cothers}{\choicetext{recipient}{2others}}
\newcommand{\csources}{\choicetext{form}{sources}}
\newcommand{\cbinaries}{\choicetext{form}{binaries}}
\newcommand{\conlylocal}{\choicetext{ioAccess}{onlyLocal}}
\newcommand{\cviainternet}{\choicetext{ioAccess}{viaInternet}}


\newcommand{\osuctxtshort}[1]{$\Rightarrow$ OSUC-#1: \textit{p.\ \pageref{OSUC-#1-DEF}}}
\newcommand{\osuctxtbreak}[1]{$\Rightarrow$ OSUC-#1\\ \textit{(see p.\ \pageref{OSUC-#1-DEF})}}
\newcommand{\osucchild}[1]{child { node[anchor=west] {#1} edge from parent[draw=none] }}

\tikzset{choice/.style={rectangle, draw, rounded corners}}

\begin{tikzpicture}[
    font=\scriptsize,
    align=left,
    grow'=right,
    level 1/.style={sibling distance=27em, level distance=18mm},
    level 2/.style={sibling distance=15em, level distance=18mm},
    level 3/.style={sibling distance=8em, level distance=24mm},
    level 4/.style={sibling distance=5em, level distance=24mm},
    level 5/.style={sibling distance=2.5em, level distance=24mm},
    level 6/.style={sibling distance=1em, level distance=18mm},
%     level 3/.style={sibling distance=7em,  level distance=18mm, anchor=west, minimum width=2cm},
%     level 4/.style={sibling distance=3em,  level distance=18mm, minimum width=1.65cm},
%     level 5/.style={sibling distance=3em,  level distance=18mm, minimum width=1.45cm},
%     level 6/.style={sibling distance=3em,  level distance=6mm},
]
\node [ellipse,draw] {OSS}
    child { node [choice] { \choicetext{type}{proapse} }
      child { node [choice] {\cunmodified}
        child { node [choice] {\cindependent}
          child { node[choice] {\cyourself} 
            \osucchild{\osuctxtshort{01}}
          }
          child { node[choice] {\cothers} 
            child { node[choice] {\csources} 
              \osucchild{\osuctxtbreak{02S}}
            }
            child { node[choice] {\cbinaries} 
             \osucchild{\osuctxtbreak{02B}}
            }
          }
        }
      }
      child { node [choice] {\cmodified}
        child { node [choice] {\cindependent}
          child { node[choice] {\cyourself} 
            child { node[choice] {\conlylocal} 
              \osucchild{\osuctxtbreak{03L}}
            }
            child { node[choice] {\cviainternet} 
              \osucchild{\osuctxtbreak{03N}}
            }
          }
          child { node[choice] {\cothers} 
            child { node[choice] {\csources} 
              \osucchild{\osuctxtbreak{04S}}
            }
            child { node[choice] {\cbinaries} 
              \osucchild{\osuctxtbreak{04B}}
            }
          }
        }
      }
    }
    child { node [choice] { \choicetext{type}{snimoli} }
      child { node [choice] {\cunmodified}
        child { node [choice] {\cindependent} 
          child { node[choice] {\cothers} 
            child { node[choice] {\csources} 
              \osucchild{\osuctxtbreak{05S}}
            }
            child { node[choice] {\cbinaries} 
              \osucchild{\osuctxtbreak{05B}}
            }
          }
        }
        child { node [choice] {\cembedded} 
          child { node[choice] {\cyourself} 
            child { node[choice] {\conlylocal} 
              \osucchild{\osuctxtbreak{06L}}
            }
            child { node[choice] {\cviainternet} 
              \osucchild{\osuctxtbreak{06N}}
            }
          }
          child { node[choice] {\cothers} 
            child { node[choice] {\csources} 
              \osucchild{\osuctxtbreak{07S}}
            }
            child { node[choice] {\cbinaries} 
              \osucchild{\osuctxtbreak{07B}}
            }
          }
        }
      }
      child { node [choice] {\cmodified}
        child { node [choice] {\cindependent} 
          child { node[choice] {\cothers} 
            child { node[choice] {\csources} 
              \osucchild{\osuctxtbreak{08S}}
            }
            child { node[choice] {\cbinaries} 
              \osucchild{\osuctxtbreak{08B}}
            }
          }
        }
        child { node [choice] {\cembedded} 
          child { node[choice] {\cyourself} 
            child { node[choice] {\conlylocal} 
              \osucchild{\osuctxtbreak{09L}}
            }
            child { node[choice] {\cviainternet} 
              \osucchild{\osuctxtbreak{09N}}
            }
          }
          child { node[choice] {\cothers} 
            child { node[choice] {\csources} 
              \osucchild{\osuctxtbreak{10S}}
            }
            child { node[choice] {\cbinaries} 
              \osucchild{\osuctxtbreak{10B}}
            }
          }
        }
      }
    };
\end{tikzpicture}
\label{OSLiCUseCaseFinder}
}


\section{The open source use cases and its to-do list references}

On the following pages, each \textbf{O}pen \textbf{S}ource \textbf{U}se
\textbf{C}ase is textually specified one more time and complemented by a list of
page numbers. Each of these pages covers the license-specific to-do list whose
items together offer a processable way for acting according to the license under
the circumstances of the described \textbf{O}pen \textbf{S}ource \textbf{U}se
\textbf{C}ase.


\begin{osucdefinitions}
\bgroup
\newcommand{\osuclinktable}[1]{%
  To see the \textit{specific, license fulfilling to-do lists}
  jump to the following pages:
  \begin{itemize}
    \item p.\ \pageref{OSUC-#1-AGPL} for the \textbf{AGPL-3.0}
      \textit{(= GNU Affero General Public License)} 
    \item p.\ \pageref{OSUC-#1-APL} for the \textbf{Apache-2.0}
      \textit{(= Apache License)}
    \item p.\ \pageref{OSUC-#1-BSD2} for the \textbf{BSD-2-Clause} License
      \textit{(= Berkeley Software Distribution)}
    \item p.\ \pageref{OSUC-#1-BSD3} for the \textbf{BSD-3-Clause} License
      \textit{(= Berkeley Software Distribution)}
    \item p.\ \pageref{OSUC-#1-CDDL} for the \textbf{CDDL-1.0}
      \textit{(= Common Develop and Distribution License)}  
    \item p.\ \pageref{OSUC-#1-EPL} for the \textbf{EPL-1.0}
      \textit{(= Eclipse Public License)}     
    \item p.\ \pageref{OSUC-#1-EUPL} for the \textbf{EUPL-1.1}
      \textit{(= European Union Public License)} 
    \item p.\ \pageref{OSUC-#1-GPL2} for the \textbf{GPL-2.0}
       \textit{(= GNU General Public License Version 2)} 
    \item p.\ \pageref{OSUC-#1-GPL3} for the \textbf{GPL-3.0}
       \textit{(= GNU General Public License Version 3)} 
    \item p.\ \pageref{OSUC-#1-LGPL2} for the \textbf{LGPL-2.1}
      \textit{(= GNU Lesser General Public License Version 2.1)}           
    \item p.\ \pageref{OSUC-#1-LGPL3} for the \textbf{LGPL-3.0}
      \textit{(= GNU Lesser General Public License Version 3)}           
    \item p.\ \pageref{OSUC-#1-MIT} for the \textbf{MIT} License
       \textit{(= Massachusetts Institute of Technology)} 
    \item p.\ \pageref{OSUC-#1-MPL} for the \textbf{MPL}
      \textit{(= Mozilla Public License)}     
    \item p.\ \pageref{OSUC-#1-MSPL} for the \textbf{MS-PL}
      \textit{(= Microsoft Public License)} 
    \item p.\ \pageref{OSUC-#1-PGL} for the \textbf{PostgreSQL}
      \textit{(= Postgres License)} 
    \item p.\ \pageref{OSUC-#1-PHP} for the \textbf{PHP-3.0} License 
  \end{itemize}}

\newcommand{\osucitem}[3]{%
  \osucdef{#1}{#2}{#3}
  \osuclinktable{#1}}

\begin{description}
\label{OSUCList}
\osucitem{01}{proapse, unmodified, independent, 4yourself}{%
Only for yourself, you are going to use an unmodified open source program,
application, or server just as you received it. But you do not combine it with
other components in the sense of software development} 

\osucitem{02S}{proapse, unmodified, independent, 2others, sources}{%
Just as you received it, you are going to distribute an unmodified open source
program, application, or server to third parties in the form of sources. In this
act of distribution, you do not combine this program, application, or server
with other software components in the sense of software development} 

\osucitem{02B}{proapse, unmodified, independent, 2others, binaries}{%
Just as you received it, you are going to distribute an unmodified open source
program, application, or server to third parties in the form of binaries. In
this act of distribution, you do not combine this program, application, or
server with other software components in the sense of software development} 
  
% \osucitem{03}{proapse, modified, independent, 4yourself}{%
% Only for yourself, you are going to modify an open source program, application,
% or server after you received it and before you will use it. But you do not
% combine it with other components in the sense of software development} 

\osucitem{03L}{proapse, modified, independent, 4yourself, onlyLocal}{% 
You are executing an open source program, application, or server which you have
modified (but not combined with other components in the sense of software
development) and which distributes its input/output only locally to you}

\osucitem{03N}{proapse, modified, independent, 4yourself, viaInternet}{% 
You are executing an open source program, application, or server which you have
modified (but not combined with other components in the sense of software
development) and which distributes its input/output to you or other users via the
internet}

\osucitem{04S}{proapse, modified, independent, 2others, sources}{%
You are going to modify an open source program, application, or server after you
received it and  before you will distribute it to third parties in the form of
sources. But you do not combine this modified program, application, or server
with other software components in the sense of software development}
  
\osucitem{04B}{proapse, modified, independent, 2others, binaries}{%
You are going to modify an open source program, application, or server after you
received it and before you will distribute it to third parties in the form of
binaries. But you do not combine this modified program, application, or server
with other software components in the sense of software development}

\osucitem{05S}{snimoli, unmodified, independent, 2others, sources}{%
Just as you received it, you are going to distribute an unmodified open source
library, code snippet, module, or plugin to third parties in the form of
sources. In this act of distribution, you do not combine this library, code
snippet, module, or plugin with other software components in the sense of
software development} 

\osucitem{05B}{snimoli, unmodified,independent, 2others, binaries}{%
Just as you received it, you are going to distribute an unmodified open source
library, code snippet, module, or plugin to third parties in the form of
binaries. In this act of distribution, you do not combine this library, code
snippet, module, or plugin with other software components in the sense of
software development} 

% \osucitem{06}{snimoli, unmodified, embedded, 4yourself}{%
% Only for yourself and just as you received it, you are going to combine an
% unmodified open source library, code snippet, module, or plugin into a larger
% software unit as one of its parts}  

\osucitem{06L}{snimoli, umodified, embedded, 4yourself, onlyLocal}{% 
You are executing any application which distributes input/output only locally to
you and which uses an unmodified embedded open source library, code snippet,
module, or plugin}

\osucitem{06N}{snimoli, umodified, embedded, 4yourself, viaInternet}{%
You are executing any application which distributes its input/output to you or
other users via the internet and which uses an unmodified embedded open source
library, code snippet, module, or plugin}


\osucitem{07S}{snimoli, unmodified, embedded, 2others, sources}{%
Just as you received it and before you will distribute it to third parties in
the form of sources and together with a larger software unit, you are going to
combine and embed an unmodified open source library, code snippet, module, or
plugin into that larger software unit in the sense of software development}

\osucitem{07B}{snimoli, unmodified, embedded, 2others, binaries}{%
Just as you received it and before you will distribute it to third parties in
the form of binaries and together with a larger software unit, you are going to
combine and embed an unmodified open source library, code snippet, module, or
plugin into that larger software unit in the sense of software development}

\osucitem{08S}{snimoli, modified, independent, 2others, sources}{%
Before you will distribute it to third parties in the form of sources, you are
going to modify an open source library, code snippet, module, or plugin. But you
do not combine it with other software components in the sense of software
development}  

\osucitem{08B}{snimoli, modified, independent, 2others, binaries}{%
Before you will distribute it to third parties in the form of binaries, you are
going to modify an open source library, code snippet, module, or plugin. But you
do not combine it with other software components in the sense of software
development} 

% \osucitem{09}{snimoli, modified, embedded, 4yourself}{%
% Only for yourself, you are going to modify an open source library, code snippet,
% module, or plugin, and you will combine it in the sense of software development
% into a larger software unit as one of its parts} 

\osucitem{09L}{snimoli, modified, embedded, 4yourself, onlyLocal}{% 
You are executing any application which distributes input/output only locally to
you and which uses an embedded open source library, code snippet, module, or
plugin -- being modified by you}

\osucitem{09N}{snimoli, modified, embedded, 4yourself, viaInternet}{%
You are executing any application which distributes its input/output to you or
other users via the internet and which uses an embedded open source library,
code snippet, module, or plugin -- being modified by you}

\osucitem{10S}{snimoli, modified, embedded, 2others, sources}{%
Before you will distribute it to third parties in the form of sources, you are
going to modify an open source library, code snippet, module, or plugin, which
you combine with other software components in the sense of software development}

\osucitem{10B}{snimoli, modified, embedded, 2others, binaries}{%
Before you will distribute it to third parties in the form of binaries, you are
going to modify an open source library, code snippet, module, or plugin, which
you combine with other software components in the sense of software development}
\end{description}
\egroup
\end{osucdefinitions}

%\bibliography{../../../bibfiles/oscResourcesEn}

% Local Variables:
% mode: latex
% fill-column: 80
% End:



%%%%%%%%%%%%%%%

% Enclose 060x files in braces, because they define local commands that should
% not interfere with each other.  Do NOT enclose 0600-common-text-blocks in an
% extra pair of braces, because it defines commands used by the other sections. 

% Telekom osCompendium 'for being included' snippet template
%
% (c) Karsten Reincke, Deutsche Telekom AG, Darmstadt 2011
%
% This LaTeX-File is licensed under the Creative Commons Attribution-ShareAlike
% 3.0 Germany License (http://creativecommons.org/licenses/by-sa/3.0/de/): Feel
% free 'to share (to copy, distribute and transmit)' or 'to remix (to adapt)'
% it, if you '... distribute the resulting work under the same or similar
% license to this one' and if you respect how 'you must attribute the work in
% the manner specified by the author ...':
%
% In an internet based reuse please link the reused parts to www.telekom.com and
% mention the original authors and Deutsche Telekom AG in a suitable manner. In
% a paper-like reuse please insert a short hint to www.telekom.com and to the
% original authors and Deutsche Telekom AG into your preface. For normal
% quotations please use the scientific standard to cite.
%
% [ File structure derived from 'mind your Scholar Research Framework' 
%   mycsrf (c) K. Reincke CC BY 3.0  http://mycsrf.fodina.de/ ]
%

% Chapter Abstract
% ----------------

\chapter{Open Source License Compliance: To-Do Lists}

\footnotesize
\begin{quote}\itshape
With respect to the defined open source use cases, this chapter lists what one
has to do for acting in accordance with the specific open source licenses.
\end{quote}
\normalsize{}

\section{Some general remarks on 'giving' someone a file}

This chapter has to be started with some general points which are relevant for
many of the to-do lists. So that the same points are not repeated too often, we
will start with these general remarks and refer to them throughout the chapter.

\label{DistributingFilesHint}
\begin{itemize}
  \item
  Sometimes when delivering a binary package containing open source software,
  the medium doesn’t allow the recipient to view all files contained in that
  package. For example, a lot of mobile devices don’t give the user access to
  the file system. But open source licenses often require ‘to give’ someone
  copies of text files, such as the license text, copyright notes, or specific
  notice file. The safe interpretation of ‘giving someone a text’ is that the
  receiver must be able to read it\footnote{To give someone anything they can't
  touch, feel or see is like not giving him the object ;-)}. Thus, on
  systems which offer a file browser and a suitable reader, it is sufficient, to
  put these file onto the files system. On the other systems, you \emph{must}
  present the content of the files  through the UI of your application---for
  example in a specific copyright screen\footnote{Additionally, in the open
  source community, it is a good tradition, to present these reference data
  voluntarily.}. The \oslic{} does not want to refine the taxonomies down to the
  level of operating systems, so it is up to the user to keep this in mind when
  reading the to-do lists.
  
  \item Sometimes a product which uses and distributes open source software
  tries to fulfill the requirement 'to give the recipients the license etc.' by
  presenting links to general versions of these licensing files hosted somewhere
  on the internet. But be aware: Although it is a good tradition---especially
  if you link to the homepages of the projects for being totally transparent---
  it is not sufficient to offer only the links. If you are required by the open
  source licenses to handover something to your users, \emph{you} must do it. It
  is not safe to delegate the task to anyone hoping that they will offer the
  files all the time your product is being distributed\footnote{Moreover, the
  advantage of doing the job oneself is that one has not to struggle with
  uncommunicated implicit modifications of the link targets.}. Even if it would
  be safe to assume that the link will remain valid forever, the point is: you
  have to fulfill the license, no one else.
\end{itemize}

\label{OSUCToDoLists}

% ==============================================================================
% Some commands common to all to-do lists

% Common footnote, used in many task lists
\newcommand{\passingFilesCorrectly}{%
  \footnote{For implementing the handover of files correctly $\rightarrow$
    OSLiC, p. \pageref{DistributingFilesHint}}}

% A LSUC that covers many OSUCs
% #1 -> List of covered OSUCS, e. g., ``OSUC-01, OSUC-07S, and OSUC-10S''
% #2 -> number of first OSUC covered, e. g., '07B'
% #3 -> number of last OSUC covered
\newcommand{\coversOsucs}[3]{\lsuccovers{#1}%
  \footnote{For details $\rightarrow$ \oslic, pp.\ \osucpageref{#2} -- \osucpageref{#3}}}

% A LSUC that maps to exactly one OSUC
% # -> number of the OSUC, e. g., '07B'
\newcommand{\mapsToOsuc}[1]{\lsuccovers{OSUC-#1}%
  \footnote{For details $\rightarrow$ \oslic, pp.\ \osucpageref{#1}}}


% Local Variables:
% mode: latex
% fill-column: 80
% End:

% Telekom osCompendium 'for being included' snippet template
%
% (c) Karsten Reincke, Deutsche Telekom AG, and Ronald Dauster, GIDO GmbH
%     Darmstadt 2014
%
% This LaTeX-File is licensed under the Creative Commons Attribution-ShareAlike
% 3.0 Germany License (http://creativecommons.org/licenses/by-sa/3.0/de/): Feel
% free 'to share (to copy, distribute and transmit)' or 'to remix (to adapt)'
% it, if you '... distribute the resulting work under the same or similar
% license to this one' and if you respect how 'you must attribute the work in
% the manner specified by the author ...':
%
% In an Internet based reuse please link the reused parts to www.telekom.com and
% mention the original authors and Deutsche Telekom AG in a suitable manner. In
% a paper-like reuse please insert a short hint to www.telekom.com and to the
% original authors and Deutsche Telekom AG into your preface. For normal
% quotations please use the scientific standard to cite.
%
% [ Framework derived from 'mind your Scholar Research Framework' 
%   mycsrf (c) K. Reincke 2012 CC BY 3.0  http://mycsrf.fodina.de/ ]
%

%% =============================================================================
%% All commands take (at least) one parameter: the name of the license, for
%% example, 'GPL-2.0'.  This parameter is always present, even if it isn't used
%% and it is always the first parameter.
%%
%% Commands with the prefix gtb (GPL Text Block) are common parametrized text
%% blocks for GPL, LGPL, and AGPL 

% ------------------------------------------------------------------------------
% Common description of license specific use cases

\newcommand{\gtbUseCaseOne}[1]{\lsucmeans{that you received #1 licensed
    software, that you will use it only for yourself, and that you do not hand
    over to any third party in any sense.}}

\newcommand{\agtbUseCaseOne}[1]{\lsucmeans{that you received #1 licensed
    software, that you will use it only for yourself, and that you do not hand
    over to any third party in any sense. Additionally you warrants that no
    other than you interacts with the executed software remotely through a
    computer network.}}

\newcommand{\gtbUseCaseTwo}[1]{\lsucmeans{that you received #1 licensed software 
    that you are now going to distribute to third parties as an independent unit
    and in the form of unmodified source code files or as an unmodified source
    code package. In this case it makes no difference if you distribute a
    program, an application, a server, a snippet, a module, a library, or a
    plugin.}}

\newcommand{\gtbUseCaseThree}[1]{\lsucmeans{that you received #1 licensed
    software, which you are now going to distribute to third parties as an
    independent unit and in the form of unmodified binary files or as an
    unmodified binary package. In this case it does not matter if you distribute
    a program, an application, a server, a snippet, a module, a library, or a
    plugin.}}

\newcommand{\gtbUseCaseFour}[2]{\lsucmeans{that you received #2 #1 licensed
    snippet, module or library that you are now going to distribute to third
    parties as an embedded component of a larger unit and in the form of
    unmodified source code files or as an unmodified source code package.}}

\newcommand{\gtbUseCaseFive}[2]{\lsucmeans{that you received #2 #1 licensed
    snippet, module or library that you are now going to distribute to third
    parties as an embedded component of a larger unit and in the form of
    unmodified binary files or as unmodified binary package.}}

\newcommand{\gtbUseCaseSix}[2]{\lsucmeans{that you received #2 #1 licensed
    program, application, or server (proapse), that you modified it, and that
    you are now going to distribute this modified version to third parties in
    the form of source code files or as a source code package.}}

\newcommand{\gtbUseCaseSeven}[2]{\lsucmeans{that you received #2 #1 licensed
    program, application, or server (proapse), that you modified it, and that
    you are now going todistribute this modified version to third parties in the
    form of binary files or as a binary package.}}

\newcommand{\gtbUseCaseEight}[2]{\lsucmeans{that you received #2 #1 licensed
code snippet, module, library, or plugin (snimoli), that you modified it, and
    that you are now going to distribute this modified version to third parties
    in the form of source code files or as a source code package, but without
    embedding it into another larger software unit.}}

\newcommand{\gtbUseCaseNine}[2]{\lsucmeans{that you received #2 #1 licensed
code snippet, module, library, or plugin (snimoli), that you modified it, and
    that you are now going to distribute this modified version to third parties
    in the form of binary files or as a binary package but without embedding it
    into another larger software unit.} }

\newcommand{\gtbUseCaseA}[2]{\lsucmeans{that you received #2 #1 licensed code
    snippet, module, library, or plugin (snimoli), that you modified it, and
    that you are now going to distribute this modified version to third parties
    in the form of source code files or as a source code package together with
    another larger software unit which contains this code snippet, module,
    library, or plugin as an embedded component.}}

\newcommand{\gtbUseCaseB}[2]{\lsucmeans{that you received #2 #1 licensed code
    snippet, module, library, or plugin (snimoli), that you modified it, and
    that you are now going to distribute this modified version to third
    partiesin the form of binary files or as a binary package together with
    another larger software unit which contains this code snippet, module,
    library, or plugin as an embedded component.}}


\newcommand{\agtbUseCaseC}[1]{\lsucmeans{that you received an #1 licensed
    program, an application, or server, that you modified it, and that you let
    this program, application, or server be executed by a computer in a way,
    that other people than you can interact with the executed software remotely
    through a computer network.}}
    
\newcommand{\agtbUseCaseD}[1]{\lsucmeans{that you received an #1 licensed
    library, snippet, or module, that you modified it or that you did not modified
    it, that you embed this modified or unm odified library, snippet, or module into
    an own overarching program, an application, or server, and that you finally
    let this own program, application, or server be executed by a computer in a
    way, that other people than you can interact with the executed software
    remotely through a computer network.}}

% ------------------------------------------------------------------------------

\newcommand{\gtbCoversOne}[1]{\coversOsucs{OSUC-01, OSUC-03L, OSUC-03N,
OSUC-06L, OSUC-06N, OSUC-09L and OSUC-09N}{01}{09N}}
\newcommand{\agtbCoversOne}[1]{\coversOsucs{OSUC-01, OSUC-03L,
OSUC-06L, and OSUC-09L}{01}{09L}}

\newcommand{\gtbCoversTwo}[1]{\coversOsucs{OSUC-02S, OSUC-05S}{02S}{05S}}
\newcommand{\gtbCoversThree}[1]{\coversOsucs{OSUC-02B, OSUC-05B}{02B}{05B}}
\newcommand{\gtbCoversFour}[1]{\mapsToOsuc{07S}}
\newcommand{\gtbCoversFive}[1]{\mapsToOsuc{07B}}
\newcommand{\gtbCoversSix}[1]{\mapsToOsuc{04S}}
\newcommand{\gtbCoversSeven}[1]{\mapsToOsuc{04B}}
\newcommand{\gtbCoversEight}[1]{\mapsToOsuc{08S}}
\newcommand{\gtbCoversNine}[1]{\mapsToOsuc{08B}}
\newcommand{\gtbCoversA}[1]{\mapsToOsuc{10S}}
\newcommand{\gtbCoversB}[1]{\mapsToOsuc{10B}}
\newcommand{\agtbCoversC}[1]{\mapsToOsuc{03N}}
\newcommand{\agtbCoversD}[1]{\coversOsucs{OSUC-06N,OSUC-09N}{06N}{09N}}

% ------------------------------------------------------------------------------
% Keep license elements 
% #1 -> the license name

\newcommand{\gtbKeepLicenseElements}[1]{Ensure that the licensing elements
  (especially all notices that refer to the #1 and to the absence of any
  warranty) are retained in your package in the form in which you have received
  them.} 

% ------------------------------------------------------------------------------
% Give a copy of the license to the recipient of the software
% #1 -> the license name

\newcommand{\gtbGiveLicense}[1]{Give the recipient a copy of the #1 license.
  If it is not already part of the software package, add it.}

% ------------------------------------------------------------------------------
% Add license elements and acknowledgement to the documemtation
% #1 -> the license name

\newcommand{\gtbAddToDocumentation}[1]{Let the documentation of your
  distribution and/or your additional material also reproduce the content of the
  existing copyright notices, a hint to the software name, a link to its
  homepage, the respective disclaimer of warranty, and a link to the #1.}

% ------------------------------------------------------------------------------
% Keep all copyright notices intact
% #1 -> the license name

\newcommand{\gtbKeepCopyrightNotices}[1]{Retain all existing copyright notices.}

% ------------------------------------------------------------------------------
% Publish the source code
% #1 -> the license name

\newcommand{\gtbSourceRepository}[1]{Push the source code package into a
  repository under your control and make it downloadable via the Internet.
  Ensure, that this repository is online for at least 3 years after you ceased
  distributing the software package.}

% program or independent library, unmodified
\newcommand{\gtbMakeUnmodifiedSourceAvailable}[1]{Make the source code of the
  distributed software publicly available (even though you did not modify it):
  \gtbSourceRepository{#1}}

% program or independent library, modified
\newcommand{\gtbMakeModifiedSourceAvailable}[1]{Make the source code of the
  distributed software publicly available: \gtbSourceRepository{#1}} 

\newcommand{\agtbMakeModifiedSourceAvailable}[1]{Make the source code of the
  executed modified program publicly available: \gtbSourceRepository{#1}}

% embedded library, modified or unmodified, GPL and AGPL
\newcommand{\gtbMakeAllSourcesAvailable}[1]{Make the \emph{complete} source code
  of the program embedding the library publicly available (and, therefore, also
  the source code of the library itself): \gtbSourceRepository{#1}}
  
\newcommand{\agtbMakeAllSourcesAvailable}[1]{Make the \emph{complete} source
  code of the excuted program embedding the (modified) library publicly
  available (and, therefore, also the source code of the (modified) library
  itself):
  \gtbSourceRepository{#1}}

% embedded library, modified or unmodified, LGPL 
\newcommand{\gtbMakeEmbeddedSourcesAvailable}[1]{Make the source code of the 
  embedded library publicly available: \gtbSourceRepository{#1}}

% ------------------------------------------------------------------------------
% Explain where to find the sources
% #1 -> the license name

\newcommand{\gtbDescribeHowToGetSource}[1]{Insert an easy to find description
  into the distribution package that explains how and where the code can be
  retrieved.}

% ------------------------------------------------------------------------------
% Create and update the modification text file
% #1 -> the license name

\newcommand{\gtbCreateChangelog}[1]{Create a \emph{modification text file,} if
  such a file does not yet exist. \emph{Add} a description of your modifications
  on a functional level to the \emph{modification text file.}}

% ------------------------------------------------------------------------------
% Mark all modifications in the source files themselves
% #1 -> the license name

\newcommand{\gtbauxMarkChanges}[1]{Mark all modifications of the source code #1
  thoroughly within the source code and include the date of the modification.}

\newcommand{\gtbMarkEmbeddedModifications}[1]{%
  \gtbauxMarkChanges{of the embedded library (snimoli)}}

\newcommand{\gtbMarkLibraryModifications}[1]{%
  \gtbauxMarkChanges{of the library (snimoli)}}

\newcommand{\gtbMarkProgramModifications}[1]{%
  \gtbauxMarkChanges{the program (proapse)}}

% ------------------------------------------------------------------------------
% Ensure the copyright notice and the disclaimer (V2.x only) are present
% #1 -> the license name
% #2 -> type of distribution (binary or source code)

% GPL-3.0/LGPL-3.0/AGPL-3.0
\newcommand{\gtbVThreeCopyrightNotice}[2]{Ensure that the
  distributed #2 package contains a conspicuous, easy to find copyright notice.
  If this element is missing, add a new file containing the main copyright
  notice.} 

% GPL-2.0/LGPL-2.1
\newcommand{\gtbVTwoCopyrightNotice}[2]{Ensure that the distributed
  #2 package contains a conspicuous, easy to find copyright notice and
  disclaimer of warranty. If these elements are missing, add a new file
  containing the main copyright notice and the disclaimer of warranty in the
  form which is textually defined by the #1 license itself. (Yes, repeat
  the disclaimer although it is also part of the license itself and although you
  are required to hand the license itself over to the receiver.)}

% ------------------------------------------------------------------------------
% Make sure licensing statements apply to your modifications
% #1 -> the license name

\newcommand{\gtbauxArrangeChanges}[2]{Arrange your modifications of the #2 in a
way that they are covered by existing #1 licensing statements. If you add new
  source code files to the #2, insert a header containing your copyright line
  and a licensing statement in the form recommended by the #1.}

\newcommand{\gtbArrangeProgramChanges}[1]{%
  \gtbauxArrangeChanges{#1}{program}}

\newcommand{\gtbArrangeLibraryChanges}[1]{%
  \gtbauxArrangeChanges{#1}{library}}

\newcommand{\gtbArrangeEmbeddedChanges}[1]{%
  \gtbauxArrangeChanges{#1}{embedded library}}

\newcommand{\gtbHowToApplyTheseTerms}[1]{%
  \footnote{For details see section `How to Apply These Terms to Your New
    Programs' in the #1 license.}} 

% ------------------------------------------------------------------------------
% Forbid patent litigation
% #1 -> license name

\newcommand{\gtbNoPatentLitigation}[1]{%
  to institute a patent litigation against anyone alleging that the software
  constitutes patent infringement.} 

% ------------------------------------------------------------------------------
% Copyright dialog
% #1 -> license name

\newcommand{\gtbauxCopyrightDialogContent}[1]{Let it reproduce the content of
  the existing copyright notices, the software name, a link to its homepage, the 
  respective disclaimer of warranty, and a link to the #1.}

\newcommand{\gtbAddToCopyrightDialogWeakCopyleft}[1]{Let the copyright dialog of 
  the on-top development clearly say that it uses the #1 licensed library. 
  \gtbauxCopyrightDialogContent{#1}}

\newcommand{\gtbAddToCopyrightDialogStrongCopyleft}[1]{Let the copyright dialog
  of the on-top development clearly say that it uses the #1 licensed library and 
  that it is itself licensed under the #1, too. 
  \gtbauxCopyrightDialogContent{#1}}

\newcommand{\gtbAddToCopyrightDialogApp}[1]{Let the copyright dialog of the
  program clearly say that it is a #1 licensed program. 
  \gtbauxCopyrightDialogContent{#1}\ 
  If these conditions are not already met, add the missing elements.}


%% =============================================================================
%% Use Case Finder

% ------------------------------------------------------------------------------
% Common license specific use case finder for GPL, LGPL
% (does not apply to AGPL because there is no version 2.x of the AGPL)

\newcommand{\gplUseCaseFinder}[3]{
\tikzstyle{nodv} = [font=\scriptsize, ellipse, draw, fill=gray!10, 
    text width=2cm, text centered, minimum height=2em]

\tikzstyle{nods} = [font=\tiny, rectangle, draw, fill=gray!20, 
    text width=1cm, text centered, rounded corners, minimum height=3em]

\tikzstyle{nodb} = [font=\tiny, rectangle, draw, fill=gray!20, 
    text width=1.5cm, text centered, rounded corners, minimum height=3em]
    
\tikzstyle{leaf} = [font=\tiny, rectangle, draw, fill=gray!30, 
    text width=1.2cm, text centered, minimum height=6em]

\tikzstyle{slimleaf} = [font=\tiny, rectangle, draw, fill=gray!30, 
    text width=1cm, text centered, minimum height=6em]


\tikzstyle{edge} = [draw, -latex']

\begin{tikzpicture}[]

  \node[nodv] (l801) at (4,11.8)   {#1};

  \node[nodv] (l701) at (0,10.2)   {#2};
  \node[nodv] (l702) at (7.5,10.2) {#3};
  

  \node[nodb] (l601) at (0,8.6) {\textit{recipient:} \\ \textbf{4yourself}};
  \node[nodb] (l602) at (7.5,8.6) {\textit{recipient:} \\ \textbf{2others}};
  
  \node[nodb] (l501) at (4,7) {\textit{state:} \\ \textbf{unmodified}};
  \node[nodb] (l502) at (11,7) {\textit{state:} \\ \textbf{modified}};
  
  \node[nodb] (l401) at (2.25,5.4) {\textit{type:} \\ \textbf{proapse or snimoli}};
  \node[nodb] (l402) at (5.4,5.4) {\textit{type:} \\ \textbf{snimoli}};
  \node[nodb] (l403) at (8.4,5.4) {\textit{type:} \\ \textbf{proapse}};
  \node[nodb] (l404) at (12.8,5.4) {\textit{type:} \\ \textbf{snimoli}};
  
  
  \node[nodb] (l301) at (2.25,3.8) {\textit{context:} \\ \textbf{independent}};
  \node[nodb] (l302) at (5.4,3.8) {\textit{context:} \\ \textbf{embedded}};
  \node[nodb] (l303) at (8.4,3.8) {\textit{context:} \\ \textbf{independent}};
  \node[nodb] (l304) at (11.3,3.8) {\textit{context:} \\ \textbf{independent}};
  \node[nodb] (l305) at (14.3,3.8) {\textit{context:} \\ \textbf{embedded}};
  
  \node[nods] (l201) at (1.45,2.2) {\textit{form:} \textbf{source}};
  \node[nods] (l202) at (3.0,2.2) {\textit{form:} \textbf{binary}};
  \node[nods] (l203) at (4.6,2.2) {\textit{form:} \textbf{source}};
  \node[nods] (l204) at (6.2,2.2) {\textit{form:} \textbf{binary}};
  \node[nods] (l205) at (7.7,2.2) {\textit{form:} \textbf{source}};
  \node[nods] (l206) at (9.1,2.2) {\textit{form:} \textbf{binary}};
  \node[nods] (l207) at (10.5,2.2) {\textit{form:} \textbf{source}};
  \node[nods] (l208) at (11.9,2.2) {\textit{form:} \textbf{binary}};
  \node[nods] (l209) at (13.4,2.2) {\textit{form:} \textbf{source}};
  \node[nods] (l210) at (15.0,2.2) {\textit{form:} \textbf{binary}};
  
  \node[slimleaf] (l101) at (0,0) {
    \textbf{#1-*-C1} 
    \textit{using software only for yourself}};
  
  \node[leaf] (l102) at (1.45,0) { 
    \textbf{#1-*-C2} 
    \textit{distributing unmodified software as independent sources}};
  
  \node[leaf] (l103) at (3.0,0) { 
    \textbf{#1-*-C3}  
    \textit{distributing unmodified software as independent binaries}};
  
  \node[leaf] (l104) at (4.6,0) { 
    \textbf{#1-*-C4} 
    \textit{distributing unmodified library as embedded sources}};
  
  \node[leaf] (l105) at (6.2,0) { 
    \textbf{#1-*-C5}  
    \textit{distributing unmodified library as embedded binaries}};
  
  \node[slimleaf] (l106) at (7.7,0) { 
    \textbf{#1-*-C6}  
    \textit{distributing modified program as sources}};
  
  \node[slimleaf] (l107) at (9.1,0) { 
    \textbf{#1-*-C7}  
    \textit{distributing modified program as binaries}};
  
  \node[slimleaf] (l108) at (10.5,0) { 
    \textbf{#1-*-C8}  
    \textit{distributing modified library as independent sources}};
  
  \node[slimleaf] (l109) at (11.9,0) { 
    \textbf{#1-*-C9}
    \textit{distributing modified library as independent binaries}};
  
  \node[leaf] (l110) at (13.4,0) { 
    \textbf{#1-*-CA}  
    \textit{distributing modified library as embedded sources}};
  
  \node[leaf] (l111) at (15,0) { 
    \textbf{#1-*-CB}  
    \textit{ distributing modified library as embedded binaries}};
  
  \path [edge] (l801) -- (l701);
  \path [edge] (l801) -- (l702);
  \path [edge] (l701) -- (l601);
  \path [edge] (l701) -- (l602);
  \path [edge] (l702) -- (l601);
  \path [edge] (l702) -- (l602);
  
  \path [edge] (l602) -- (l501);
  \path [edge] (l602) -- (l502);
  
  \path [edge] (l501) -- (l401);
  \path [edge] (l501) -- (l402);
  \path [edge] (l502) -- (l403);
  \path [edge] (l502) -- (l404);
  
  \path [edge] (l401) -- (l301);
  \path [edge] (l402) -- (l302);
  \path [edge] (l403) -- (l303);
  \path [edge] (l404) -- (l304);
  \path [edge] (l404) -- (l305);
  
  \path [edge] (l301) -- (l201);
  \path [edge] (l301) -- (l202);
  \path [edge] (l302) -- (l203);
  \path [edge] (l302) -- (l204);
  \path [edge] (l303) -- (l205);
  \path [edge] (l303) -- (l206);
  \path [edge] (l304) -- (l207);
  \path [edge] (l304) -- (l208);
  \path [edge] (l305) -- (l209);
  \path [edge] (l305) -- (l210);
  
  \path [edge] (l601) -- (l101);
  \path [edge] (l201) -- (l102);
  \path [edge] (l202) -- (l103);
  \path [edge] (l203) -- (l104);
  \path [edge] (l204) -- (l105);
  \path [edge] (l205) -- (l106);
  \path [edge] (l206) -- (l107);
  \path [edge] (l207) -- (l108);
  \path [edge] (l208) -- (l109);
  \path [edge] (l209) -- (l110);
  \path [edge] (l210) -- (l111);
  
\end{tikzpicture}
} % end of gplUseCaseFinder

\newcommand{\agplUseCaseFinder}[2]{
\tikzstyle{nodv} = [font=\scriptsize, ellipse, draw, fill=gray!10, 
    text width=2cm, text centered, minimum height=2em]

\tikzstyle{nods} = [font=\tiny, rectangle, draw, fill=gray!20, 
    text width=0.8cm, text centered, rounded corners, minimum height=3em]

\tikzstyle{nodz} = [font=\tiny, rectangle, draw, fill=gray!20, 
    text width=0.6cm, text centered, rounded corners, minimum height=3em]

\tikzstyle{nodb} = [font=\tiny, rectangle, draw, fill=gray!20, 
    text width=1cm, text centered, rounded corners, minimum height=3em]
    
\tikzstyle{leaf} = [font=\tiny, rectangle, draw, fill=gray!30, 
    text width=1cm, text centered, minimum height=7em]

\tikzstyle{slimleaf} = [font=\tiny, rectangle, draw, fill=gray!30, 
    text width=0.84cm, text centered, minimum height=7em]


\tikzstyle{edge} = [draw, -latex']

\begin{tikzpicture}[]

  \node[nodv] (l801) at (4,12)   {#1 #2};

  \node[nodb] (l601) at (0,11.6) {\textit{recipient:} \\ \textbf{4your-} \\\textbf{self}};
  \node[nodb] (l602) at (8.8,11.6) {\textit{recipient:} \\ \textbf{2others}};

  \node[nodb] (l551) at (0,9) {\textit{ioAccess:} \\ \textbf{only-} \\ \textbf{Local}}; 
  \node[nodb] (l552) at (2,9) {\textit{ioAccess:} \\ \textbf{via-} \\ \textbf{Internet}};   
  
   
  \node[nods] (l503) at (1.2,7) {\textit{state:} \\ \textbf{unmo-} \\ \textbf{dified}}; 
  \node[nods] (l504) at (3.2,7) {\textit{state:} \\ \textbf{mo-} \\ \textbf{dified}}; 
  \node[nodb] (l501) at (6.6,7) {\textit{state:} \\ \textbf{unmo-} \\
  \textbf{dified}}; \node[nodb] (l502) at (11.5,7) {\textit{state:} \\ \textbf{mo-} \\ \textbf{dified}}; 
  
  \node[nodz] (l407) at (0.6,5.4) {\textit{type:} \\ \textbf{pro-apse}};
  \node[nodz] (l408) at (1.6,5.4) {\textit{type:} \\ \textbf{sni-moli}};
  \node[nodz] (l405) at (2.8,5.4) {\textit{type:} \\ \textbf{pro-apse}};
  \node[nodz] (l406) at (3.8,5.4) {\textit{type:} \\ \textbf{sni-moli}};
  
  \node[nodb] (l401) at (5.8,5.4) {\textit{type:} \\ \textbf{proapse or snimoli}};
  \node[nodb] (l402) at (7.6,5.4) {\textit{type:} \\ \textbf{snimoli}};
  \node[nodb] (l403) at (9.6,5.4) {\textit{type:} \\ \textbf{proapse}};
  \node[nodb] (l404) at (13.3,5.4) {\textit{type:} \\ \textbf{snimoli}};
  
  \node[nodb] (l306) at (1.3,3.8) {\textit{context:} \\ \textbf{inde-} \\ \textbf{pendent}};
  \node[nodb] (l307) at (2.7,3.8) {\textit{context:} \\ \textbf{em-} \\ \textbf{bedded}};
    
  \node[nodb] (l301) at (4.6,3.8) {\textit{context:} \\ \textbf{inde-} \\ \textbf{pendent}};
  \node[nodb] (l302) at (7.2,3.8) {\textit{context:} \\ \textbf{em-} \\ \textbf{bedded}};
  \node[nodb] (l303) at (9.6,3.8) {\textit{context:} \\ \textbf{inde-} \\ \textbf{pendent}};
  \node[nodb] (l304) at (11.9,3.8) {\textit{context:} \\ \textbf{inde-} \\ \textbf{pendent}};
  \node[nodb] (l305) at (14.3,3.8) {\textit{context:} \\ \textbf{em-} \\ \textbf{bedded}};
  
  \node[nods] (l201) at (4.0,2.2) {\textit{form:} \textbf{source}};
  \node[nods] (l202) at (5.3,2.2) {\textit{form:} \textbf{binary}};
  \node[nods] (l203) at (6.6,2.2) {\textit{form:} \textbf{source}};
  \node[nods] (l204) at (7.8,2.2) {\textit{form:} \textbf{binary}};
  \node[nods] (l205) at (9.0,2.2) {\textit{form:} \textbf{source}};
  \node[nods] (l206) at (10.2,2.2) {\textit{form:} \textbf{binary}};
  \node[nods] (l207) at (11.4,2.2) {\textit{form:} \textbf{source}};
  \node[nods] (l208) at (12.6,2.2) {\textit{form:} \textbf{binary}};
  \node[nods] (l209) at (13.8,2.2) {\textit{form:} \textbf{source}};
  \node[nods] (l210) at (15.0,2.2) {\textit{form:} \textbf{binary}};
  
  \node[slimleaf] (l101) at (0,0) {
    \textbf{#1-C1} 
    \textit{using apps \& libs only for yourself (+ sub conditions)}};
    
   \node[leaf] (l112) at (1.3,0) { 
    \textbf{#1-CC}
    \textit{execu-ting a modified #1 program with net-io-Access}};   
    
  \node[leaf] (l113) at (2.7,0) { 
    \textbf{#1-CD}
    \textit{execu-ting any app with net-io-Access using a (modified) library}}; 
 
  \node[slimleaf] (l102) at (4.0,0) { 
    \textbf{#1-C2} 
    \textit{distri-buting unmo-dified software as independent sources}};
  
  \node[leaf] (l103) at (5.3,0) { 
    \textbf{#1-C3}  
    \textit{distribu-ting unmo-dified software as independent binaries}};
  
  \node[slimleaf] (l104) at (6.6,0) { 
    \textbf{#1-C4} 
    \textit{distribu-ting an un-modified library as embedded sources}};
  
  \node[slimleaf] (l105) at (7.8,0) { 
    \textbf{#1-C5}  
    \textit{distribu-ting an un-modified library as embedded binaries}};
  
  \node[slimleaf] (l106) at (9,0) { 
    \textbf{#1-C6}  
    \textit{distri-buting a modi-fied program as sources}};
  
  \node[slimleaf] (l107) at (10.2,0) { 
    \textbf{#1-C7}  
    \textit{distri-buting a modi-fied program as binaries}};
  
  \node[slimleaf] (l108) at (11.4,0) { 
    \textbf{#1-C8}  
    \textit{distri-buting a modi-fied library as independent sources}};
  
  \node[slimleaf] (l109) at (12.6,0) { 
    \textbf{#1-C9}
    \textit{distri-buting a modi-fied library as independent binaries}};
  
  \node[slimleaf] (l110) at (13.8,0) { 
    \textbf{#1-CA}  
    \textit{distri-buting a modi-fied library as embedded sources}};
  
  \node[slimleaf] (l111) at (15,0) { 
    \textbf{#1-CB}  
    \textit{distri-buting a modi-fied library as embedded binaries}};
  
  \path [edge] (l801) -- (l601);
  \path [edge] (l801) -- (l602);
  
  \path [edge] (l601) -- (l551);
  \path [edge] (l601) -- (l552);

  \path [edge] (l602) -- (l501);
  \path [edge] (l602) -- (l502);
  \path [edge] (l552) -- (l503);
  \path [edge] (l552) -- (l504); 

  
  \path [edge] (l501) -- (l401);
  \path [edge] (l501) -- (l402);
  \path [edge] (l502) -- (l403);
  \path [edge] (l502) -- (l404);
  
  \path [edge] (l503) -- (l407);  
  \path [edge] (l503) -- (l408); 
  \path [edge] (l504) -- (l405);  
  \path [edge] (l504) -- (l406);  
   
    
  \path [edge] (l401) -- (l301);
  \path [edge] (l402) -- (l302);
  \path [edge] (l403) -- (l303);
  \path [edge] (l404) -- (l304);
  \path [edge] (l404) -- (l305);

  
  \path [edge] (l405) -- (l306);
  \path [edge] (l406) -- (l307);
  \path [edge] (l407) -- (l101);   
  \path [edge] (l408) -- (l307);  
  
  \path [edge] (l301) -- (l201);
  \path [edge] (l301) -- (l202);
  \path [edge] (l302) -- (l203);
  \path [edge] (l302) -- (l204);
  \path [edge] (l303) -- (l205);
  \path [edge] (l303) -- (l206);
  \path [edge] (l304) -- (l207);
  \path [edge] (l304) -- (l208);
  \path [edge] (l305) -- (l209);
  \path [edge] (l305) -- (l210);
  
  \path [edge] (l551) -- (l101);

  \path [edge] (l306) -- (l112);
  \path [edge] (l307) -- (l113);
        
  \path [edge] (l201) -- (l102);
  \path [edge] (l202) -- (l103);
  \path [edge] (l203) -- (l104);
  \path [edge] (l204) -- (l105);
  \path [edge] (l205) -- (l106);
  \path [edge] (l206) -- (l107);
  \path [edge] (l207) -- (l108);
  \path [edge] (l208) -- (l109);
  \path [edge] (l209) -- (l110);
  \path [edge] (l210) -- (l111);
  
\end{tikzpicture}
} % end of agplUseCaseFinder


{% Telekom osCompendium 'for being included' snippet template
%
% (c) Karsten Reincke, Deutsche Telekom AG, Darmstadt 2011
%
% This LaTeX-File is licensed under the Creative Commons Attribution-ShareAlike
% 3.0 Germany License (http://creativecommons.org/licenses/by-sa/3.0/de/): Feel
% free 'to share (to copy, distribute and transmit)' or 'to remix (to adapt)'
% it, if you '... distribute the resulting work under the same or similar
% license to this one' and if you respect how 'you must attribute the work in
% the manner specified by the author ...':
%
% In an internet based reuse please link the reused parts to www.telekom.com and
% mention the original authors and Deutsche Telekom AG in a suitable manner. In
% a paper-like reuse please insert a short hint to www.telekom.com and to the
% original authors and Deutsche Telekom AG into your preface. For normal
% quotations please use the scientific standard to cite.
%
% [ Framework derived from 'mind your Scholar Research Framework' 
%   mycsrf (c) K. Reincke 2012 CC BY 3.0  http://mycsrf.fodina.de/ ]
%


%% use all entries of the bibliography
%\nocite{*}

\section{AGPL licensed software}

\agplUseCaseFinder{AGPL}{3.0}

%% ============================================================================= 
%% Common Building Blocks

\newcommand{\useCaseOne}{%
  \agtbUseCaseOne{AGPL-\ver}
  \agtbCoversOne{AGPL-\ver}}

\newcommand{\useCaseTwo}{%
  \gtbUseCaseTwo{AGPL-\ver}
  \gtbCoversTwo{AGPL-\ver}}

\newcommand{\useCaseThree}{%
  \gtbUseCaseThree{AGPL-\ver}
  \gtbCoversThree{AGPL-\ver}}

\newcommand{\useCaseFour}{%
  \gtbUseCaseFour{AGPL-\ver}{an}
  \gtbCoversFour{AGPL-\ver}}

\newcommand{\useCaseFive}{%
  \gtbUseCaseFive{AGPL-\ver}{an}
  \gtbCoversFive{AGPL-\ver}}

\newcommand{\useCaseSix}{%
  \gtbUseCaseSix{AGPL-\ver}{an}
  \gtbCoversSix{AGPL-\ver}}

\newcommand{\useCaseSeven}{%
  \gtbUseCaseSeven{AGPL-\ver}{an}
  \gtbCoversSeven{GPL-\ver}}

\newcommand{\useCaseEight}{%
  \gtbUseCaseEight{AGPL-\ver}{an}
  \gtbCoversEight{AGPL-\ver}}

\newcommand{\useCaseNine}{%
  \gtbUseCaseNine{AGPL-\ver}{an}
  \gtbCoversNine{AGPL-\ver}}

\newcommand{\useCaseA}{%
  \gtbUseCaseA{AGPL-\ver}{an}
  \gtbCoversA{AGPL-\ver}}

\newcommand{\useCaseB}{%
  \gtbUseCaseB{AGPL-\ver}{an}
  \gtbCoversB{AGPL-\ver}}

\newcommand{\useCaseC}{%
  \agtbUseCaseC{AGPL-\ver}
  \agtbCoversC{AGPL-\ver}}

\newcommand{\useCaseD}{%
  \agtbUseCaseD{AGPL-\ver}
  \agtbCoversD{AGPL-\ver}}
  
% ------------------------------------------------------------------------------
% Common Text Blocks from 0600-commomn-text-blocks.tex

\newcommand{\keepLicenseElements}{\gtbKeepLicenseElements{AGPL-\ver}}
\newcommand{\addToDocumentation}{\gtbAddToDocumentation{AGPL-\ver}}
\newcommand{\giveLicense}{\gtbGiveLicense{AGPL-\ver}}
\newcommand{\retainCopyrightNotices}{\gtbKeepCopyrightNotices{AGPL-\ver}}
\newcommand{\describeHowToGetSource}{\gtbDescribeHowToGetSource{AGPL-\ver}}
\newcommand{\createChangelog}{\gtbCreateChangelog{AGPL-\ver}}
\newcommand{\markEmbeddedModifications}{\gtbMarkEmbeddedModifications{AGPL-\ver}}
\newcommand{\markLibraryModifications}{\gtbMarkLibraryModifications{AGPL-\ver}}
\newcommand{\markProgramModifications}{\gtbMarkProgramModifications{AGPL-\ver}}
\newcommand{\gpltwoEnsureCopyrightNoticeSource}{\gtbVTwoCopyrightNotice{AGPL-2.0}{source code}}
\newcommand{\gpltwoEnsureCopyrightNoticeBinary}{\gtbVTwoCopyrightNotice{AGPL-2.0}{binary}}
\newcommand{\gplthreeEnsureCopyrightNoticeSource}{\gtbVThreeCopyrightNotice{AGPL-3.0}{source code}}
\newcommand{\gplthreeEnsureCopyrightNoticeBinary}{\gtbVThreeCopyrightNotice{AGPL-3.0}{binary}}
\newcommand{\makeUnmodifiedSourceAvailable}{\gtbMakeUnmodifiedSourceAvailable{AGPL-\ver}} 
\newcommand{\makeModifiedSourceAvailable}{\gtbMakeModifiedSourceAvailable{AGPL-\ver}}
\newcommand{\makeExecModifiedSourceAvailable}{\agtbMakeModifiedSourceAvailable{AGPL-\ver}} 
\newcommand{\makeAllSourcesAvailable}{\gtbMakeAllSourcesAvailable{AGPL-\ver}}
\newcommand{\makeExecAllSourcesAvailable}{\agtbMakeAllSourcesAvailable{AGPL-\ver}}
\newcommand{\arrangeProgramChanges}{\gtbArrangeProgramChanges{AGPL-\ver}}
\newcommand{\arrangeLibraryChanges}{\gtbArrangeLibraryChanges{AGPL-\ver}}
\newcommand{\arrangeEmbeddedChanges}{\gtbArrangeEmbeddedChanges{AGPL-\ver}}
\newcommand{\howToApplyTheseTerms}{\gtbHowToApplyTheseTerms{AGPL-\ver}}
\newcommand{\noPatentLitigation}{\gtbNoPatentLitigation{AGPL-\ver}}
\newcommand{\addToCopyrightDialogLib}{\gtbAddToCopyrightDialogStrongCopyleft{AGPL-\ver}}
\newcommand{\addToCopyrightDialogApp}{\gtbAddToCopyrightDialogApp{AGPL-\ver}}

% ------------------------------------------------------------------------------
% Make sure, licensing statements apply to enclosing program

\newcommand{\auxArrange}[1]{Arrange the #1 of the on-top development in a way
  that they are covered by the AGPL-\ver{} licensing statements.} 

\newcommand{\arrangeEnclosingBinaries}{%
  \auxArrange{the binaries of the on-top development}}

\newcommand{\arrangeEnclosingSources}{%
  \auxArrange{the sources of the on-top development}}


%% =============================================================================
%% AGPL-3.0 Use Cases

\newcommand{\ver}{3.0}

\begin{license}{AGPL3} 
\licensename{AGPL-3.0}
\licensespec{GNU Affero General Public License Version 3}
\licenseabbrev{AGPL}
%\licenseversion{3.0}

% ------------------------------------------------------------------------------
\subsection{AGPL-\ver-C1: Using the software only for yourself under additional restrictions}
\begin{lsuc}{AGPL-\ver-C1}
  \linkosuc{01}
  \linkosuc{03L} 
  \linkosuc{06L}
  \linkosuc{09L}
  \label{OSUC-01-AGPL}
  \label{OSUC-03L-AGPL}
  \label{OSUC-06L-AGPL}
  \label{OSUC-09L-AGPL}
      
  \useCaseOne

  \begin{lsucrequiresnothing}
    \lsucitem{You are allowed to execute an unmodified AGPL program without
    being obliged to do anything, as long as you do not give the program to
    third parties. And you are allowed to embed any AGPL licensed library, 
    snippet or module into your own program and to execute that program without
    being obliged to do anything, as long as no other than you can interact with
    it remotely through a computer network and as long as you do not give the
    library or your program to third parties.}
  \end{lsucrequiresnothing}

  \begin{lsucprohibits}
    \lsucitem{\noPatentLitigation}
  \end{lsucprohibits}
\end{lsuc}

% ------------------------------------------------------------------------------
\subsection{AGPL-\ver-C2: Passing the unmodified software as independent sources}
\begin{lsuc}{AGPL-\ver-C2}
  \linkosuc{02S}
  \linkosuc{05S}
  \label{OSUC-02S-AGPL}
  \label{OSUC-05S-AGPL}

  \useCaseTwo

  \begin{lsucrequires}
    \lsucmandatory{\keepLicenseElements}
    \lsucmandatory{\gplthreeEnsureCopyrightNoticeSource}
    \lsucmandatory{\giveLicense}\passingFilesCorrectly
    \lsucmandatory{\retainCopyrightNotices}
    \lsucoptional{\addToDocumentation}
  \end{lsucrequires}

  \begin{lsucprohibits}
    \lsucitem{\noPatentLitigation}
  \end{lsucprohibits}
\end{lsuc}

% ------------------------------------------------------------------------------
\subsection{AGPL-\ver-C3: Passing the unmodified software as independent binaries} 
\begin{lsuc}{AGPL-\ver-C3}
  \linkosuc{02B} 
  \linkosuc{05B}
  \label{OSUC-02B-AGPL}
  \label{OSUC-05B-AGPL}


  \useCaseThree

  \begin{lsucrequires}
    \lsucmandatory{\keepLicenseElements}
    \lsucmandatory{\gplthreeEnsureCopyrightNoticeBinary}
    \lsucmandatory{\giveLicense}\passingFilesCorrectly  
    \lsucmandatory{\makeUnmodifiedSourceAvailable}
    \lsucmandatory{\describeHowToGetSource}
    \lsucmandatory{\retainCopyrightNotices}
    \lsucsourcedist{AGPL-\ver-C2}
    \lsucoptional{\addToDocumentation}
  \end{lsucrequires}

  \begin{lsucprohibits}
    \lsucitem{\noPatentLitigation}
  \end{lsucprohibits}
\end{lsuc}

% ------------------------------------------------------------------------------
\subsection{AGPL-\ver-C4: Passing the unmodified library as embedded sources}
\begin{lsuc}{AGPL-\ver-C4}
  \linkosuc{07S} 
  \label{OSUC-07S-AGPL}

  \useCaseFour

  \begin{lsucrequires}
    \lsucmandatory{\keepLicenseElements}
    \lsucmandatory{\gplthreeEnsureCopyrightNoticeSource}
    \lsucmandatory{\giveLicense}\passingFilesCorrectly
    \lsucmandatory{\retainCopyrightNotices}
    \lsucmandatory{\addToCopyrightDialogLib}
    \lsucmandatory{\arrangeEnclosingSources}
    \lsucoptional{\addToDocumentation}
  \end{lsucrequires}

  \begin{lsucprohibits}
    \lsucitem{\noPatentLitigation}
  \end{lsucprohibits}
\end{lsuc}

% ------------------------------------------------------------------------------
\subsection{AGPL-\ver-C5: Passing the unmodified library as embedded binaries} 
\begin{lsuc}{AGPL-\ver-C5}
  \linkosuc{07B} 
  \label{OSUC-07B-AGPL}

  \useCaseFive

  \begin{lsucrequires}
    \lsucmandatory{\keepLicenseElements}
    \lsucmandatory{\gplthreeEnsureCopyrightNoticeBinary}
    \lsucmandatory{\giveLicense}\passingFilesCorrectly
    \lsucmandatory{\makeAllSourcesAvailable}
    \lsucmandatory{\describeHowToGetSource}
    \lsucmandatory{\addToCopyrightDialogLib}
    \lsucmandatory{\arrangeEnclosingBinaries}
    \lsucmandatory{\retainCopyrightNotices}
    \lsucsourcedist{AGPL-\ver-C4}
    \lsucoptional{\addToDocumentation}
  \end{lsucrequires}

  \begin{lsucprohibits}
    \lsucitem{\noPatentLitigation}
  \end{lsucprohibits}
\end{lsuc}

% ------------------------------------------------------------------------------
\subsection{AGPL-\ver-C6: Passing a modified program as source code}
\begin{lsuc}{AGPL-\ver-C6}
  \linkosuc{04S} 
  \label{OSUC-04S-AGPL}

  \useCaseSix

  \begin{lsucrequires}
    \lsucmandatory{\keepLicenseElements}
    \lsucmandatory{\gplthreeEnsureCopyrightNoticeSource}
    \lsucmandatory{\giveLicense}\passingFilesCorrectly
    \lsucmandatory{\retainCopyrightNotices}
    \lsucmandatory{\addToCopyrightDialogApp}
    \lsucmandatory{\markProgramModifications}
    \lsucmandatory{\arrangeProgramChanges}\howToApplyTheseTerms
    \lsucoptional{\createChangelog}
    \lsucoptional{\addToDocumentation}
  \end{lsucrequires}

  \begin{lsucprohibits}
    \lsucitem{\noPatentLitigation}
  \end{lsucprohibits}
\end{lsuc}

% ------------------------------------------------------------------------------
\subsection{AGPL-\ver-C7: Passing a modified program as binary}
\begin{lsuc}{AGPL-\ver-C7}
  \linkosuc{04B}
  \label{OSUC-04B-AGPL}

  \useCaseSeven

  \begin{lsucrequires}
    \lsucmandatory{\keepLicenseElements}
    \lsucmandatory{\gplthreeEnsureCopyrightNoticeBinary}
    \lsucmandatory{\giveLicense}\passingFilesCorrectly
    \lsucmandatory{\retainCopyrightNotices}
    \lsucmandatory{\markProgramModifications}
    \lsucmandatory{\addToCopyrightDialogApp}
    \lsucmandatory{\arrangeProgramChanges}\howToApplyTheseTerms
    \lsucmandatory{\makeModifiedSourceAvailable}
    \lsucmandatory{\describeHowToGetSource}
    \lsucsourcedist{AGPL-\ver-C6}
    \lsucoptional{\createChangelog}
    \lsucoptional{\addToDocumentation}
  \end{lsucrequires}

  \begin{lsucprohibits}
    \lsucitem{\noPatentLitigation}
  \end{lsucprohibits}
\end{lsuc}

% ------------------------------------------------------------------------------
\subsection{AGPL-\ver-C8: Passing a modified library as independent source code}
\begin{lsuc}{AGPL-\ver-C8}
  \linkosuc{08S}
  \label{OSUC-08S-AGPL}

  \useCaseEight

  \begin{lsucrequires}
     \lsucmandatory{\keepLicenseElements}
    \lsucmandatory{\gplthreeEnsureCopyrightNoticeSource}
    \lsucmandatory{\giveLicense}\passingFilesCorrectly
    \lsucmandatory{\retainCopyrightNotices}
    \lsucmandatory{\markLibraryModifications}
    \lsucmandatory{\arrangeLibraryChanges}\howToApplyTheseTerms
    \lsucoptional{\createChangelog}
    \lsucoptional{\addToDocumentation}
  \end{lsucrequires}

  \begin{lsucprohibits}
    \lsucitem{\noPatentLitigation}
  \end{lsucprohibits}
\end{lsuc}

% ------------------------------------------------------------------------------
\subsection{AGPL-\ver-C9: Passing a modified library as independent binary}
\begin{lsuc}{AGPL-\ver-C9}
  \linkosuc{08B}
  \label{OSUC-08B-AGPL}

  \useCaseNine

  \begin{lsucrequires}
    \lsucmandatory{\keepLicenseElements}
    \lsucmandatory{\gplthreeEnsureCopyrightNoticeSource}  
    \lsucmandatory{\giveLicense}\passingFilesCorrectly
    \lsucmandatory{\retainCopyrightNotices}
    \lsucmandatory{\makeModifiedSourceAvailable}
    \lsucmandatory{\describeHowToGetSource}
    \lsucsourcedist{AGPL-\ver-C8}
    \lsucmandatory{\markLibraryModifications}
    \lsucmandatory{\arrangeLibraryChanges}\howToApplyTheseTerms
    \lsucoptional{\createChangelog}
    \lsucoptional{\addToDocumentation}
  \end{lsucrequires}

  \begin{lsucprohibits}
    \lsucitem{\noPatentLitigation}
  \end{lsucprohibits}
\end{lsuc}

% ------------------------------------------------------------------------------
\subsection{AGPL-\ver-CA: Passing a modified library as embedded source code}
\begin{lsuc}{AGPL-\ver-CA}
  \linkosuc{10S}
  \label{OSUC-10S-AGPL}

  \useCaseA

  \begin{lsucrequires}
    \lsucmandatory{\keepLicenseElements}
    \lsucmandatory{\gplthreeEnsureCopyrightNoticeSource}
    \lsucmandatory{\giveLicense}\passingFilesCorrectly
    \lsucmandatory{\retainCopyrightNotices}
    \lsucmandatory{\addToCopyrightDialogLib}
    \lsucmandatory{\markEmbeddedModifications}
    \lsucmandatory{\arrangeEmbeddedChanges}\howToApplyTheseTerms
    \lsucmandatory{\arrangeEnclosingSources}
    \lsucoptional{\createChangelog}
    \lsucoptional{\addToDocumentation}
  \end{lsucrequires}

  \begin{lsucprohibits}
    \lsucitem{\noPatentLitigation}
  \end{lsucprohibits}
\end{lsuc}

% ------------------------------------------------------------------------------
\subsection{AGPL-\ver-CB: Passing a modified library as embedded binary}
\begin{lsuc}{AGPL-\ver-CB}
  \linkosuc{10B}
  \label{OSUC-10B-AGPL}
  
  \useCaseB

  \begin{lsucrequires}
    \lsucmandatory{\keepLicenseElements}
    \lsucmandatory{\gplthreeEnsureCopyrightNoticeBinary}
    \lsucmandatory{\giveLicense}\passingFilesCorrectly
    \lsucmandatory{\retainCopyrightNotices}
    \lsucmandatory{\makeAllSourcesAvailable}
    \lsucmandatory{\describeHowToGetSource}
    \lsucsourcedist{AGPL-\ver-CA}
    \lsucmandatory{\addToCopyrightDialogLib}
    \lsucmandatory{\markEmbeddedModifications}
    \lsucmandatory{\arrangeEmbeddedChanges}\howToApplyTheseTerms
    \lsucmandatory{\arrangeEnclosingBinaries}
    \lsucoptional{\createChangelog}
    \lsucoptional{\addToDocumentation}
  \end{lsucrequires}

  \begin{lsucprohibits}
    \lsucitem{\noPatentLitigation}
  \end{lsucprohibits}
\end{lsuc}


% ------------------------------------------------------------------------------
\subsection{AGPL-\ver-CC: Executing a modified program with network interaction}
\begin{lsuc}{AGPL-\ver-CC}
  \linkosuc{03N}
  \label{OSUC-03N-AGPL}
  
  \useCaseC

  \begin{lsucrequires}
    \lsucmandatory{\keepLicenseElements}
    \lsucmandatory{\gplthreeEnsureCopyrightNoticeBinary}
    \lsucmandatory{\giveLicense}\passingFilesCorrectly
    \lsucmandatory{\retainCopyrightNotices}
    \lsucmandatory{\markProgramModifications}
    \lsucmandatory{\addToCopyrightDialogApp}
    \lsucmandatory{\arrangeProgramChanges}\howToApplyTheseTerms
    \lsucmandatory{\makeExecModifiedSourceAvailable}
    \lsucmandatory{\describeHowToGetSource}
    \lsucsourcedist{AGPL-\ver-C6}
    \lsucoptional{\createChangelog}
    \lsucoptional{\addToDocumentation}
  \end{lsucrequires}

  \begin{lsucprohibits}
    \lsucitem{\noPatentLitigation}
  \end{lsucprohibits}
\end{lsuc}

\subsection{AGPL-\ver-CD: Executing a (modified) library as embedded component
with network interaction}
\begin{lsuc}{AGPL-\ver-CD}
  \linkosuc{09N}
  \linkosuc{06N}

  \label{OSUC-06N-AGPL}
  \label{OSUC-09N-AGPL}
  
  
  \useCaseD

  \begin{lsucrequires}
    \lsucmandatory{\keepLicenseElements}
    \lsucmandatory{\gplthreeEnsureCopyrightNoticeBinary}
    \lsucmandatory{\giveLicense}\passingFilesCorrectly
    \lsucmandatory{\retainCopyrightNotices}
    \lsucmandatory{\makeExecAllSourcesAvailable}
    \lsucmandatory{\describeHowToGetSource}
    \lsucsourcedist{AGPL-\ver-CA}
    \lsucmandatory{\addToCopyrightDialogLib}
    \lsucmandatory{\markEmbeddedModifications}
    \lsucmandatory{\arrangeEmbeddedChanges}\howToApplyTheseTerms
    \lsucmandatory{\arrangeEnclosingBinaries}
    \lsucoptional{\createChangelog}
    \lsucoptional{\addToDocumentation}
  \end{lsucrequires}

  \begin{lsucprohibits}
    \lsucitem{\noPatentLitigation}
  \end{lsucprohibits}
\end{lsuc}
% ------------------------------------------------------------------------------
\end{license}

%% =============================================================================
%% Discussion

\subsection{Discussions and Explanations}
\label{AGPL3Discussion}
  
For simplifying the justifications of our AGPL interpretation, we can state,
that the AGPL-3.0 and the GPL-3.0 are very similar: apart from some differences
caused by the varying names and passings remarks\footnote{Very similar are the
preamble and §0.  Compare \cite[][\nopage wp.]{Gpl30OsiLicense2007a} versus
\cite[][\nopage wp.]{Agpl30OsiLicense2007a}}, the most paragraphs of the two
licenses exactly offer the same text\footnote{Equal are §1 - 12 and §14 - §17.
Compare \cite[][\nopage wp.]{Gpl30OsiLicense2007a} versus \cite[][\nopage
wp.]{Agpl30OsiLicense2007a}}. Only the §13 of the AGPL-3.0 does not match to the
§13 of the GPL-3.0: §13 of the GPL-3.0 permits \enquote{[\ldots] to link or
combine any covered work with a work lincesed under version 3 of the GNU Affero
Generasl Public License}\footcite[cf.][\nopage wp. §13]{Gpl30OsiLicense2007a};
while §13 of the AGPL-3.0 deals with the \enquote{remote network
interaction}\footcite[cf.][\nopage wp. §13]{Agpl30OsiLicense2007a}. Therefore,
the analysis of the GPL-3.0 lincense\footnote{$\rightarrow$ p.
\pageref{GPL3Discussion}} is also valid for the AGPL-3.0; it is not necessary to
repeat that discussion here.

So, we can focus on the difference. The AGPL-3.0 tries to close a gap of the
GPL-3.0:

Purpose of all GNU licenses is to preserve the freedom to use, to study, to
share, and to modify the GNU programs and libraries\footcite[cf.][\nopage
wp.]{FsfFreeSoftware2015a}. These licenses want to prevent that users circumvent
the tasks which establish and maintain this freedom: Only if someone uses the
program / library only for himself, he shall not be obliged to do anything. But
if any third party was involved into the use of the GNU software, this third
party should receive all those rights and possibilities to use the software
which all the other users already have got.

In a time, where using the benefits of a program meant \emph{executing the
software on ons's own machine} (and hence \emph{having received the program at
least as a binary}), it was enough to let the obligations of -- for example --
handing over the license or the source code be triggered by the act of
'distributing the software'. Nowadays, in the times of cloud software systems,
users can let profit other users from the free software without conveying the
software. In these cases, they  execute the free program on their own machines,
but they nevertheless do not use the free program any longer only for
themselves. So, in time of cloud service technologies, the trigger of executing
the license fulfilling tasks must be complemented by a criterion which indicates
that a third party is involved into the context of using the software. And this
criteroin must no longer presuppose that this third party has received the
software itself.

For that purpose, the AGPL-3.0 states, that such an executed AGPL program must
\enquote{[\ldots] prominently offer all useres \emph{interacting with it
remotely through a computer network} [\ldots] an opportunity to receive the
Corresponding Source of your version by providing access to the Corresponding
Source from a network server at no charge, through some standard or customary
means of facilitating copying of software}\footcite[cf.][\nopage wp.
§13]{Agpl30OsiLicense2007a}. Obviously, the trigger of \emph{distributing the
AGPL software} now has been expanded by the feature \emph{being able to interact
with the AGPL software remotely through a computer network}.

The first consequence of this analysis is, that we can take over all the GPL
uses cases which deal with \emph{distributing the software} (2others) and all
the corresponding license fulfilling tasklists of
GPL-3-C2\footnote{$\rightarrow$ OSLiC, p. \pageref{OSUC-02S-GPL3}} until
GPL-3-CB\footnote{$\rightarrow$ OSLiC, p. \pageref{OSUC-10B-GPL3}} -- as we have
defined them in the GPL chapter.

The second consequence is, that we now have to subclassify the open source use
case \emph{recipient:4yourself}: we have to distinguish the use with internet
input-output access from that with only local input-output acesss.

Additionally, the AGPL limits the requirement to the condition, that the used
program is modified. The license exactly says that \enquote{[\ldots] if you
\emph{modify} the Program, your modfied version must prominently offer all
useres \emph{interacting with it remotely through a computer network} [\ldots]
an opportunity to receive the Corresponding Source of your version
[\ldots]}\footcite[cf.][\nopage wp. §13]{Agpl30OsiLicense2007a}. Thus, the third
consequence is, that we have to subclassify the open source use case 
\emph{recipient:4yourself} not only by the features \emph{ioAccess:viaInternet}
and \emph{ioAccess:onlyLocal}, but also by the features
\emph{state:modified} and \emph{state:unmodified}.

Finally, there is another little complication: One can only execute a program. A
library can not be directly executed. So, the question arises, what the user has
to be do if executes an own program which uses an unmodified AGPL licensed
library or module?

On the first glance, the license in §13 says only that he has to publish the
sources too, if executes a modified program. But on further reflection, one has
also to consider the other paragraphs of the AGPL: If one embeds an AGPL
licensed library, snippet or module into an own program, then -- due to the
Copyleft effect of the AGPL -- this program which uses the library, snippet or
module, has to be licensed under the AGPL too. And finally, every new program
has to be regarded as a modification of the first empty file. In other words:
one can only execute an own program using an unmodified AGPL library compliantly,
if one respects the §13 for the complete software complex being comprised of the
library itself and the pure code of the overarching program.

Based on this analysis, we had only to introduce two new AGPL specific open
source use cases and could recycle the complete set of GPL specific open source
use cases:

\begin{itemize}
  \item All GPL-3.0 use cases triggered by the distribution of the software
  \emph{recipient:2others} are transfered into the AGPL-3.0 finder and the
  AGPL-3.0 tasklist chapter as they have been defined in the GPL finder and the
  GPL-3.0 tasklist chapter.
  \item All combinations of \emph{recipient:4yourself} and
  \emph{ioAccess:onlyLocally} are covered by the old GPL 'yourself' use case
  which says, that one has not do anything as long as one uses the software only
  for oneself. But in the context of AGPL, this use case has additional
  conditions: one has not do anything if one does not distribute the software to
  other parties in any thing and if one executes this software on one's own
  machines in an environment which does not allow anyone else than
  oneself to interact with it remotely through a computer network.
  \item If one executes an unmodified AGPL program, which one has received and
  which one has not modified, then one also has not do anything.
  \item If one 'executes' an unmodified library as an embedded component of the
  really executed overarching program, then one has also to license this
  overarching program under the AGPL and hence has to fulfill the conditions of §13.
  \item If one executes a modified AGPL program, which one has received, has to
  fulfill the conditions of §13.
  \item If one executes an modified library as an embedded component of the
  really executed overarching program, then one has also to license this
  overarching program under the AGPL and hence has to fulfill the conditions of
  §13 with respect to bot parts, to the overarching program and the library.
\end{itemize}

There is a last point, which should also be discussed here. It concerns the
question of granularity:

The AGPL-3.0 requires that the \enquote{[\ldots] modified version (of an
[executed] program) must prominently offer all users interacting with it remotely through a
computer network [\ldots] an opportunity to receive the Corresponding Source of
your version by providing access to the Corresponding Source from a network
server at no charge [\ldots]}\footcite[cf.][\nopage wp.
§13]{Agpl30OsiLicense2007a}. For respecting this rule, one has to know what the
term \emph{Corresponding Source} means: how many of the embedded components of the
program must be conveyed together with the overarching program.

Fortunately, the AGPL-3.0 (and the GPL-3.0) defines the used terms: \enquote{The
\enquote{Corresponding Source} for a work in object code form means all the
source code needed to generate, install, and (for an executable work) run the
object code and to modify the work, including scripts to control those
activities.\footcite[cf.][\nopage wp. §1]{Agpl30OsiLicense2007a}} If one took
this statements seriously, one would have to \enquote{provide access to} the
complete software stack of the executed AGPL program -- just down to the glibc.

But the AGPL does not want to be to greedy. Therefore it limits the scope by
determining, that the \emph{Corresponding Source} \enquote{[\ldots] does not
include the work's System Libraries, or general-purpose tools or generally
available free programs which are used unmodified in performing those activities
but which are not part of the work}\footcite[cf.][\nopage wp.
§1]{Agpl30OsiLicense2007a}. For understanding this rule, one has to know, what
the term \emph{System Libraries} means. The AGPl says, that \enquote{the
\enquote{System Libraries} of an executable work include anything, other than
the work as a whole, that (a) is included in the normal form of packaging a
Major Component, but which is not part of that Major Component, and (b) serves
only to enable use of the work with that Major Component, or to implement a
Standard Interface for which an implementation is available to the public in
source code form.\footcite[cf.][\nopage wp. §1]{Agpl30OsiLicense2007a}}
Unfortunately, one has now to analyse, what the AGPL defines as a \emph{Major
Component}: \enquote{A enquote{Major Component}, in this context, means a major
essential component (kernel, window system, and so on) of the specific operating
system (if any) on which the executable work runs, or a compiler used to produce
the work, or an object code interpreter used to run it\footcite[cf.][\nopage wp.
§1]{Agpl30OsiLicense2007a}}.

Based on these specifications, one can give some rule of thumbs concerning the
question down to which level one has to give access to the corresponding source
code of an an executed AGPL program:
\begin{itemize}
  \item If one lets execute a modified AGPL licensed binary program, then one has
  to give access to the code of
  \begin{itemize}
  \item the executed program itself
  \item every modified embedded component of that program
  \item every not freely accessible embedded component of that program
  \item all not freely accessible tools, scripts, data which are necessary to
  compile the sources of the program in a freely accessible compilation /
  developement environment
  \end{itemize}
  But it is not necessary to give access to unmodified standard libraries,
  compilers, or tools which can freely be downloaded from their standard
  repositories.
  \item If one lets execute a modified AGPL licensed script, then one has
  to give access to the code of
  \begin{itemize}
  \item the executed script itself
  \item every modified embedded script component included by the main script
  \item every not freely accessible embedded script component included by the main script
  \item all not freely accessible tools, scripts, data which are necessary to
  to let that main script be executed by a freely accessible interpreter
  \item the interpreter itself if it is not freely accessible.
  \end{itemize}
  But it is not necessary to give access to unmodified standard script
  libraries, interpreters, or tools which can freely be downloaded from their
  standard repositories
  
\end{itemize}



}
{% Telekom osCompendium 'for being included' snippet template
%
% (c) Karsten Reincke, Deutsche Telekom AG, Darmstadt 2011
%
% This LaTeX-File is licensed under the Creative Commons Attribution-ShareAlike
% 3.0 Germany License (http://creativecommons.org/licenses/by-sa/3.0/de/): Feel
% free 'to share (to copy, distribute and transmit)' or 'to remix (to adapt)'
% it, if you '... distribute the resulting work under the same or similar
% license to this one' and if you respect how 'you must attribute the work in
% the manner specified by the author ...':
%
% In an internet based reuse please link the reused parts to www.telekom.com and
% mention the original authors and Deutsche Telekom AG in a suitable manner. In
% a paper-like reuse please insert a short hint to www.telekom.com and to the
% original authors and Deutsche Telekom AG into your preface. For normal
% quotations please use the scientific standard to cite.
%
% [ Framework derived from 'mind your Scholar Research Framework' 
%   mycsrf (c) K. Reincke 2012 CC BY 3.0  http://mycsrf.fodina.de/ ]
%


%% use all entries of the bibliography
%\nocite{*}

\section{Apache-2.0 licensed software}

\begin{license}{APL} % ends at end of file
\licensename{Apache-2.0}
\licensespec{Apache License 2.0}
\licenseversion{2.0}
\licenseabbrev{Apache}

Today, the current release of the Apache open source license is version 2.0,
older versions are deprecated.\footnote{For details $\rightarrow$ \oslic, pp.\
\protectionpageref{APL}} Because it focusses primarily on the
\enquote{redistribution,}\footcite[cf.][\nopage wp.\ §4]{Apl20OsiLicense2004a}
the following simplified Apache specific open source use case
finder\footnote{For details of the general OSUC finder $\rightarrow$ \oslic, pp.\
\pageref{OsucTokens} and \pageref{OsucDefinitionTree}} can be used:
 
\tikzstyle{nodv} = [font=\small, ellipse, draw, fill=gray!10, 
    text width=2cm, text centered, minimum height=2em]

\tikzstyle{nods} = [font=\footnotesize, rectangle, draw, fill=gray!20, 
    text width=1.2cm, text centered, rounded corners, minimum height=3em]

\tikzstyle{nodb} = [font=\footnotesize, rectangle, draw, fill=gray!20, 
    text width=2.2cm, text centered, rounded corners, minimum height=3em]
    
\tikzstyle{leaf} = [font=\tiny, rectangle, draw, fill=gray!30, 
    text width=1.2cm, text centered, minimum height=6em]

\tikzstyle{edge} = [draw, -latex']

\begin{tikzpicture}[]

\node[nodv] (l71) at (3,10) {Apache-2.0};

\node[nodb] (l61) at (0,8.6) {\textit{recipient:} \\ \textbf{4yourself}};
\node[nodb] (l62) at (6.5,8.6) {\textit{recipient:} \\ \textbf{2others}};

\node[nodb] (l51) at (2.5,7) {\textit{state:} \\ \textbf{unmodified}};
\node[nodb] (l52) at (9.3,7) {\textit{state:} \\ \textbf{modified}};

\node[nods] (l41) at (1.8,5.4) {\textit{form:} \textbf{source}};
\node[nods] (l42) at (3.6,5.4) {\textit{form:} \textbf{binary}};
\node[nodb] (l43) at (6.5,5.4) {\textit{type:} \\ \textbf{proapse}};
\node[nodb] (l44) at (12,5.4) {\textit{type:} \\ \textbf{snimoli}};


\node[nods] (l31) at (5.4,3.8) {\textit{form:} \textbf{source}};
\node[nods] (l32) at (7.2,3.8) {\textit{form:} \textbf{binary}};
\node[nodb] (l33) at (10,3.8) {\textit{context:} \\ \textbf{independent}};
\node[nodb] (l34) at (13.5,3.8) {\textit{context:} \\ \textbf{embedded}};

\node[nods] (l21) at (9,2.2) {\textit{form:} \textbf{source}};
\node[nods] (l22) at (10.8,2.2) {\textit{form:} \textbf{binary}};
\node[nods] (l23) at (12.6,2.2) {\textit{form:} \textbf{source}};
\node[nods] (l24) at (14.4,2.2) {\textit{form:} \textbf{binary}};

\node[leaf] (l11) at (0,0) {\textbf{Apache-2.0-C1} \textit{using software only
for yourself}};

\node[leaf] (l12) at (1.8,0) { \textbf{Apache-2.0-C2} \textit{ distributing unmodified
software as sources}};

\node[leaf] (l13) at (3.6,0) { \textbf{Apache-2.0-C3}  \textit{ distributing unmodified
software as binaries}};

\node[leaf] (l14) at (5.4,0) { \textbf{Apache-2.0-C4}  \textit{ distributing modified
program as sources}};

\node[leaf] (l15) at (7.2,0) { \textbf{Apache-2.0-C5}  \textit{ distributing modified
program as binaries}};

\node[leaf] (l16) at (9,0) { \textbf{Apache-2.0-C6}  \textit{ distributing modified
library as independent sources}};

\node[leaf] (l17) at (10.8,0) { \textbf{Apache-2.0-C7} \textit{distributing modified
library as independent binaries}};

\node[leaf] (l18) at (12.6,0) { \textbf{Apache-2.0-C8}  \textit{distributing
modified library as embedded sources}};

\node[leaf] (l19) at (14.4,0) { \textbf{Apache-2.0-C9}  \textit{ distributing
modified library as embedded binaries}};


\path [edge] (l71) -- (l61);
\path [edge] (l71) -- (l62);
\path [edge] (l61) -- (l11);
\path [edge] (l62) -- (l51);
\path [edge] (l62) -- (l52);
\path [edge] (l51) -- (l41);
\path [edge] (l51) -- (l42);
\path [edge] (l52) -- (l43);
\path [edge] (l52) -- (l44);
\path [edge] (l41) -- (l12);
\path [edge] (l42) -- (l13);
\path [edge] (l43) -- (l31);
\path [edge] (l43) -- (l32);
\path [edge] (l44) -- (l33);
\path [edge] (l44) -- (l34);
\path [edge] (l31) -- (l14);
\path [edge] (l32) -- (l15);
\path [edge] (l33) -- (l21);
\path [edge] (l33) -- (l22);
\path [edge] (l34) -- (l23);
\path [edge] (l34) -- (l24);
\path [edge] (l21) -- (l16);
\path [edge] (l22) -- (l17);
\path [edge] (l23) -- (l18);
\path [edge] (l24) -- (l19);

\end{tikzpicture}


%% ==============================================================================
%% Common building blocks
%%

% ------------------------------------------------------------------------------
% Give a copy of the license
\newcommand{\auxGiveLicenseFor}[1]{%
  Give the recipient a copy of the Apache 2.0 license. If it is not already part
  of the #1 package, add it.} 
\newcommand{\copyLicenseWithBinary}{\auxGiveLicenseFor{binary}}
\newcommand{\copyLicenseWithSource}{\auxGiveLicenseFor{software}}

% ------------------------------------------------------------------------------
% Keep licensing elements intact
\newcommand{\keepLicenseElements}{
  Ensure that the licensing elements (especially the specific copyright notice
  of the original author(s)) are retained in your package in the form you have
  received them.} 
\newcommand{\keepLicenseElementsBinary}{%
  \keepLicenseElements{}
  If you compile the binary from the sources, ensure that all the
  licensing elements are also incorporated into the package.} 

% ------------------------------------------------------------------------------
% Keep notice file intact
\newcommand{\auxCreateAndAddToNoticeFile}{%
  Create a \emph{notice text file}, if it still does not exist. \emph{Add} a
  description of your modifications into the \emph{notice text file.}}

\newcommand{\auxKeepNoticeFile}[1]{%
  Ensure that the \emph{notice text file} is #1 your package in the form you
  have initially received it.}  
\newcommand{\keepNoticeFile}{\auxKeepNoticeFile{retained in}}
\newcommand{\includeNoticeFile}{\auxKeepNoticeFile{retained in or integrated into}}

\newcommand{\keepAllNotices}{%
  Ensure that the \emph{notice text file} contains at least all the information
  in the \emph{notice text file} that you have received.} 
\newcommand{\expandNotices}{%
  \keepAllNotices{} 
  \auxCreateAndAddToNoticeFile{}}

% ------------------------------------------------------------------------------
% Display the notice file
\newcommand{\auxDisplayNotice}{%
  Ensure that the \emph{notice text file} is also reproduced if and whereever
  such third-party notices normally appear.}

\newcommand{\displayNotice}{%
  \auxDisplayNotice{}
  If the program already displays a copyright dialog, update it in an
  appropriate manner.} 

\newcommand{\displayNoticeOfEmbeddedLibrary}{%
  \auxDisplayNotice{}
  If the software that embeds this library displays its own copyright dialog,
  insert this information there.}

% ------------------------------------------------------------------------------
% Mark your modifications
\newcommand{\auxMarkModifiedSource}[1]{%
  Inside of the #1 source code, mark all your modifications thoroughly. 
  \auxCreateAndAddToNoticeFile{}}
\newcommand{\markModifiedSource}{\auxMarkModifiedSource{}}
\newcommand{\markModifiedLibrarySource}{\auxMarkModifiedSource{library}}

\newcommand{\auxMarkUndistributedChanges}[1]{%
  Even if you do not want to distribute your modified source code, mark all your
  modifications #1 thoroughly.} 
\newcommand{\markUndistributedChanges}{%
  \auxMarkUndistributedChanges{}}
\newcommand{\markUndistributedLibraryChanges}{%
  \auxMarkUndistributedChanges{of the embedded libary}}

% ------------------------------------------------------------------------------
% Add info to documentation
\newcommand{\auxDocumentation}{%
  Let the documentation of your distribution and/or your additional material
  also reproduce the content of the \emph{notice text file}, a hint to the
  software name, a link to its homepage, and a link to the Apache 2.0 license}

\newcommand{\documentation}{%
  \auxDocumentation.} 
\newcommand{\documentationBinary}{%
  \auxDocumentation, especially as a subsection of your own copyright notice.} 

% ------------------------------------------------------------------------------
% Forbid promoting services using trademarks, etc.
\newcommand{\noTrademarks}{%
  to promote any of your services based on the this software by trademarks,
  service marks, or product names linked to the software except as required for
  reasonable and customary use in describing the software file.
  }
\newcommand{\noTrademarksExceptNotice}{%
  to promote any of your services or products based on the this software by
  trademarks, service marks, or product names linked to this Apache software,
  except as required for reasonable and customary use in describing the origin
  of the Work and reproducing the content of the NOTICE file.
  }

% ------------------------------------------------------------------------------
% Forbid patent litigation
\newcommand{\noPatentLitigation}{%
  to institute a patent litigation against anyone alleging that the software
  constitutes patent infringement.} 

% ------------------------------------------------------------------------------

\subsection{Apache-2.0-C1: Using the software only for yourself}
\begin{lsuc}{Apache-2.0-C1}
  \linkosuc{01}
  \linkosuc{03L} 
  \linkosuc{03N} 
  \linkosuc{06L}
  \linkosuc{06N}
  \linkosuc{09L}
  \linkosuc{09N}

  \lsucmeans{that you received Apache-2.0 licensed software, that you will use it
  only for yourself, and that you do not hand it over to any 3rd party in any
  sense.}

  \coversOsucs{OSUC-01, OSUC-03L, OSUC-03N, OSUC-06L, OSUC-06N, OSUC-09L, and
  OSUC-09N}{01}{09N}

  \begin{lsucrequiresnothing}
    \lsucitem{You are allowed to use any kind of Apache software in any sense
      and in any context without being obliged to do anything as long as you do
      not give the software to third parties.}
  \end{lsucrequiresnothing}
  
  \begin{lsucprohibits}
    \lsucitem{\noTrademarks}
    \lsucitem{\noPatentLitigation}
  \end{lsucprohibits}
\end{lsuc}

\subsection{Apache-2.0-C2: Passing the unmodified software as source code}
\begin{lsuc}{Apache-2.0-C2}
  \linkosuc{02S}
  \linkosuc{05S}
  \linkosuc{07S}

  \lsucmeans{that you received Apache-2.0 licensed software which you are now
  going to distribute to third parties in the form of unmodified source code
  files or as unmodified source code package. In this case it makes no
  difference if you distribute a program, an application, a server, a snippet, a
  module, a library, or a plugin as an independent or as an embedded unit.}

  \coversOsucs{OSUC-02S, OSUC-05S, OSUC-07S}{02S}{07S}

  \begin{lsucrequires}
    \lsucmandatory{\copyLicenseWithSource}\passingFilesCorrectly
    \lsucmandatory{\keepLicenseElements}
    \lsucmandatory{\keepNoticeFile}%
    \footnote{The Apache license seems purposely to be a bit ambiguous: it
      uses the term \enquote{\enquote{Notice} text file}. In its strict sense,
      the term refers to a file named
      `NOTICE.[txt\textbar{}pdf\textbar{}\ldots]'. In a weaker sense, it may 
      denote any (text) file containing (licensing) notices. To be sure to act
      according to this requirement you should also read this term in the
      broader sense if there is no text file named `NOTICE'}
      
    \lsucoptional{\documentation}
  \end{lsucrequires}

  \begin{lsucprohibits}
    \lsucitem{\noTrademarks}
    \lsucitem{\noPatentLitigation}
  \end{lsucprohibits}
\end{lsuc}


\subsection{Apache-2.0-C3: Passing the unmodified software as binaries}
\begin{lsuc}{Apache-2.0-C3}
  \linkosuc{02B} 
  \linkosuc{05B} 
  \linkosuc{07B} 

  \lsucmeans{that you received Apache-2.0 licensed software which you are now
  going to distribute to third parties in the form of unmodified binary files or
  as unmodified binary package. In this case it does not matter if you distribute
  a program, an application, a server, a snippet, a module, a library, or a
  plugin as an independent or an embedded unit.}

  \coversOsucs{OSUC-02B, OSUC-05B, OSUC-07B}{02B}{07B}

  \begin{lsucrequires}
    \lsucmandatory{\copyLicenseWithBinary}\passingFilesCorrectly
    \lsucmandatory{\keepLicenseElementsBinary}
    \lsucmandatory{\includeNoticeFile}

    \lsucmandatory{Ensure that the \emph{notice text file} is also reproduced if
      and whereever such third-party notices normally appear (especially, if you
      are distributing an unmodified Apache-2.0 licensed library as embedded
      component of your own work which displays its own copyright notice.)} 
    
    \lsucoptional{\documentationBinary}
  \end{lsucrequires}

  \begin{lsucprohibits}
    \lsucitem{\noTrademarks}
    \lsucitem{\noPatentLitigation}
  \end{lsucprohibits}

\end{lsuc}

\subsection{Apache-2.0-C4: Passing a modified program as source code}
\begin{lsuc}{Apache-2.0-C4}
  \linkosuc{04S} 
  
  \lsucmeans{that you received an Apache-2.0 licensed program, application, or
  server (proapse), that you modified it, and that you are now going to
  distribute this modified version to third parties in the form of source code files or as
  a source code package.}
 
  \mapsToOsuc{04S}

  \begin{lsucrequires}
    \lsucmandatory{\copyLicenseWithSource}\passingFilesCorrectly
    \lsucmandatory{\keepLicenseElements}
    \lsucmandatory{\keepAllNotices}
    \lsucmandatory{\displayNotice}
  
    \lsucmandatory{Inside of the source code, mark all your modifications
      thoroughly. Generate a \emph{notice text file}, if it still does not
      exist. \emph{Add} a description of your modifications into the
      \emph{notice text file.}}

    \lsucoptional{\documentationBinary}
  \end{lsucrequires}
 
  \begin{lsucprohibits}
    \lsucitem{\noTrademarks}
    \lsucitem{\noPatentLitigation}
  \end{lsucprohibits}
\end{lsuc}

\subsection{Apache-2.0-C5: Passing a modified program as binary}
\begin{lsuc}{Apache-2.0-C5}
  \linkosuc{04B}

  \lsucmeans{that you received an Apache-2.0 licensed program, application, or
  server (proapse), that you modified it, and that you are now going to
  distribute this modified version to third parties in the form of binary files or as a
  binary package.}

  \mapsToOsuc{04B}

  \begin{lsucrequires}
    \lsucmandatory{\copyLicenseWithBinary}\passingFilesCorrectly
    \lsucmandatory{\keepLicenseElementsBinary}
    \lsucmandatory{\expandNotices}
    \lsucmandatory{\displayNotice}
 
    \lsucoptional{Even if you do not want to distribute your modified source
      code, mark all your modifications thoroughly.} 

    \lsucoptional{\documentationBinary}
  \end{lsucrequires}

  \begin{lsucprohibits}
    \lsucitem{\noTrademarks}
    \lsucitem{\noPatentLitigation}
  \end{lsucprohibits}
\end{lsuc}

\subsection{Apache-2.0-C6: Passing a modified library as independent source code}
\begin{lsuc}{Apache-2.0-C6}
  \linkosuc{08S}

  \lsucmeans{that you received an Apache-2.0 licensed code snippet, module, library,
  or plugin (snimoli), that you modified it, and that you are now going to
  distribute this modified version to third parties in the form of source code
  files or as a source code package, but without embedding it into another
  larger software unit.}

  \mapsToOsuc{08S}

  \begin{lsucrequires}
    \lsucmandatory{\copyLicenseWithSource}\passingFilesCorrectly
    \lsucmandatory{\keepLicenseElements}
    \lsucmandatory{\keepAllNotices}
 
    \lsucmandatory{Inside of the source code, mark all your modifications
      thoroughly. Generate a \emph{notice text file}, if it still does not
      exist. \emph{Expand} the \emph{notice text file} by a description of your
      modifications.}

    \lsucoptional{\documentation}
  \end{lsucrequires}

  \begin{lsucprohibits}
    \lsucitem{\noTrademarks}
    \lsucitem{\noPatentLitigation}
  \end{lsucprohibits}

\end{lsuc}


\subsection{Apache-2.0-C7: Passing a modified library as independent binary}
\begin{lsuc}{Apache-2.0-C7}
  \linkosuc{08B}

  \lsucmeans{that you received an Apache-2.0 licensed code snippet, module, library,
  or plugin (snimoli), that you modified it, and that you are now going to
  distribute this modified version to third parties in the form of binary files
  or as a binary package but without embedding it into another larger software
  unit.}

  \mapsToOsuc{08B}

  \begin{lsucrequires}
    \lsucmandatory{\copyLicenseWithBinary}\passingFilesCorrectly
    \lsucmandatory{\keepLicenseElementsBinary}
    \lsucmandatory{\expandNotices}
   
    \lsucoptional{Even if you do not want to distribute your modified source
      code, mark all your modifications thoroughly.} 

    \lsucoptional{\documentationBinary}
  \end{lsucrequires}

  \begin{lsucprohibits}
    \lsucitem{\noTrademarks}
    \lsucitem{\noPatentLitigation}
  \end{lsucprohibits}
\end{lsuc}

\subsection{Apache-2.0-C8: Passing a modified library as embedded source code}
\begin{lsuc}{Apache-2.0-C8}
  \linkosuc{10S}

  \lsucmeans{that you received an Apache-2.0 licensed code snippet, module, library,
  or plugin (snimoli), that you modified it, and that you are now going to
  distribute this modified version to third parties in the form of source code
  files or as a source code package together with another larger software unit
  which contains this code snippet, module, library, or plugin as an embedded
  component.}

  \mapsToOsuc{10S}

  \begin{lsucrequires}
    \lsucmandatory{\copyLicenseWithSource}\passingFilesCorrectly
    \lsucmandatory{\keepLicenseElements}
    \lsucmandatory{\keepAllNotices}
 
    \lsucmandatory{\displayNoticeOfEmbeddedLibrary}
 
    \lsucmandatory{Inside of the library source code, mark all your
      modifications thoroughly. Generate a \emph{notice text file}, if it still
      does not exist. \emph{Expand} the \emph{notice text file} by a description
      of your modifications.}%
    \footnote{The term library also includes snippet, module, and plugin.} 

    \lsucoptional{\documentation}

    \lsucoptional{Arrange your source code distribution so that the integrated
      Apache license and the \emph{notice text file} clearly refer only to the
      embedded library and do not disturb the licensing of your own overarching
      work. It's a good tradition to keep embedded components like
      libraries, modules, snippets, or plugins in a specific directory which
      contains also all additional licensing elements.}
  \end{lsucrequires}

  \begin{lsucprohibits}
    \lsucitem{\noTrademarks}
    \lsucitem{\noPatentLitigation}
  \end{lsucprohibits}
\end{lsuc}


\subsection{Apache-2.0-C9: Passing a modified library as embedded binary}
\begin{lsuc}{Apache-2.0-C9}
  \linkosuc{10B}

  \lsucmeans{that you received an Apache-2.0 licensed code snippet, module, library,
  or plugin (snimoli), that you modified it, and that you are now going to
  distribute this modified version to third parties in the form of binary files
  or as a binary package together with another larger software unit which
  contains this code snippet, module, library, or plugin as an embedded component.}

  \mapsToOsuc{10B}

  \begin{lsucrequires}
    \lsucmandatory{\copyLicenseWithBinary}\passingFilesCorrectly
    \lsucmandatory{\keepLicenseElementsBinary}
    \lsucmandatory{\expandNotices}
 
    \lsucmandatory{\displayNoticeOfEmbeddedLibrary}
     
    \lsucoptional{Even if you do not want to distribute your modified source
      code, mark all your modifications of the embedded libary thoroughly.}
    \footnote{library or snippet, or module, or plugin} 

    \lsucoptional{\documentationBinary}
  
    \lsucoptional{Arrange your binary distribution so that the integrated Apache
      license and the \emph{notice text file} clearly refer only to the embedded
      library and do not disturb the licensing of your own overarching
      work. It's a good tradition to keep the libraries, modules, snippet, or
      plugins in specific directories which contain also all licensing elements.}
  \end{lsucrequires}

  \begin{lsucprohibits}
    \lsucitem{\noTrademarks}
    \lsucitem{\noPatentLitigation}
  \end{lsucprohibits}
\end{lsuc}

\subsection{Discussions and Explanations}
\label{APLDiscussion}
\begin{itemize}
  \item On the one hand, the Apache 2.0 license does not permit
  \enquote{[\ldots] to use the trade names, trademarks, service marks, or
  product names of the Licensor, except as required for reasonable and customary
  use in describing the origin of the Work and reproducing the content of the
  NOTICE file}\footcite[cf.][\nopage wp.\ §6]{Apl20OsiLicense2004a}. On the other
  hand, this license alerts that all the patent licenses granted to those who
  \enquote{[\ldots] institute a patent litigation} will terminate
  automatically\footcite[cf.][\nopage wp.\ §3]{Apl20OsiLicense2004a}. Hence, the
  \oslic{} generally (Apache-2.0-C1 - Apache-2.0-C9) interdicts to promote products or services by
  these elements and to legally fight against patents linked to the software.
  
  \item The Apache-2.0 also requires to \enquote{[\ldots] give any other recipients of
  the Work or Derivative Works a copy of this License}\footcite[cf.][\nopage wp.\
  §4.1]{Apl20OsiLicense2004a}. Therefore, all \emph{2others} use cases contain
  the respective mandatory condition (Apache-2.0-C2 - Apache-2.0-C9).
   
  \item Additionally, the Apache-2.0 requires, that modifications must be
  marked\footcite[cf.][\nopage wp.\ §4.2]{Apl20OsiLicense2004a}. Thus, in all
  cases of passing the modified software in the form of source code the \oslic{}
  requires to mark the modifications and to integrate a hint into the notice
  file---while in all the cases of passing the modified software in the form of
  binaries it inserts only a voluntary condition (Apache-2.0-C4 - Apache-2.0-C9).
  
  \item Furthermore, the Apache-2.0 requires that one must \enquote{[\ldots] retain, in
  the Source form of any Derivative Works that You distribute, all copyright,
  patent, trademark, and attribution notices from the Source form of the Work}
  So, the OSLIC requires in all contexts (Apache-2.0-C1 - Apache-2.0-C9) that the licensing
  elements are retained in the form you have received them\footnote{This might
  confuse some readers: Yes, even if you distribute a modified version in the
  form of binaries you must fulfill this condition. Moreover, you must also hand
  the license over to your receipient. But, nevertheless, you are not obliged to
  publish the modified source code, too. ($\rightarrow$ \oslic, p.
  \protectionpageref{APL})}.
  
  \item Finally, the Apache-2.0 requires that the received ``NOTICE text file''
  must be integrated as readable copy to each package distributed in the form of
  source code, or---in case of binary distibutions---must be displayed
  \enquote{[\ldots] if and wherever such third-party notices normally
  appear}\footcite[cf.][\nopage wp.\ §4.4]{Apl20OsiLicense2004a}. Thus, the
  \oslic{} requires mandatorily that all source code distributions must include
  the notice text file (Apache-2.0-C2, Apache-2.0-C4, Apache-2.0-C6,
  Apache-2.0-C8) and that all distributions of binary applications which
  normally show such a copyrigth screen must integrate the content of the notice
  file into this screen (Apache-2.0-C5, Apache-2.0-C9). For libraries
  distributed in the form of binaries it is assumed that they normally do not
  contain such copyright dialogs (Apache-2.0-C7)
\end{itemize}

\end{license}

%\bibliography{../../../bibfiles/oscResourcesEn}

% Local Variables:
% mode: latex
% fill-column: 80
% End:
}
{% Telekom osCompendium 'for being included' snippet template
%
% (c) Karsten Reincke, Deutsche Telekom AG, Darmstadt 2011
%
% This LaTeX-File is licensed under the Creative Commons Attribution-ShareAlike
% 3.0 Germany License (http://creativecommons.org/licenses/by-sa/3.0/de/): Feel
% free 'to share (to copy, distribute and transmit)' or 'to remix (to adapt)'
% it, if you '... distribute the resulting work under the same or similar
% license to this one' and if you respect how 'you must attribute the work in
% the manner specified by the author ...':
%
% In an internet based reuse please link the reused parts to www.telekom.com and
% mention the original authors and Deutsche Telekom AG in a suitable manner. In
% a paper-like reuse please insert a short hint to www.telekom.com and to the
% original authors and Deutsche Telekom AG into your preface. For normal
% quotations please use the scientific standard to cite.
%
% [ Framework derived from 'mind your Scholar Research Framework' 
%   mycsrf (c) K. Reincke 2012 CC BY 3.0  http://mycsrf.fodina.de/ ]
%


%% use all entries of the bibliography
%\nocite{*}

\section{BSD licensed software}

As an approved open source license, the BSD license exists in two
versions%
  \footnote{Following the OSI, there is another `ancient' BSD
    license---containing a fourth clause known as advertising clause---which
    \enquote{(\ldots) officially was rescinded by the Director of the Office of
      Technology Licensing of the University of California on July 22nd, 1999}. 
    Because of that cancellation you can simply act according the  
    \cite[cf.][\nopage wp.]{BsdLicense3Clause} 
    if you have to fulfill the oldest of the BSD licenses.}  
The latest release is the \textit{BSD 2-Clause license,}\citeBSDsimple{}, the
older release is the \textit{BSD 3-Clause license.}\citeBSDnew{} The very little
differences between the two versions have to be respected exactly. 

All BSD open source licenses focus explicitely on the (re-)distribution
\textit{open source use cases,} which we have specified by our token
\textit{2others}. Conditions for the other use cases specified by the token
\textit{4yourself} can be derived.%
  \footnote{For details of the \textit{open source use case tokens} see
    p.\ \pageref{OsucTokens}. For details of the \textit{open source use cases}
    based on these token see p. \pageref{OsucDefinitionTree}} 
Additionally the BSD licenses distinguishes between different forms of distribution,
esp.\ whether the work is distributed as a (set of) source code file(s) or as a
set of binary file(s). Use the following tree to find the BSD license fulfilling
to-do lists. 

\tikzstyle{nodv} = [font=\small, ellipse, draw, fill=gray!10, 
    text width=2cm, text centered, minimum height=2em]


\tikzstyle{nods} = [font=\footnotesize, rectangle, draw, fill=gray!20, 
    text width=1.2cm, text centered, rounded corners, minimum height=3em]

\tikzstyle{nodb} = [font=\footnotesize, rectangle, draw, fill=gray!20, 
    text width=2.2cm, text centered, rounded corners, minimum height=3em]
    
\tikzstyle{leaf} = [font=\tiny, rectangle, draw, fill=gray!30, 
    text width=1.2cm, text centered, minimum height=6em]

\tikzstyle{edge} = [draw, -latex']

\begin{tikzpicture}[]

\node[nodv] (l81) at (4,11.8) {BSD};

\node[nodv] (l71) at (0,10.2) {3-Clause License};
\node[nodv] (l72) at (6.5,10.2) {2-Clause License};


\node[nodb] (l61) at (0,8.6) {\textit{recipient:} \\ \textbf{4yourself}};
\node[nodb] (l62) at (6.5,8.6) {\textit{recipient:} \\ \textbf{2others}};

\node[nodb] (l51) at (2.5,7) {\textit{state:} \\ \textbf{unmodified}};
\node[nodb] (l52) at (9.3,7) {\textit{state:} \\ \textbf{modified}};

\node[nods] (l41) at (1.8,5.4) {\textit{form:} \textbf{source}};
\node[nods] (l42) at (3.6,5.4) {\textit{form:} \textbf{binary}};
\node[nodb] (l43) at (6.5,5.4) {\textit{type:} \\ \textbf{proapse}};
\node[nodb] (l44) at (12,5.4) {\textit{type:} \\ \textbf{snimoli}};


\node[nods] (l31) at (5.4,3.8) {\textit{form:} \textbf{source}};
\node[nods] (l32) at (7.2,3.8) {\textit{form:} \textbf{binary}};
\node[nodb] (l33) at (10,3.8) {\textit{context:} \\ \textbf{independent}};
\node[nodb] (l34) at (13.5,3.8) {\textit{context:} \\ \textbf{embedded}};

\node[nods] (l21) at (9,2.2) {\textit{form:} \textbf{source}};
\node[nods] (l22) at (10.8,2.2) {\textit{form:} \textbf{binary}};
\node[nods] (l23) at (12.6,2.2) {\textit{form:} \textbf{source}};
\node[nods] (l24) at (14.4,2.2) {\textit{form:} \textbf{binary}};

\node[leaf] (l11) at (0,0) {
  \textbf{BSD2-C1} 
  \textbf{BSD3-C1} 
  \textit{using software only for yourself}};

\node[leaf] (l12) at (1.8,0) { 
  \textbf{BSD2-C2} 
  \textbf{BSD3-C2} 
  \textit{ distributing unmodified software as sources}};

\node[leaf] (l13) at (3.6,0) { 
  \textbf{BSD2-C3}  
  \textbf{BSD3-C3}  
  \textit{ distributing unmodified software as binaries}};

\node[leaf] (l14) at (5.4,0) { 
  \textbf{BSD2-C4}  
  \textbf{BSD3-C4}  
  \textit{ distributing modified program as sources}};

\node[leaf] (l15) at (7.2,0) { 
  \textbf{BSD2-C5}  
  \textbf{BSD3-C5}  
  \textit{ distributing modified program as binaries}};

\node[leaf] (l16) at (9,0) { 
  \textbf{BSD2-C6}  
  \textbf{BSD3-C6}  
  \textit{ distributing modified library as independent sources}};

\node[leaf] (l17) at (10.8,0) { 
  \textbf{BSD2-C7} 
  \textbf{BSD3-C7} 
  \textit{distributing modified library as independent binaries}};

\node[leaf] (l18) at (12.6,0) { 
  \textbf{BSD2-C8}  
  \textbf{BSD3-C8}  
  \textit{distributing modified library as embedded sources}};

\node[leaf] (l19) at (14.4,0) { 
  \textbf{BSD2-C9}  
  \textbf{BSD3-C9}  
  \textit{ distributing modified library as embedded binaries}};

\path [edge] (l81) -- (l71);
\path [edge] (l81) -- (l72);
\path [edge] (l71) -- (l61);
\path [edge] (l71) -- (l62);
\path [edge] (l72) -- (l61);
\path [edge] (l72) -- (l62);
\path [edge] (l61) -- (l11);
\path [edge] (l62) -- (l51);
\path [edge] (l62) -- (l52);
\path [edge] (l51) -- (l41);
\path [edge] (l51) -- (l42);
\path [edge] (l52) -- (l43);
\path [edge] (l52) -- (l44);
\path [edge] (l41) -- (l12);
\path [edge] (l42) -- (l13);
\path [edge] (l43) -- (l31);
\path [edge] (l43) -- (l32);
\path [edge] (l44) -- (l33);
\path [edge] (l44) -- (l34);
\path [edge] (l31) -- (l14);
\path [edge] (l32) -- (l15);
\path [edge] (l33) -- (l21);
\path [edge] (l33) -- (l22);
\path [edge] (l34) -- (l23);
\path [edge] (l34) -- (l24);
\path [edge] (l21) -- (l16);
\path [edge] (l22) -- (l17);
\path [edge] (l23) -- (l18);
\path [edge] (l24) -- (l19);

\end{tikzpicture}

%% ============================================================================= 
%% Common Building Blocks

% ------------------------------------------------------------------------------
% Common description of license specific use cases

\newcommand{\useCaseOne}{that you received BSD licensed software, that you will
  use it only for yourself and that you do not hand it over to any 3rd party in
  any sense.}

\newcommand{\useCaseTwo}{that you received BSD licensed software which you are
  now going to distribute to third parties in the form of unmodified source code
  files or as unmodified source code package. In this case it makes no
  difference if you distribute a program, an application, a server, a snippet, a
  module, a library, or a plugin as an independent or as an embedded unit.}

\newcommand{\useCaseThree}{that you received BSD licensed software which you are
  now going to distribute to third parties in the form of unmodified binary
  files or as unmodified binary package. In this case it does not matter if you
  distribute a program, an application, a server, a snippet, a module, a
  library, or a plugin as an independent or an embedded unit.}

\newcommand{\useCaseFour}{that you received a BSD licensed program, application,
  or server (proapse), that you modified it, and that you are now going to
  distribute this modified version to third parties in the form of source code
  files or as a source code package.}

\newcommand{\useCaseFive}{that you received a BSD licensed program, application,
  or server (proapse), that you modified it, and that you are now going to
  distribute this modified version to third parties in the form of binary files
  or as a binary package.}

\newcommand{\useCaseSix}{that you received a BSD licensed code snippet, module,
  library, or plugin (snimoli), that you modified it, and that you are now going
  to distribute this modified version to third parties in the form of source
  code files or as a source code package, but without embedding it into another
  larger software unit.}

\newcommand{\useCaseSeven}{that you received a BSD licensed code snippet,
  module, library, or plugin (snimoli), that you modified it, and that you are
  now going to distribute this modified version to third parties in the form of
  binary files or as a binary package but without embedding it into another
  larger software unit.}

\newcommand{\useCaseEight}{that you received a BSD licensed code snippet,
  module, library, or plugin (snimoli), that you modified it, and that you are
  now going to distribute this modified version to third parties in the form of
  source code files or as a source code package together with another larger
  software unit which contains this code snippet, module, library, or plugin as
  an embedded component.}

\newcommand{\useCaseNine}{that you received a BSD licensed code snippet, module,
  library, or plugin (snimoli), that you modified it, and that you are now going
  to distribute this modified version to third parties in the form of binary
  files or as a binary package together with another larger software unit which
  contains this code snippet, module, library, or plugin as an embedded
  component.}

% ------------------------------------------------------------------------------
% Common mapping to generic use cases

\newcommand{\coversOne}{\coversOsucs{OSUC-01, OSUC-03L, OSUC-03N, OSUC-06L,
OSUC-06N, OSUC-09L and OSUC-09N}{01}{09N}}
\newcommand{\coversTwo}{\coversOsucs{OSUC-02S, OSUC-05S, OSUC-07S}{02S}{07S}}
\newcommand{\coversThree}{\coversOsucs{OSUC-02B, OSUC-05B, OSUC-07B}{02B}{07B}}
\newcommand{\coversFour}{\mapsToOsuc{04S}}
\newcommand{\coversFive}{\mapsToOsuc{04B}}
\newcommand{\coversSix}{\mapsToOsuc{08S}}
\newcommand{\coversSeven}{\mapsToOsuc{08B}}
\newcommand{\coversEight}{\mapsToOsuc{10S}}
\newcommand{\coversNine}{\mapsToOsuc{10B}}

% ------------------------------------------------------------------------------
% Keep license elements

\newcommand{\keepLicenseElements}{Ensure that the licensing elements
  (particularly the BSD license text, the specific copyright notice of the
  original author(s), and the BSD disclaimer) are retained in your package in
  the form you have received them.}

\newcommand{\insertLicenseIntoBinary}{Ensure that your distribution contains the
  original copyright notice, the BSD license, and the BSD disclaimer in the form
  you have received them. 
  If you build the binary package from the source code package and if this does
  not automatically generate and integrate the licensing files then create the
  copyright notice, the BSD conditions, and the BSD disclaimer in the form found
  to the in the source code package and insert these files into your
  distribution manually.} 

% ------------------------------------------------------------------------------
% Include copyright notice and license conditions in documentation

\newcommand{\includeLicenseInDocumentation}{Let the documentation of your
  distribution or your additional material also contain the original copyright
  notice, the BSD conditions, and the BSD disclaimer.}

% ------------------------------------------------------------------------------
% Add name of the projecty and link to its homepage to copyright dialog

\newcommand{\addLicenseToCopyrightMessage}{%
  It is a good practice of the open source community to let the copyright
  message that is shown by the running program also state that the program is 
  licensed under the BSD license. 
  Because you are already modifying the program you can also add such a hint if
  the presented original copyright notice lacks such a statement.}

\newcommand{\addLibraryLicenseToCopyrightMessage}{%
  It is a good practice of the open source community to let the copyright
  message that is shown by the running program also state that it contains
  components licensed under the BSD license. 
  Because you are embedding this snimoli into a larger software unit, you are
  developing this larger unit. 
  Hence, you can also expand the copyright notice of this larger unit by such a
  hint to its BSD components.}

% ------------------------------------------------------------------------------
% Keep components separate so that it is clear which license applies
\newcommand{\auxKeepComponentsSeparate}[1]{Arrange your #1 distribution so that
  the licensing elements (particularly, the BSD license text, the copyright
  notice of the original author(s), and the BSD disclaimer) clearly refer only
  to the embedded library and do not affect the licensing of your own
  overarching work. 
  It's a good tradition to keep the embedded components like libraries, modules,
  snippets, or plugins in separate directories, which also contains all their
  licensing elements.}  

\newcommand{\keepSourcesSeparate}{\auxKeepComponentsSeparate{source code}}
\newcommand{\keepBinariesSeparate}{\auxKeepComponentsSeparate{binary}}

% ------------------------------------------------------------------------------
% No advertising using the authors' names (BSD3)

\newcommand{\dontUseAuthorNames}{to use the name of the licensing organization
  or the names of the licensing contributors to promote your own work.}

%% =============================================================================
%% BSD-3-Clause
%% =============================================================================

\begin{license}{BSD3} 
\licensename{BSD-3-Clause}
\licensespec{New BSD (3 Clauses)}
\licenseabbrev{BSD-3-Clause}

%% ============================================================================= 
%% Use Cases

\subsection{BSD-3-Clause-C1: Using the software only for yourself}
\begin{lsuc}{BSD-3-Clause-C1}
  \linkosuc{01}
  \linkosuc{03L} 
  \linkosuc{03N} 
  \linkosuc{06L}
  \linkosuc{06N}
  \linkosuc{09L}
  \linkosuc{09N}
  
  \lsucmeans{\useCaseOne} 
  \lsuccovers{\coversOne}

  \begin{lsucrequiresnothing}
    \lsucitem{You are allowed to use any kind of BSD software in any sense and
      in any context without any obligations as long as you do not give the
      software to 3rd parties.}
  \end{lsucrequiresnothing}

  \begin{lsucprohibits}
    \lsucitem{\dontUseAuthorNames}%
    \footnote{which may be, for example, an internet service based on this BSD
      software used in your own data center}. 
  \end{lsucprohibits}
\end{lsuc}

% ------------------------------------------------------------------------------
\subsection{BSD-3-Clause-C2: Passing the unmodified software as source code}
\begin{lsuc}{BSD-3-Clause-C2}
  \linkosuc{02S} 
  \linkosuc{05S} 
  \linkosuc{07S} 

  \lsucmeans{\useCaseTwo}
  \lsuccovers{\coversTwo}

  \begin{lsucrequires}
    \lsucmandatory{\keepLicenseElements}
    \lsucoptional{\includeLicenseInDocumentation}
  \end{lsucrequires}

  \begin{lsucprohibits}
    \lsucitem{\dontUseAuthorNames}%
  \end{lsucprohibits}
\end{lsuc}

% ------------------------------------------------------------------------------
\subsection{BSD-3-Clause-C3: Passing the unmodified software as binary}
\begin{lsuc}{BSD-3-Clause-C3}
  \linkosuc{02B} 
  \linkosuc{05B} 
  \linkosuc{07B} 

  \lsucmeans{\useCaseThree}
  \lsuccovers{\coversThree}

  \begin{lsucrequires}  
    \lsucmandatory{\insertLicenseIntoBinary}\passingFilesCorrectly
    \lsucmandatory{\includeLicenseInDocumentation}
  \end{lsucrequires}

  \begin{lsucprohibits}
    \lsucitem{\dontUseAuthorNames}%
  \end{lsucprohibits}
\end{lsuc}

% ------------------------------------------------------------------------------
\subsection{BSD-3-Clause-C4: Passing a modified program as source code}
\begin{lsuc}{BSD-3-Clause-C4}
  \linkosuc{04S}

  \lsucmeans{\useCaseFour}
  \lsuccovers{\coversFour}

  \begin{lsucrequires}
    \lsucmandatory{\keepLicenseElements}
    \lsucoptional{\includeLicenseInDocumentation}
    \lsucoptional{\addLicenseToCopyrightMessage}
  \end{lsucrequires}

  \begin{lsucprohibits}
    \lsucitem{\dontUseAuthorNames}%
  \end{lsucprohibits}
\end{lsuc}

% ------------------------------------------------------------------------------
\subsection{BSD-3-Clause-C5: Passing a modified program as binary}
\begin{lsuc}{BSD-3-Clause-C5}
  \linkosuc{04B}

  \lsucmeans{\useCaseFive}
  \lsuccovers{\coversFive}

  \begin{lsucrequires}
    \lsucmandatory{\insertLicenseIntoBinary}\passingFilesCorrectly
    \lsucmandatory{\includeLicenseInDocumentation}
    \lsucoptional{\addLicenseToCopyrightMessage}
  \end{lsucrequires}

  \begin{lsucprohibits}
    \lsucitem{\dontUseAuthorNames}%
  \end{lsucprohibits}
\end{lsuc}

% ------------------------------------------------------------------------------
\subsection{BSD-3-Clause-C6: Passing a modified library as independent source code}
\begin{lsuc}{BSD-3-Clause-C6}
  \linkosuc{08S}

  \lsucmeans{\useCaseSix}
  \lsuccovers{\coversSix}

  \begin{lsucrequires}
    \lsucmandatory{\keepLicenseElements}
    \lsucoptional{\includeLicenseInDocumentation}
  \end{lsucrequires}

  \begin{lsucprohibits}
    \lsucitem{\dontUseAuthorNames}%
  \end{lsucprohibits}
\end{lsuc}

% ------------------------------------------------------------------------------
\subsection{BSD-3-Clause-C7: Passing a modified library as independent binary}
\begin{lsuc}{BSD-3-Clause-C7}
  \linkosuc{08B}

  \lsucmeans{\useCaseSeven}
  \lsuccovers{\coversSeven}

  \begin{lsucrequires}
    \lsucmandatory{\insertLicenseIntoBinary}\passingFilesCorrectly
    \lsucmandatory{\includeLicenseInDocumentation}
  \end{lsucrequires}

  \begin{lsucprohibits}
    \lsucitem{\dontUseAuthorNames}%
  \end{lsucprohibits}
\end{lsuc}

% ------------------------------------------------------------------------------
\subsection{BSD-3-Clause-C8: Passing a modified library as embedded source code}
\begin{lsuc}{BSD-3-Clause-C8}
  \linkosuc{10S}

  \lsucmeans{\useCaseEight}
  \lsuccovers{\coversEight}

  \begin{lsucrequires}
    \lsucmandatory{\keepLicenseElements}
    \lsucoptional{\includeLicenseInDocumentation}
    \lsucoptional{\addLibraryLicenseToCopyrightMessage}
    \lsucoptional{\keepSourcesSeparate}
  \end{lsucrequires}

  \begin{lsucprohibits}
    \lsucitem{\dontUseAuthorNames}%
  \end{lsucprohibits}
\end{lsuc}

% ------------------------------------------------------------------------------
\subsection{BSD-3-Clause-C9: Passing a modified library as embedded binary}
\begin{lsuc}{BSD-3-Clause-C9}
  \linkosuc{10B}

  \lsucmeans{\useCaseNine}
  \lsuccovers{\coversNine}

  \begin{lsucrequires}
    \lsucmandatory{\insertLicenseIntoBinary}\passingFilesCorrectly
    \lsucmandatory{\includeLicenseInDocumentation}
    \lsucoptional{\addLibraryLicenseToCopyrightMessage}
    \lsucoptional{\keepBinariesSeparate}
  \end{lsucrequires}

  \begin{lsucprohibits}
    \lsucitem{\dontUseAuthorNames}%
  \end{lsucprohibits}
\end{lsuc}

\end{license}

%% =============================================================================
%% BSD-2-Clause
%% =============================================================================

\begin{license}{BSD2}
\licensename{BSD-2-Clause}
\licensespec{Simplified BSD (2 Clauses)}
\licenseabbrev{BSD-2-Clause}

%% =============================================================================
%% Use Cases

\subsection{BSD-2-Clause-C1: Using the software only for yourself}
\begin{lsuc}{BSD-2-Clause-C1}
  \linkosuc{01}
  \linkosuc{03L} 
  \linkosuc{03N} 
  \linkosuc{06L}
  \linkosuc{06N}
  \linkosuc{09L}
  \linkosuc{09N}
  
  \lsucmeans{\useCaseOne} 
  \lsuccovers{\coversOne}

  \begin{lsucrequiresnothing}
    \lsucitem{You are allowed to use any kind of BSD software in any sense and
      in any context without any obligations as long as you do not give the
      software to 3rd parties.}
  \end{lsucrequiresnothing}

  \lsucprohibitsnothing
\end{lsuc}

% ------------------------------------------------------------------------------
\subsection{BSD-2-Clause-C2: Passing the unmodified software as source code}
\begin{lsuc}{BSD-2-Clause-C2}
  \linkosuc{02S} 
  \linkosuc{05S} 
  \linkosuc{07S} 

  \lsucmeans{\useCaseTwo}
  \lsuccovers{\coversTwo}

  \begin{lsucrequires}
    \lsucmandatory{\keepLicenseElements}
    \lsucoptional{\includeLicenseInDocumentation}
  \end{lsucrequires}

  \lsucprohibitsnothing
\end{lsuc}

% ------------------------------------------------------------------------------
\subsection{BSD-2-Clause-C3: Passing the unmodified software as binary}
\begin{lsuc}{BSD-2-Clause-C3}
  \linkosuc{02B} 
  \linkosuc{05B} 
  \linkosuc{07B} 

  \lsucmeans{\useCaseThree}
  \lsuccovers{\coversThree}

  \begin{lsucrequires}  
    \lsucmandatory{\insertLicenseIntoBinary}\passingFilesCorrectly
    \lsucmandatory{\includeLicenseInDocumentation}
  \end{lsucrequires}

  \lsucprohibitsnothing
\end{lsuc}

% ------------------------------------------------------------------------------
\subsection{BSD-2-Clause-C4: Passing a modified program as source code}
\begin{lsuc}{BSD-2-Clause-C4}
  \linkosuc{04S}

  \lsucmeans{\useCaseFour}
  \lsuccovers{\coversFour}

  \begin{lsucrequires}
    \lsucmandatory{\keepLicenseElements}
    \lsucoptional{\includeLicenseInDocumentation}
    \lsucoptional{\addLicenseToCopyrightMessage}
  \end{lsucrequires}

  \lsucprohibitsnothing
\end{lsuc}

% ------------------------------------------------------------------------------
\subsection{BSD-2-Clause-C5: Passing a modified program as binary}
\begin{lsuc}{BSD-2-Clause-C5}
  \linkosuc{04B}

  \lsucmeans{\useCaseFive}
  \lsuccovers{\coversFive}

  \begin{lsucrequires}
    \lsucmandatory{\insertLicenseIntoBinary}\passingFilesCorrectly
    \lsucmandatory{\includeLicenseInDocumentation}
    \lsucoptional{\addLicenseToCopyrightMessage}
  \end{lsucrequires}

  \lsucprohibitsnothing
\end{lsuc}

% ------------------------------------------------------------------------------
\subsection{BSD-2-Clause-C6: Passing a modified library as independent source code}
\begin{lsuc}{BSD-2-Clause-C6}
  \linkosuc{08S}

  \lsucmeans{\useCaseSix}
  \lsuccovers{\coversSix}

  \begin{lsucrequires}
    \lsucmandatory{\keepLicenseElements}
    \lsucoptional{\includeLicenseInDocumentation}
  \end{lsucrequires}

  \lsucprohibitsnothing
\end{lsuc}

% ------------------------------------------------------------------------------
\subsection{BSD-2-Clause-C7: Passing a modified library as independent binary}
\begin{lsuc}{BSD-2-Clause-C7}
  \linkosuc{08B}

  \lsucmeans{\useCaseSeven}
  \lsuccovers{\coversSeven}

  \begin{lsucrequires}
    \lsucmandatory{\insertLicenseIntoBinary}\passingFilesCorrectly
    \lsucmandatory{\includeLicenseInDocumentation}
  \end{lsucrequires}

  \lsucprohibitsnothing
\end{lsuc}

% ------------------------------------------------------------------------------
\subsection{BSD-2-Clause-C8: Passing a modified library as embedded source code}
\begin{lsuc}{BSD-2-Clause-C8}
  \linkosuc{10S}

  \lsucmeans{\useCaseEight}
  \lsuccovers{\coversEight}

  \begin{lsucrequires}
    \lsucmandatory{\keepLicenseElements}
    \lsucoptional{\includeLicenseInDocumentation}
    \lsucoptional{\addLibraryLicenseToCopyrightMessage}
    \lsucoptional{\keepSourcesSeparate}
  \end{lsucrequires}

  \lsucprohibitsnothing
\end{lsuc}

% ------------------------------------------------------------------------------
\subsection{BSD-2-Clause-C9: Passing a modified library as embedded binary}
\begin{lsuc}{BSD-2-Clause-C9}
  \linkosuc{10B}

  \lsucmeans{\useCaseNine}
  \lsuccovers{\coversNine}

  \begin{lsucrequires}
    \lsucmandatory{\insertLicenseIntoBinary}\passingFilesCorrectly
    \lsucmandatory{\includeLicenseInDocumentation}
    \lsucoptional{\addLibraryLicenseToCopyrightMessage}
    \lsucoptional{\keepBinariesSeparate}
  \end{lsucrequires}

  \lsucprohibitsnothing
\end{lsuc}

\end{license}

%% =============================================================================
%% Discussion
%% =============================================================================

\subsection{Discussions and Explanations}
\label{BSD2Discussion}%
\label{BSD3Discussion}

The \textit{BSD 2-Clause license} has a simple structure: In the beginning, it
generally \enquote{(permits) redistribution and use in source and binary forms,
with or without modification, [\ldots]}, if one fulfills the two rules of the
license.\citeBSDsimple{} The first rule concerns the (re)distribution in the
form of source code, the second the (re)distribution of binary packages. Here are
some explanations why we translated the rules into different sets of executable
tasks:

\begin{itemize}
\item For the \enquote{redistribution of source code}, the license requires
  that the package must \enquote{ [\ldots] retain the above copyright notice,
  this list of conditions and the following disclaimer.}\citeBSDsimple{}
  Hence, you are not allowed to modify any of the copyright notes which are
  already embedded in the (source) files. And from a logical point of view,
  there must exist an explicit or implicit assertion that the software is
  licensed under the \textit{BSD 2-Clause license}%
  \footnote{The BSD license requires that a re-distributed software package must 
    contain the (package specific) copyright notice, the (license specific)
    conditions and the BSD disclaimer.\cite[cf.][\nopage wp]{BsdLicense2Clause} 
    You might ask, what you should do, if these elements are missing in the
    package you received. If so, the package you received had not been licensed
    adequately. Hence, you do not know reliably whether you have received it
    under a BSD license. In other words: If you have received a BSD licensed
    software package, it must contain sufficient license fulfilling elements, or
    it is not BSD licensed software.}. 
  This is often implemented by simply adding a copy of the license into the
  package. Hence, you are furthermore not allowed to modify these files or
  corresponding text snippets. For our purposes, we translated the bans into the
  following executable task: 

\begin{quote}\textit{\keepLicenseElements}\end{quote}

\item For the redistribution in the form of binary files, the license requires,
  that the licensing elements must be \enquote{[\ldots] (reproduced) in the
  documentation and/or other materials provided with the distribution.}%
  \citeBSDsimple{}
  Hence, this is not required as a necessary condition for the (re)distribution
  as source code package. But nevertheless, even for a distribution in the form
  of source code, it is often possible to fulfill this rule, too---e.g., if you
  offer your own download site for source code packages.  In such cases, it is a
  sign of respect to mention the licensing not only inside the packages, but
  also in the text of your site. Because of that, we added the following
  voluntary task for all BSD open source use cases which deal with the
  redistribution in the form of source code: 

\begin{quote}\textit{\includeLicenseInDocumentation}\end{quote}

\item Naturally, because the reproduction of the licensing elements \enquote{in
  the documentation and/or other materials provided with the distribution} is
  explicitly required for the \enquote{redistribution in binary form},%
  \citeBSDsimple{} 
  we had to rewrite the facultative task for a distribution in the form of
  source code as a mandatory task for all BSD open source use cases which deals
  with the redistribution in binary form.

\item In case of (re)distributing the program in the form of binary files, it is
  sometimes not enough, to pass the licensing elements as one has received them.
  If you compile the binary package from the source code, it is not necessarily
  true, that the licensing elements are also automatically generated and
  embedded into the `binary package.' But nevertheless, you have to add the
  copyright notice, the conditions and the disclaimer to this package for acting
  according to the BSD license. Therefore we chose the following form of an
  executable, license fulfilling task for all binary distributions:

\begin{quote}\textit{\insertLicenseIntoBinary}\end{quote}

\item Finally, we wished to insert a hint to the general (open source) tradition
  to mention the open source software used and their licenses as part of the
  `copyright widget' of an application. This is not required by the BSD
  license. But it is a general, good tradition. Naturally, because of the
  freedom to use and modify open source software and to redistribute a modified
  version of it, you are also allowed to insert such references, even if they
  are missing. Therefore we added a third voluntary task to honor this tradition
  for all relevant open source use cases.

\end{itemize}


%\bibliography{../../../bibfiles/oscResourcesEn}

% Local Variables:
% mode: latex
% fill-column: 80
% End:
}
{% Telekom osCompendium 'for being included' snippet template
%
% (c) Karsten Reincke, Deutsche Telekom AG, Darmstadt 2011
%
% This LaTeX-File is licensed under the Creative Commons Attribution-ShareAlike
% 3.0 Germany License (http://creativecommons.org/licenses/by-sa/3.0/de/): Feel
% free 'to share (to copy, distribute and transmit)' or 'to remix (to adapt)'
% it, if you '... distribute the resulting work under the same or similar
% license to this one' and if you respect how 'you must attribute the work in
% the manner specified by the author ...':
%
% In an internet based reuse please link the reused parts to www.telekom.com and
% mention the original authors and Deutsche Telekom AG in a suitable manner. In
% a paper-like reuse please insert a short hint to www.telekom.com and to the
% original authors and Deutsche Telekom AG into your preface. For normal
% quotations please use the scientific standard to cite.
%
% [ Framework derived from 'mind your Scholar Research Framework' 
%   mycsrf (c) K. Reincke 2012 CC BY 3.0  http://mycsrf.fodina.de/ ]
%


%% use all entries of the bibliography
%\nocite{*}

\section{CDDL licensed software [tbd]}

Also, [\ldots]

Thus, for
finding the relevant, simply processable task lists, also the following CDDL
specific open source use case structure\footnote{For details of the general OSUC
finder $\rightarrow$ \oslic, pp.\ \pageref{OsucTokens} and
\pageref{OsucDefinitionTree}} can be used:
 
\tikzstyle{nodv} = [font=\small, ellipse, draw, fill=gray!10, 
    text width=2cm, text centered, minimum height=2em]

\tikzstyle{nods} = [font=\footnotesize, rectangle, draw, fill=gray!20, 
    text width=1.2cm, text centered, rounded corners, minimum height=3em]

\tikzstyle{nodb} = [font=\footnotesize, rectangle, draw, fill=gray!20, 
    text width=2.2cm, text centered, rounded corners, minimum height=3em]
    
\tikzstyle{leaf} = [font=\tiny, rectangle, draw, fill=gray!30, 
    text width=1.2cm, text centered, minimum height=6em]

\tikzstyle{edge} = [draw, -latex']

\begin{tikzpicture}[]

\node[nodv] (l71) at (4,10) {CDDL};

\node[nodb] (l61) at (0,8.6) {\textit{recipient:} \\ \textbf{4yourself}};
\node[nodb] (l62) at (6.5,8.6) {\textit{recipient:} \\ \textbf{2others}};

\node[nodb] (l51) at (2.5,7) {\textit{state:} \\ \textbf{unmodified}};
\node[nodb] (l52) at (9.3,7) {\textit{state:} \\ \textbf{modified}};

\node[nods] (l41) at (1.8,5.4) {\textit{form:} \textbf{source}};
\node[nods] (l42) at (3.6,5.4) {\textit{form:} \textbf{binary}};
\node[nodb] (l43) at (6.5,5.4) {\textit{type:} \\ \textbf{proapse}};
\node[nodb] (l44) at (12,5.4) {\textit{type:} \\ \textbf{snimoli}};


\node[nods] (l31) at (5.4,3.8) {\textit{form:} \textbf{source}};
\node[nods] (l32) at (7.2,3.8) {\textit{form:} \textbf{binary}};
\node[nodb] (l33) at (10,3.8) {\textit{context:} \\ \textbf{independent}};
\node[nodb] (l34) at (13.5,3.8) {\textit{context:} \\ \textbf{embedded}};

\node[nods] (l21) at (9,2.2) {\textit{form:} \textbf{source}};
\node[nods] (l22) at (10.8,2.2) {\textit{form:} \textbf{binary}};
\node[nods] (l23) at (12.6,2.2) {\textit{form:} \textbf{source}};
\node[nods] (l24) at (14.4,2.2) {\textit{form:} \textbf{binary}};

\node[leaf] (l11) at (0,0) {\textbf{CDDL-1} \textit{using software only
for yourself}};

\node[leaf] (l12) at (1.8,0) { \textbf{CDDL-2} \textit{ distributing unmodified
software as sources}};

\node[leaf] (l13) at (3.6,0) { \textbf{CDDL-3}  \textit{ distributing unmodified
software as binaries}};

\node[leaf] (l14) at (5.4,0) { \textbf{CDDL-4}  \textit{ distributing modified
program as sources}};

\node[leaf] (l15) at (7.2,0) { \textbf{CDDL-5}  \textit{ distributing modified
program as binaries}};

\node[leaf] (l16) at (9,0) { \textbf{CDDL-6}  \textit{ distributing modified
library as independent sources}};

\node[leaf] (l17) at (10.8,0) { \textbf{CDDL-7} \textit{distributing modified
library as independent binaries}};

\node[leaf] (l18) at (12.6,0) { \textbf{CDDL-8}  \textit{distributing
modified library as embedded sources}};

\node[leaf] (l19) at (14.4,0) { \textbf{CDDL-9}  \textit{ distributing modified
library as embedded binaries}};


\path [edge] (l71) -- (l61);
\path [edge] (l71) -- (l62);
\path [edge] (l61) -- (l11);
\path [edge] (l62) -- (l51);
\path [edge] (l62) -- (l52);
\path [edge] (l51) -- (l41);
\path [edge] (l51) -- (l42);
\path [edge] (l52) -- (l43);
\path [edge] (l52) -- (l44);
\path [edge] (l41) -- (l12);
\path [edge] (l42) -- (l13);
\path [edge] (l43) -- (l31);
\path [edge] (l43) -- (l32);
\path [edge] (l44) -- (l33);
\path [edge] (l44) -- (l34);
\path [edge] (l31) -- (l14);
\path [edge] (l32) -- (l15);
\path [edge] (l33) -- (l21);
\path [edge] (l33) -- (l22);
\path [edge] (l34) -- (l23);
\path [edge] (l34) -- (l24);
\path [edge] (l21) -- (l16);
\path [edge] (l22) -- (l17);
\path [edge] (l23) -- (l18);
\path [edge] (l24) -- (l19);

\end{tikzpicture}


\subsection{CDDL-1: Using the software only for yourself}
\label{OSUC-01-CDDL} \label{OSUC-03L-CDDL} \label{OSUC-03N-CDDL}
\label{OSUC-06L-CDDL} \label{OSUC-06N-CDDL} \label{OSUC-09L-CDDL}
\label{OSUC-09N-CDDL}


\begin{description}

\item[means] that you are going to use a received CDDL licensed software only
for yourself and that you do not hand it over to any 3rd party in any sense.

\item[covers] OSUC-01, OSUC-03, OSUC-06, and OSUC-09\footnote{For details 
$\rightarrow$ \oslic, pp.\ \pageref{OSUC-01-DEF} - \pageref{OSUC-09-DEF}}

\item[requires] \ldots

\item[prohibits] \ldots


\end{description}

\subsection{CDDL-2: Passing the unmodified software as source code}
\label{OSUC-02S-CDDL} \label{OSUC-05S-CDDL} \label{OSUC-07S-CDDL} 

\begin{description}

\item[means] that you are going to distribute an unmodified version of the
received CDDL software to 3rd parties - in the form of source code files or as a
source code package. In this case it is not discriminating to distribute a
program, an application, a server, a snippet, a module, a library, or a plugin
as an independent or an embedded unit

\item[covers] OSUC-02S, OSUC-05S, OSUC-07S\footnote{For details $\rightarrow$
\oslic, pp.\ \pageref{OSUC-02S-DEF} - \pageref{OSUC-07S-DEF}}

\item[requires] the following tasks in order to fulfill the license conditions:
\begin{itemize}
  
  \item \ldots
  
\end{itemize}

\item[prohibits] \ldots
\begin{itemize}
  \item \ldots
\end{itemize}
\end{description}


\subsection{CDDL-3: Passing the unmodified software as binaries} 
\label{OSUC-02B-CDDL} \label{OSUC-05B-CDDL} \label{OSUC-07B-CDDL}

\begin{description}
\item[means] that you are going to distribute an unmodified version of the
received CDDL software to 3rd parties -- in the form of binary files or as a
bi\-na\-ry package. In this case it is not discriminating to distribute a
program, an application, a server, a snippet, a module, a library, or a plugin
as an independent or an embedded unit.

\item[covers] OSUC-02B, OSUC-05B, OSUC-07B\footnote{For details $\rightarrow$
\oslic, pp.\ \pageref{OSUC-02B-DEF} - \pageref{OSUC-07B-DEF}}

\item[requires] the following tasks in order to fulfill the license conditions:
\begin{itemize}
  
  \item \ldots
  
\end{itemize}

\item[prohibits] \ldots
\begin{itemize}
  \item \ldots
\end{itemize}
\end{description}


\subsection{CDDL-4: Passing a modified program as source code}
\label{OSUC-04S-CDDL} 

\begin{description}
\item[means] that you are going to distribute a modified version of the received
CDDL licensed program, application, or server (proapse) to 3rd parties -- in the
form of source code files or a source code package.
\item[covers] OSUC-04S\footnote{For details $\rightarrow$ \oslic, pp.\
\pageref{OSUC-04S-DEF}}
\item[requires] the following tasks in order to fulfill the license conditions:
\begin{itemize}
  
  \item \ldots
  
\end{itemize}

\item[prohibits] \ldots
\begin{itemize}
  \item \ldots
\end{itemize}
\end{description}

\subsection{CDDL-5: Passing a modified program as binary}
\label{OSUC-04B-CDDL} 

\begin{description}
\item[means] that you are going to distribute a modified version of the received
CDDL licensed pro\-gram, application, or server (proapse) to 3rd parties -- in
the form of binary files or as a binary package.
\item[covers] OSUC-04B\footnote{For details $\rightarrow$ \oslic, pp.\
\pageref{OSUC-04B-DEF}}
\item[requires] the following tasks in order to fulfill the license conditions:
\begin{itemize}
  
  \item \ldots
  
\end{itemize}

\item[prohibits] \ldots
\begin{itemize}
  \item \ldots
\end{itemize}
\end{description}

\subsection{CDDL-6: Passing a modified library as independent source code}
\label{OSUC-08S-CDDL}

\begin{description}
\item[means] that you are going to distribute a modified version of the received
CDDL licensed code snippet, module, library, or plugin (snimoli) to 3rd parties
-- in the form of source code files or as a source code package, but without
embedding it into another larger software unit.
\item[covers] OSUC-08S\footnote{For details $\rightarrow$ \oslic, pp.\
\pageref{OSUC-08S-DEF}}
\item[requires] the following tasks in order to fulfill the license conditions:
\begin{itemize}
  
  \item \ldots
  
\end{itemize}

\item[prohibits] \ldots
\begin{itemize}
  \item \ldots
\end{itemize}
\end{description}

\subsection{CDDL-7: Passing a modified library as independent binary}
\label{OSUC-08B-CDDL}

\begin{description}
\item[means] that you are going to distribute a modified version of the received
CDDL licensed code snippet, module, library, or plugin (snimoli) to 3rd parties
-- in the form of binary files or as a binary package but without embedding it
into another larger software unit.
\item[covers] OSUC-08B\footnote{For details $\rightarrow$ \oslic, pp.\
\pageref{OSUC-08B-DEF}}
\item[requires] the following tasks in order to fulfill the license conditions:
\begin{itemize}
  
  \item \ldots
  
\end{itemize}

\item[prohibits] \ldots
\begin{itemize}
  \item \ldots
\end{itemize}
\end{description}

\subsection{CDDL-8: Passing a modified library as embedded source code}
\label{OSUC-10S-CDDL}

\begin{description}
\item[means] that you are going to distribute a modified version of the received
CDDL licensed code snippet, module, library, or plugin (snimoli) to 3rd parties
-- in the form of source code files or as a source code package together with
another larger software unit which contains this code snippet, module, library,
or plugin as an embedded component.
\item[covers] OSUC-10S\footnote{For details $\rightarrow$ \oslic, pp.\
\pageref{OSUC-10S-DEF}}
\item[requires] the following tasks in order to fulfill the license conditions:
\begin{itemize}
  
  \item \ldots
  
\end{itemize}

\item[prohibits] \ldots
\begin{itemize}
  \item \ldots
\end{itemize}
\end{description}

\subsection{CDDL-9: Passing a modified library as embedded binary}
\label{OSUC-10B-CDDL}

\begin{description}
\item[means] that you are going to distribute a modified version of the received
CDDL licensed code snippet, module, library, or plugin to 3rd parties -- in the
form of binary files or as a binary package together with another larger
software unit which contains this code snippet, module, library, or plugin as an
embedded component.
\item[covers] OSUC-10B\footnote{For details $\rightarrow$ \oslic, pp.\
\pageref{OSUC-10B-DEF}}
\item[requires] the following tasks in order to fulfill the license conditions:
\begin{itemize}
  
  \item \ldots
  
\end{itemize}

\item[prohibits] \ldots
\begin{itemize}
  \item \ldots
\end{itemize}
\end{description}

\subsection{Discussions and Explanations}
\label{CDDLDiscussion}

The CDDL offers \ldots which contains nearly all
requirements\footcite[cf.][\nopage wp.\ §3]{Cddl10OsiLicense2004a}. Only for some

\begin{itemize}

  \item 

\end{itemize}


%\bibliography{../../../bibfiles/oscResourcesEn}

% Local Variables:
% mode: latex
% fill-column: 80
% End:
}
{% Telekom osCompendium 'for being included' snippet template
%
% (c) Karsten Reincke, Deutsche Telekom AG, Darmstadt 2011
%
% This LaTeX-File is licensed under the Creative Commons Attribution-ShareAlike
% 3.0 Germany License (http://creativecommons.org/licenses/by-sa/3.0/de/): Feel
% free 'to share (to copy, distribute and transmit)' or 'to remix (to adapt)'
% it, if you '... distribute the resulting work under the same or similar
% license to this one' and if you respect how 'you must attribute the work in
% the manner specified by the author ...':
%
% In an internet based reuse please link the reused parts to www.telekom.com and
% mention the original authors and Deutsche Telekom AG in a suitable manner. In
% a paper-like reuse please insert a short hint to www.telekom.com and to the
% original authors and Deutsche Telekom AG into your preface. For normal
% quotations please use the scientific standard to cite.
%
% [ Framework derived from 'mind your Scholar Research Framework' 
%   mycsrf (c) K. Reincke 2012 CC BY 3.0  http://mycsrf.fodina.de/ ]
%


%% use all entries of the bibliography
%\nocite{*}

\section{EPL-1.0 licensed software}
\begin{license}{EPL}
\licensename{EPL-1.0}
\licensespec{Eclipse Public License 1.0}
\licenseabbrev{EPL}

The Eclipse Public License clearly distinguishes the distribution in the form of
source code from that in the form of binaries: First, it allows to
\enquote{distribute} Eclipse licensed programs \enquote{in source code and in
object code}.\citeEPL{§3} Then it specifies under which conditions one may
distribute the program as a set of binaries.\citeEPL{§3 top area} One of
these conditions is---roughly speaking---that the distributor makes the sources
available too.\citeEPL{§3 mid area} More precisely, the EPL-1.0 has to be taken as a
license with weak copyleft ($\rightarrow$ \oslic, p.\ \protectionpageref{EPL}).
The other conditions refer to the distribution in general---no matter what form
or state is used.\citeEPL{§3 bottom area} So, taken as whole, the EPL-1.0 mainly
focusses on the distribution of software.  Thus, for finding the relevant, easy
to process task lists, the following EPL-1.0 specific open source use case 
structure%
  \footnote{For details of the general OSUC finder $\rightarrow$ \oslic,
    pp.\ \pageref{OsucTokens} and \pageref{OsucDefinitionTree}}
can be used:
 
\tikzstyle{nodv} = [font=\small, ellipse, draw, fill=gray!10, 
    text width=2cm, text centered, minimum height=2em]

\tikzstyle{nods} = [font=\footnotesize, rectangle, draw, fill=gray!20, 
    text width=1.2cm, text centered, rounded corners, minimum height=3em]

\tikzstyle{nodb} = [font=\footnotesize, rectangle, draw, fill=gray!20, 
    text width=2.2cm, text centered, rounded corners, minimum height=3em]
    
\tikzstyle{leaf} = [font=\tiny, rectangle, draw, fill=gray!30, 
    text width=1.2cm, text centered, minimum height=6em]

\tikzstyle{edge} = [draw, -latex']

\begin{tikzpicture}[]

\node[nodv] (l71) at (4,10) {EPL-1.0};

\node[nodb] (l61) at (0,8.6) {\textit{recipient:} \\ \textbf{4yourself}};
\node[nodb] (l62) at (6.5,8.6) {\textit{recipient:} \\ \textbf{2others}};

\node[nodb] (l51) at (2.5,7) {\textit{state:} \\ \textbf{unmodified}};
\node[nodb] (l52) at (9.3,7) {\textit{state:} \\ \textbf{modified}};

\node[nods] (l41) at (1.8,5.4) {\textit{form:} \textbf{source}};
\node[nods] (l42) at (3.6,5.4) {\textit{form:} \textbf{binary}};
\node[nodb] (l43) at (6.5,5.4) {\textit{type:} \\ \textbf{proapse}};
\node[nodb] (l44) at (12,5.4) {\textit{type:} \\ \textbf{snimoli}};


\node[nods] (l31) at (5.4,3.8) {\textit{form:} \textbf{source}};
\node[nods] (l32) at (7.2,3.8) {\textit{form:} \textbf{binary}};
\node[nodb] (l33) at (10,3.8) {\textit{context:} \\ \textbf{independent}};
\node[nodb] (l34) at (13.5,3.8) {\textit{context:} \\ \textbf{embedded}};

\node[nods] (l21) at (9,2.2) {\textit{form:} \textbf{source}};
\node[nods] (l22) at (10.8,2.2) {\textit{form:} \textbf{binary}};
\node[nods] (l23) at (12.6,2.2) {\textit{form:} \textbf{source}};
\node[nods] (l24) at (14.4,2.2) {\textit{form:} \textbf{binary}};

\node[leaf] (l11) at (0,0) {\textbf{EPL-1.0-C1} \textit{using software only
for yourself}};

\node[leaf] (l12) at (1.8,0) { \textbf{EPL-1.0-C2} \textit{ distributing unmodified
software as sources}};

\node[leaf] (l13) at (3.6,0) { \textbf{EPL-1.0-C3}  \textit{ distributing unmodified
software as binaries}};

\node[leaf] (l14) at (5.4,0) { \textbf{EPL-1.0-C4}  \textit{ distributing modified
program as sources}};

\node[leaf] (l15) at (7.2,0) { \textbf{EPL-1.0-C5}  \textit{ distributing modified
program as binaries}};

\node[leaf] (l16) at (9,0) { \textbf{EPL-1.0-C6}  \textit{ distributing modified
library as independent sources}};

\node[leaf] (l17) at (10.8,0) { \textbf{EPL-1.0-C7} \textit{distributing modified
library as independent binaries}};

\node[leaf] (l18) at (12.6,0) { \textbf{EPL-1.0-C8}  \textit{distributing
modified library as embedded sources}};

\node[leaf] (l19) at (14.4,0) { \textbf{EPL-1.0-C9}  \textit{ distributing modified
library as embedded binaries}};


\path [edge] (l71) -- (l61);
\path [edge] (l71) -- (l62);
\path [edge] (l61) -- (l11);
\path [edge] (l62) -- (l51);
\path [edge] (l62) -- (l52);
\path [edge] (l51) -- (l41);
\path [edge] (l51) -- (l42);
\path [edge] (l52) -- (l43);
\path [edge] (l52) -- (l44);
\path [edge] (l41) -- (l12);
\path [edge] (l42) -- (l13);
\path [edge] (l43) -- (l31);
\path [edge] (l43) -- (l32);
\path [edge] (l44) -- (l33);
\path [edge] (l44) -- (l34);
\path [edge] (l31) -- (l14);
\path [edge] (l32) -- (l15);
\path [edge] (l33) -- (l21);
\path [edge] (l33) -- (l22);
\path [edge] (l34) -- (l23);
\path [edge] (l34) -- (l24);
\path [edge] (l21) -- (l16);
\path [edge] (l22) -- (l17);
\path [edge] (l23) -- (l18);
\path [edge] (l24) -- (l19);

\end{tikzpicture}

%%
%% Common Building Blocks
%%

% ------------------------------------------------------------------------------
% Forbid to change copyright notices
\newcommand{\dontChangeCopyrightNotices}{to remove or to alter any copyright
  notices that were contained in the software package when you received it.} 
  
% ------------------------------------------------------------------------------
% Forbid patent litigation
\newcommand{\noPatentLitigation}{%
  to institute a patent litigation against anyone alleging that the software
  constitutes patent infringement.}
  

% ------------------------------------------------------------------------------
% Require to keep licensing elements
\newcommand{\keepLicensingElements}{Ensure that the licensing elements
  (particularly all copyright notices and the disclaimer of warranty and
  disclaimer of liability) are retained in your package in exactly the form you
  have received them.}
\newcommand{\addWhenCompiling}{If you compile the binary from the sources,
  ensure that all these licensing elements are also incorporated into the
  package.}

% ------------------------------------------------------------------------------
% Give recipient a copy of the license
\newcommand{\giveLicenseFile}{Give the recipient a copy of the EPL-1.0 license.
  If it is not already part of the software package, add it. If the licensing
  statement in the licensing file of the package does still not clearly state
  that the package is licensed under the EPL-1.0, additionally insert your own
  correct EPL-1.0 licensing file.}

% ------------------------------------------------------------------------------
% No warranty
\newcommand{\noWarranty}[1]{If still not existing, integrate an explicit, very 
  prominently placed `No warranty' statement into the distributed #1 package.
  Let this statement clearly say that all (other) contributors to the software
  do not accept any responsibility for the quality of the software. Then, copy
  the no-warranty clause and the disclaimer of liability from the EPL-1.0 itself
  into that file.} 

\newcommand{\includeInCopyrightScreen}{Let the copyright screen of your own
  overarching program show the same information as a specification for the
  embedded component.}

\newcommand{\updateCopyrightScreen}{Update an existing copyright screen
  presented by the program so that it shows the same information.}

% ------------------------------------------------------------------------------
% Add license info to documentation
\newcommand{\auxAddToDoc}[1]{%
  Let the documentation of your distribution or your additional material
  reproduce the content of an existing \emph{copyright notice text files}, a
  hint to the software name, a link to its homepage, and a link to the EPL-1.0
  license#1.}   

\newcommand{\addToDocumentation}{\auxAddToDoc{}}
\newcommand{\addToYourCopyrightNotice}{%
  \auxAddToDoc{, preferably as a subsection of your own copyright notice}}

% ------------------------------------------------------------------------------
% Publish source code
\newcommand{\auxPublishSourceCode}[2]{Make the source code of #1 accessible
  through a repository under your own control#2: Push the source code package
  into an internet repository and enable the download function. Ensure that this
  respository is available for a reasonable period of time.}

\newcommand{\publishUnmodifiedSourceCode}[1]{%
  \auxPublishSourceCode{#1}{, even if you did not modify it}}
\newcommand{\publishSourceCode}[1]{%
  \auxPublishSourceCode{#1}{}}

% ------------------------------------------------------------------------------
% Include link to source repository
\newcommand{\linkToRepo}{Insert a prominent hint to the download repository
  into your distribution or your additional material and explain how the code
  can be obtained.}
 
% ------------------------------------------------------------------------------
% Create a modification text file...
\newcommand{\describeModifications}{Create a \emph{modification text file} if
  such a file does not exist. \emph{Add} a general description of your
  modifications to the \emph{modification text file.} Incorporate it into your
  distribution package.} 

% ------------------------------------------------------------------------------
% mark all modifications in the source
\newcommand{\markAllModifications}{Mark all modifications of the source code of
  the program thoroughly; namely within the modified source code.}

% ------------------------------------------------------------------------------
% Organize your sources
\newcommand{\organizeYourModifications}{Organize your modifications in a way
  that they are covered by the existing EPL-1.0 licensing statements.}
\newcommand{\addHeaderToNewFiles}{If you add new source code files, insert a
  header containing your copyright line and an EPL-1.0 adequate licensing the
  statement.} 

% ------------------------------------------------------------------------------
% Use separate directories
\newcommand{\useSeparateDirectory}[1]{Arrange your #1 distribution so that the
  integrated EPL-1.0 and the \emph{licensing files} clearly refer only to the
  embedded library and do not disturb the licensing of your own overarching
  work. It's a good tradition to keep the embedded components like libraries,
  modules, snippets, or plugins in separate directories which also contains all 
  additional licensing elements.}

% ------------------------------------------------------------------------------
% EPL-C1
% ------------------------------------------------------------------------------
\subsection{EPL-1.0-C1: Using the software only for yourself}
\begin{lsuc}{EPL-1.0-C1}
  \linkosuc{01}
  \linkosuc{03L} 
  \linkosuc{03N} 
  \linkosuc{06L}
  \linkosuc{06N}
  \linkosuc{09L}
  \linkosuc{09N}

  \lsucmeans{that you received EPL-1.0 licensed software, that you will use it
  only for yourself, and that you do not hand it over to any 3rd party in any
  sense.} 

  \coversOsucs{OSUC-01, OSUC-03L, OSUC-03N, OSUC-06L, OSUC-06N, OSUC-09L, and
  OSUC-09N}{01}{09N}

  \begin{lsucrequiresnothing}
    \lsucitem{You are allowed to use any kind of EPL-1.0 software in any sense
      and in any context without being obliged to do anything as long as you do
      not give the software to third parties.}
  \end{lsucrequiresnothing}

  \begin{lsucprohibits}
    \lsucitem{\dontChangeCopyrightNotices}
    \lsucitem{\noPatentLitigation}
  \end{lsucprohibits}

\end{lsuc}

% ------------------------------------------------------------------------------
% EPL-C2
% ------------------------------------------------------------------------------
\subsection{EPL-1.0-C2: Passing the unmodified software as source code}
\begin{lsuc}{EPL-1.0-C2}
  \linkosuc{02S} 
  \linkosuc{05S}
  \linkosuc{07S} 

  \lsucmeans{that you received EPL-1.0 licensed software which you are now
  going to distribute to third parties in the form of unmodified source code
  files or as unmodified source code package. In this case it makes no
  difference if you distribute a program, an application, a server, a snippet, a
  module, a library, or a plugin as an independent or as an embedded unit.} 

  \coversOsucs{OSUC-02S, OSUC-05S, OSUC-07S}{02S}{07S}

  \begin{lsucrequires}
    \lsucmandatory{\keepLicensingElements}
    \lsucmandatory{\giveLicenseFile}\passingFilesCorrectly
    \lsucmandatory{\noWarranty{source code}}  
    \lsucoptional{\addToDocumentation}
  \end{lsucrequires}

  \begin{lsucprohibits}
    \lsucitem{\dontChangeCopyrightNotices}
    \lsucitem{\noPatentLitigation}
  \end{lsucprohibits}

\end{lsuc}

% ------------------------------------------------------------------------------
% EPL-C3
% ------------------------------------------------------------------------------
\subsection{EPL-1.0-C3: Passing the unmodified software as binaries} 
\begin{lsuc}{EPL-1.0-C3}
  \linkosuc{02B} 
  \linkosuc{05B} 
  \linkosuc{07B} 

  \lsucmeans{that you received EPL-1.0 licensed software which you are now
  going to distribute to third parties in the form of unmodified binary files or
  as unmodified binary package. In this case it does not matter if you distribute
  a program, an application, a server, a snippet, a module, a library, or a
  plugin as an independent or an embedded unit.}

  \coversOsucs{OSUC-02B, OSUC-05B, OSUC-07B}{02B}{07B}

  \begin{lsucrequires}
  
    \lsucmandatory{\keepLicensingElements\ \addWhenCompiling}
    \lsucmandatory{\noWarranty{binary}}
    \lsucmandatory{\publishUnmodifiedSourceCode{the software}}
    \lsucmandatory{\linkToRepo}
    \lsucsourcedist{EPL-1.0-C2}
    \lsucoptional{\addToDocumentation}
  \end{lsucrequires}

  \begin{lsucprohibits}
    \lsucitem{\dontChangeCopyrightNotices}
    \lsucitem{\noPatentLitigation}
  \end{lsucprohibits}

\end{lsuc}

% ------------------------------------------------------------------------------
% EPL-C4
% ------------------------------------------------------------------------------
\subsection{EPL-1.0-C4: Passing a modified program as source code}
\begin{lsuc}{EPL-1.0-C4}
  \linkosuc{04S} 

  \lsucmeans{that you received an EPL-1.0 licensed program, application, or
  server (proapse), that you modified it, and that you are now going to
  distribute this modified version to third parties in the form of source code files or as
  a source code package.}

  \mapsToOsuc{04S}

  \begin{lsucrequires}
    \lsucmandatory{\keepLicensingElements}
    \lsucmandatory{\describeModifications}
    \lsucmandatory{\markAllModifications}
    \lsucmandatory{\giveLicenseFile}\passingFilesCorrectly
    \lsucmandatory{\organizeYourModifications\ \addHeaderToNewFiles}
    \lsucmandatory{\noWarranty{source code} \updateCopyrightScreen}
    \lsucoptional{\addToDocumentation}
  \end{lsucrequires}
 
  \begin{lsucprohibits}
    \lsucitem{\dontChangeCopyrightNotices}
    \lsucitem{\noPatentLitigation}
  \end{lsucprohibits}

\end{lsuc}

% ------------------------------------------------------------------------------
% EPL-C5
% ------------------------------------------------------------------------------
\subsection{EPL-1.0-C5: Passing a modified program as binary}
\begin{lsuc}{EPL-1.0-C5}
  \linkosuc{04B}

  \lsucmeans{that you received an EPL-1.0 licensed program, application, or
  server (proapse), that you modified it, and that you are now going to
  distribute this modified version to third parties in the form of binary files or as a
  binary package.}

  \mapsToOsuc{04B}

  \begin{lsucrequires}
    \lsucmandatory{\keepLicensingElements\ \addWhenCompiling}
    \lsucmandatory{\describeModifications}
    \lsucmandatory{\markAllModifications}
    \lsucmandatory{\organizeYourModifications}
    \lsucmandatory{\noWarranty{binary} \updateCopyrightScreen}
    \lsucmandatory{\publishSourceCode{the program}}
    \lsucmandatory{\linkToRepo}
    \lsucsourcedist{EPL-1.0-C4}
    \lsucoptional{\addToYourCopyrightNotice}
  \end{lsucrequires}

  \begin{lsucprohibits}
    \lsucitem{\dontChangeCopyrightNotices}
    \lsucitem{\noPatentLitigation}
  \end{lsucprohibits}

\end{lsuc}

% ------------------------------------------------------------------------------
% EPL-C6
% ------------------------------------------------------------------------------
\subsection{EPL-1.0-C6: Passing a modified library as independent source code}
\begin{lsuc}{EPL-1.0-C6}
  \linkosuc{08S}

  \lsucmeans{that you received an EPL-1.0 licensed code snippet, module, library,
  or plugin (snimoli), that you modified it, and that you are now going to
  distribute this modified version to third parties in the form of source code
  files or as a source code package, but without embedding it into another
  larger software unit.}

  \mapsToOsuc{08S}

  \begin{lsucrequires}
    \lsucmandatory{\keepLicensingElements}
    \lsucmandatory{\describeModifications}
    \lsucmandatory{\markAllModifications}
    \lsucmandatory{\giveLicenseFile}\passingFilesCorrectly
    \lsucmandatory{\organizeYourModifications\ \addHeaderToNewFiles}
    \lsucmandatory{\noWarranty{source code}}
    \lsucoptional{\addToDocumentation}
  \end{lsucrequires}

  \begin{lsucprohibits}
    \lsucitem{\dontChangeCopyrightNotices}
    \lsucitem{\noPatentLitigation}
  \end{lsucprohibits}

\end{lsuc}

% ------------------------------------------------------------------------------
% EPL-C7
% ------------------------------------------------------------------------------
\subsection{EPL-1.0-C7: Passing a modified library as independent binary}
\begin{lsuc}{EPL-1.0-C7}
  \linkosuc{08B}

  \lsucmeans{that you received an EPL-1.0 licensed code snippet, module, library,
  or plugin (snimoli), that you modified it, and that you are now going to
  distribute this modified version to third parties in the form of binary files
  or as a binary package but without embedding it into another larger software
  unit.}

  \mapsToOsuc{08B}

  \begin{lsucrequires}
    \lsucmandatory{\keepLicensingElements\ \addWhenCompiling}
    \lsucmandatory{\describeModifications}
    \lsucmandatory{\markAllModifications}
    \lsucmandatory{\organizeYourModifications}
    \lsucmandatory{\noWarranty{binary}}
    \lsucmandatory{\publishSourceCode{the modified library}}
    \lsucmandatory{\linkToRepo}
    \lsucsourcedist{EPL-1.0-C6}
    \lsucoptional{\addToDocumentation}
  \end{lsucrequires}

  \begin{lsucprohibits}
    \lsucitem{\dontChangeCopyrightNotices}
    \lsucitem{\noPatentLitigation}
  \end{lsucprohibits}

\end{lsuc}

% ------------------------------------------------------------------------------
% EPL-C8
% ------------------------------------------------------------------------------
\subsection{EPL-1.0-C8: Passing a modified library as embedded source code}
\begin{lsuc}{EPL-1.0-C8}
  \linkosuc{10S}

  \lsucmeans{that you received an EPL-1.0 licensed code snippet, module, library,
  or plugin (snimoli), that you modified it, and that you are now going to
  distribute this modified version to third parties in the form of source code
  files or as a source code package together with another larger software unit
  which contains this code snippet, module, library, or plugin as an embedded
  component.}

  \mapsToOsuc{10S}

  \begin{lsucrequires}
    \lsucmandatory{\keepLicensingElements}
    \lsucmandatory{\describeModifications}
    \lsucmandatory{\markAllModifications}
    \lsucmandatory{\giveLicenseFile}\passingFilesCorrectly
    \lsucmandatory{\noWarranty{source code} \includeInCopyrightScreen}
    \lsucmandatory{\organizeYourModifications\ \addHeaderToNewFiles}
    \lsucoptional{\useSeparateDirectory{source code}}
    \lsucoptional{\addToYourCopyrightNotice}
  \end{lsucrequires}

  \begin{lsucprohibits}
    \lsucitem{\dontChangeCopyrightNotices}
    \lsucitem{\noPatentLitigation}
  \end{lsucprohibits}

\end{lsuc}

% ------------------------------------------------------------------------------
% EPL-C9
% ------------------------------------------------------------------------------
\subsection{EPL-1.0-C9: Passing a modified library as embedded binary}
\begin{lsuc}{EPL-1.0-C9}
  \linkosuc{10B}

  \lsucmeans{that you received an EPL-1.0 licensed code snippet, module, library,
  or plugin (snimoli), that you modified it, and that you are now going to
  distribute this modified version to third parties in the form of binary files
  or as a binary package together with another larger software unit which
  contains this code snippet, module, library, or plugin as an embedded component.}

  \mapsToOsuc{10B}

  \begin{lsucrequires}
    \lsucmandatory{\keepLicensingElements\ \addWhenCompiling}
    \lsucmandatory{\describeModifications}
    \lsucmandatory{\markAllModifications}
    \lsucmandatory{\noWarranty{binary} \includeInCopyrightScreen}
    \lsucmandatory{\publishSourceCode{the embedded library}}
    \lsucmandatory{\linkToRepo}
    \lsucmandatory{\organizeYourModifications}
    \lsucsourcedist{EPL-1.0-C8}
    \lsucoptional{\useSeparateDirectory{binary}}
    \lsucoptional{\addToYourCopyrightNotice}
  \end{lsucrequires}

  \begin{lsucprohibits}
    \lsucitem{\dontChangeCopyrightNotices}
    \lsucitem{\noPatentLitigation}
  \end{lsucprohibits}

\end{lsuc}

% ------------------------------------------------------------------------------
\subsection{Discussions and Explanations}
\label{EPLDiscussion}

The EPL-1.0 contains a succinct section \enquote{Requirements}\citeEPL{§3}
complemented by some definitions concerning a \enquote{Commercial
Distribution}\citeEPL{§4}: First, it describes what a distributor must do for
correctly distributing an Eclipse licensed program as a set of binaries. Then,
it explains, what must be done to comply with the license when distributing the
software as source code.  Finally, it lists two conditions which must be
fulfilled in any case.\citeEPL{§3}  
With respect to this structure, we can discover the following tasks:

\begin{itemize}

  \item The EPL-1.0 generally requires that \enquote{Contributors may not remove or
    alter any copyright notices contained within the Program}\citeEPL{§3} where
    the word `Contributor' has to be read as \enquote{any person or entity that
    distributes the Program}, and the word `Program' denotes the
    \enquote{initial contribution} and all its modifications.\citeEPL{§1} 
    Similar to the EUPL and at least in a very strict reading, the EPL-1.0 does not
    limit these requirements to the distribution of the software
    (\emph{2others}). But in practice it will be difficult to control the
    compliant use of the software in those cases where one uses the software
    only for oneself. But opposite to, for example, the EUPL, the EPL-1.0 clearly
    contains this interdiction. The \oslic{} solves this practical inconsistence
    duplicating the message: On the one hand, it rewrites the negative condition
    as a mandatory positive assertion for the \emph{2others} use cases (EPL-1.0-C2 --
    EPL-1.0-C9). This should emphasize the \emph{activity} to retain the copyright
    notes in exact the form one has received them. On the other hand, the \oslic{}
    inserts the corresponding interdiction into the `prohibits' section of the
    \emph{4yourself} use cases (EPL-1.0-C1 -- EPL-1.0-C9).
  
  \item Furthermore, the EPL-1.0 requires that \enquote{each Contributor must
    identify itself as the originator of its Contributions [\ldots] in a manner
    that reasonably allows subsequent Recipients to identify the originator of
    the Contribution},\citeEPL{§3} In this case, `Contribution' has to be read
    as the \enquote{initial code and documention} together with all subsequent
    modifications of these parts.\citeEPL{§1} To fulfill this condition
    faithfully, a developer must mark and describe his modifications of the
    source code within this source code; and the distributor must describe these
    modifications on the more general level of software features in a file
    sometimes called CHANGES. At a first glance, the requirement to document the
    source code modifications within the source code seems to be restricted to
    the use cases which concern the distribution of a modified EPL-1.0 software in
    the form of source code. But the EPL-1.0 allows the distribution in the form of
    binaries only if the distributor also states where one can obtain the
    correspoding code.\citeEPL{§3} So, distributing the binaries implies the
    distribution of the source code.  Therefore the \oslic{} inserts the two
    requirements as mandatory clauses into all the use cases concerning the
    distribution of a modified EPL-1.0 software (EPL-1.0-C4 -- EPL-1.0-C9).
  
  \item For all distributions in the form of source code the EPL-1.0 requires that
    the software \enquote{[\ldots] must be made available under this (Eclipse
    Public License 1.0) Agreement} and that \enquote{[\ldots] a copy of this
    Agreement must be included with each copy of the Program.}\citeEPL{§3} 
    Thus, the \oslic{} inserts a respective mandatory clause into the use cases
    (EPL-1.0-C4, EPL-1.0-C6, EPL-1.0-C8). But the EPL-1.0 is a license with a weak copyleft%
    \footnote{($\rightarrow$ \oslic, p.\ \protectionpageref{EPL})}. 
    Therefore, this conditions does not cover the overarching program which uses
    the embedded library (EPL-1.0-C8).
    
  \item Additionally, the EPL-1.0 allows to distribute the software in the form
    of binaries if the distributor \enquote{[\ldots] effectively disclaims on
    behalf of all Contributors all warranties and conditions [\ldots] (and)
    effectively excludes on behalf of all Contributors all liability for
    damages [\ldots]} in the broadest sense.\citeEPL{§3} This limitation of
    liability is very important to the EPL-1.0. Thus, it further specifies and
    explains this aspect once more in another section titled \enquote{Commercial
    Distribution}. There, this aspect is no longer focussed only on a
    distribution in the form of binaries.\citeEPL{§4} So the \oslic{} inserts a
    mandatory clause into all use cases concerning the distribution that the
    paragraph of \enquote{No Warranty}\citeEPL{§5} and the \enquote{Disclaimer
    of Liability}\citeEPL{§6} of the EPL-1.0 must explicitly be present in the
    documentation of distribution package and---if technically possible---%
    presented by the copyright screen.   
  
  \item Aside from that, the EPL-1.0 allows the distribution of the software in the
    form of binaries only if the distributor clearly \enquote{[\ldots] states that
    the source code for the program is available from such Contributor
    (distributor) [\ldots]} and if he additionally \enquote{[\ldots] informs
    licensees how to obtain it in a reasonable manner [\ldots]}\citeEPL{§3} 
    This requirement can only be fulfilled seriously if the distributor himself
    offers the source code via a repository. It is not sufficient to point to
    any external download repository in the world wide web. Thus,---for all use
    cases concerning the distribution in the form of binaries---the \oslic{}
    follows the respective requirement introduced by the EPL-1.0 (EPL-1.0-C3, EPL-1.0-C5,
    EPL-1.0-C7, EPL-1.0-C9).  
  
  \item Moreover, one has clearly to state that the previous rule implies a real
    source code distribution which therefore must follow the rules of
    distributing the software. Thus, the \oslic{} requires in all cases of a binary 
    distribution to execute also the task-lists of the respective source code
    use cases. 
 
 	\item Finally, the EPL-1.0 contains a patent clause stating that \enquote{if
 	any recipient institutes patent litigation against any entity [\ldots]
 	alleging that the Program itself [\ldots] infringes such Recipient's
 	patent(s), then such Recipient's rights granted [\ldots by the EPL-1.0] shall
 	terminate [\ldots]}\citeEPL{§7}. Based on this fact, the \oslic{} generally
 	(EPL-1.0-C1 -- EPL-1.0-C9) interdicts to legally fight against patents linked to the software.
\end{itemize}
\end{license}

%\bibliography{../../../bibfiles/oscResourcesEn}

% Local Variables:
% mode: latex
% fill-column: 80
% End:
}
{% Telekom osCompendium 'for being included' snippet template
%
% (c) Karsten Reincke, Deutsche Telekom AG, Darmstadt 2011
%
% This LaTeX-File is licensed under the Creative Commons Attribution-ShareAlike
% 3.0 Germany License (http://creativecommons.org/licenses/by-sa/3.0/de/): Feel
% free 'to share (to copy, distribute and transmit)' or 'to remix (to adapt)'
% it, if you '... distribute the resulting work under the same or similar
% license to this one' and if you respect how 'you must attribute the work in
% the manner specified by the author ...':
%
% In an internet based reuse please link the reused parts to www.telekom.com and
% mention the original authors and Deutsche Telekom AG in a suitable manner. In
% a paper-like reuse please insert a short hint to www.telekom.com and to the
% original authors and Deutsche Telekom AG into your preface. For normal
% quotations please use the scientific standard to cite.
%
% [ Framework derived from 'mind your Scholar Research Framework' 
%   mycsrf (c) K. Reincke 2012 CC BY 3.0  http://mycsrf.fodina.de/ ]
%


%% use all entries of the bibliography
%\nocite{*}

\section{EUPL-1.1 licensed software}
\begin{license}{EUPL} % ends at end of file
\licensename{EUPL-1.1}
\licensespec{European Union Public License 1.1}
\licenseversion{1.1}
\licenseabbrev{EUPL}

The European Union Public License explicitly distinguishes the distribution of
the source code from that of the binaries: In the chapter \enquote{Communication
of the Source Code,} it allows to \enquote{provide the Work either in its Source
Code form, or as Executable Code.}\citeEUPL{§3} But if a piece of EUPL-1.1 licensed
software is distributed as binary package, then the license additionally
requires that the distributor either \enquote{[\ldots] provides a
machine-readable copy of the Source Code [\ldots]} directly together with the
binaries\citeEUPL{§5} or that he \enquote{[\ldots] indicates [\ldots] a
repository where the Source Code is easily and freely accessible for as long 
as the Licensor continues to distribute [\ldots] the Work.}\citeEUPL{§3} For
respecting this conditions it is irrelevant whether the software has been
modified or not and all the other \enquote{obligations of the licensee} refer to
both forms.\citeEUPL{§5}

There is a particular aspect which has to be considered for acting in
accordance to the EUPL-1.1: Taken literally, the EUPL is a license with a weak
copyleft, no doubt. But this happens only a result of the fact that the EUPL-1.1
allows the licensee to relicense the software by following the conditions of the
\enquote{Compatibility clause}\citeEUPL{§5} and an license listed in an
appendix, which also includes some licenses with a weak copyleft.%
  \footnote{($\rightarrow$ \oslic, p.\ \protectionpageref{EUPL})} 
But, with respect to question how to fulfill the license best, it is safer to
treat the EUPL-1.1 as a license with a strong copyleft. Concerning the use of an
unmodified or a modified library as an embedded component, a license with 
a strong copyleft implies that the application which is using the
(un)modified library has also to be licensed under the same conditions as the
library itself. 
Thus, to find a simple to process task lists, use the following EUPL-1.1
specific open source use case structure:%
  \footnote{For details of the general OSUC finder $\rightarrow$ \oslic,
    pp.\ \pageref{OsucTokens} and \pageref{OsucDefinitionTree}} 

 
\tikzstyle{nodv} = [font=\scriptsize, ellipse, draw, fill=gray!10, 
    text width=2cm, text centered, minimum height=2em]

\tikzstyle{nods} = [font=\tiny, rectangle, draw, fill=gray!20, 
    text width=1cm, text centered, rounded corners, minimum height=3em]

\tikzstyle{nodb} = [font=\tiny, rectangle, draw, fill=gray!20, 
    text width=1.5cm, text centered, rounded corners, minimum height=3em]
    
\tikzstyle{leaf} = [font=\tiny, rectangle, draw, fill=gray!30, 
    text width=1.2cm, text centered, minimum height=6em]

\tikzstyle{slimleaf} = [font=\tiny, rectangle, draw, fill=gray!30, 
    text width=1cm, text centered, minimum height=6em]


\tikzstyle{edge} = [draw, -latex']

\begin{tikzpicture}[]

\node[nodv] (l801) at (4,10.2) {EUPL-1.1};

\node[nodb] (l601) at (0,8.6) {\textit{recipient:} \\ \textbf{4yourself}};
\node[nodb] (l602) at (7.5,8.6) {\textit{recipient:} \\ \textbf{2others}};

\node[nodb] (l501) at (4,7) {\textit{state:} \\ \textbf{unmodified}};
\node[nodb] (l502) at (11,7) {\textit{state:} \\ \textbf{modified}};

\node[nodb] (l401) at (2.25,5.4) {\textit{type:} \\ \textbf{proapse or snimoli}};
\node[nodb] (l402) at (5.4,5.4) {\textit{type:} \\ \textbf{snimoli}};
\node[nodb] (l403) at (8.4,5.4) {\textit{type:} \\ \textbf{proapse}};
\node[nodb] (l404) at (12.8,5.4) {\textit{type:} \\ \textbf{snimoli}};


\node[nodb] (l301) at (2.25,3.8) {\textit{context:} \\ \textbf{independent}};
\node[nodb] (l302) at (5.4,3.8) {\textit{context:} \\ \textbf{embedded}};
\node[nodb] (l303) at (8.4,3.8) {\textit{context:} \\ \textbf{independent}};
\node[nodb] (l304) at (11.3,3.8) {\textit{context:} \\ \textbf{independent}};
\node[nodb] (l305) at (14.3,3.8) {\textit{context:} \\ \textbf{embedded}};

\node[nods] (l201) at (1.45,2.2) {\textit{form:} \textbf{source}};
\node[nods] (l202) at (3.0,2.2) {\textit{form:} \textbf{binary}};
\node[nods] (l203) at (4.6,2.2) {\textit{form:} \textbf{source}};
\node[nods] (l204) at (6.2,2.2) {\textit{form:} \textbf{binary}};
\node[nods] (l205) at (7.7,2.2) {\textit{form:} \textbf{source}};
\node[nods] (l206) at (9.1,2.2) {\textit{form:} \textbf{binary}};
\node[nods] (l207) at (10.5,2.2) {\textit{form:} \textbf{source}};
\node[nods] (l208) at (11.9,2.2) {\textit{form:} \textbf{binary}};
\node[nods] (l209) at (13.4,2.2) {\textit{form:} \textbf{source}};
\node[nods] (l210) at (15.0,2.2) {\textit{form:} \textbf{binary}};

\node[slimleaf] (l101) at (0,0) {\textbf{EUPL-1.1-C1} \textit{using software only
for yourself}};

\node[leaf] (l102) at (1.45,0) { \textbf{EUPL-1.1-C2} \textit{ distributing unmodified
software as independent sources}};

\node[leaf] (l103) at (3.0,0) { \textbf{EUPL-1.1-C3}  \textit{ distributing unmodified
software as independent binaries}};

\node[leaf] (l104) at (4.6,0) { \textbf{EUPL-1.1-C4} \textit{ distributing unmodified
library as embedded sources}};

\node[leaf] (l105) at (6.2,0) { \textbf{EUPL-1.1-C5}  \textit{ distributing unmodified
library as embedded binaries}};

\node[slimleaf] (l106) at (7.7,0) { \textbf{EUPL-1.1-C6}  \textit{ distributing modified
program as sources}};

\node[slimleaf] (l107) at (9.1,0) { \textbf{EUPL-1.1-C7}  \textit{ distributing modified
program as binaries}};

\node[slimleaf] (l108) at (10.5,0) { \textbf{EUPL-1.1-C8}  \textit{ distributing modified
library as independent sources}};

\node[slimleaf] (l109) at (11.9,0) { \textbf{EUPL-1.1-C9} \textit{distributing modified
library as independent binaries}};

\node[leaf] (l110) at (13.4,0) { \textbf{EUPL-1.1-CA}  \textit{distributing
modified library as embedded sources}};

\node[leaf] (l111) at (15,0) { \textbf{EUPL-1.1-CB}  \textit{ distributing modified
library as embedded binaries}};

\path [edge] (l801) -- (l601);
\path [edge] (l801) -- (l602);


\path [edge] (l602) -- (l501);
\path [edge] (l602) -- (l502);

\path [edge] (l501) -- (l401);
\path [edge] (l501) -- (l402);
\path [edge] (l502) -- (l403);
\path [edge] (l502) -- (l404);

\path [edge] (l401) -- (l301);
\path [edge] (l402) -- (l302);
\path [edge] (l403) -- (l303);
\path [edge] (l404) -- (l304);
\path [edge] (l404) -- (l305);

\path [edge] (l301) -- (l201);
\path [edge] (l301) -- (l202);
\path [edge] (l302) -- (l203);
\path [edge] (l302) -- (l204);
\path [edge] (l303) -- (l205);
\path [edge] (l303) -- (l206);
\path [edge] (l304) -- (l207);
\path [edge] (l304) -- (l208);
\path [edge] (l305) -- (l209);
\path [edge] (l305) -- (l210);

\path [edge] (l601) -- (l101);
\path [edge] (l201) -- (l102);
\path [edge] (l202) -- (l103);
\path [edge] (l203) -- (l104);
\path [edge] (l204) -- (l105);
\path [edge] (l205) -- (l106);
\path [edge] (l206) -- (l107);
\path [edge] (l207) -- (l108);
\path [edge] (l208) -- (l109);
\path [edge] (l209) -- (l110);
\path [edge] (l210) -- (l111);

\end{tikzpicture}

%%
%% Common Building Blocks
%%

% ------------------------------------------------------------------------------
% Text of repeated footnotes

\newcommand{\reasonForOtherUseCase}{Making the code accessible via a repository
  means distributing the software in the form of source code. Hence, you must
  also fulfill all tasks of the corresponding use case.}

% ------------------------------------------------------------------------------
% Don't use trademarks, etc for advertising
\newcommand{\noTrademarks}{to promote any of your services or products based on
  the this software by trade names, trademarks, service marks, or names linked
  to this EUPL-1.1 software, except as required for reasonable and customary use in
  describing the origin of the software and reproducing the copyright notice.}

% ------------------------------------------------------------------------------
% Do not remove copyright notices and license files
\newcommand{\keepLicensingElements}{Ensure that the licensing elements
  (particularly the copyright, patent, and trademark notices and all notices
  that refer to the license or to the disclaimer of warranties) are retained in
  your package in the form you have received them.}

\newcommand{\addWhenCompiling}{If you compile the binary from the sources,
  ensure that all the licensing elements are also incorporated into the
  package.} 

% ------------------------------------------------------------------------------
% Give receiver a copy of the license text
\newcommand{\giveLicense}{Give the recipient a copy of the EUPL-1.1 license. If
  it is not already part of the software package, add it.} 

% ------------------------------------------------------------------------------
% Make the source code available
\newcommand{\auxMakeSourceAvailable}[1]{Make the source code of #1 accessible
  via a repository under your own control (even if you did not modify it): 
  Push the source code package into a repository, make it downloadable via the 
  internet, and include an easy to find description in the distribution package,
  which explains how and where the code can be received. Ensure, that this
  repository is online for as long as you continue to distribute the software.} 

% TODO: replace 'overarching'
\newcommand{\makeSourceAvailable}{\auxMakeSourceAvailable{%
    the distributed software}}
\newcommand{\makeAllSourcesAvailable}{\auxMakeSourceAvailable{%
    the embedded library \emph{and} your overarching program}} 

% ------------------------------------------------------------------------------
% Add location of source repository to documentation
\newcommand{\mentionRepositoryInDocumentation}{Insert a prominent hint to the
  download repository into your distribution or your additional material.}

% ------------------------------------------------------------------------------
% Add links and license to documentation
\newcommand{\auxAddToDoc}{Let the documentation of your distribution or
  your additional material also reproduce the content of the existing
  \emph{copyright notice text files}, a hint to the software name, a link to its
  homepage, and a link to the EUPL-1.1 license}

\newcommand{\addToDocumentation}{\auxAddToDoc.}
\newcommand{\addToYourCopyrightNotice}{\auxAddToDoc, preferably as a subsection
  of your own copyright notice.}

% ------------------------------------------------------------------------------
% mark all modifications
\newcommand{\auxMarkAllModifications}[1]{Mark all modifications of source code
  of the #1 thoroughly within the source code and include the date of the
  modification.}  

\newcommand{\markAllProgramModifications}{
  \auxMarkAllModifications{program}}
\newcommand{\markAllLibraryModifications}{%
  \auxMarkAllModifications{library}}
\newcommand{\markAllEmbeddedModifications}{%
  \auxMarkAllModifications{embedded library}}

% ------------------------------------------------------------------------------
% Copyleft
\newcommand{\auxApplyCopyleft}[1]{License your program, which includes the
  library, also under the EUPL-1.1.  Arrange the #1 of the on-top development in 
  a way that they are also covered by the EUPL-1.1 licensing statements.}

\newcommand{\applyCopyleftToSources}{\auxApplyCopyleft{sources}}
\newcommand{\applyCopyleftToBinaries}{\auxApplyCopyleft{binaries}}

% ------------------------------------------------------------------------------
% Marking modification

\newcommand{\auxNewSources}{If you add new source
  code files, insert a header containing your copyright line and an EUPL-1.1
  adequate licensing the statement.}

\newcommand{\auxArrangeModifications}{Arrange your modifications in a way that
  they are covered by the existing EUPL-1.1 licensing statements.}

\newcommand{\arrangeBinaryModifications}{\auxArrangeModifications}
\newcommand{\arrangeSourceModifications}{\auxArrangeModifications\ \auxNewSources}

% ------------------------------------------------------------------------------
% Modification text file

\newcommand{\addModificationTextFile}{Create a \emph{modification text file}, 
  if such a file still does not exist. \emph{Add} a description of your
  modifications to the \emph{modification text file.}}

% ------------------------------------------------------------------------------
% Copyright dialog

\newcommand{\copyrightDialog}{Let the copyright dialog of the on-top development
  clearly say, that it uses the EUPL-1.1 licensed library and that it is itself
  licensed under the EUPL-1.1, too.}

% ------------------------------------------------------------------------------
\subsection{EUPL-1.1-C1: Using the software only for yourself}
\begin{lsuc}{EUPL-1.1-C1}
  \linkosuc{01}
  \linkosuc{03L} 
  \linkosuc{03N} 
  \linkosuc{06L}
  \linkosuc{06N}
  \linkosuc{09L}
  \linkosuc{09N}

  \lsucmeans{that you received EUPL-1.1 licensed software, that you will use it
  only for yourself and that you do not hand it over to any 3rd party in any
  sense.}

  \lsuccovers{OSUC-01, OSUC-03L, OSUC-03N, OSUC-06L, OSUC-06N, OSUC-09L, and
  OSUC-09N\footnote{For details $\rightarrow$ \oslic, pp.\ \pageref{OSUC-01-DEF}
  - \pageref{OSUC-09N-DEF}}}

  \begin{lsucrequiresnothing}
    \lsucitem{You are allowed to use any kind of EUPL-1.1 software in any sense
      and in any context without being obliged to do anything as long as you do
      not give the software to third parties.}
  \end{lsucrequiresnothing}
  
  \begin{lsucprohibits}
    \lsucitem{\noTrademarks}
  \end{lsucprohibits}
\end{lsuc}

% ------------------------------------------------------------------------------
\subsection{EUPL-1.1-C2: Passing the unmodified software as independent sources}
\begin{lsuc}{EUPL-1.1-C2}
  \linkosuc{02S}
  \linkosuc{05S}

  \lsucmeans{that you received EUPL-1.1 licensed software which you are now going
  to distribute to third parties as an independent unit and in the form of
  unmodified source code files or as unmodified source code package. In this
  case it makes no difference if you distribute a program, an application, a
  server, a snippet, a module, a library, or a plugin as an independent or as an
  embedded unit.}

  \lsuccovers{OSUC-02S, OSUC-05S\footnote{For details $\rightarrow$ \oslic,
      pp.\ \pageref{OSUC-02S-DEF} - \pageref{OSUC-05S-DEF}}} 

  \begin{lsucrequires}
    \lsucmandatory{\keepLicensingElements}
    \lsucmandatory{\giveLicense}\passingFilesCorrectly
    \lsucoptional{\addToDocumentation}
  \end{lsucrequires}

  \begin{lsucprohibits}
    \lsucitem{\noTrademarks}
  \end{lsucprohibits}

\end{lsuc}

% ------------------------------------------------------------------------------
\subsection{EUPL-1.1-C3: Passing the unmodified software as independent binaries} 
\begin{lsuc}{EUPL-1.1-C3}
  \linkosuc{02B}
  \linkosuc{05B}

  \lsucmeans{that you received EUPL-1.1 licensed software which you are now going to
  distribute to third parties as an independent unit and in the form of
  unmodified binary files or as unmodified binary package. In this case it does
  not matter if you distribute a program, an application, a server, a snippet, a
  module, a library, or a plugin as an independent or an embedded unit.}

  \lsuccovers{OSUC-02B, OSUC-05B\footnote{For details $\rightarrow$ \oslic,
      pp.\ \pageref{OSUC-02B-DEF} - \pageref{OSUC-05B-DEF}}} 

  \begin{lsucrequires}
    \lsucmandatory{\keepLicensingElements\ \addWhenCompiling}
    \lsucmandatory{\giveLicense}\passingFilesCorrectly
    \lsucmandatory{\makeSourceAvailable}
    \lsucmandatory{\mentionRepositoryInDocumentation}
    \lsucsourcedist{EUPL-1.1-C2}
    \lsucoptional{\addToDocumentation}
  \end{lsucrequires}

  \begin{lsucprohibits}
    \lsucitem{\noTrademarks}
  \end{lsucprohibits}

\end{lsuc}

% ------------------------------------------------------------------------------
\subsection{EUPL-1.1-C4: Passing the unmodified library as embedded sources}
\begin{lsuc}{EUPL-1.1-C4}
  \linkosuc{07S}

  \lsucmeans{that you received a EUPL-1.1 licensed snippet, module or
  library which you are now going to distribute to third parties as an embedded
  component of a larger unit and in the form of unmodified source code files or as
  unmodified source code package.}

  \lsuccovers{OSUC-07S\footnote{For details $\rightarrow$ \oslic,
      pp.\ \pageref{OSUC-07S-DEF}}} 

  \begin{lsucrequires}
    \lsucmandatory{\keepLicensingElements}
    \lsucmandatory{\giveLicense}\passingFilesCorrectly
    \lsucmandatory{\applyCopyleftToSources}
    \lsucoptional{\copyrightDialog}
    \lsucoptional{\addToYourCopyrightNotice}
  \end{lsucrequires}

  \begin{lsucprohibits}
    \lsucitem{\noTrademarks}
  \end{lsucprohibits}

\end{lsuc}

% ------------------------------------------------------------------------------
\subsection{EUPL-1.1-C5: Passing the unmodified library as embedded binaries} 
\begin{lsuc}{EUPL-1.1-C5}
  \linkosuc{07B}

  \lsucmeans{that you received a EUPL-1.1 licensed snippet, module or
  library which you are now going to distribute to third parties as an embedded
  component of a larger unit and in the form of unmodified binary files or as
  unmodified binary package.}

  \lsuccovers{OSUC-07B\footnote{For details $\rightarrow$ \oslic,
      pp.\ \pageref{OSUC-07B-DEF}}} 

  \begin{lsucrequires}
    \lsucmandatory{\keepLicensingElements\ \addWhenCompiling}
    \lsucmandatory{\giveLicense}\passingFilesCorrectly
    \lsucmandatory{\makeAllSourcesAvailable}
    \lsucmandatory{\mentionRepositoryInDocumentation}
    \lsucmandatory{\applyCopyleftToBinaries}
    \lsucsourcedist{EUPL-1.1-C4}
    \lsucoptional{\copyrightDialog}
    \lsucoptional{\addToYourCopyrightNotice}
  \end{lsucrequires}

  \begin{lsucprohibits}
    \lsucitem{\noTrademarks}
  \end{lsucprohibits}

\end{lsuc}

% ------------------------------------------------------------------------------
\subsection{EUPL-1.1-C6: Passing a modified program as source code}
\begin{lsuc}{EUPL-1.1-C6}
  \linkosuc{04S} 

  \lsucmeans{that you received a EUPL-1.1 licensed program, application, or
  server (proapse), that you modified it, and that you are now going to
  distribute this modified version to third parties in the form of source code files or as
  a source code package.} 

  \lsuccovers{OSUC-04S\footnote{For details $\rightarrow$ \oslic,
      pp.\ \pageref{OSUC-04S-DEF}}} 

  \begin{lsucrequires}
    \lsucmandatory{\keepLicensingElements}
    \lsucmandatory{\giveLicense}\passingFilesCorrectly
    \lsucmandatory{\addModificationTextFile}
    \lsucmandatory{\markAllProgramModifications}
    \lsucmandatory{\arrangeSourceModifications}
    \lsucoptional{\addToDocumentation}
  \end{lsucrequires}
 
  \begin{lsucprohibits}
    \lsucitem{\noTrademarks}
  \end{lsucprohibits}

\end{lsuc}

% ------------------------------------------------------------------------------
\subsection{EUPL-1.1-C7: Passing a modified program as binary}
\begin{lsuc}{EUPL-1.1-C7}
  \linkosuc{04B}

  \lsucmeans{that you received a EUPL-1.1 licensed program, application, or
  server (proapse), that you modified it, and that you are now going to
  distribute this modified version to third parties in the form of binary files or as a
  binary package.}

  \lsuccovers{OSUC-04B\footnote{For details $\rightarrow$ \oslic,
      pp.\ \pageref{OSUC-04B-DEF}}} 

  \begin{lsucrequires}
    \lsucmandatory{\keepLicensingElements\ \addWhenCompiling}
    \lsucmandatory{\giveLicense}\passingFilesCorrectly
    \lsucmandatory{\addModificationTextFile}
    \lsucmandatory{\arrangeBinaryModifications}
    \lsucmandatory{\makeSourceAvailable}
    \lsucmandatory{\mentionRepositoryInDocumentation}
    \lsucsourcedist{EUPL-1.1-C6}
    \lsucoptional{\markAllProgramModifications}
    \lsucoptional{\addToDocumentation}
  \end{lsucrequires}

  \begin{lsucprohibits}
    \lsucitem{\noTrademarks}
  \end{lsucprohibits}

\end{lsuc}

% ------------------------------------------------------------------------------
\subsection{EUPL-1.1-C8: Passing a modified library as independent source code}
\begin{lsuc}{EUPL-1.1-C8}
  \linkosuc{08S}

  \lsucmeans{that you received a EUPL-1.1 licensed code snippet, module, library,
  or plugin (snimoli), that you modified it, and that you are now going to
  distribute this modified version to third parties in the form of source code
  files or as a source code package, but without embedding it into another
  larger software unit.}

  \lsuccovers{OSUC-08S\footnote{For details $\rightarrow$ \oslic,
      pp.\ \pageref{OSUC-08S-DEF}}} 

  \begin{lsucrequires}
    \lsucmandatory{\keepLicensingElements}
    \lsucmandatory{\giveLicense}\passingFilesCorrectly
    \lsucmandatory{\addModificationTextFile}
    \lsucmandatory{\markAllLibraryModifications}
    \lsucmandatory{\arrangeSourceModifications}
    \lsucoptional{\addToDocumentation}
  \end{lsucrequires}

  \begin{lsucprohibits}
    \lsucitem{\noTrademarks}
  \end{lsucprohibits}

\end{lsuc}

% ------------------------------------------------------------------------------
\subsection{EUPL-1.1-C9: Passing a modified library as independent binary}
\begin{lsuc}{EUPL-1.1-C9}
  \linkosuc{08B}

  \lsucmeans{that you received a EUPL-1.1 licensed code snippet, module, library,
  or plugin (snimoli), that you modified it, and that you are now going to
  distribute this modified version to third parties in the form of binary files
  or as a binary package but without embedding it into another larger software
  unit.}

  \lsuccovers{OSUC-08B\footnote{For details $\rightarrow$ \oslic,
      pp.\ \pageref{OSUC-08B-DEF}}} 

  \begin{lsucrequires}
    \lsucmandatory{\keepLicensingElements\ \addWhenCompiling}
    \lsucmandatory{\giveLicense}\passingFilesCorrectly
    \lsucmandatory{\addModificationTextFile}
    \lsucmandatory{\arrangeBinaryModifications}
    \lsucmandatory{\makeSourceAvailable}
    \lsucmandatory{\mentionRepositoryInDocumentation}
    \lsucsourcedist{EUPL-1.1-C8}
    \lsucoptional{\markAllLibraryModifications}
    \lsucoptional{\addToDocumentation}
  \end{lsucrequires}

  \begin{lsucprohibits}
    \lsucitem{\noTrademarks}
  \end{lsucprohibits}

\end{lsuc}

% ------------------------------------------------------------------------------
\subsection{EUPL-1.1-CA: Passing a modified library as embedded source code}
\begin{lsuc}{EUPL-1.1-CA}
  \linkosuc{10S}

  \lsucmeans{that you received a EUPL-1.1 licensed code snippet, module, library,
  or plugin (snimoli), that you modified it, and that you are now going to
  distribute this modified version to third parties in the form of source code
  files or as a source code package together with another larger software unit
  which contains this code snippet, module, library, or plugin as an embedded
  component.}

  \lsuccovers{OSUC-10S\footnote{For details $\rightarrow$ \oslic,
      pp.\ \pageref{OSUC-10S-DEF}}} 

  \begin{lsucrequires}
    \lsucmandatory{\keepLicensingElements}
    \lsucmandatory{\giveLicense}\passingFilesCorrectly
    \lsucmandatory{\addModificationTextFile}
    \lsucmandatory{\arrangeSourceModifications}
    \lsucmandatory{\applyCopyleftToSources}
    \lsucmandatory{\markAllEmbeddedModifications}
    \lsucoptional{\copyrightDialog}
    \lsucoptional{\addToYourCopyrightNotice}
  \end{lsucrequires}

  \begin{lsucprohibits}
    \lsucitem{\noTrademarks}
  \end{lsucprohibits}

\end{lsuc}

% ------------------------------------------------------------------------------
\subsection{EUPL-1.1-CB: Passing a modified library as embedded binary}
\begin{lsuc}{EUPL-1.1-CB}
  \linkosuc{10B}

  \lsucmeans{that you received a EUPL-1.1 licensed code snippet, module, library,
  or plugin (snimoli), that you modified it, and that you are now going to
  distribute this modified version to third parties in the form of binary files
  or as a binary package together with another larger software unit which
  contains this code snippet, module, library, or plugin as an embedded component.}

  \lsuccovers{OSUC-10B\footnote{For details $\rightarrow$ \oslic,
      pp.\ \pageref{OSUC-10B-DEF}}} 

  \begin{lsucrequires}
    \lsucmandatory{\keepLicensingElements \addWhenCompiling}
    \lsucmandatory{\giveLicense}\passingFilesCorrectly
    \lsucmandatory{\addModificationTextFile}
    \lsucmandatory{\makeAllSourcesAvailable}
    \lsucmandatory{\mentionRepositoryInDocumentation}
    \lsucsourcedist{EUPL-1.1-CA}
    \lsucmandatory{\arrangeBinaryModifications}
    \lsucmandatory{\applyCopyleftToBinaries}
    \lsucoptional{\markAllEmbeddedModifications}
    \lsucoptional{\addToYourCopyrightNotice}
  \end{lsucrequires}

  \begin{lsucprohibits}
    \lsucitem{\noTrademarks}
  \end{lsucprohibits}

\end{lsuc}

% ------------------------------------------------------------------------------
\subsection{Discussions and Explanations}
\label{EUPLDiscussion}
\begin{itemize}
  
\item The EUPL-1.1 generally \enquote{[\ldots] does not grant permission to use
  the trade names, trademarks, service marks, or names of the Licensor, except
  as required for reasonable and customary use in describing the origin of the
  Work and reproducing the content of the copyright notice.}\citeEUPL{§5} 
  Therefore, the \oslic{} genreally interdicts (EUPL-1.1-C1 -- EUPL-1.1-CB) to promote any
  service or product based on this software by such elements. 

\item The EUPL-1.1 generally requires that \enquote{[\ldots] the Licensee shall
  keep intact all copyright, patent or trademarks notices and all notices that
  refer to the Licence and to the disclaimer of warranties.}\citeEUPL{§5} 
  In a very strict reading, the EUPL-1.1 does not limit this requirement to the
  distribution of the software. But in practise, it will be impossible to
  control the compliant use of the software in those cases (\emph{4yourself})
  unless you also start to distribute the software. Therefore the \oslic{} only
  inserts this requirement as a mandatory clause only for the \emph{2others} use 
  cases (EUPL-1.1-C2 -- EUPL-1.1-CB). 
  
\item The EUPL-1.1 also requires to \enquote{[\ldots] include [\ldots] a copy of
  the (EUPL-1.1) Licence with every (distributed) copy of the Work}.\citeEUPL{§5}
  Therefore, all \emph{2others} use cases contain the respective mandatory
  condition (EUPL-1.1-C2 -- EUPL-1.1-CB).
  
\item Additionally, the EUPL-1.1 requires that the \enquote{licensee} who
  distributes a modified work \enquote{[\ldots] must cause any Derivative Work 
  to carry prominent notices stating that the Work has been modified and the
  date of modification.}\citeEUPL{§5} 
  Thus, the \oslic{} integrates the mandatory requirement to generate (update) a
  respective notice file into all `modification' use cases and recommends to mark
  all modifications in the source code (EUPL-1.1-C6 -- EUPL-1.1-CB).
  
\item Furthermore, the EUPL-1.1 requires that any distributor of the software
  \enquote{[\ldots] provide a machine-readable copy of the Source Code [\ldots]}
  by \enquote{[\ldots] (indicating) a repository where this Source will be
  easily and freely available for as long as the Licensee continues to
  distribute [\ldots] the Work.}%
  \footnote{\cite[cf.][\nopage wp.\ §5]{EuplLicense2007en}. To be precise, the
    EUPL-1.1 also allows to directly distribute the source code together with the
    binary packages (\cite[cf.][\nopage wp.\ §3]{EuplLicense2007en}). With
    respect to the \oslic{} principle to offer only one reliable way, the \oslic{}
    simplifies this option: It `only' asks for the repository solution.} 
  Therefore the \oslic{} inserts a respective requirement into the task list of all
  cases concerning a binary distribution (EUPL-1.1-C3, EUPL-1.1-C7, EUPL-1.1-C9, and EUPL-1.1-CB)
  
\item Finally, the EUPL-1.1 contains a \enquote{copyleft clause} stating that if a
  \enquote{[\ldots] Licensee distributes [\ldots] copies of the Original Works
  or Derivative Works based upon the Original Work, this Distribution [...] will
  be done under the terms of this (EUPL-1.1) Licence [\ldots]}. In all the use cases
  which do not concern the use of an embedded component (EUPL-1.1-C2 -- EUPL-1.1-C9) this
  copyleft clause is already fulfilled by either distributing the modified
  sources themselves or by making them accessible via a repository. In those
  cases where the licensee distributes an program that uses an embedded EUPL-1.1
  licensed component (EUPL-1.1-CA -- EUPL-1.1-CB), in general, the code of the embedding
  program must also be distributed. Thus, with respect to the use case (EUPL-1.1-CA)
  this is already fulfilled by definition. Therefore, the \oslic{} only mentions
  this default view in the case EUPL-1.1-CB implying a strong copyleft effect.%
  \footnote{Formally, the EUPL-1.1 is only a license with weak copyleft. 
    But this is only a result of allowing to relicense the software
    ($\rightarrow$ \oslic, p.\ \protectionpageref{EUPL}). So, as long as
    you do not relicense the embedded library with respect to the list of
    \enquote{compatible licenses according to article 5 EUPL-1.1} 
    (\cite[cf.][\nopage wp §5 and Appendix]{EuplLicense2007en}), 
    you also have to publish the code of your overarching work.}

\end{itemize}

\end{license}

%\bibliography{../../../bibfiles/oscResourcesEn}

% Local Variables:
% mode: latex
% fill-column: 80
% End:
}
{% Telekom osCompendium 'for being included' snippet template
%
% (c) Karsten Reincke, Deutsche Telekom AG, Darmstadt 2011
%
% This LaTeX-File is licensed under the Creative Commons Attribution-ShareAlike
% 3.0 Germany License (http://creativecommons.org/licenses/by-sa/3.0/de/): Feel
% free 'to share (to copy, distribute and transmit)' or 'to remix (to adapt)'
% it, if you '... distribute the resulting work under the same or similar
% license to this one' and if you respect how 'you must attribute the work in
% the manner specified by the author ...':
%
% In an Internet based reuse please link the reused parts to www.telekom.com and
% mention the original authors and Deutsche Telekom AG in a suitable manner. In
% a paper-like reuse please insert a short hint to www.telekom.com and to the
% original authors and Deutsche Telekom AG into your preface. For normal
% quotations please use the scientific standard to cite.
%
% [ Framework derived from 'mind your Scholar Research Framework' 
%   mycsrf (c) K. Reincke 2012 CC BY 3.0  http://mycsrf.fodina.de/ ]
%


%% use all entries of the bibliography
%\nocite{*}

\section{GPL licensed software}

Both versions of the GNU General Public License explicitly distinguish the
distribution of the source code from that of the binaries: On the one hand, the
GPL-2.0 mainly talks about copying and distributing the source
code,\citeGPLtwo{§1, §2} but also mentions the specific conditions for
\enquote{[\ldots] (copying) and (distributing) the Program [\ldots] in object
code or executable form [\ldots]}\citeGPLtwo{§3} On the other hand, the GPL-3.0
describes the \enquote{Basic Permissions} and the conditions for
\enquote{Conveying Verbatim Copies} or for \enquote{Conveying Modified Source
Versions}\citeGPLtwo{§2, §4, §5} before it explains the rules for
\enquote{Conveying Non-Source-Forms}.\citeGPLtwo{§2, §4, §5}  

GPL-2.0 and GPL-3.0 mainly talk about copying \emph{and} distributing the
software; private use is nearly completely unspecified: The GPL-2.0 lists its
`restrictions' only with respect to the act of copying \emph{and} distributing
\enquote{copies of the program}\citeGPLtwo{§1, §2, §4 et passim; emphasize by
KR} while the GPL-3.0 explicitly specifies that one \enquote{[\ldots] may
make, run and propagate covered works that (one does) not convey, without
conditions so long as (the) license otherwise remains in
force.}\citeGPLthree{§2} 

As licenses with a strong copyleft, they require that any application that
contains a GPL-licensed library must itself be licensed under the same
conditions as the library.
 
Finally, the GPL-2.0 and the GPL-3.0 aim for the same results and share the
same spirit by requiring nearly the same task to be performed for fulfilling the
license conditions.  Therefore it is appropriate to cover both versions in the
same chapter and to offer a common specialized GPL open source use case
structure for quickly finding the appropriate task list.%
  \footnote{For details of the general OSUC finder $\rightarrow$ \oslic,
    pp.\ \pageref{OsucTokens} and \pageref{OsucDefinitionTree}}
However, the task lists themselves will be kept separate.

In the following diagram, GPL-*-C1 (GPL-*-C2, \ldots, GPL-*-CB) is either
GPL-2.0-C1 (and so forth), if you are looking for the GPL-2.0 use case, or
GPL-3.0-C1, \ldots for the GPL-3.0 use case.

%% ============================================================================= 
%% Use-Case Finder

\gplUseCaseFinder{GPL}{2.0}{3.0}

%% ============================================================================= 
%% Common Building Blocks

\newcommand{\useCaseOne}{%
  \gtbUseCaseOne{GPL-\ver}
  \gtbCoversOne{GPL-\ver}}

\newcommand{\useCaseTwo}{%
  \gtbUseCaseTwo{GPL-\ver}
  \gtbCoversTwo{GPL-\ver}}

\newcommand{\useCaseThree}{%
  \gtbUseCaseThree{GPL-\ver}
  \gtbCoversThree{GPL-\ver}}

\newcommand{\useCaseFour}{%
  \gtbUseCaseFour{GPL-\ver}{a}
  \gtbCoversFour{GPL-\ver}}

\newcommand{\useCaseFive}{%
  \gtbUseCaseFive{GPL-\ver}{a}
  \gtbCoversFive{GPL-\ver}}

\newcommand{\useCaseSix}{%
  \gtbUseCaseSix{GPL-\ver}{a}
  \gtbCoversSix{GPL-\ver}}

\newcommand{\useCaseSeven}{%
  \gtbUseCaseSeven{GPL-\ver}{a}
  \gtbCoversSeven{GPL-\ver}}

\newcommand{\useCaseEight}{%
  \gtbUseCaseEight{GPL-\ver}{a}
  \gtbCoversEight{GPL-\ver}}

\newcommand{\useCaseNine}{%
  \gtbUseCaseNine{GPL-\ver}{a}
  \gtbCoversNine{GPL-\ver}}

\newcommand{\useCaseA}{%
  \gtbUseCaseA{GPL-\ver}{a}
  \gtbCoversA{GPL-\ver}}

\newcommand{\useCaseB}{%
  \gtbUseCaseB{GPL-\ver}{a}
  \gtbCoversB{GPL-\ver}}

% ------------------------------------------------------------------------------
% Common Text Blocks from 0600-commomn-text-blocks.tex

\newcommand{\keepLicenseElements}{\gtbKeepLicenseElements{GPL-\ver}}
\newcommand{\addToDocumentation}{\gtbAddToDocumentation{GPL-\ver}}
\newcommand{\giveLicense}{\gtbGiveLicense{GPL-\ver}}
\newcommand{\retainCopyrightNotices}{\gtbKeepCopyrightNotices{GPL-\ver}}
\newcommand{\describeHowToGetSource}{\gtbDescribeHowToGetSource{GPL-\ver}}
\newcommand{\createChangelog}{\gtbCreateChangelog{GPL-\ver}}
\newcommand{\markEmbeddedModifications}{\gtbMarkEmbeddedModifications{GPL-\ver}}
\newcommand{\markLibraryModifications}{\gtbMarkLibraryModifications{GPL-\ver}}
\newcommand{\markProgramModifications}{\gtbMarkProgramModifications{GPL-\ver}}
\newcommand{\gpltwoEnsureCopyrightNoticeSource}{\gtbVTwoCopyrightNotice{GPL-2.0}{source code}}
\newcommand{\gpltwoEnsureCopyrightNoticeBinary}{\gtbVTwoCopyrightNotice{GPL-2.0}{binary}}
\newcommand{\gplthreeEnsureCopyrightNoticeSource}{\gtbVThreeCopyrightNotice{GPL-3.0}{source code}}
\newcommand{\gplthreeEnsureCopyrightNoticeBinary}{\gtbVThreeCopyrightNotice{GPL-3.0}{binary}}
\newcommand{\makeUnmodifiedSourceAvailable}{\gtbMakeUnmodifiedSourceAvailable{GPL-\ver}} 
\newcommand{\makeModifiedSourceAvailable}{\gtbMakeModifiedSourceAvailable{GPL-\ver}} 
\newcommand{\makeAllSourcesAvailable}{\gtbMakeAllSourcesAvailable{GPL-\ver}}
\newcommand{\arrangeProgramChanges}{\gtbArrangeProgramChanges{GPL-\ver}}
\newcommand{\arrangeLibraryChanges}{\gtbArrangeLibraryChanges{GPL-\ver}}
\newcommand{\arrangeEmbeddedChanges}{\gtbArrangeEmbeddedChanges{GPL-\ver}}
\newcommand{\howToApplyTheseTerms}{\gtbHowToApplyTheseTerms{GPL-\ver}}
\newcommand{\noPatentLitigation}{\gtbNoPatentLitigation{GPL-\ver}}
\newcommand{\addToCopyrightDialogLib}{\gtbAddToCopyrightDialogStrongCopyleft{GPL-\ver}}
\newcommand{\addToCopyrightDialogApp}{\gtbAddToCopyrightDialogApp{GPL-\ver}}

% ------------------------------------------------------------------------------
% Make sure, licensing statements apply to enclosing program

\newcommand{\auxArrange}[1]{Arrange the #1 in a way
  that they are covered by the GPL-\ver{} licensing statements.} 

\newcommand{\arrangeEnclosingBinaries}{%
  \auxArrange{the binaries of the on-top development}}

\newcommand{\arrangeEnclosingSources}{%
  \auxArrange{the sources of the on-top development}}

%% ============================================================================= 
%% GPL-2.0 Use Cases

\newcommand{\ver}{2.0}

\begin{license}{GPL2} 
\licensename{GPL-2.0}
\licensespec{General Public License Version 2}
\licenseabbrev{GPL}
%\licenseversion{2.0}

% ------------------------------------------------------------------------------
\subsection{GPL-\ver-C1: Using the software only for yourself}
\begin{lsuc}{GPL-\ver-C1}
  \linkosuc{01}
  \linkosuc{03L} 
  \linkosuc{03N} 
  \linkosuc{06L}
  \linkosuc{06N}
  \linkosuc{09L}
  \linkosuc{09N}

  \useCaseOne

  \begin{lsucrequiresnothing}
    \lsucitem{You are allowed to use any kind of GPL-\ver software in any sense
      and in any context without being obliged to do anything as long as you do
      not give the software to third parties.}
  \end{lsucrequiresnothing}

  \lsucprohibitsnothing
\end{lsuc}

% ------------------------------------------------------------------------------
\subsection{GPL-\ver-C2: Passing the unmodified software as independent sources}
\begin{lsuc}{GPL-\ver-C2}
  \linkosuc{02S}
  \linkosuc{05S}

  \useCaseTwo

  \begin{lsucrequires}
    \lsucmandatory{\keepLicenseElements}
    \lsucmandatory{\gpltwoEnsureCopyrightNoticeSource}
    \lsucmandatory{\giveLicense}\passingFilesCorrectly
    \lsucmandatory{\retainCopyrightNotices}
    \lsucoptional{\addToDocumentation}
  \end{lsucrequires}

  \lsucprohibitsnothing
\end{lsuc}

% ------------------------------------------------------------------------------
\subsection{GPL-\ver-C3: Passing the unmodified software as independent binaries} 
\begin{lsuc}{GPL-\ver-C3}
  \linkosuc{02B} 
  \linkosuc{05B}

  \useCaseThree

  \begin{lsucrequires}
    \lsucmandatory{\keepLicenseElements}
    \lsucmandatory{\gpltwoEnsureCopyrightNoticeBinary}
    \lsucmandatory{\giveLicense}\passingFilesCorrectly  
    \lsucmandatory{\makeUnmodifiedSourceAvailable}
    \lsucmandatory{\describeHowToGetSource}
    \lsucmandatory{\retainCopyrightNotices}
    \lsucsourcedist{GPL-\ver-C2}
    \lsucoptional{\addToDocumentation}
  \end{lsucrequires}

  \lsucprohibitsnothing
\end{lsuc}

% ------------------------------------------------------------------------------
\subsection{GPL-\ver-C4: Passing the unmodified library as embedded sources}
\begin{lsuc}{GPL-\ver-C4}
  \linkosuc{07S} 

  \useCaseFour

  \begin{lsucrequires}
    \lsucmandatory{\keepLicenseElements}
    \lsucmandatory{\gpltwoEnsureCopyrightNoticeSource}
    \lsucmandatory{\giveLicense}\passingFilesCorrectly
    \lsucmandatory{\retainCopyrightNotices}
    \lsucmandatory{\addToCopyrightDialogLib}
    \lsucmandatory{\arrangeEnclosingSources}
    \lsucoptional{\addToDocumentation}
  \end{lsucrequires}

  \lsucprohibitsnothing
\end{lsuc}

% ------------------------------------------------------------------------------
\subsection{GPL-\ver-C5: Passing the unmodified library as embedded binaries} 
\begin{lsuc}{GPL-\ver-C5}
  \linkosuc{07B} 

  \useCaseFive

  \begin{lsucrequires}
    \lsucmandatory{\keepLicenseElements}
    \lsucmandatory{\gpltwoEnsureCopyrightNoticeBinary}
    \lsucmandatory{\giveLicense}\passingFilesCorrectly
    \lsucmandatory{\makeAllSourcesAvailable}
    \lsucmandatory{\describeHowToGetSource}
    \lsucmandatory{\addToCopyrightDialogLib}
    \lsucmandatory{\arrangeEnclosingBinaries}
    \lsucmandatory{\retainCopyrightNotices}
    \lsucsourcedist{GPL-\ver-C4}
    \lsucoptional{\addToDocumentation}
  \end{lsucrequires}

  \lsucprohibitsnothing
\end{lsuc}

% ------------------------------------------------------------------------------
\subsection{GPL-\ver-C6: Passing a modified program as source code}
\begin{lsuc}{GPL-\ver-C6}
  \linkosuc{04S} 

  \useCaseSix

  \begin{lsucrequires}
    \lsucmandatory{\keepLicenseElements}
    \lsucmandatory{\gpltwoEnsureCopyrightNoticeSource}
    \lsucmandatory{\giveLicense}\passingFilesCorrectly
    \lsucmandatory{\retainCopyrightNotices}
    \lsucmandatory{\addToCopyrightDialogApp}
    \lsucmandatory{\markProgramModifications}
    \lsucmandatory{\arrangeProgramChanges}\howToApplyTheseTerms
    \lsucoptional{\createChangelog}
    \lsucoptional{\addToDocumentation}
  \end{lsucrequires}

  \lsucprohibitsnothing
\end{lsuc}

% ------------------------------------------------------------------------------
\subsection{GPL-\ver-C7: Passing a modified program as binary}
\begin{lsuc}{GPL-\ver-C7}
  \linkosuc{04B}

  \useCaseSeven

  \begin{lsucrequires}
    \lsucmandatory{\keepLicenseElements}
    \lsucmandatory{\gpltwoEnsureCopyrightNoticeBinary}
    \lsucmandatory{\giveLicense}\passingFilesCorrectly
    \lsucmandatory{\retainCopyrightNotices}
    \lsucmandatory{\markProgramModifications}
    \lsucmandatory{\addToCopyrightDialogApp}
    \lsucmandatory{\arrangeProgramChanges}\howToApplyTheseTerms
    \lsucmandatory{\makeModifiedSourceAvailable}
    \lsucmandatory{\describeHowToGetSource}
    \lsucsourcedist{GPL-\ver-C6}
    \lsucoptional{\createChangelog}
    \lsucoptional{\addToDocumentation}
  \end{lsucrequires}

  \lsucprohibitsnothing
\end{lsuc}

% ------------------------------------------------------------------------------
\subsection{GPL-\ver-C8: Passing a modified library as independent source code}
\begin{lsuc}{GPL-\ver-C8}
  \linkosuc{08S}

  \useCaseEight

  \begin{lsucrequires}
    \lsucmandatory{\keepLicenseElements}
    \lsucmandatory{\gpltwoEnsureCopyrightNoticeSource}
    \lsucmandatory{\giveLicense}\passingFilesCorrectly
    \lsucmandatory{\retainCopyrightNotices}
    \lsucmandatory{\markLibraryModifications}
    \lsucmandatory{\arrangeLibraryChanges}\howToApplyTheseTerms
    \lsucoptional{\createChangelog}
    \lsucoptional{\addToDocumentation}
  \end{lsucrequires}

  \lsucprohibitsnothing
\end{lsuc}

% ------------------------------------------------------------------------------
\subsection{GPL-\ver-C9: Passing a modified library as independent binary}
\begin{lsuc}{GPL-\ver-C9}
  \linkosuc{08B}

  \useCaseNine

  \begin{lsucrequires}
    \lsucmandatory{\keepLicenseElements}
    \lsucmandatory{\gpltwoEnsureCopyrightNoticeSource}  
    \lsucmandatory{\giveLicense}\passingFilesCorrectly
    \lsucmandatory{\retainCopyrightNotices}
    \lsucmandatory{\makeModifiedSourceAvailable}
    \lsucmandatory{\describeHowToGetSource}
    \lsucsourcedist{GPL-\ver-C8}
    \lsucmandatory{\markLibraryModifications}
    \lsucmandatory{\arrangeLibraryChanges}\howToApplyTheseTerms
    \lsucoptional{\createChangelog}
    \lsucoptional{\addToDocumentation}
  \end{lsucrequires}

  \lsucprohibitsnothing
\end{lsuc}

% ------------------------------------------------------------------------------
\subsection{GPL-\ver-CA: Passing a modified library as embedded source code}
\begin{lsuc}{GPL-\ver-CA}
  \linkosuc{10S}

  \useCaseA

  \begin{lsucrequires}
    \lsucmandatory{\keepLicenseElements}
    \lsucmandatory{\gpltwoEnsureCopyrightNoticeSource}
    \lsucmandatory{\giveLicense}\passingFilesCorrectly
    \lsucmandatory{\retainCopyrightNotices}
    \lsucmandatory{\addToCopyrightDialogLib}
    \lsucmandatory{\markEmbeddedModifications}
    \lsucmandatory{\arrangeEmbeddedChanges}\howToApplyTheseTerms
    \lsucmandatory{\arrangeEnclosingSources}
    \lsucoptional{\createChangelog}
    \lsucoptional{\addToDocumentation}
  \end{lsucrequires}

  \lsucprohibitsnothing
\end{lsuc}

% ------------------------------------------------------------------------------
\subsection{GPL-\ver-CB: Passing a modified library as embedded binary}
\begin{lsuc}{GPL-\ver-CB}
  \linkosuc{10B}

  \useCaseB

  \begin{lsucrequires}
    \lsucmandatory{\keepLicenseElements}
    \lsucmandatory{\gpltwoEnsureCopyrightNoticeBinary}
    \lsucmandatory{\giveLicense}\passingFilesCorrectly
    \lsucmandatory{\retainCopyrightNotices}
    \lsucmandatory{\makeAllSourcesAvailable}
    \lsucmandatory{\describeHowToGetSource}
    \lsucsourcedist{GPL-\ver-CA}
    \lsucmandatory{\addToCopyrightDialogLib}
    \lsucmandatory{\markEmbeddedModifications}
    \lsucmandatory{\arrangeEmbeddedChanges}\howToApplyTheseTerms
    \lsucmandatory{\arrangeEnclosingBinaries}
    \lsucoptional{\createChangelog}
    \lsucoptional{\addToDocumentation}
  \end{lsucrequires}

  \lsucprohibitsnothing
\end{lsuc}

% ------------------------------------------------------------------------------
\end{license}

%% =============================================================================
%% GPL-3.0 Use Cases

\renewcommand{\ver}{3.0}

\begin{license}{GPL3} 
\licensename{GPL-3.0}
\licensespec{General Public License Version 3}
\licenseabbrev{GPL}
%\licenseversion{3.0}

% ------------------------------------------------------------------------------
\subsection{GPL-\ver-C1: Using the software only for yourself}
\begin{lsuc}{GPL-\ver-C1}
  \linkosuc{01}
  \linkosuc{03L} 
  \linkosuc{03N} 
  \linkosuc{06L}
  \linkosuc{06N}
  \linkosuc{09L}
  \linkosuc{09N}

  \useCaseOne

  \begin{lsucrequiresnothing}
    \lsucitem{You are allowed to use any kind of GPL software in any sense and in
      any context without being obliged to do anything as long as you do not
      give the software to third parties.}
  \end{lsucrequiresnothing}

  \begin{lsucprohibits}
    \lsucitem{\noPatentLitigation}
  \end{lsucprohibits}
\end{lsuc}

% ------------------------------------------------------------------------------
\subsection{GPL-\ver-C2: Passing the unmodified software as independent sources}
\begin{lsuc}{GPL-\ver-C2}
  \linkosuc{02S}
  \linkosuc{05S}

  \useCaseTwo

  \begin{lsucrequires}
    \lsucmandatory{\keepLicenseElements}
    \lsucmandatory{\gplthreeEnsureCopyrightNoticeSource}
    \lsucmandatory{\giveLicense}\passingFilesCorrectly
    \lsucmandatory{\retainCopyrightNotices}
    \lsucoptional{\addToDocumentation}
  \end{lsucrequires}

  \begin{lsucprohibits}
    \lsucitem{\noPatentLitigation}
  \end{lsucprohibits}
\end{lsuc}

% ------------------------------------------------------------------------------
\subsection{GPL-\ver-C3: Passing the unmodified software as independent binaries} 
\begin{lsuc}{GPL-\ver-C3}
  \linkosuc{02B} 
  \linkosuc{05B}

  \useCaseThree

  \begin{lsucrequires}
    \lsucmandatory{\keepLicenseElements}
    \lsucmandatory{\gplthreeEnsureCopyrightNoticeBinary}
    \lsucmandatory{\giveLicense}\passingFilesCorrectly  
    \lsucmandatory{\makeUnmodifiedSourceAvailable}
    \lsucmandatory{\describeHowToGetSource}
    \lsucmandatory{\retainCopyrightNotices}
    \lsucsourcedist{GPL-\ver-C2}
    \lsucoptional{\addToDocumentation}
  \end{lsucrequires}

  \begin{lsucprohibits}
    \lsucitem{\noPatentLitigation}
  \end{lsucprohibits}
\end{lsuc}

% ------------------------------------------------------------------------------
\subsection{GPL-\ver-C4: Passing the unmodified library as embedded sources}
\begin{lsuc}{GPL-\ver-C4}
  \linkosuc{07S} 

  \useCaseFour

  \begin{lsucrequires}
    \lsucmandatory{\keepLicenseElements}
    \lsucmandatory{\gplthreeEnsureCopyrightNoticeSource}
    \lsucmandatory{\giveLicense}\passingFilesCorrectly
    \lsucmandatory{\retainCopyrightNotices}
    \lsucmandatory{\addToCopyrightDialogLib}
    \lsucmandatory{\arrangeEnclosingSources}
    \lsucoptional{\addToDocumentation}
  \end{lsucrequires}

  \begin{lsucprohibits}
    \lsucitem{\noPatentLitigation}
  \end{lsucprohibits}
\end{lsuc}

% ------------------------------------------------------------------------------
\subsection{GPL-\ver-C5: Passing the unmodified library as embedded binaries} 
\begin{lsuc}{GPL-\ver-C5}
  \linkosuc{07B} 

  \useCaseFive

  \begin{lsucrequires}
    \lsucmandatory{\keepLicenseElements}
    \lsucmandatory{\gplthreeEnsureCopyrightNoticeBinary}
    \lsucmandatory{\giveLicense}\passingFilesCorrectly
    \lsucmandatory{\makeAllSourcesAvailable}
    \lsucmandatory{\describeHowToGetSource}
    \lsucmandatory{\addToCopyrightDialogLib}
    \lsucmandatory{\arrangeEnclosingBinaries}
    \lsucmandatory{\retainCopyrightNotices}
    \lsucsourcedist{GPL-\ver-C4}
    \lsucoptional{\addToDocumentation}
  \end{lsucrequires}

  \begin{lsucprohibits}
    \lsucitem{\noPatentLitigation}
  \end{lsucprohibits}
\end{lsuc}

% ------------------------------------------------------------------------------
\subsection{GPL-\ver-C6: Passing a modified program as source code}
\begin{lsuc}{GPL-\ver-C6}
  \linkosuc{04S} 

  \useCaseSix

  \begin{lsucrequires}
    \lsucmandatory{\keepLicenseElements}
    \lsucmandatory{\gplthreeEnsureCopyrightNoticeSource}
    \lsucmandatory{\giveLicense}\passingFilesCorrectly
    \lsucmandatory{\retainCopyrightNotices}
    \lsucmandatory{\addToCopyrightDialogApp}
    \lsucmandatory{\markProgramModifications}
    \lsucmandatory{\arrangeProgramChanges}\howToApplyTheseTerms
    \lsucoptional{\createChangelog}
    \lsucoptional{\addToDocumentation}
  \end{lsucrequires}

  \begin{lsucprohibits}
    \lsucitem{\noPatentLitigation}
  \end{lsucprohibits}
\end{lsuc}

% ------------------------------------------------------------------------------
\subsection{GPL-\ver-C7: Passing a modified program as binary}
\begin{lsuc}{GPL-\ver-C7}
  \linkosuc{04B}

  \useCaseSeven

  \begin{lsucrequires}
    \lsucmandatory{\keepLicenseElements}
    \lsucmandatory{\gplthreeEnsureCopyrightNoticeBinary}
    \lsucmandatory{\giveLicense}\passingFilesCorrectly
    \lsucmandatory{\retainCopyrightNotices}
    \lsucmandatory{\markProgramModifications}
    \lsucmandatory{\addToCopyrightDialogApp}
    \lsucmandatory{\arrangeProgramChanges}\howToApplyTheseTerms
    \lsucmandatory{\makeModifiedSourceAvailable}
    \lsucmandatory{\describeHowToGetSource}
    \lsucsourcedist{GPL-\ver-C6}
    \lsucoptional{\createChangelog}
    \lsucoptional{\addToDocumentation}
  \end{lsucrequires}

  \begin{lsucprohibits}
    \lsucitem{\noPatentLitigation}
  \end{lsucprohibits}
\end{lsuc}

% ------------------------------------------------------------------------------
\subsection{GPL-\ver-C8: Passing a modified library as independent source code}
\begin{lsuc}{GPL-\ver-C8}
  \linkosuc{08S}

  \useCaseEight

  \begin{lsucrequires}
     \lsucmandatory{\keepLicenseElements}
    \lsucmandatory{\gplthreeEnsureCopyrightNoticeSource}
    \lsucmandatory{\giveLicense}\passingFilesCorrectly
    \lsucmandatory{\retainCopyrightNotices}
    \lsucmandatory{\markLibraryModifications}
    \lsucmandatory{\arrangeLibraryChanges}\howToApplyTheseTerms
    \lsucoptional{\createChangelog}
    \lsucoptional{\addToDocumentation}
  \end{lsucrequires}

  \begin{lsucprohibits}
    \lsucitem{\noPatentLitigation}
  \end{lsucprohibits}
\end{lsuc}

% ------------------------------------------------------------------------------
\subsection{GPL-\ver-C9: Passing a modified library as independent binary}
\begin{lsuc}{GPL-\ver-C9}
  \linkosuc{08B}

  \useCaseNine

  \begin{lsucrequires}
    \lsucmandatory{\keepLicenseElements}
    \lsucmandatory{\gplthreeEnsureCopyrightNoticeBinary}
    \lsucmandatory{\giveLicense}\passingFilesCorrectly
    \lsucmandatory{\retainCopyrightNotices}
    \lsucmandatory{\makeModifiedSourceAvailable}
    \lsucmandatory{\describeHowToGetSource}
    \lsucsourcedist{GPL-\ver-C8}
    \lsucmandatory{\markLibraryModifications}
    \lsucmandatory{\arrangeLibraryChanges}\howToApplyTheseTerms
    \lsucoptional{\createChangelog}
    \lsucoptional{\addToDocumentation}
  \end{lsucrequires}

  \begin{lsucprohibits}
    \lsucitem{\noPatentLitigation}
  \end{lsucprohibits}
\end{lsuc}

% ------------------------------------------------------------------------------
\subsection{GPL-\ver-CA: Passing a modified library as embedded source code}
\begin{lsuc}{GPL-\ver-CA}
  \linkosuc{10S}

  \useCaseA

  \begin{lsucrequires}
    \lsucmandatory{\keepLicenseElements}
    \lsucmandatory{\gplthreeEnsureCopyrightNoticeSource}
    \lsucmandatory{\giveLicense}\passingFilesCorrectly
    \lsucmandatory{\retainCopyrightNotices}
    \lsucmandatory{\addToCopyrightDialogLib}
    \lsucmandatory{\markEmbeddedModifications}
    \lsucmandatory{\arrangeEmbeddedChanges}\howToApplyTheseTerms
    \lsucmandatory{\arrangeEnclosingSources}
    \lsucoptional{\createChangelog}
    \lsucoptional{\addToDocumentation}
  \end{lsucrequires}

  \begin{lsucprohibits}
    \lsucitem{\noPatentLitigation}
  \end{lsucprohibits}
\end{lsuc}

% ------------------------------------------------------------------------------
\subsection{GPL-\ver-CB: Passing a modified library as embedded binary}
\begin{lsuc}{GPL-\ver-CB}
  \linkosuc{10B}

  \useCaseB

  \begin{lsucrequires}
    \lsucmandatory{\keepLicenseElements}
    \lsucmandatory{\gplthreeEnsureCopyrightNoticeBinary}
    \lsucmandatory{\giveLicense}\passingFilesCorrectly
    \lsucmandatory{\retainCopyrightNotices}
    \lsucmandatory{\makeAllSourcesAvailable}
    \lsucmandatory{\describeHowToGetSource}
    \lsucsourcedist{GPL-\ver-CA}
    \lsucmandatory{\addToCopyrightDialogLib}
    \lsucmandatory{\markEmbeddedModifications}
    \lsucmandatory{\arrangeEmbeddedChanges}\howToApplyTheseTerms
    \lsucmandatory{\arrangeEnclosingBinaries}
    \lsucoptional{\createChangelog}
    \lsucoptional{\addToDocumentation}
  \end{lsucrequires}

  \begin{lsucprohibits}
    \lsucitem{\noPatentLitigation}
  \end{lsucprohibits}
\end{lsuc}

% ------------------------------------------------------------------------------
\end{license}

%% =============================================================================
%% Discussion

\subsection{Discussions and Explanations}
\label{GPL2Discussion}%
\label{GPL3Discussion}

\newcommand{\gplTwoAndThree}[2]{\footnote{%
    For GPL-2.0 see \cite[cf.][\nopage wp.\ #1]{Gpl20OsiLicense1991a}.\par\noindent 
    For GPL-3.0 see \cite[cf.][\nopage wp.\ #2]{Gpl30OsiLicense2007a}.}} 

The GPL-2.0 allows to \enquote{[\ldots] copy and (to) distribute verbatim copies
of the Program's complete source code as you receive it [...] provided that
you [a] conspicuously and appropriately publish on each copy an appropriate
copyright notice and disclaimer of warranty; [b] keep intact all the notices
that refer to this License and to the absence of any warranty; and [c]
distribute a copy of this License along with the Program.}\citeGPLtwo{§1} The
GPL-2.0 also allows to \enquote{[\ldots] copy and distribute [\ldots]
modifications (of the Program or any portion of it) [\ldots] under the terms
of Section~1}\citeGPLtwo{§2} while it allows to distribute binaries
\enquote{under the terms of Sections 1 and~2}.\citeGPLtwo{§4} But the GPL-2.0
does not require any tasks if you are using the work only for yourself. Thus,
the quoted conditions of \enquote{Section~1} are mandatory for all use cases
concerning the distribution of an GPL-2.0 licensed work (GPL-2.0-C2 --
GPL-2.0-CB)
  
\label{Gpl3ConditionsDistri}
The GPL-3.0 uses a similar structure to establish the same requirements: In §4
it allows to \enquote{[\ldots] convey verbatim copies of the Program's source
code as you receive it [\ldots] provided that you conspicuously and
appropriately publish on each copy an appropriate copyright notice; keep
intact all notices stating that this License and any non-permissive terms
added in accord with section 7 apply to the code; keep intact all notices of
the absence of any warranty; and give all recipients a copy of this License
along with the Program}. §5 also allows to \enquote{[\ldots] convey [\ldots]
modifications [\ldots] under the terms of section 4 [\ldots]} and §6 gives
permission to \enquote{[\ldots] convey a covered work in object form under the
terms of sections of 4 and 5}.\citeGPLthree{§4, §5, §6} In contrast to the
GPL-2.0, the GPL-3.0 explicitly states that one \enquote{[\ldots] may make, run
and propagate covered works that (one) (does) not convey [distribute], without
conditions so long as (the GPL-3.0) license otherwise remains in
force.}\citeGPLthree{§2}
% TODO: rephrase, including hosting the software, and name the conditions:
% exclusively for 'you' and under 'your' direction and control, and no copies
% of your own copyrighted material
Moreover, giving a package to a third party for getting a modified version back
has not to be taken as a case of distribution if the modification has only been
executed on behalf and only for the purpose of the purchaser and if the modified
version is not distributed to any third party.\citeGPLthree{§2} If one collects
all these GPL-3.0 statements together, than one may conclude that the tasks
which fulfill the corresponding GPL-2.0 requirements together also fit the
GPL-3.0 conditions.
  
The GPL-2.0 allows to \enquote{[\ldots] copy and (to) distribute the Program (or
a work based on it [\ldots]) in object code or executable form [\ldots] provided
that you accompany it with the complete corresponding machine-readable source
code [\ldots] on a medium customarily used for software
interchange}.\citeGPLtwo{§3, §3a} As a substitution for this basic condition,
the GPL-2.0 allows to \enquote{accompany} the binary distribution package
\enquote{[\ldots] with a written offer, valid for at least three years, to give
any third party, for a charge no more than your cost of physically performing
source distribution, a complete machine-readable copy of the corresponding
source code [\ldots] on a medium customarily used for software
interchange}.\citeGPLtwo{§3b} The \oslic{} construes the common technique to
download files from the Internet as a distribution \emph{on a medium [being
today] customarily used for software interchange}. Therefore, the \oslic{} requires
for all open source use cases that refer to the distribution of binaries
(GPL-2.0-C3, GPL-2.0-C7, GPL-2.0-C9, GPL-2.0-CB) to make the corresponding
source code of the library itself accessible via an Internet repository under
your own control. 
  
\label{Gpl3CondCopyleft}
The GPL-3.0 also explicitly requires to make the source code accessible in case
of distributing binaries. But opposite to the GPL-2.0, the GPL-3.0 explicitly
offers the option of giving \enquote{[\ldots] access to copy the Corresponding
Source from a network server at no charge} as a means to fulfill the
conditions.\citeGPLthree{§6 and §6b} So again, the tasks which ensure to act in
accordance to the GPL-2.0 license in case of distributing binaries, also fulfill
the conditions of the GPL-3.0.

The weakness that in this case \enquote{third parties [which have received the
binaries] are not compelled to copy the source code [\ldots]} is a concession
made by the GPL-2.0.\citeGPLtwo{§3, at the end} But the necessity to offer the
source code via a repository controlled by yourself may generally not be
circumvented: The GPL-2.0 allows to redistribute a link to an external source
code repository only in case of \enquote{noncommercial
distributions}.\citeGPLtwo{§3c} 
  
Both, the GPL-2.0 and the GPL-3.0 allow you to \enquote{[\ldots] modify your
copy or copies of the Program or any portion of it [\ldots] and (to) copy and
distribute such modifications [\ldots]} only under very similar restrictions and
conditions:\citeGPLtwo{§2} 
\begin{itemize}
\item First, modified files must be marked as modifications and the date of the
  modification.\gplTwoAndThree{§2}{§5} These conditions must be respected by all
  open source use cases concerning the distribution of the modified work
  [GPL-2.0-C6/GPL-3.0C6 -- GPL-2.0-C9/GPL-3.0-C9], because even if one primarily
  intends to distribute binaries, one has also to deliver the source code. The
  \oslic{} captures this requirement in the mandatory condition to mark each
  modified file and the voluntary condition to update / generate a general
  changelog.
    
\item Second, both versions of the GPL require that all copies of the modified
  software which are using an interactive interface or a method to display
  messages must \enquote{[\ldots] print or display an announcement including an
  appropriate copyright notice and a notice that there is no warranty [\ldots]
  and that users may redistribute the program under these conditions, and
  telling the user how to view a copy of this License.}\gplTwoAndThree{§2c}{§5d}
  The \oslic{} rewrites this condition in the form that the work shall let its
  copyright dialog clearly reproduce the content of the existing copyright
  notices, the software name, a link to its homepage, the respective disclaimer
  of warranty, and a link to the GPL-2.0-file (or GPL-3.0-file, resp.), which
  has to be delivered together with the software.
  % TODO: actually, the task does not refer to the _file_
  These conditions have to be respected if one redistributes the received and
  then modified programs (GPL-2.0-C6, GPL-2.0-C7, GPL-3.0-C6, GPL-3.0-C7) or if
  one distributes one's own programs which are using (modified) libraries as
  embedded components (GPL-2.0-CA, GPL-2.0-CB, GPL-3.0-CA, GPL-3.0-CB). For
  those open source use cases that concern the redistribution of received and
  modified libraries, etc., the \oslic{} does not mention these requirements
  because libraries, plugins, or snippets normally do not have their own
  copyright dialogs.  
    
\item Third, the GPL requires to \enquote{ [\ldots] cause any work (being
  distributed or published), that in whole or in part contains or is derived
  from the Program or any part thereof, to be licensed as a whole at no charge
  to all third parties under the terms of this (GPL.)}\gplTwoAndThree{§2b}{§5c}
  This requirement does not depend of the form in which the software is
  distributed. The \oslic{} adopts this statement in the following way:
  \begin{itemize}
  \item For all open source use cases which concern the distribution (GPL-2.0-C2
    \ldots GPL-2.0-CB, GPL-3.0-C2 \ldots GPL-3.0-CB), the \oslic{} rewrites this
    condition as the mandatory requirement to retain all existing licensing
    elements.
      
  \item For all use cases which deal with the distribution of a modified version
    of the software (GPL-2.0-C6 \ldots GPL-2.0-CB, GPL-3.0-C6 \ldots
    GPL-3.0-CB), the OSliC adds the requirement to organize the modifications in
    a way that they are covered by the respective GPL-2.0 or GPL-3.0 licensing
    statements.
      
  \item For the use case which deal with the distribution of an embedded library
    (GPL-2.0-C4, GPL-2.0-C5, GPL-2.0-CA, GPL-2.0-CB, GPL-3.0-C4, GPL-3.0-C5,
    GPL-3.0-CA, GPL-3.0-CB) the \oslic{} requires also to license the on-top
    development under the terms of the respective GPL-2.0 or GPL-3.0 license.
    \end{itemize}
   
\item Finally, as parts of those task lists which concern the distribution in
  the form of binaries, the \oslic{} reminds the reader also to execute the
  corresponding source code use cases because distributing the binaries without
  making the corresponding sources accessible is not allowed by the GPL.
\end{itemize}

And a last issue should be addressed here. It concerns the problem of
granularity.

The GPL-3.0 allows \enquote{[\ldots] to convey a covered work in object code
form [\ldots] provided that [one] also conveys the [\ldots] Corresponding
Source [\ldots]}\citeGPLthree{§6}. For understanding the scope of the sources
one has to convey, one must known, what the term \emph{Corresponding Source}
means. Fortunately, the GPL-3.0 assists its readers to understand this term in
the right way:

\begin{itemize}
  \item  \enquote{The \enquote{Corresponding Source} for a work in object code
  form means all the source code needed to generate, install, and (for an
  executable work) run the object code and to modify the work, including scripts
  to control those activities.\footcite[cf.][\nopage wp.
  §1]{Gpl30OsiLicense2007a}} Thus, if one took this statements seriously, one
  would have to \enquote{provide access to} the complete software stack of the
  executed AGPL program, just down to the glibc. But the GPL does not want to
  be to greedy. Therefore it limits the scope:
  \item To limit the sope, the GPL states, that the \emph{Corresponding Source}
  \enquote{[\ldots] does not include the work's System Libraries, or
  general-purpose tools or generally available free programs which are used
  unmodified in performing those activities but which are not part of the
  work}\footcite[cf.][\nopage wp. §1]{Gpl30OsiLicense2007a}. Unfortunately, one
  now has to analyze, what the term \emph{System Libraries} means, if one wants
  to understand this rule correctly.
  \item Therefore, the GPl says also, that \enquote{the \enquote{System
  Libraries} of an executable work include anything, other than the work as a
  whole, that (a) is included in the normal form of packaging a Major Component,
  but which is not part of that Major Component, and (b) serves only to enable
  use of the work with that Major Component, or to implement a Standard
  Interface for which an implementation is available to the public in source
  code form.\footcite[cf.][\nopage wp. §1]{Gpl30OsiLicense2007a}}. And for
  understing this sentence adequately, one has to know, what a \emph{Major Component}
  is.
  \item So, finally, the GPL defines as \enquote{enquote{Major Component}
  [\ldots as] a major essential component (kernel, window system, and so on) of
  the specific operating system (if any) on which the executable work runs, or a
  compiler used to produce the work, or an object code interpreter used to run
  it\footcite[cf.][\nopage wp. §1]{Gpl30OsiLicense2007a}}.
\end{itemize}

Based on these specifications, one can give some rule of thumbs concerning the
question down to which level one has to give access to the corresponding source
code of an conveyed GPL binary program:
\begin{itemize}
  \item If one conveys a GPL licensed binary program, then one has
  also to deliver the code of
  \begin{itemize}
  \item the dlivered program itself
  \item every modified embedded component of that program
  \item every not freely accessible embedded component of that program
  \item all not freely accessible tools, scripts, data which are necessary to
  compile the sources of the program in a freely accessible compilation /
  developement environment
  \end{itemize}
  But it is not necessary to deliver the code of unmodified standard libraries,
  compilers, or tools which can freely be downloaded from their standard
  repository.
  \item If one conveys a GPL licensed script, then one has also to deliver the
  code of
  \begin{itemize}
  \item every modified embedded script component included by the main script
  \item every not freely accessible embedded script component included by the main script
  \item all not freely accessible tools, scripts, data which are necessary to
  to let that main script be executed by a freely accessible interpreter
  \item the interpreter itself if it is not freely accessible.
  \end{itemize}
  But it is not necessary to give access to unmodified standard script
  libraries, interpreters, or tools which can freely be downloaded from their
  standard repository.
\end{itemize}


  
%\bibliography{../../../bibfiles/oscResourcesEn}

% Local Variables:
% mode: latex
% fill-column: 80
% End:
}
{% Telekom osCompendium 'for being included' snippet template
%
% (c) Karsten Reincke, Deutsche Telekom AG, Darmstadt 2011
%
% This LaTeX-File is licensed under the Creative Commons Attribution-ShareAlike
% 3.0 Germany License (http://creativecommons.org/licenses/by-sa/3.0/de/): Feel
% free 'to share (to copy, distribute and transmit)' or 'to remix (to adapt)'
% it, if you '... distribute the resulting work under the same or similar
% license to this one' and if you respect how 'you must attribute the work in
% the manner specified by the author ...':
%
% In an Internet based reuse please link the reused parts to www.telekom.com and
% mention the original authors and Deutsche Telekom AG in a suitable manner. In
% a paper-like reuse please insert a short hint to www.telekom.com and to the
% original authors and Deutsche Telekom AG into your preface. For normal
% quotations please use the scientific standard to cite.
%
% [ Framework derived from 'mind your Scholar Research Framework' 
%   mycsrf (c) K. Reincke 2012 CC BY 3.0  http://mycsrf.fodina.de/ ]
%


%% use all entries of the bibliography
%\nocite{*}

\section{LGPL licensed software}

Both versions of the GNU Lesser General Public License explicitly distinguish
the distribution of the source code from that of the binaries: On the one hand,
the LGPL-2.1 mainly talks about copying and distributing the source
code.\citeLGPLtwo{§1, §2, §5, §6} But it also directly mentions the specific
conditions for \enquote{[\ldots] (copying) and (distributing) the Library
[\ldots] in object code or executable form [\ldots]}\citeLGPLtwo{§4} On the other
hand, the LGPL-3.0 and the GPL-3.0---which have to be considered together
because the GPL-3.0 is included into the LGPL-3.0\citeLGPLthree{just before
  §0}--- treat the distribution of source code and the distribution of object
code as different aspects of the same phenomenon%
  \footnote{The GPL-3.0 contains a specific section named 
    \enquote{Conveying Non-Source Forms} which describes the conditions to
    \enquote{[\ldots] convey a covered work in object code form [\ldots]}
    (\cite[cf.][\nopage wp.\ §6]{Gpl30OsiLicense2007a}), while the LGPL-3.0
    explicitly deals with the 
    \enquote{object code incorporating material from (the) library header files}
    (\cite[cf.][\nopage wp.\ §3]{Lgpl30OsiLicense2007a}).}
Additionally, LGPL-2.1 and LGPL-3.0 mainly talk about copying and distributing
the software; the private use is almost complete unspecified.%
  \footnote{The LGPL-2.1 lists its `restrictions' only with respect to the act
    of copying and distributing \enquote{copies of the library} 
    (\cite[cf.][\nopage wp.\ §1, §2, §4 et passim]{Lgpl21OsiLicense1999a}) 
    while the GPL-3.0 explicitly specifies that one \enquote{[\ldots] may make,
    run and propagate covered works that (one does) not convey, without
    conditions so long as (the) license otherwise remains in force} 
    (\cite[cf.][\nopage wp.\ §2]{Gpl30OsiLicense2007a}).}
Finally, the LGPL-2.1 and the LGPL-3.0 aim for the same results and share the
same spirit by requiring nearly the same license fulfilling tasks. Therefore it
seems appropriate to cover both versions in one chapter%
  \footnote{The exception concerns the distribution of a modified program,
    application, or server under the terms of the LGPL} 
and to offer the same LGPL specific open source use case structure%
  \footnote{For details of the general OSUC finder $\rightarrow$ \oslic,
    pp.\ \pageref{OsucTokens} and \pageref{OsucDefinitionTree}} 
for finding the corresponding task lists: 
 
%% ============================================================================= 
%% Use-Case Finder

\gplUseCaseFinder{LGPL}{2.1}{3.0}

%% ============================================================================= 
%% Common Building Blocks

% ------------------------------------------------------------------------------
% Common description of license specific use cases

\newcommand{\useCaseOne}{%
  \gtbUseCaseOne{LGPL-\ver}
  \gtbCoversOne{LGPL-\ver}}

\newcommand{\useCaseTwo}{%
  \gtbUseCaseTwo{LGPL-\ver}
  \gtbCoversTwo{LGPL-\ver}}
 
\newcommand{\useCaseThree}{%
  \gtbUseCaseThree{LGPL-\ver}
  \gtbCoversThree{LGPL-\ver}}

\newcommand{\useCaseFour}{%
  \gtbUseCaseFour{LGPL-\ver}{an}
  \gtbCoversFour{LGPL-\ver}}
 
\newcommand{\useCaseFive}{%
  \gtbUseCaseFive{LGPL-\ver}{an}
  \gtbCoversFive{LGPL-\ver}}

\newcommand{\useCaseSix}{%
  \gtbUseCaseSix{LGPL-\ver}{an}
  \gtbCoversSix{LGPL-\ver}}

\newcommand{\useCaseSeven}{%
  \gtbUseCaseSeven{LGPL-\ver}{an}
  \gtbCoversSeven{LGPL-\ver}}

\newcommand{\useCaseEight}{%
  \gtbUseCaseEight{LGPL-\ver}{an}
  \gtbCoversEight{LGPL-\ver}}

\newcommand{\useCaseNine}{%
  \gtbUseCaseNine{LGPL-\ver}{an}
  \gtbCoversNine{LGPL-\ver}}

\newcommand{\useCaseA}{%
  \gtbUseCaseA{LGPL-\ver}{an}
  \gtbCoversA{LGPL-\ver}}

\newcommand{\useCaseB}{%
  \gtbUseCaseB{LGPL-\ver}{an}
  \gtbCoversB{LGPL-\ver}}

% ------------------------------------------------------------------------------
% Common Text Blocks from 0600-common-text-blocks.tex

\newcommand{\keepLicensingElements}{\gtbKeepLicenseElements{LGPL-\ver}}
\newcommand{\giveLicense}{\gtbGiveLicense{LGPL-\ver}}
\newcommand{\addToDocumentation}{\gtbAddToDocumentation{LGPL-\ver}} 
\newcommand{\makeUnmodifiedSourceAvailable}{\gtbMakeUnmodifiedSourceAvailable{LGPL-\ver}}
\newcommand{\makeModifiedSourceAvailable}{\gtbMakeModifiedSourceAvailable{LGPL-\ver}}
\newcommand{\makeEmbeddedSourceAvailable}{\gtbMakeEmbeddedSourcesAvailable{LGPL-\ver}}
\newcommand{\describeHowToGetSource}{\gtbDescribeHowToGetSource{LGPL-\ver}}
\newcommand{\createChangelog}{\gtbCreateChangelog{LGPL-\ver}}
\newcommand{\retainCopyrightNotices}{\gtbKeepCopyrightNotices{LGPL-\ver}}
\newcommand{\markEmbeddedModifications}{\gtbMarkEmbeddedModifications{LGPL-\ver}}
\newcommand{\markLibraryModifications}{\gtbMarkLibraryModifications{LGPL-\ver}}
\newcommand{\markProgramModifications}{\gtbMarkProgramModifications{LGPL-\ver}}
\newcommand{\lgpltwoEnsureCopyrightNoticeSource}{\gtbVTwoCopyrightNotice{LGPL-2.1}{source code}}
\newcommand{\lgpltwoEnsureCopyrightNoticeBinary}{\gtbVTwoCopyrightNotice{LGPL-2.1}{binary}}
\newcommand{\lgplthreeEnsureCopyrightNoticeSource}{\gtbVThreeCopyrightNotice{LGPL-3.0}{source code}}
\newcommand{\lgplthreeEnsureCopyrightNoticeBinary}{\gtbVThreeCopyrightNotice{LGPL-3.0}{binary}}
\newcommand{\arrangeProgramChanges}{\gtbArrangeProgramChanges{LGPL-\ver}}
\newcommand{\arrangeLibraryChanges}{\gtbArrangeLibraryChanges{LGPL-\ver}}
\newcommand{\arrangeEmbeddedChanges}{\gtbArrangeEmbeddedChanges{LGPL-\ver}}
\newcommand{\howToApplyTheseTerms}{\gtbHowToApplyTheseTerms{LGPL-\ver}}
\newcommand{\noPatentLitigation}{\gtbNoPatentLitigation{LGPL-\ver}}
\newcommand{\addToCopyrightDialogLibWeak}{\gtbAddToCopyrightDialogWeakCopyleft{LGPL-\ver}}
\newcommand{\addToCopyrightDialogApp}{\gtbAddToCopyrightDialogApp{LGPL-\ver}}

% ------------------------------------------------------------------------------
% Relicense programs under GPL (for LGPL-2.1 only)

\newcommand{\relicenseUnderGPL}{Change all the notices in all files that refer
  to the LGPL-2.1, so that they refer to the ordinary GNU General Public
  License, version 2, instead of to this License.}
  
% ------------------------------------------------------------------------------
% Modified code must be a library

\newcommand{\dontDistributeProgram}{%
  to modify the received work in a way that the resulting 
  \enquote{modified work} is no longer a software library (but a program).
  \textbf{You are not allowed to distribute a modified program under the
    terms of LGPL-2.1.}}

\newcommand{\mustBeALibrary}{%
  \footnote{The LGPL-2.1 explictly requires that 
    \enquote{the modified work must itself be a software library} 
    (\cite[cf.][\nopage wp.\ §2a]{Lgpl21OsiLicense1999a}). 
    For details $\rightarrow$ \oslic, p.\ \pageref{para:libislib}}} 

% ------------------------------------------------------------------------------
% Make sure user can relink the application with a modified library

\newcommand{\allowRelinking}{Either distribute the on-top development and the
  library in the form of dynamically linkable parts or distribute the statically
  linked application together with a written offer, valid for at least three
  years, to give the user all object-files of the on-top development and the
  library, so that he can relink the application himself.}

% ------------------------------------------------------------------------------
% RPD: TODO: What does this actually mean?

\newcommand{\keepStructuralIndependence}{Maintain the structural independence of
  the library.} 

%% ============================================================================= 
%% LGPL-2.1 Use Cases

\newcommand{\ver}{2.1}

\begin{license}{LGPL2} % ends at end of file
\licensename{LGPL-\ver}
\licensespec{GNU Lesser General Public License \ver}
%\licenseversion{2.1}
\licenseabbrev{LGPL}

% ------------------------------------------------------------------------------
\subsection{LGPL-\ver-C1: Using the software only for yourself}
\begin{lsuc}{LGPL-\ver-C1}
  \linkosuc{01}
  \linkosuc{03L} 
  \linkosuc{03N} 
  \linkosuc{06L}
  \linkosuc{06N}
  \linkosuc{09L}
  \linkosuc{09N}
  
  \useCaseOne

  \begin{lsucrequiresnothing}
    \lsucitem{You are allowed to use any kind of LGPL-\ver{} licensed software
      in any sense and in any context without being obliged to do anything as
      long as you do not give the software to third parties.}
  \end{lsucrequiresnothing}

  \lsucprohibitsnothing
\end{lsuc}

% ------------------------------------------------------------------------------
\subsection{LGPL-\ver-C2: Passing the unmodified software as independent source code}
\begin{lsuc}{LGPL-\ver-C2}
  \linkosuc{02S} 
  \linkosuc{05S} 

  \useCaseTwo

  \begin{lsucrequires}
    \lsucmandatory{\keepLicensingElements}
    \lsucmandatory{\lgpltwoEnsureCopyrightNoticeSource}
    \lsucmandatory{\giveLicense}\passingFilesCorrectly
    \lsucmandatory{\retainCopyrightNotices}
    \lsucoptional{\addToDocumentation}
  \end{lsucrequires}

  \lsucprohibitsnothing
\end{lsuc}

% ------------------------------------------------------------------------------
\subsection{LGPL-\ver-C3: Passing the unmodified software as independent binaries}
\begin{lsuc}{LGPL-\ver-C3} 
  \linkosuc{02B} 
  \linkosuc{05B} 

  \useCaseThree

  \begin{lsucrequires}
    \lsucmandatory{\keepLicensingElements}
    \lsucmandatory{\lgpltwoEnsureCopyrightNoticeBinary}
    \lsucmandatory{\giveLicense}\passingFilesCorrectly
    \lsucmandatory{\makeUnmodifiedSourceAvailable}
    \lsucmandatory{\describeHowToGetSource}
    \lsucoptional{\retainCopyrightNotices}
    \lsucsourcedist{LGPL-\ver-C2}
    \lsucoptional{\addToDocumentation}
  \end{lsucrequires}

  \lsucprohibitsnothing
\end{lsuc}

% ------------------------------------------------------------------------------
\subsection{LGPL-\ver-C4: Passing the unmodified library as embedded source code}
\begin{lsuc}{LGPL-\ver-C4}
  \linkosuc{07S} 

  \useCaseFour

  \begin{lsucrequires}
    \lsucmandatory{\keepLicensingElements}
    \lsucmandatory{\lgpltwoEnsureCopyrightNoticeSource}
    \lsucmandatory{\giveLicense}\passingFilesCorrectly
    \lsucoptional{\addToDocumentation}
    \lsucoptional{\retainCopyrightNotices}
  \end{lsucrequires}

  \lsucprohibitsnothing
\end{lsuc}

% ------------------------------------------------------------------------------
\subsection{LGPL-\ver-C5: Passing the unmodified library as embedded binaries}
\begin{lsuc}{LGPL-\ver-C5}
  \linkosuc{07B} 

  \useCaseFive

  \begin{lsucrequires}
    \lsucmandatory{\keepLicensingElements}
    \lsucmandatory{\lgpltwoEnsureCopyrightNoticeBinary}
    \lsucmandatory{\giveLicense}\passingFilesCorrectly
    \lsucmandatory{\makeUnmodifiedSourceAvailable}
    \lsucmandatory{\describeHowToGetSource}
    \lsucmandatory{\allowRelinking}
    \lsucsourcedist{LGPL-\ver-C4}
    \lsucoptional{\addToDocumentation}
    \lsucoptional{\retainCopyrightNotices}
  \end{lsucrequires}

  \lsucprohibitsnothing
\end{lsuc}


% ------------------------------------------------------------------------------
\subsection{LGPL-2.1-C6: Passing a modified program as source code}
\begin{lsuc}{LGPL-2.1-C6}
  \linkosuc{04S} 
  
  \useCaseSix

  \begin{lsucrequires}
    \lsucmandatory{\relicenseUnderGPL}
  \end{lsucrequires}

  \begin{lsucprohibits}
    \lsucitem{\dontDistributeProgram}\mustBeALibrary
  \end{lsucprohibits}

\end{lsuc}

% ------------------------------------------------------------------------------
\subsection{LGPL-2.1-C7: Passing a modified program as binary}
\begin{lsuc}{LGPL-2.1-C7}
  \linkosuc{04B} 

  \useCaseSeven

  \begin{lsucrequires}
    \lsucmandatory{\relicenseUnderGPL}
  \end{lsucrequires}

  \begin{lsucprohibits}
    \lsucitem{\dontDistributeProgram}\mustBeALibrary
  \end{lsucprohibits}

\end{lsuc}

% ------------------------------------------------------------------------------
\subsection{LGPL-\ver-C8: Passing a modified library as independent source code}
\begin{lsuc}{LGPL-\ver-C8}
  \linkosuc{08S}

  \useCaseEight

  \begin{lsucrequires}
    \lsucmandatory{\keepLicensingElements}
    \lsucmandatory{\lgpltwoEnsureCopyrightNoticeSource}
    \lsucmandatory{\giveLicense}\passingFilesCorrectly
    \lsucmandatory{\markLibraryModifications}
    \lsucmandatory{\arrangeLibraryChanges}\howToApplyTheseTerms
    \lsucoptional{\createChangelog}  
    \lsucoptional{\addToDocumentation}
    \lsucoptional{\retainCopyrightNotices}
  \end{lsucrequires}

  \begin{lsucprohibits}
    \lsucitem{to modify the library in a way that it is no longer a library}
  \end{lsucprohibits}
\end{lsuc}

% ------------------------------------------------------------------------------
\subsection{LGPL-\ver-C9: Passing a modified library as independent binary}
\begin{lsuc}{LGPL-\ver-C9}
  \linkosuc{08B}

  \useCaseNine

  \begin{lsucrequires}
    \lsucmandatory{\keepLicensingElements}
    \lsucmandatory{\lgpltwoEnsureCopyrightNoticeBinary}
    \lsucmandatory{\giveLicense}\passingFilesCorrectly
    \lsucmandatory{\makeModifiedSourceAvailable}
    \lsucmandatory{\describeHowToGetSource}
    \lsucsourcedist{LGPL-\ver-C8}
    \lsucmandatory{\markLibraryModifications}
    \lsucmandatory{\arrangeLibraryChanges}\howToApplyTheseTerms
    \lsucoptional{\createChangelog}
    \lsucoptional{\addToDocumentation}  
    \lsucoptional{\retainCopyrightNotices}  
  \end{lsucrequires}

  \begin{lsucprohibits}
    \lsucitem{to modify the library in a way that it is no longer a library.}
  \end{lsucprohibits}
\end{lsuc}

% ------------------------------------------------------------------------------
\subsection{LGPL-\ver-CA: Passing a modified library as embedded source code}
\begin{lsuc}{LGPL-\ver-CA}
  \linkosuc{10S}

  \useCaseA

  \begin{lsucrequires}
    \lsucmandatory{\keepLicensingElements}
    \lsucmandatory{\lgpltwoEnsureCopyrightNoticeSource}
    \lsucmandatory{\giveLicense}\passingFilesCorrectly
    \lsucmandatory{\markEmbeddedModifications}
    \lsucmandatory{\arrangeEmbeddedChanges}\howToApplyTheseTerms
    \lsucmandatory{\keepStructuralIndependence}
    \lsucmandatory{\addToCopyrightDialogLibWeak}
    \lsucoptional{\createChangelog}  
    \lsucoptional{\addToDocumentation}
    \lsucoptional{\retainCopyrightNotices}
  \end{lsucrequires}

  \begin{lsucprohibits}
    \lsucitem{to modify the library in a way that it is no longer a library.}
  \end{lsucprohibits}
\end{lsuc}

% ------------------------------------------------------------------------------
\subsection{LGPL-\ver-CB: Passing a modified library as embedded binary}
\begin{lsuc}{LGPL-\ver-CB}
  \linkosuc{10B}

  \useCaseB

  \begin{lsucrequires}
    \lsucmandatory{\keepLicensingElements}
    \lsucmandatory{\lgpltwoEnsureCopyrightNoticeBinary}
    \lsucmandatory{\giveLicense}\passingFilesCorrectly
    \lsucmandatory{\makeEmbeddedSourceAvailable}
    \lsucmandatory{\describeHowToGetSource}
    \lsucsourcedist{LGPL-\ver-CA}
    \lsucmandatory{\markEmbeddedModifications}
    \lsucmandatory{\arrangeEmbeddedChanges}\howToApplyTheseTerms
    \lsucmandatory{\keepStructuralIndependence}
    \lsucmandatory{\addToCopyrightDialogLibWeak}
    \lsucmandatory{\allowRelinking}
    \lsucoptional{\createChangelog}  
    \lsucoptional{\addToDocumentation}
    \lsucoptional{\retainCopyrightNotices}
  \end{lsucrequires}

  \begin{lsucprohibits}
    \lsucitem{to modify the library in a way that it is no longer a library.}
  \end{lsucprohibits}
\end{lsuc}

% ------------------------------------------------------------------------------
\end{license}

%% ============================================================================= 
%% LGPL-3.0 Use Cases

\renewcommand{\ver}{3.0}

\begin{license}{LGPL3} % ends at end of file
\licensename{LGPL-\ver}
\licensespec{GNU Lesser General Public License \ver}
%\licenseversion{3.0}
\licenseabbrev{LGPL}

% ------------------------------------------------------------------------------
\subsection{LGPL-\ver-C1: Using the software only for yourself}
\begin{lsuc}{LGPL-\ver-C1}
  \linkosuc{01}
  \linkosuc{03L} 
  \linkosuc{03N} 
  \linkosuc{06L}
  \linkosuc{06N}
  \linkosuc{09L}
  \linkosuc{09N}
  
  \useCaseOne

  \begin{lsucrequiresnothing}
    \lsucitem{You are allowed to use any kind of LGPL-\ver{} licensed software
      in any sense and in any context without being obliged to do anything as
      long as you do not give the software to third parties.}
  \end{lsucrequiresnothing}

  \begin{lsucprohibits}
    \lsucitem{\noPatentLitigation}
  \end{lsucprohibits}
\end{lsuc}

% ------------------------------------------------------------------------------
\subsection{LGPL-\ver-C2: Passing the unmodified software as independent source code}
\begin{lsuc}{LGPL-\ver-C2}
  \linkosuc{02S} 
  \linkosuc{05S} 

  \useCaseTwo

  \begin{lsucrequires}
    \lsucmandatory{\keepLicensingElements}
    \lsucmandatory{\lgplthreeEnsureCopyrightNoticeSource}
    \lsucmandatory{\giveLicense}\passingFilesCorrectly
    \lsucoptional{\addToDocumentation}
    \lsucoptional{\retainCopyrightNotices}
  \end{lsucrequires}

  \begin{lsucprohibits}
    \lsucitem{\noPatentLitigation}
  \end{lsucprohibits}
\end{lsuc}

% ------------------------------------------------------------------------------
\subsection{LGPL-\ver-C3: Passing the unmodified software as independent binaries}
\begin{lsuc}{LGPL-\ver-C3} 
  \linkosuc{02B} 
  \linkosuc{05B} 

  \useCaseThree

  \begin{lsucrequires}
    \lsucmandatory{\keepLicensingElements}
    \lsucmandatory{\lgplthreeEnsureCopyrightNoticeBinary}
    \lsucmandatory{\giveLicense}\passingFilesCorrectly
    \lsucmandatory{\makeUnmodifiedSourceAvailable}
    \lsucmandatory{\describeHowToGetSource}
    \lsucmandatory{\retainCopyrightNotices}
    \lsucsourcedist{LGPL-\ver-C2}
    \lsucoptional{\addToDocumentation}
  \end{lsucrequires}

  \begin{lsucprohibits}
    \lsucitem{\noPatentLitigation}
  \end{lsucprohibits}
\end{lsuc}

% ------------------------------------------------------------------------------
\subsection{LGPL-\ver-C4: Passing the unmodified library as embedded source code}
\begin{lsuc}{LGPL-\ver-C4}
  \linkosuc{07S} 

  \useCaseFour

  \begin{lsucrequires}
    \lsucmandatory{\keepLicensingElements}
    \lsucmandatory{\lgplthreeEnsureCopyrightNoticeSource}
    \lsucmandatory{\giveLicense}\passingFilesCorrectly
    \lsucmandatory{\retainCopyrightNotices}
    \lsucmandatory{\addToCopyrightDialogLibWeak}
    \lsucoptional{\addToDocumentation}
  \end{lsucrequires}

  \begin{lsucprohibits}
    \lsucitem{\noPatentLitigation}
  \end{lsucprohibits}
\end{lsuc}

% ------------------------------------------------------------------------------
\subsection{LGPL-\ver-C5: Passing the unmodified library as embedded binaries}
\begin{lsuc}{LGPL-\ver-C5}
  \linkosuc{07B} 

  \useCaseFive

  \begin{lsucrequires}
    \lsucmandatory{\keepLicensingElements}
    \lsucmandatory{\lgplthreeEnsureCopyrightNoticeBinary}
    \lsucmandatory{\giveLicense}\passingFilesCorrectly
    \lsucmandatory{\makeUnmodifiedSourceAvailable}
    \lsucmandatory{\describeHowToGetSource}
    \lsucmandatory{\addToCopyrightDialogLibWeak}
    \lsucmandatory{\allowRelinking}
    \lsucsourcedist{LGPL-\ver-C4}
    \lsucoptional{\addToDocumentation}
    \lsucoptional{\retainCopyrightNotices}
  \end{lsucrequires}

  \begin{lsucprohibits}
    \lsucitem{\noPatentLitigation}
  \end{lsucprohibits}
\end{lsuc}


% ------------------------------------------------------------------------------
\subsection{LGPL-3.0-C6: Passing a modified program as source code}
\begin{lsuc}{LGPL-3.0-C6}
  \linkosuc{04S} 
  
  \useCaseSix

  \begin{lsucrequires}
    \lsucmandatory{\keepLicensingElements}
    \lsucmandatory{\lgplthreeEnsureCopyrightNoticeSource}
    \lsucmandatory{\giveLicense}\passingFilesCorrectly
    \lsucmandatory{\retainCopyrightNotices}
    \lsucmandatory{\addToCopyrightDialogApp}
    \lsucmandatory{\markProgramModifications}
    \lsucmandatory{\arrangeProgramChanges}\howToApplyTheseTerms
    \lsucoptional{\createChangelog}
    \lsucoptional{\addToDocumentation}
  \end{lsucrequires}
 
  \begin{lsucprohibits}
    \lsucitem{\noPatentLitigation}
  \end{lsucprohibits}
\end{lsuc}

% ------------------------------------------------------------------------------
\subsection{LGPL-3.0-C7: Passing a modified program as binary}
\begin{lsuc}{LGPL-3.0-C7}
  \linkosuc{04B} 

  \useCaseSeven

  \begin{lsucrequires}
    \lsucmandatory{\keepLicensingElements}
    \lsucmandatory{\lgplthreeEnsureCopyrightNoticeBinary}
    \lsucmandatory{\giveLicense}\passingFilesCorrectly
    \lsucmandatory{\retainCopyrightNotices}
    \lsucmandatory{\markProgramModifications}
    \lsucmandatory{\addToCopyrightDialogApp}
    \lsucmandatory{\arrangeProgramChanges}\howToApplyTheseTerms
    \lsucmandatory{\makeModifiedSourceAvailable}
    \lsucmandatory{\describeHowToGetSource}  
    \lsucsourcedist{LGPL-\ver-C4}
    \lsucoptional{\createChangelog}  
    \lsucoptional{\addToDocumentation}
  \end{lsucrequires}

  \begin{lsucprohibits}
    \lsucitem{\noPatentLitigation}
  \end{lsucprohibits}
\end{lsuc}

% ------------------------------------------------------------------------------
\subsection{LGPL-\ver-C8: Passing a modified library as independent source code}
\begin{lsuc}{LGPL-\ver-C8}
  \linkosuc{08S}

  \useCaseEight

  \begin{lsucrequires}
    \lsucmandatory{\keepLicensingElements}
    \lsucmandatory{\lgplthreeEnsureCopyrightNoticeSource}
    \lsucmandatory{\giveLicense}\passingFilesCorrectly
    \lsucmandatory{\retainCopyrightNotices}
    \lsucmandatory{\markLibraryModifications}
    \lsucmandatory{\arrangeLibraryChanges}\howToApplyTheseTerms
    \lsucoptional{\createChangelog}  
    \lsucoptional{\addToDocumentation}
  \end{lsucrequires}

  \begin{lsucprohibits}
    \lsucitem{\noPatentLitigation}
  \end{lsucprohibits}
\end{lsuc}

% ------------------------------------------------------------------------------
\subsection{LGPL-\ver-C9: Passing a modified library as independent binary}
\begin{lsuc}{LGPL-\ver-C9}
  \linkosuc{08B}

  \useCaseNine

  \begin{lsucrequires}
    \lsucmandatory{\keepLicensingElements}
    \lsucmandatory{\lgplthreeEnsureCopyrightNoticeBinary}
    \lsucmandatory{\giveLicense}\passingFilesCorrectly
    \lsucmandatory{\retainCopyrightNotices}  
    \lsucmandatory{\makeModifiedSourceAvailable}
    \lsucmandatory{\describeHowToGetSource}
    \lsucsourcedist{LGPL-\ver-C8}
    \lsucmandatory{\markLibraryModifications}
    \lsucmandatory{\arrangeLibraryChanges}\howToApplyTheseTerms
    \lsucoptional{\createChangelog}
    \lsucoptional{\addToDocumentation}  
  \end{lsucrequires}

  \begin{lsucprohibits}
    \lsucitem{\noPatentLitigation}
  \end{lsucprohibits}
\end{lsuc}

% ------------------------------------------------------------------------------
\subsection{LGPL-\ver-CA: Passing a modified library as embedded source code}
\begin{lsuc}{LGPL-\ver-CA}
  \linkosuc{10S}

  \useCaseA

  \begin{lsucrequires}
    \lsucmandatory{\keepLicensingElements}
    \lsucmandatory{\lgplthreeEnsureCopyrightNoticeSource}
    \lsucmandatory{\giveLicense}\passingFilesCorrectly
    \lsucmandatory{\addToCopyrightDialogLibWeak}
    \lsucmandatory{\markEmbeddedModifications}
    \lsucmandatory{\arrangeEmbeddedChanges}\howToApplyTheseTerms
    \lsucmandatory{\keepStructuralIndependence}
    \lsucoptional{\createChangelog}  
    \lsucoptional{\addToDocumentation}
    \lsucoptional{\retainCopyrightNotices}
  \end{lsucrequires}

  \begin{lsucprohibits}
    \lsucitem{\noPatentLitigation}
  \end{lsucprohibits}
\end{lsuc}

% ------------------------------------------------------------------------------
\subsection{LGPL-\ver-CB: Passing a modified library as embedded binary}
\begin{lsuc}{LGPL-\ver-CB}
  \linkosuc{10B}

  \useCaseB

  \begin{lsucrequires}
    \lsucmandatory{\keepLicensingElements}
    \lsucmandatory{\lgplthreeEnsureCopyrightNoticeBinary}
    \lsucmandatory{\giveLicense}\passingFilesCorrectly
    \lsucmandatory{\makeEmbeddedSourceAvailable}
    \lsucmandatory{\describeHowToGetSource}
    \lsucsourcedist{LGPL-\ver-CA}
    \lsucmandatory{\addToCopyrightDialogLibWeak}
    \lsucmandatory{\markEmbeddedModifications}
    \lsucmandatory{\arrangeEmbeddedChanges}\howToApplyTheseTerms
    \lsucmandatory{\keepStructuralIndependence}
    \lsucmandatory{\allowRelinking}
    \lsucoptional{\createChangelog}  
    \lsucoptional{\addToDocumentation}
    \lsucoptional{\retainCopyrightNotices}
  \end{lsucrequires}

  \begin{lsucprohibits}
    \lsucitem{\noPatentLitigation}
  \end{lsucprohibits}
\end{lsuc}

% ------------------------------------------------------------------------------
\end{license}

%% =============================================================================
%% Discussion

\subsection{Discussions and Explanations}
\label{LGPL2Discussion}%
\label{LGPL3Discussion}
\newcommand{\lgplTwoAndGplThree}[2]{\footnote{%
    For LGPL-2.1 see \cite[cf.][\nopage wp.\ #1]{Lgpl21OsiLicense1999a}.
    \par\noindent
    For GPL-3.0, which is included in the LGPL-3.0, see \cite[cf.][\nopage
      wp.\ #2]{Gpl30OsiLicense2007a}.}} 

\newcommand{\lgplTwoAndThree}[2]{\footnote{%
    For LGPL-2.1 see \cite[cf.][\nopage wp.\ #1]{Lgpl21OsiLicense1999a}.
    \par\noindent
    For LGPL-3.0 see \cite[cf.][\nopage wp.\ #2]{Lgpl30OsiLicense2007a}.}}

\begin{itemize}
  
\item The LGPL-2.1 allows to \enquote{[\ldots] copy and (to) distribute
  verbatim copies of the Library's complete source code as you receive it [...]
  provided that you 
  [a] conspicuously and appropriately publish on each copy an appropriate
  copyright notice and disclaimer of warranty; 
  [b] keep intact all the notices that refer to this License and to the absence
  of any warranty; and 
  [c] distribute a copy of this License along with the Library.}\citeLGPLtwo{§1}
  Additionally, the LGPL-2.1 allows the distribution of the modified source code
  \enquote{under the terms of Section~1}\citeLGPLtwo{§2} and the distribution
  of binaries \enquote{under the terms of Sections~1 and~2}.\citeLGPLtwo{§4}
  But the LGPL does not require any tasks if you are using the work only for
  yourself. Thus, the quoted conditions of \enquote{Section 1} are mandatory for
  all use cases concerning the distribution of an LGPL licensed work
  (LGPL-2.1-C2 -- LGPL-2.1-CB).
  \footnote{The GPL-3.0, which is included into the LGPL-3.0, uses a similar
    structure to establish the same requirements ($\rightarrow$ \oslic, p.\ 
    \pageref{Gpl3ConditionsDistri}). Based on this fact, one may conclude that
    the tasks which fulfill the corresponding LGPL-2.1 requirements together
    also fit the GPL-3.0 conditions and hence those of the LGPL-3.0.}

\item Although the LGPL-2.1 does not explicitly require to retain the
  copyright notices in the form you have received them, it is nevertheless a
  very good idea not to modify these elements (LGPL-2.1-C2 - LGPL-2.1-CB). The
  LGPL-3.0, on the other hand, inherits the clauses that require all notices to
  be kept intact from the GPL-3.0 (LGPL-3.0-C2 -- LGPL-3.0-CB).\citeGPLthree{§4} 
  
\item The LGPL-2.1 allows to \enquote{[\ldots] copy and (to) distribute the
  Library (or a portion or derivative of it [\ldots]) in object code or
  executable form [\ldots] provided that you accompany it with the complete
  corresponding machine-readable source code [\ldots] on a medium customarily
  used for software interchange.} And the license further states that, if one
  makes the object code accessible without distributing it directly, then the
  same `download' method for the source code fulfills this
  condition.\citeLGPLtwo{§4}  So, no doubt: Taken literally, the LGPL requires
  you to distribute the source code and the object code together and by the same
  method: either both on (for example) DVD or both offered for download; but not
  the one on DVD and the other by a download from a repository. But the first
  specification also says, that the \enquote{complete corresponding machine
  readable source code} has to be distributed \enquote{on a medium customarily
  used for software interchange.}\citeLGPLtwo{§4}  The \oslic{} considers the
  possibility to download files from the Internet as a distribution \emph{on a
  medium [today] customarily used for software interchange.} Therefore, the
  \oslic{} requires for all open source use cases that refer to the distribution
  of binaries (LGPL-2.1-C3, LGPL-2.1-C5, LGPL-2.1-C7, LGPL-2.1-C9, and
  LGPL-2.1-CA) to make the source code of the corresponding library accessible
  via an Internet repository.
  
  In contrast to the LGPL-2.1, the GPL-3.0, which is included in the LGPL-3.0,
  explictily offers the option to distribute the sources via an Internet server
  ($\rightarrow$ \oslic, p.\ \pageref{Gpl3CondCopyleft}). So, one may again
  conclude that the tasks that fulfill the corresponding LGPL-2.1 requirements
  together also fit the GPL-3.0 and the LGPL-3.0 conditions. 
  
\item The LGPL allows to \enquote{[\ldots] modify your copy or copies of the
  Library or any portion of it [\ldots] and (to) copy and distribute such
  modifications [\ldots]} only under some restrictions and
  condtions:\citeLGPLtwo{§2} 
  \begin{itemize}
  \item First, modified files must be marked as modifications and this must
    include the date of the modification.\lgplTwoAndGplThree{§2}{§5} This
    condition must be respected by all open source use cases concerning the
    distribution of the modified work [LGPL-*-C6 - LGPL-*-CB], because even if
    one primarily intends to distribute binaries, one has also to deliver the
    source code. The \oslic{} `replaces' this requirement by the mandatory
    condition to mark each modified file and by the voluntary condition to
    update or create a general changelog file.
    
  \item Second, the license requires that the modified version does not depend
    on external data structures without \enquote{[\ldots] (making) a good faith
    effort to ensure that, in the event an application does not supply such (a)
    function or table, the facility still operates, and performs whatever part
    of its purpose remains meaningful.}\lgplTwoAndThree{§2d}{§2a} The \oslic{}
    rewrites this condition as the obligation to maintain the structural
    independence of the library in case of using the modified library as
    embedded component [LGPL-*-CA - LGPL-*-CB]. 
    
  \item \label{para:libislib}Third, the LGPL-2.1 definitely requires, that
    \enquote{the modified work must itself be a software
    library.}\citeLGPLtwo{§2}  This conditions can directly be incorprated as an
    interdiction into all use cases which refer to the modification of a library
    [LGPL-2.1-C8 - LGPL-2.1-CB]. But it is difficult to respect this condition
    if one wants to modify a program which one has received under the terms of
    the LGPL-2.1. In principal, one can write an application and license it
    under the LGPL-2.1. But, as a consequence, that impedes the modification of
    this work because the result must be a library. 
 
    The LGPL-3.0 does not contain any such requirement. Hence, the \oslic{} allows
    the distribution of modified programs (LGPL-*-C6, LGPL-*-C7) only if they
    are licensed under the terms of LGPL-3.0. For programs licensed under
    LGPL-2.1, the only option is to relicense the software under the terms of
    the regular GPL-2.0 (or, at your discretion, GPL-3.0).  This is explicitely
    allowed by the LGPL-2.1: \enquote{You may opt to apply the terms of the
    ordinary GNU General Public License instead of this License to a given
    copy of the Library.}\citeLGPLtwo{§3} 
  \end{itemize}
  
\item Additionally, the LGPL-2.1 allows the licensee to distribute a
  program\footnote{or another library} developed on-top of the library (what the
  LGPL-2.1 calls a \enquote{work that uses the libary}\citeLGPLtwo{§5, §6})
  \enquote{as an exception to the Sections above} in \enquote{combination} 
  with the library \enquote{under terms of your choice,}\citeLGPLtwo{§6},
  provided that the licensee fulfills additional conditions:
  
  First, it must clearly be stated that the on-top development depends on the
  (modified) library. Second, the LGPL must be added into the distributed
  package.\citeLGPLtwo{§6} In the LGPL-3.0, this condition is similarily
  integrated: On the one hand, the \enquote{combined work} is defined as
  \enquote{a work produced by combining or linking an Application with the
  Library}.\citeLGPLthree{§0} On the other hand, the LGPL-3.0 states that one
  \enquote{[\ldots] may convey a Combined Work under terms of (his own) choice}
  provided that one [a] clearly says that the on-top development uses the LGPL
  licensed library, [b] distributes the LGPL-3.0 and the GPL-3.0 license as part
  of the package, [c] includes all these (licensing) information in an existing
  copyright dialog, if any, [d] requires an appropriate shared library mechanism,
  and [e] offers the respective installion information.\citeLGPLthree{§4} These
  requirements can directly be inserted as conditions into the respective use
  cases for both LGPL versions (LGPL-*-CA, LGPL-*-CB).
  
\item The most difficult requirements of the LGPL-2.1 concern the distribution
  in the form of binaries. In a very strict reading, the LGPL does not require
  to link the on-top development and the libary only dynamically. At first, the
  LGPL mentions, that the \enquote{[..] work (that uses the Library), in
  isolation, is not a derivative work of the Library [\ldots]}. But if it is
  linked to the library the resulting executable program becomes
  \enquote{a derivative of the Library} and that it is therefore
  \enquote{[\ldots] covered by this License (LGPL-2.1)}. But the LGPL-2.1 
  directly continues this statement with the hint, that \enquote{Section 6
  states terms for distribution of such executables.}\citeLGPLtwo{§5} Finally,
  section 6 directly starts with the statement: \enquote{As an exception to the
  Sections above, you may also combine or link a `work that uses the
  Library' with the Library to produce a work containing portions of the
  Library, and distribute that work under terms of your choice}.\citeLGPLtwo{§6}
  
  This is important to know, because until this section 6 one can not directly
  read or indirectly infer that the LGPL-2.1 distinguished the act of
  dynamically linking a program and a library from that of statically linking
  these parts. The LGPL only wants to ensure that the binaries of the library
  itself can be replaced by a newer version. And that is required by
  section~6.\citeLGPLtwo{§6} 
  From a practical point of view, this can only be guaranteed, if the binaries of
  the on-top development and the library are linked using a \enquote{suitable
  shared library mechanism}\citeLGPLtwo{§6} or if one also gets all compiled,
  but not linked object-files of the on-top development and the library, either
  directly, or via using a \enquote{a written offer, valid for at least three
  years, to give the same user the (respective) materials}.\citeLGPLtwo{§6}  In
  the first case, the user can replace the received version of the library and
  can let the application be relinked  automatically. In the second case, he has
  to do it manually. It is important to know that both these ways exist if one
  wants or must distribute statically linked works. The LGPL-2.1 does not forbid
  to distribute statically linked applications. But it requires to enable the
  receiver to relink the work. 
  
  The LGPL-3.0 has reduced these complex conditions in a special way: First, it
  does not use the words `statically linked' or `dynamically' linked at
  all. Second it defines the combined work `only' as the result of
  \enquote{combining or linking an Application with the
  Library}.\citeLGPLthree{§0}  But then it requires for the distribution of the
  combined works that one has either to \enquote{convey the Minimal
  Corresponding Source under the terms of this License, and the Corresponding
  Application Code in a form suitable for, and under terms that permit, the user
  to recombine or relink the Application with a modified version of the Linked
  Version to produce a modified Combined Work [\ldots]} or that one must
  presuppose that the receiver uses \enquote{[\ldots] suitable shared library
  mechanism for linking with the Library [\ldots] that [\ldots] operate properly
  with a modified version of the Library [\ldots]}\citeLGPLthree{§4} Finally,
  the LGPL-3.0 adds that in the first case the these materials which enables the
  relinking must be distributed \enquote{[\ldots] in the manner specified by
  section~6 of the GNU GPL[-3.0] for conveying Corresponding
  Source.}\citeLGPLthree{§4}  And this section~6 of the GPL-3.0 allows the well
  known method to \enquote{convey the object code [\ldots] accompanied by a
  written offer [\ldots] to give anyone [\ldots] access to copy the
  Corresponding Source from a network server at no charge}.\citeGPLthree{§6}

  Therefore, the \oslic{} can condense these conditions into the requirement,
  either to distribute dynamically linkable parts, or to distribute statically
  linked applications \enquote{(accompanied) [\ldots] with a written offer,
  valid for at least three years, to give the same user the [complete]
  materials,}\citeLGPLtwo{§6} so that he can relink the application. It is
  clear, that this condition only applies to the use cases LGPL-*-C5 and
  LGPL-*-CB. 
  
\end{itemize}

%\bibliography{../../../bibfiles/oscResourcesEn}

% Local Variables:
% mode: latex
% fill-column: 80
% End:
}
{% Telekom osCompendium 'for being included' snippet template
%
% (c) Karsten Reincke, Deutsche Telekom AG, Darmstadt 2011
%
% This LaTeX-File is licensed under the Creative Commons Attribution-ShareAlike
% 3.0 Germany License (http://creativecommons.org/licenses/by-sa/3.0/de/): Feel
% free 'to share (to copy, distribute and transmit)' or 'to remix (to adapt)'
% it, if you '... distribute the resulting work under the same or similar
% license to this one' and if you respect how 'you must attribute the work in
% the manner specified by the author ...':
%
% In an internet based reuse please link the reused parts to www.telekom.com and
% mention the original authors and Deutsche Telekom AG in a suitable manner. In
% a paper-like reuse please insert a short hint to www.telekom.com and to the
% original authors and Deutsche Telekom AG into your preface. For normal
% quotations please use the scientific standard to cite.
%
% [ Framework derived from 'mind your Scholar Research Framework' 
%   mycsrf (c) K. Reincke 2012 CC BY 3.0  http://mycsrf.fodina.de/ ]
%


%% use all entries of the bibliography
%\nocite{*}

\section{MIT licensed software}

\begin{license}{MIT}
\licensename{MIT}
\licensespec{MIT License}
\licenseabbrev{MIT}

The MIT license is known as one of the most permissive licenses. Thus, the
MIT specific finder can be simplified:

\tikzstyle{nodv} = [font=\small, ellipse, draw, fill=gray!10, 
    text width=2cm, text centered, minimum height=2em]

\tikzstyle{nods} = [font=\footnotesize, rectangle, draw, fill=gray!20, 
    text width=1.2cm, text centered, rounded corners, minimum height=3em]

\tikzstyle{nodb} = [font=\footnotesize, rectangle, draw, fill=gray!20, 
    text width=2.2cm, text centered, rounded corners, minimum height=3em]

\tikzstyle{nodx} = [font=\footnotesize, rectangle, draw, fill=gray!20, 
    text width=2.4cm, text centered, rounded corners, minimum height=3em]
    
\tikzstyle{leaf} = [font=\tiny, rectangle, draw, fill=gray!30, 
    text width=1.2cm, text centered, minimum height=6em]

\tikzstyle{edge} = [draw, -latex']

\begin{tikzpicture}[]

\node[nodv] (l61) at ( 2.4, 9.2) {MIT};

\node[nodb] (l51) at ( 0.0, 7.8) {\textit{recipient:} \\ \textbf{4yourself}};
\node[nodb] (l52) at ( 4.8, 7.8) {\textit{recipient:} \\ \textbf{2others}};

\node[nodb] (l41) at ( 2.5, 6.2) {\textit{state:} \\ \textbf{unmodified}};
\node[nodb] (l42) at ( 7.0, 6.2) {\textit{state:} \\ \textbf{modified}};

\node[nodb] (l31) at ( 5.0, 4.6) {\textit{type:} \\ \textbf{proapse}};
\node[nodb] (l32) at ( 9.0, 4.6) {\textit{type:} \\ \textbf{snimoli}};

\node[nodx] (l21) at ( 7.5, 2.8) {\textit{context:} \\ \textbf{independent}};
\node[nodx] (l22) at (10.5, 2.8) {\textit{context:} \\ \textbf{embedded}};

\node[leaf] (l11) at ( 0.0, 0.0) {\textbf{MIT-C1} \textit{using software only for yourself}};
\node[leaf] (l12) at ( 2.5, 0.0) {\textbf{MIT-C2} \textit{distributing unmodified package}};
\node[leaf] (l13) at ( 5.0, 0.0) {\textbf{MIT-C3} \textit{distributing modified program}};
\node[leaf] (l14) at ( 7.5, 0.0) {\textbf{MIT-C4} \textit{distributing modified library as independent package}};
\node[leaf] (l15) at (10.5, 0.0) {\textbf{MIT-C5} \textit{distributing modified library as embedded package}};


\path [edge] (l61) -- (l51);
\path [edge] (l61) -- (l52);
\path [edge] (l51) -- (l11);
\path [edge] (l52) -- (l41);
\path [edge] (l52) -- (l42);
\path [edge] (l41) -- (l12);
\path [edge] (l42) -- (l31);
\path [edge] (l42) -- (l32);
\path [edge] (l31) -- (l13);
\path [edge] (l32) -- (l21);
\path [edge] (l32) -- (l22);
\path [edge] (l21) -- (l14);
\path [edge] (l22) -- (l15);

\end{tikzpicture}

%%
%% Common building blocks
%%

% ------------------------------------------------------------------------------
% License elements must be preserved
\newcommand{\keepLicensingElements}{Ensure that the licensing elements
  (especially the MIT license text containing the specific copyright notices of
  the original author(s), the permission notices and the MIT disclaimer) are
  retained in your package in the form you have received them.}

% ------------------------------------------------------------------------------
% Add a link to the project home page
\newcommand{\linkToProject}{It's a good tradition to let the documentation of
  your distribution and/or your additional material also contain a link to the 
  original software (project) and its homepage.}

% ------------------------------------------------------------------------------
% Mark your modifications of the software
\newcommand{\markYourModifications}{Mark your modifications in the source code,
  regardless whether you want to distribute the code or not.}

% ------------------------------------------------------------------------------
% Add a copyright notice for your modifications 
\newcommand{\addYourCopyrightNotice}{You can augment an existing copyright 
  notice presented by the program with information about your own work or
  modifications.}

% ------------------------------------------------------------------------------
% Acknowledge the use of the OSS
\newcommand{\displayAcknowledement}{It is a good practice of the open source  
  community to let the copyright notice that is shown by the running program  
  also state that the program uses a component licensed under the MIT
  license. And it is a good tradition to insert links to the homepage or 
  download page of this component.} 

% ------------------------------------------------------------------------------

\subsection{MIT-C1: Using the software only for yourself}
\begin{lsuc}{MIT-C1}
  \linkosuc{01}
  \linkosuc{03L} 
  \linkosuc{03N} 
  \linkosuc{06L}
  \linkosuc{06N}
  \linkosuc{09L}
  \linkosuc{09N}
  
  \lsucmeans{that you received MIT licensed software, that you will use it
  only for yourself and that you do not hand it over to any 3rd party in any
  sense.}

  \lsuccovers{OSUC-01, OSUC-03L, OSUC-03N, OSUC-06L, OSUC-06N, OSUC-09L, and
  OSUC-09N\footnote{For details $\rightarrow$ \oslic, pp.\ \pageref{OSUC-01-DEF}
  - \pageref{OSUC-09N-DEF}}}

  \begin{lsucrequiresnothing}
    \lsucitem{You are allowed to use any kind of MIT licensed software in any
      sense and in any context without any obligations if you do not give the
      software to third parties and if you do not modify the existing copyright
      notices and the existing permission notice.}
  \end{lsucrequiresnothing}

  \lsucprohibitsnothing
\end{lsuc}

\subsection{MIT-C2: Passing the unmodified software}
\begin{lsuc}{MIT-C2}
  \linkosuc{02S} 
  \linkosuc{05S} 
  \linkosuc{07S} 
  \linkosuc{02B} 
  \linkosuc{05B} 
  \linkosuc{07B} 

  \lsucmeans{that you received MIT licensed software which you are now going to
  distribute to third parties in the form of unmodified binaries or as unmodifed
  source code files. In this case it makes no difference if you distribute a
  program, an application, a server, a snippet, a module, a library, or a plugin
  as an independent package.}

  \lsuccovers{OSUC-02S,  OSUC-02B, OSUC-05S, OSUC-05B, OSUC-07S,
    OSUC-07B\footnote{For details $\rightarrow$ \oslic,
      pp.\ \pageref{OSUC-02S-DEF} - \pageref{OSUC-07B-DEF}}} 

  \begin{lsucrequires}
    \lsucmandatory{\keepLicensingElements}
    \lsucoptional{\linkToProject}
  \end{lsucrequires}

  \lsucprohibitsnothing
\end{lsuc}


\subsection{MIT-C3: Passing a modified program}
\begin{lsuc}{MIT-C3}
  \linkosuc{04S}
  \linkosuc{04B}

  \lsucmeans{that you received an MIT licensed program, application, or server
  (proapse), that you modified it, and that you are now going to distribute this
  modified version to third parties in the form binaries or as source code
  files.}

  \lsuccovers{OSUC-04S, OSUC-04B\footnote{For details $\rightarrow$ \oslic,
      pp.\ \pageref{OSUC-04S-DEF}}}

  \begin{lsucrequires}
    \lsucmandatory{\keepLicensingElements}
    \lsucoptional{\markYourModifications}
    \lsucoptional{\linkToProject}
    \lsucoptional{\addYourCopyrightNotice} 
    \lsucoptional{\displayAcknowledement}
  \end{lsucrequires}

  \lsucprohibitsnothing
\end{lsuc}

\subsection{MIT-C4: Passing a modified library independently}
\begin{lsuc}{MIT-C4}
  \linkosuc{08S}
  \linkosuc{08B}

  \lsucmeans{that you received an MIT licensed code snippet, module, library, or
  plugin (snimoli), that you modified it, and that you are now going to
  distribute this modified version to third parties in the the form of binaries
  or as source code files, but without embedding it into another larger software
  unit.}

  \lsuccovers{OSUC-08S, OSUC-08B\footnote{For details $\rightarrow$ \oslic,
      pp.\ \pageref{OSUC-08S-DEF}}}

  \begin{lsucrequires}
    \lsucmandatory{\keepLicensingElements}
    \lsucoptional{\markYourModifications}
    \lsucoptional{\linkToProject}
  \end{lsucrequires}

  \lsucprohibitsnothing
\end{lsuc}


\subsection{MIT-C5: Passing a modified library as embedded component}
\begin{lsuc}{MIT-C5}
  \linkosuc{10S}
  \linkosuc{10B}

  \lsucmeans{that you received an MIT licensed code snippet, module, library, or
  plugin (snimoli), that you modified it, and that you are now going to
  distribute this modified version to third parties in the form of binaries or
  as source code files together with another larger software unit which contains
  this code snippet, module, library, or plugin as an embedded component,
  regardless whether you distribute it in the form of binaries or as source code
  files.}

  \lsuccovers{OSUC-10S, OSUC-10B\footnote{For details $\rightarrow$ \oslic,
      pp.\ \pageref{OSUC-10S-DEF}}}

  \begin{lsucrequires}
    \lsucmandatory{\keepLicensingElements}
    \lsucoptional{\markYourModifications}
    \lsucoptional{\displayAcknowledement}
    \lsucoptional{\linkToProject}

    \lsucoptional{Arrange your distribution so that the original licensing
      elements (especially the MIT license text containing the specific
      copyright notices of the original author(s), the permission notices and
      the MIT disclaimer) clearly refer only to the embedded library and do not
      disturb the licensing of your own overarching work. It's a good tradition
      to keep the libraries, modules, snippet, or plugins in separate
      directories, which contain also all licensing elements.}
  \end{lsucrequires}

  \lsucprohibitsnothing
\end{lsuc}

\subsection{Discussions and Explanations}
\label{MITDiscussion}
The MIT-License is known as one of the most permissive licenses. It is a very
short license containing (0) a copyright notice, (1) a paragraph saying that you
are allowed to do almost anything you want, followed (2) by the condition that
you have to \enquote{include} the existing copyright notes and the permission
notes \enquote{[\ldots] in all copies or substantial portions of the software},
and (3) closed by the well known disclaimer.\citeMIT{} But the license doesn't
talk about the difference of source code and object code. So, you have to find
the right way by yourself. 
Here are our readings:

\begin{itemize}
  \item If you do not modify the received MIT licensed application, neither for
    your own purposes, nor for handing over the program to 3rd parties, you can 
    conclude that all copyright notices and permission notices are already
    correct.
  \item Nevertheless, we added the hint not to modify these licensing elements
    in the context of the use case \emph{used by yourself}. This is implied by
    the MIT license itself. It requires explicitly that \enquote{the above
      copyright notice and this permission notice shall be included in all
      copies or substantial portions of the Software}\citeMIT---thus also into
    those copies you make for your own purposes on your own machines. 
  \item If you modify the MIT licensed application, regardless for which
    purpose, you are simply not allowed to erase or modify existing copyright
    notes and permission notices. You may add your own modifications under new
    conditions, but the old notices must survive. 
  \item We request that you also keep the MIT disclaimer. This is not
    explicitely required by the license. The permission notices, which is
    required to be preserved, most likely refers to the text \emph{between} the
    copyright notice and the disclaimer and, hence, does not include the latter.
    But another possible, although less likely interpretation is that the whole
    text of the license is what permission notice refers to.  
\end{itemize}

\end{license}

%\bibliography{../../../bibfiles/oscResourcesEn}

% Local Variables:
% mode: latex
% fill-column: 80
% End:
}
{% Telekom osCompendium 'for being included' snippet template
%
% (c) Karsten Reincke, Deutsche Telekom AG, Darmstadt 2011
%
% This LaTeX-File is licensed under the Creative Commons Attribution-ShareAlike
% 3.0 Germany License (http://creativecommons.org/licenses/by-sa/3.0/de/): Feel
% free 'to share (to copy, distribute and transmit)' or 'to remix (to adapt)'
% it, if you '... distribute the resulting work under the same or similar
% license to this one' and if you respect how 'you must attribute the work in
% the manner specified by the author ...':
%
% In an internet based reuse please link the reused parts to www.telekom.com and
% mention the original authors and Deutsche Telekom AG in a suitable manner. In
% a paper-like reuse please insert a short hint to www.telekom.com and to the
% original authors and Deutsche Telekom AG into your preface. For normal
% quotations please use the scientific standard to cite.
%
% [ Framework derived from 'mind your Scholar Research Framework' 
%   mycsrf (c) K. Reincke 2012 CC BY 3.0  http://mycsrf.fodina.de/ ]
%


%% use all entries of the bibliography
%\nocite{*}

\section{MPL-2.0 licensed software}

\begin{license}{MPL} % ends at end of file
\licensename{MPL-2.0}
\licensespec{Mozilla Public License 2.0}
\licenseversion{2.0}
\licenseabbrev{MPL}

The Mozilla Public License clearly distinguishes the distribution of source code
from the distribution of binaries: First, it allows the \enquote{Distribution of
Source Form}.\citeMPL{§3.1} Then, it specifies the conditions for a
\enquote{Distribution of Executable Form}.\citeMPL{§3.2} Additionally, the
MPL-2.0 contrasts the \enquote{distribution of Covered Software} with the
\enquote{distribution of a Larger Work}.\citeMPL{§3.3} So, taken as whole, the
MPL-2.0 mainly focusses on the distribution of software. Thus, for finding the
relevant executable task lists, the following MPL-2.0 specific open source use
case structure%
  \footnote{For details of the general OSUC finder $\rightarrow$ \oslic, 
    pp.\ \pageref{OsucTokens} and \pageref{OsucDefinitionTree}} 
can be used:
 
\tikzstyle{nodv} = [font=\small, ellipse, draw, fill=gray!10, 
    text width=2cm, text centered, minimum height=2em]

\tikzstyle{nods} = [font=\footnotesize, rectangle, draw, fill=gray!20, 
    text width=1.2cm, text centered, rounded corners, minimum height=3em]

\tikzstyle{nodb} = [font=\footnotesize, rectangle, draw, fill=gray!20, 
    text width=2.2cm, text centered, rounded corners, minimum height=3em]
    
\tikzstyle{leaf} = [font=\tiny, rectangle, draw, fill=gray!30, 
    text width=1.2cm, text centered, minimum height=6em]

\tikzstyle{edge} = [draw, -latex']

\begin{tikzpicture}[]

\node[nodv] (l71) at (4,10) {MPL-2.0};

\node[nodb] (l61) at (0,8.6) {\textit{recipient:} \\ \textbf{4yourself}};
\node[nodb] (l62) at (6.5,8.6) {\textit{recipient:} \\ \textbf{2others}};

\node[nodb] (l51) at (2.5,7) {\textit{state:} \\ \textbf{unmodified}};
\node[nodb] (l52) at (9.3,7) {\textit{state:} \\ \textbf{modified}};

\node[nods] (l41) at (1.8,5.4) {\textit{form:} \textbf{source}};
\node[nods] (l42) at (3.6,5.4) {\textit{form:} \textbf{binary}};
\node[nodb] (l43) at (6.5,5.4) {\textit{type:} \\ \textbf{proapse}};
\node[nodb] (l44) at (12,5.4) {\textit{type:} \\ \textbf{snimoli}};


\node[nods] (l31) at (5.4,3.8) {\textit{form:} \textbf{source}};
\node[nods] (l32) at (7.2,3.8) {\textit{form:} \textbf{binary}};
\node[nodb] (l33) at (10,3.8) {\textit{context:} \\ \textbf{independent}};
\node[nodb] (l34) at (13.5,3.8) {\textit{context:} \\ \textbf{embedded}};

\node[nods] (l21) at (9,2.2) {\textit{form:} \textbf{source}};
\node[nods] (l22) at (10.8,2.2) {\textit{form:} \textbf{binary}};
\node[nods] (l23) at (12.6,2.2) {\textit{form:} \textbf{source}};
\node[nods] (l24) at (14.4,2.2) {\textit{form:} \textbf{binary}};

\node[leaf] (l11) at (0,0) {\textbf{MPL-2.0-C1} \textit{using software only
for yourself}};

\node[leaf] (l12) at (1.8,0) { \textbf{MPL-2.0-C2} \textit{ distributing unmodified
software as sources}};

\node[leaf] (l13) at (3.6,0) { \textbf{MPL-2.0-C3}  \textit{ distributing unmodified
software as binaries}};

\node[leaf] (l14) at (5.4,0) { \textbf{MPL-2.0-C4}  \textit{ distributing modified
program as sources}};

\node[leaf] (l15) at (7.2,0) { \textbf{MPL-2.0-C5}  \textit{ distributing modified
program as binaries}};

\node[leaf] (l16) at (9,0) { \textbf{MPL-2.0-C6}  \textit{ distributing modified
library as independent sources}};

\node[leaf] (l17) at (10.8,0) { \textbf{MPL-2.0-C7} \textit{distributing modified
library as independent binaries}};

\node[leaf] (l18) at (12.6,0) { \textbf{MPL-2.0-C8}  \textit{distributing
modified library as embedded sources}};

\node[leaf] (l19) at (14.4,0) { \textbf{MPL-2.0-C9}  \textit{ distributing modified
library as embedded binaries}};


\path [edge] (l71) -- (l61);
\path [edge] (l71) -- (l62);
\path [edge] (l61) -- (l11);
\path [edge] (l62) -- (l51);
\path [edge] (l62) -- (l52);
\path [edge] (l51) -- (l41);
\path [edge] (l51) -- (l42);
\path [edge] (l52) -- (l43);
\path [edge] (l52) -- (l44);
\path [edge] (l41) -- (l12);
\path [edge] (l42) -- (l13);
\path [edge] (l43) -- (l31);
\path [edge] (l43) -- (l32);
\path [edge] (l44) -- (l33);
\path [edge] (l44) -- (l34);
\path [edge] (l31) -- (l14);
\path [edge] (l32) -- (l15);
\path [edge] (l33) -- (l21);
\path [edge] (l33) -- (l22);
\path [edge] (l34) -- (l23);
\path [edge] (l34) -- (l24);
\path [edge] (l21) -- (l16);
\path [edge] (l22) -- (l17);
\path [edge] (l23) -- (l18);
\path [edge] (l24) -- (l19);

\end{tikzpicture}

%% =============================================================================
%% Common building blocks
%%

% ------------------------------------------------------------------------------
% Ensure license elements are present

\newcommand{\keepLicenseElements}{Ensure that the licensing elements (especially
  all copyright notices, patent notices, disclaimers of warranty, or limitations
  of liability) are retained in your package in exactly the form that you have
  received.}

\newcommand{\addWhenCompiling}{If you compile the binary from the sources,
  ensure that all these licensing elements are also incorporated into the
  package.}

% ------------------------------------------------------------------------------
% Give the recipient a copy of the license

\newcommand{\giveLicenseText}{Give the recipient a copy of the MPL-2.0 license.
  If it is not already part of the software package, add it. If the licensing
  statement in the licensing file of the package does still not clearly state
  that the package is licensed under the MPL-2.0, additionally insert your own
  correct MPL-2.0 licensing file containing the sentence: 
  \emph{This Source Code Form is subject to the terms of the Mozilla Public
    License, v. 2.0. If a copy of the MPL was not distributed with this file,
    You can obtain one at http://mozilla.org/MPL/2.0/.}}

% ------------------------------------------------------------------------------
% Add license, name, and link to homepage to documentation

\newcommand{\auxAddToDoc}[1]{Let the documentation of your distribution
  and/or your additional material also reproduce the content of the existing
  \emph{copyright notice text files,} the name of #1, a link to its homepage,
  and a link to the MPL-2.0 license.}  

\newcommand{\acknowledgeMPLSoftware}{
  \auxAddToDoc{the software}}

\newcommand{\acknowlegdeEmbeddedLibrary}{
  \auxAddToDoc{the embedded MPL-2.0 licensed component}}

% ------------------------------------------------------------------------------
% Make the source code available

\newcommand{\auxMakeSourceAvailable}[1]{Make the source code of #1 accessible
  via a repository under your own control: Push the source code package into the
  repository and make it downloadable via the Internet. Do no charge any fees
  from the user for downloading the source. Ensure, that this repository is
  online for a reasonable period of time after you ceased distributing the
  software.} 

\newcommand{\makeSourceAvailable}{\auxMakeSourceAvailable{%
    the distributed software}}

\newcommand{\makeEmbeddedSourceAvailable}{\auxMakeSourceAvailable{%
    the embedded library}}

\newcommand{\describeHowToGetSource}{Insert an easy to find description into the 
  distribution package that explains how and where the code can be retrieved.}

% ------------------------------------------------------------------------------
% Ensure modifications are covered by MPL

\newcommand{\auxPlaceModificationsUnderMPL}[1]{Organize your modifications #1
  in such a way that they are covered by the existing MPL-2.0 licensing
  statements.}

\newcommand{\auxPlaceNewFilesUnderMPL}[1]{If you add new source code files#1,
  insert a header containing your copyright line and an MPL-2.0 adequate
  licensing the statement.}

\newcommand{\placeBinaryModificationsUnderMPL}{%
  \auxPlaceModificationsUnderMPL{}}

\newcommand{\placeSourceModificationsUnderMPL}{%
  \auxPlaceModificationsUnderMPL{}
  \auxPlaceNewFilesUnderMPL{}}

\newcommand{\placeEmbeddedBinaryUnderMPL}{%
  \auxPlaceModificationsUnderMPL{of the embedded library}}

\newcommand{\placeEmbeddedSourceUnderMPL}{%
  \auxPlaceModificationsUnderMPL{of the embedded library}
  \auxPlaceNewFilesUnderMPL{ to the library itself}}

% ------------------------------------------------------------------------------
% Create modification text file

\newcommand{\createChangeLog}{Create a \emph{modification text file}, if such a
  notice file still does not exist. \emph{Add} a general description of your
  modifications to the \emph{modification text file}. Incorporate the file into
  your distribution package.}

% ------------------------------------------------------------------------------
% Mark all modifications

\newcommand{\markAllModifications}{Mark all modifications of the source code
  thoroughly, preferably in the modified source itself.}

% ------------------------------------------------------------------------------
% Separate embedded library from enclosing program

\newcommand{\auxKeepSeparate}[1]{Arrange your #1 distribution so that the
  licensing elements (especially the MPL-2.0 license text and the
  \emph{licensing files}) clearly refer only to the embedded library and do not
  affect the licensing of your own overarching work. It's a good tradition to
  keep embedded components like libraries, modules, snippets, or plugins in
  separate directories, which contain also all additional licensing elements.}

\newcommand{\keepSourceSeparate}{\auxKeepSeparate{source code}}
\newcommand{\keepBinarySeparate}{\auxKeepSeparate{binary}}

% ------------------------------------------------------------------------------
% Do not modify or remove license elements

\newcommand{\dontAlterLicenseElement}{to remove or to alter any license elements
  (including copyright notices, patent notices, disclaimers of warranty, or
  limitations of liability) contained within the software package you have
  received.}

% ------------------------------------------------------------------------------
% Do not use trademarks and logos to promote your own work

\newcommand{\dontUseTrademarks}{to promote any of your services based on the
    this software by trademarks, service marks, or logos linked to this MPL-2.0
    software, except as required for reasonable and customary use in describing
    the origin of the software and reproducing the copyright notice.}

%% =============================================================================
%% Use Cases
%%

\subsection{MPL-2.0-C1: Using the software only for yourself}
\begin{lsuc}{MPL-2.0-C1}
  \linkosuc{01}
  \linkosuc{03L} 
  \linkosuc{03N} 
  \linkosuc{06L}
  \linkosuc{06N}
  \linkosuc{09L}
  \linkosuc{09N}

  \lsucmeans{that you received MPL-2.0 licensed software, that you will use it
    only for yourself, and that you do not hand it over to any third party in
    any sense.}

  \coversOsucs{OSUC-01, OSUC-03L, OSUC-03N, OSUC-06L, OSUC-06N, OSUC-09L, and
  OSUC-09N}{01}{09N}

  \begin{lsucrequiresnothing}
    \lsucitem{You are allowed to use any kind of MPL-2.0 software in any sense
      and in any context without being obliged to do anything as long as you do
      not give the software to third parties.}
  \end{lsucrequiresnothing}

  \begin{lsucprohibits}
    \lsucitem{\dontAlterLicenseElement}
    \lsucitem{\dontUseTrademarks}
  \end{lsucprohibits}
\end{lsuc}

% ------------------------------------------------------------------------------
\subsection{MPL-2.0-C2: Passing the unmodified software as source code}
\begin{lsuc}{MPL-2.0-C2}
  \linkosuc{02S} 
  \linkosuc{05S} 
  \linkosuc{07S} 

  \lsucmeans{that you received MPL-2.0 licensed software which you are now going
    to distribute to third parties in the form of unmodified source code files
    or as unmodified source code package. In this case it makes no difference if
    you distribute a program, an application, a server, a snippet, a module, a
    library, or a plugin as an independent or as an embedded unit.}

  \coversOsucs{OSUC-02S, OSUC-05S, OSUC-07S}{02S}{07S}

  \begin{lsucrequires}
    \lsucmandatory{\keepLicenseElements}
    \lsucmandatory{\giveLicenseText}\passingFilesCorrectly
    \lsucoptional{\acknowledgeMPLSoftware}
  \end{lsucrequires}

  \begin{lsucprohibits}
    \lsucitem{\dontAlterLicenseElement}
    \lsucitem{\dontUseTrademarks}
  \end{lsucprohibits}
\end{lsuc}

% ------------------------------------------------------------------------------
\subsection{MPL-2.0-C3: Passing the unmodified software as binaries} 
\begin{lsuc}{MPL-2.0-C3}
  \linkosuc{02B} 
  \linkosuc{05B} 
  \linkosuc{07B}

  \lsucmeans{that you received MPL-2.0 licensed software which you are now going
    to distribute to third parties in the form of unmodified binary files or as
    unmodified binary package. In this case it does not matter if you distribute
    a program, an application, a server, a snippet, a module, a library, or a
    plugin as an independent or an embedded unit.}

  \coversOsucs{OSUC-02B, OSUC-05B, OSUC-07B}{02B}{07B}

  \begin{lsucrequires}
    \lsucmandatory{\keepLicenseElements\ \addWhenCompiling}
    \lsucmandatory{\makeSourceAvailable}
    \lsucmandatory{\describeHowToGetSource}
  
    \lsucsourcedist{MPL-2.0-C2}
    \lsucoptional{\giveLicenseText}\passingFilesCorrectly
    \lsucoptional{\acknowledgeMPLSoftware}
  \end{lsucrequires}

  \begin{lsucprohibits}
    \lsucitem{\dontAlterLicenseElement}
    \lsucitem{\dontUseTrademarks}
  \end{lsucprohibits}
\end{lsuc}

% ------------------------------------------------------------------------------
\subsection{MPL-2.0-C4: Passing a modified program as source code}
\begin{lsuc}{MPL-2.0-C4}
  \linkosuc{04S} 

  \lsucmeans{that you received an MPL-2.0 licensed program, application, or
    server (proapse), that you modified it, and that you are now going to
    distribute this modified version to third parties in the form of source code
    files or as a source code package.}

  \mapsToOsuc{04S}

  \begin{lsucrequires}
    \lsucmandatory{\keepLicenseElements}
    \lsucmandatory{\giveLicenseText}\passingFilesCorrectly
    \lsucmandatory{\placeSourceModificationsUnderMPL}
    \lsucoptional{\createChangeLog}
    \lsucoptional{\markAllModifications}

    \lsucoptional{\acknowledgeMPLSoftware}
  \end{lsucrequires}
 
  \begin{lsucprohibits}
    \lsucitem{\dontAlterLicenseElement}
    \lsucitem{\dontUseTrademarks}
  \end{lsucprohibits}
\end{lsuc}

% ------------------------------------------------------------------------------
\subsection{MPL-2.0-C5: Passing a modified program as binary}
\begin{lsuc}{MPL-2.0-C5}
  \linkosuc{04B} 

  \lsucmeans{that you received an MPL-2.0 licensed program, application, or
    server (proapse), that you modified it, and that you are now going to
    distribute this modified version to third parties in the form of binary
    files or as a binary package.}

  \mapsToOsuc{04B}

  \begin{lsucrequires}
    \lsucmandatory{\keepLicenseElements\ \addWhenCompiling}
    \lsucmandatory{\makeSourceAvailable}
    \lsucmandatory{\describeHowToGetSource}
    \lsucsourcedist{MPL-2.0-C4}
    \lsucmandatory{\placeBinaryModificationsUnderMPL}
    \lsucoptional{\createChangeLog}
    \lsucoptional{\giveLicenseText}\passingFilesCorrectly
    \lsucoptional{\acknowledgeMPLSoftware}
  \end{lsucrequires}


  \begin{lsucprohibits}
    \lsucitem{\dontAlterLicenseElement}
    \lsucitem{\dontUseTrademarks}
  \end{lsucprohibits}
\end{lsuc}

% ------------------------------------------------------------------------------
\subsection{MPL-2.0-C6: Passing a modified library as independent source code}
\begin{lsuc}{MPL-2.0-C6}
  \linkosuc{08S}

  \lsucmeans{that you received an MPL-2.0 licensed code snippet, module,
    library, or plugin (snimoli), that you modified it, and that you are now
    going to distribute this modified version to third parties in the form of
    source code files or as a source code package, but without embedding it into
    another larger software unit.}

  \mapsToOsuc{08S}

  \begin{lsucrequires}
    \lsucmandatory{\keepLicenseElements}
    \lsucmandatory{\giveLicenseText}\passingFilesCorrectly
    \lsucmandatory{\placeSourceModificationsUnderMPL}
    \lsucoptional{\createChangeLog}
    \lsucoptional{\markAllModifications}
    \lsucoptional{\acknowledgeMPLSoftware}
  \end{lsucrequires}

  \begin{lsucprohibits}
    \lsucitem{\dontAlterLicenseElement}
    \lsucitem{\dontUseTrademarks}
  \end{lsucprohibits}
\end{lsuc}

% ------------------------------------------------------------------------------
\subsection{MPL-2.0-C7: Passing a modified library as independent binary}
\begin{lsuc}{MPL-2.0-C7}
  \linkosuc{08B}

  \lsucmeans{that you received an MPL-2.0 licensed code snippet, module,
    library, or plugin (snimoli), that you modified it, and that you are now
    going to distribute this modified version to third parties in the form of
    binary files or as a binary package but without embedding it into another
    larger software unit.}

  \mapsToOsuc{08B}

  \begin{lsucrequires}
    \lsucmandatory{\keepLicenseElements\ \addWhenCompiling}
    \lsucmandatory{\makeSourceAvailable}
    \lsucmandatory{\describeHowToGetSource}
    \lsucsourcedist{MPL-2.0-C6}
    \lsucmandatory{\placeBinaryModificationsUnderMPL}
    \lsucoptional{\createChangeLog}
    \lsucoptional{\giveLicenseText}\passingFilesCorrectly
    \lsucoptional{\acknowledgeMPLSoftware}
  \end{lsucrequires}

  \begin{lsucprohibits}
    \lsucitem{\dontAlterLicenseElement}
    \lsucitem{\dontUseTrademarks}
  \end{lsucprohibits}
\end{lsuc}

% ------------------------------------------------------------------------------
\subsection{MPL-2.0-C8: Passing a modified library as embedded source code}
\begin{lsuc}{MPL-2.0-C8}
  \linkosuc{10S}

  \lsucmeans{that you received an MPL-2.0 licensed code snippet, module,
    library, or plugin (snimoli), that you modified it, and that you are now
    going to distribute this modified version to third parties in the form of
    source code files or as a source code package together with another larger
    software unit which contains this code snippet, module, library, or plugin
    as an embedded component.}

  \mapsToOsuc{10S}

  \begin{lsucrequires}

    \lsucmandatory{\keepLicenseElements}
    \lsucmandatory{\giveLicenseText}\passingFilesCorrectly
    \lsucmandatory{\placeEmbeddedSourceUnderMPL}
    \lsucoptional{\keepSourceSeparate}
    \lsucoptional{\createChangeLog}
    \lsucoptional{\markAllModifications}
    \lsucoptional{\acknowlegdeEmbeddedLibrary}
  \end{lsucrequires}

  \begin{lsucprohibits}
    \lsucitem{\dontAlterLicenseElement}
    \lsucitem{\dontUseTrademarks}
  \end{lsucprohibits}
\end{lsuc}

% ------------------------------------------------------------------------------
\subsection{MPL-2.0-C9: Passing a modified library as embedded binary}
\begin{lsuc}{MPL-2.0-C9}
  \linkosuc{10B}

  \lsucmeans{that you received an MPL-2.0 licensed code snippet, module,
    library, or plugin (snimoli), that you modified it, and that you are now
    going to distribute this modified version to third parties in the form of
    binary files or as a binary package together with another larger software
    unit which contains this code snippet, module, library, or plugin as an
    embedded component.}

  \mapsToOsuc{10B}

  \begin{lsucrequires}
    \lsucmandatory{\keepLicenseElements \addWhenCompiling}
    \lsucmandatory{\makeEmbeddedSourceAvailable}
    \lsucmandatory{\describeHowToGetSource}
    \lsucsourcedist{MPL-2.0-C8}
    \lsucmandatory{\placeEmbeddedBinaryUnderMPL}
    \lsucoptional{\createChangeLog}
    \lsucoptional{\giveLicenseText}\passingFilesCorrectly
    \lsucoptional{\keepBinarySeparate}
    \lsucoptional{\acknowlegdeEmbeddedLibrary}
  \end{lsucrequires}

  \begin{lsucprohibits}
    \lsucitem{\dontAlterLicenseElement}
    \lsucitem{\dontUseTrademarks}
  \end{lsucprohibits}
\end{lsuc}

% ------------------------------------------------------------------------------

\subsection{Discussions and Explanations}
\label{MPLDiscussion}
The MPL-2.0 offers a section \enquote{Responsibilities} which contains nearly all
requirements.\citeMPL{§3} Only for some subordinate aspects, one has also to
reflect other paragraphs.\citeMPL{pars pro to cf.}{§3 - concerning the trademarks}
With respect to this structure, we can detect the following tasks:

\begin{itemize}

\item In a more general attitude, the MPL-2.0 states that it \enquote{[\ldots]
  does not grant any rights in the trademarks, service marks, or logos of any
  Contributor}---except as it may be necessary \enquote{to comply with} other
  requirements of the license.\citeMPL{§2.3} The \oslic{} rewrites the message
  as the interdiction to promote own services and products by and with such
  elements. 
  
\item The MPL-2.0 also generally prescribes that \enquote{you may not remove or
  alter the substance of any license notice (including copyright notices, patent
  notices, disclaimer of warranties, or limitations of liabiliy) contained
  within the Source Code Form [\ldots]}\citeMPL{§3.4} This focussing to the
  \enquote{substance of any license notice} refers to the allowance to
  \enquote{[\ldots] alter any license notices to the extent required to remedy
  known factual innacuracies}.\citeMPL{§3.4}  Following its principle to offer
  one reliable way and to ignore variants of secondary importance, the \oslic{}
  simplifies this condition to the general proscription to modify any licensing
  material for all use cases [MPL-2.0-C1 -- MPL-2.0-C9]. But for emphasizing
  that this is a job which must be activily done, the \oslic{} additionally
  rewrites this interdiction into all \emph{2others} use cases [MPL-2.0-C2 --
  MPL-2.0-C9] as the task to retain the licensing elements in the form one has
  obtained them. 
  
\item Moreover, the MPL-2.0 requires for all \enquote{distributions of [the]
  source [code] form} that all modifications of the software \enquote{[\ldots] 
  must be under the terms of (the MPL-2.0)} and that the distributor
  \enquote{[\ldots] must inform} all \enquote{recipients} that the software
  \enquote{[\ldots] is governed by the terms of (the MPL-2.0), and how (the
  recipients) can obtain a copy of this license}.\citeMPL{§3.1}  For the
  respective use case (MPL-2.0-C2, MPL-2.0-C4, MPL-2.0-C6, MPL-2.0-C8), the
  \oslic{} rewrites these conditions so that each MPL-2.0 source code package
  must neccessarily contain the MPL-2.0 itself as textfile and an additional 
  licensing file or statement strictly following the text given by the addendum
  of the MPL-2.0.\citeMPL{Exhibit A} Because the MPL-2.0 is only a license with
  weak copyleft, the \oslic{} proposes to separate the MPL-2.0 licensed,
  embedded component from the enclosing program (MPL-2.0-C8). 
  
\item But the MPL-2.0 does not explicitly require marking all modifications.
  Nevertheless, this is state of the art in computer emgineering. Therefore,
  with respect to the cases of distributing modified source code (MPL-2.0-C4,
  MPL-2.0-C6 and MPL-2.0-C8), the \oslic{} proposes to mark all modifications
  inside of the source code and to update the description of the functional
  changes. In case of distributing the modified software in the form of
  binaries, it should be sufficient to describe the modifications only on the
  functional level. 
  
\item Furthermore, the MPL-2.0 requires that the \enquote{Covered Software}---in 
  all cases of distributing it in an \enquote{Executable Form} (MPL-2.0-C3,
  MPL-2.0-C5, MPL-2.0-C7, MPL-2.0-C9)---\enquote{[\ldots] must also be made
  available in Source Code Form [\ldots]} and that the distributor
  \enquote{[\ldots] must inform recipients of the Executable Form how they can
  obtain a copy of such Source Code Form by reasonable means in a timely manner,
  at a charge no more than the cost of distribution to the
  recipient}.\citeMPL{§3.2.a}  The \oslic{} rewrites these conditions as the
  obligation to offer a download service at no charge and to point towards this
  services inside of the distributed package.
  
\item In this context, the MPL-2.0 allows to distribute the binaries under terms
  of another license \enquote{[\ldots] provided that that the license for the
  Executable Form does not attempt to limit or alter the recipients’ rights in
  the Source Code Form under this License.}\citeMPL{§3.2.b} This possibility
  might become important for those cases where the license compatibility must
  explicitly be managed. Normally, it should be sufficient also to distribute
  the binaries under the MPL-2.0. Thus, in case of distributing binaries
  (MPL-2.0-C3, MPL-2.0-C5, MPL-2.0-C7, MPL-2.0-C9), the \oslic{} proposes to
  insert into the distribution packages the MPL-2.0 itself and an additional
  licensing file or statement strictly following the text given by the addendum
  of the MPL-2.0.\citeMPL{Exhibit A} But again, because the MPL-2.0 is only a
  license with weak copyleft, the \oslic{} proposes to separate the MPL-2.0
  licensed embedded component from the overarching program (MPL-2.0-C9).
  
\item Finally, one clearly has to state that the distribution of the source code
  required by the previous rule must, of course, follow the rules of distributing
  the software. Thus, the \oslic{} requires in all cases of a binary distribution
  to execute also the task-lists of the respective source code use cases.

\end{itemize}

% ------------------------------------------------------------------------------

\end{license}
%\bibliography{../../../bibfiles/oscResourcesEn}

% Local Variables:
% mode: latex
% fill-column: 80
% End:
}
{% Telekom osCompendium 'for being included' snippet template
%
% (c) Karsten Reincke, Deutsche Telekom AG, Darmstadt 2011
%
% This LaTeX-File is licensed under the Creative Commons Attribution-ShareAlike
% 3.0 Germany License (http://creativecommons.org/licenses/by-sa/3.0/de/): Feel
% free 'to share (to copy, distribute and transmit)' or 'to remix (to adapt)'
% it, if you '... distribute the resulting work under the same or similar
% license to this one' and if you respect how 'you must attribute the work in
% the manner specified by the author ...':
%
% In an internet based reuse please link the reused parts to www.telekom.com and
% mention the original authors and Deutsche Telekom AG in a suitable manner. In
% a paper-like reuse please insert a short hint to www.telekom.com and to the
% original authors and Deutsche Telekom AG into your preface. For normal
% quotations please use the scientific standard to cite.
%
% [ Framework derived from 'mind your Scholar Research Framework' 
%   mycsrf (c) K. Reincke 2012 CC BY 3.0  http://mycsrf.fodina.de/ ]
%


%% use all entries of the bibliography
%\nocite{*}

\section{Microsoft Public License}

\begin{license}{MSPL} % ends at end of file
\licensename{MS-PL}
\licensespec{Microsoft Public License}
\licenseabbrev{MS-PL}

The MS-PL license is also one of the most permissive licenses. Thus, the
MS-PL specific finder can be simplified:


\tikzstyle{nodv} = [font=\small, ellipse, draw, fill=gray!10, 
    text width=2cm, text centered, minimum height=2em]

\tikzstyle{nods} = [font=\footnotesize, rectangle, draw, fill=gray!20, 
    text width=1.2cm, text centered, rounded corners, minimum height=3em]

\tikzstyle{nodb} = [font=\footnotesize, rectangle, draw, fill=gray!20, 
    text width=2.2cm, text centered, rounded corners, minimum height=3em]
    
\tikzstyle{leaf} = [font=\tiny, rectangle, draw, fill=gray!30, 
    text width=1.2cm, text centered, minimum height=6em]

\tikzstyle{edge} = [draw, -latex']

\begin{tikzpicture}[]

\node[nodv] (l71) at (4,10) {MS-PL};

\node[nodb] (l61) at (0,8.6) {\textit{recipient:} \\ \textbf{4yourself}};
\node[nodb] (l62) at (6.5,8.6) {\textit{recipient:} \\ \textbf{2others}};

\node[nodb] (l51) at (2.5,7) {\textit{state:} \\ \textbf{unmodified}};
\node[nodb] (l52) at (9.3,7) {\textit{state:} \\ \textbf{modified}};

\node[nodb] (l43) at (6.5,5.4) {\textit{type:} \\ \textbf{proapse}};
\node[nodb] (l44) at (12,5.4) {\textit{type:} \\ \textbf{snimoli}};


\node[nods] (l31) at (5.4,3.8) {\textit{form:} \textbf{source}};
\node[nods] (l32) at (7.2,3.8) {\textit{form:} \textbf{binary}};
\node[nodb] (l33) at (10,3.8) {\textit{context:} \\ \textbf{independent}};
\node[nodb] (l34) at (13.5,3.8) {\textit{context:} \\ \textbf{embedded}};

\node[nods] (l21) at (9,2.2) {\textit{form:} \textbf{source}};
\node[nods] (l22) at (10.8,2.2) {\textit{form:} \textbf{binary}};
\node[nods] (l23) at (12.6,2.2) {\textit{form:} \textbf{source}};
\node[nods] (l24) at (14.4,2.2) {\textit{form:} \textbf{binary}};

\node[leaf] (l11) at (0,0) {\textbf{MS-PL-C1} \textit{using software only
for yourself}};

\node[leaf] (l12) at (2.5,0) { \textbf{MS-PL-C2} \textit{ distributing unmodified
software}};

\node[leaf] (l14) at (5.4,0) { \textbf{MS-PL-C3}  \textit{ distributing modified
program as sources}};

\node[leaf] (l15) at (7.2,0) { \textbf{MS-PL-C4}  \textit{ distributing modified
program as binaries}};

\node[leaf] (l16) at (9,0) { \textbf{MS-PL-C5}  \textit{ distributing modified
library as independent sources}};

\node[leaf] (l17) at (10.8,0) { \textbf{MS-PL-C6} \textit{distributing modified
library as independent binaries}};

\node[leaf] (l18) at (12.6,0) { \textbf{MS-PL-C7}  \textit{distributing
modified library as embedded sources}};

\node[leaf] (l19) at (14.4,0) { \textbf{MS-PL-C8}  \textit{ distributing modified
library as embedded binaries}};


\path [edge] (l71) -- (l61);
\path [edge] (l71) -- (l62);
\path [edge] (l61) -- (l11);
\path [edge] (l62) -- (l51);
\path [edge] (l62) -- (l52);
\path [edge] (l52) -- (l43);
\path [edge] (l52) -- (l44);
\path [edge] (l51) -- (l12);


\path [edge] (l43) -- (l31);
\path [edge] (l43) -- (l32);
\path [edge] (l44) -- (l33);
\path [edge] (l44) -- (l34);
\path [edge] (l31) -- (l14);
\path [edge] (l32) -- (l15);
\path [edge] (l33) -- (l21);
\path [edge] (l33) -- (l22);
\path [edge] (l34) -- (l23);
\path [edge] (l34) -- (l24);
\path [edge] (l21) -- (l16);
\path [edge] (l22) -- (l17);
\path [edge] (l23) -- (l18);
\path [edge] (l24) -- (l19);

\end{tikzpicture}

%%
%% Common Building Blocks
%%

% ------------------------------------------------------------------------------
% Forbid using name, logo, and trademarks
\newcommand{\noLogoOrTrademark}{to use any contributors' name, logo, or
  trademarks (without an additional or general legally based approval).}

% ------------------------------------------------------------------------------
\subsection{MS-PL-C1: Using the software only for yourself}
\begin{lsuc}{MS-PL-C1}
  \linkosuc{01}
  \linkosuc{03L} 
  \linkosuc{03N} 
  \linkosuc{06L}
  \linkosuc{06N}
  \linkosuc{09L}
  \linkosuc{09N}
  
  \lsucmeans{that you received MS-PL licensed software, that you will use it
  only for yourself and that you do not hand it over to any 3rd party in any
  sense.}

  \lsuccovers{OSUC-01, OSUC-03L, OSUC-03N, OSUC-06L, OSUC-06N, OSUC-09L, and
  OSUC-09N\footnote{For details see pp.\ \pageref{OSUC-01-DEF} -
  \pageref{OSUC-09N-DEF}}}

  \begin{lsucrequiresnothing}
    \lsucitem{You are allowed to use any kind of MS-PL licensed software in any 
      sense and in any context without any other obligations if you do not give
      the software to 3rd parties.} 
  \end{lsucrequiresnothing}

  \begin{lsucprohibits}
    \lsucitem{\noLogoOrTrademark} 
  \end{lsucprohibits}
\end{lsuc}

% ------------------------------------------------------------------------------
\subsection{MS-PL-C2: Passing the unmodified software}
\begin{lsuc}{MS-PL-C2}
  \linkosuc{02S} 
  \linkosuc{05S} 
  \linkosuc{07S} 
  \linkosuc{02B} 
  \linkosuc{05B} 
  \linkosuc{07B} 

  \lsucmeans{that you received MS-PL licensed software which you are now going to
  distribute to third parties in the form of unmodified binaries or as unmodifed
  source code files. In this case it makes no difference if you distribute a
  program, an application, a server, a snippet, a module, a library, or a plugin
  as an independent package.}

  \lsuccovers{OSUC-02S, OSUC-02B, OSUC-05S,  OSUC-05B, OSUC-07S,
    OSUC-07B\footnote{For details $\rightarrow$ \oslic,
      pp.\ \pageref{OSUC-02B-DEF} -- \pageref{OSUC-07B-DEF}}}

  \begin{lsucrequires}
    \lsucmandatory{Ensure that all licensing elements (particularly all
      copyright, patent, trademark, and attribution notices that are part of the
      version you received) are completely retained in your package.} 
  
    \lsucmandatory{Incorporate a complete copy of the MS-PL license into your 
      package, regardless whether you distribute a source code or a binary
      package.}%
    \footnote{$\rightarrow$ \oslic, p.\ \pageref{MsplSourceBinHint}} 
  
    \lsucoptional{It's a good tradition to let the documentation of your
      distribution and/or your additional material also contain a link to the
      original software (project) and its homepage.} 
  \end{lsucrequires}

  \begin{lsucprohibits}
    \lsucitem{\noLogoOrTrademark} 
  \end{lsucprohibits}
\end{lsuc}

% ------------------------------------------------------------------------------
\subsection{MS-PL-C3: Passing a modified program as source code}
\begin{lsuc}{MS-PL-C3}
  \linkosuc{04S}

  \lsucmeans{that you received an MS-PL licensed program, application, or
  server (proapse), that you modified it, and that you are now going to
  distribute this modified version to third parties in the form of source code files or as
  a source code package.}

  \lsuccovers{OSUC-04S\footnote{For details $\rightarrow$ \oslic,
      pp.\ \pageref{OSUC-04S-DEF}}} 

  \begin{lsucrequires}
    \lsucmandatory{Ensure that all licensing elements (particularly all
      copyright, patent, trademark, and attribution notices that are part of the
      version you received) are completely retained in your package.} 
 
    \lsucmandatory{Incorporate a complete copy of the MS-PL license into your
      package.} 
  
    \lsucmandatory{If you do not want to publish your modifications under the
      MS-PL too, then cleanly separate your own sources and licensing documents
      from original elements of the adopted work.} 
  
    \lsucoptional{Mark your modifications in the sourcecode.}
  
    \lsucoptional{It's a good tradition to let the documentation of your
      distribution or your additional material also contain a link to the
      original software (project) and its homepage (as far as this does not
      clashes with the prohibitions stated below).} 
  
    \lsucoptional{You are allowed to expand an existing copyright notice of the
      program to mention your own contributions.} 
  
    \lsucoptional{It is a good practice of the open source community, to let the
      copyright notice which is shown by the running program also state that the
      program is licensed under the MS-PL license (as far as this does not
      clashes with the prohibitions stated below). Because you are already
      modifying the program, you can also add such a hint, if the  original
      copyright notice lacks such a statement.} 
  \end{lsucrequires}

  \begin{lsucprohibits}
    \lsucitem{\noLogoOrTrademark} 
  \end{lsucprohibits}
\end{lsuc}

% ------------------------------------------------------------------------------
\subsection{MS-PL-C4: Passing a modified program as binary}
\begin{lsuc}{MS-PL-C4}
  \linkosuc{04B}

  \lsucmeans{that you received an MS-PL licensed program, application, or
  server (proapse), that you modified it, and that you are now going to
  distribute this modified version to third parties in the form of binary files or as a
  binary package.}

  \lsuccovers{OSUC-04B\footnote{For details $\rightarrow$ \oslic,
      pp.\ \pageref{OSUC-04B-DEF}}} 

  \begin{lsucrequires}
  
    \lsucoptional{Mark your modifications in the source code even if you do not
      intend to distribute it.} 
  
    \lsucoptional{It's a good tradition to let the documentation of your
      distribution or your additional material also contain a link to the
      original software (project) and its homepage (as far as this does not
      clashes with with the prohibitions stated below).} 
  
    \lsucoptional{It is a good practice of the open source community, to let the
      copyright notice which is shown by the running program also state that the
      derivative work is based on a version originally licensed under the MS-PL
      license (as far as this does not clashes with the prohibitions stated
      below), perhaps by linking to the project homepage of the original.
      Because you are already modifying the program, you can also add such a
      hint, if the original copyright notice lacks such a statement.} 
  \end{lsucrequires}

  \begin{lsucprohibits}
    \lsucitem{\noLogoOrTrademark} 
  \end{lsucprohibits}
\end{lsuc}

% ------------------------------------------------------------------------------
\subsection{MS-PL-C5: Passing a modified library independently as source code}
\begin{lsuc}{MS-PL-C5}
  \linkosuc{08S}

  \lsucmeans{that you received an MS-PL licensed code snippet, module, library,
  or plugin (snimoli), that you modified it, and that you are now going to
  distribute this modified version to third parties in the form of source code
  files or as a source code package, but without embedding it into another
  larger software unit.}

  \lsuccovers{OSUC-08S\footnote{For details $\rightarrow$ \oslic, pp.\ \pageref{OSUC-08S-DEF}}}

  \begin{lsucrequires}
    \lsucmandatory{Ensure that all licensing elements (particularly all
      copyright, patent, trademark, and attribution notices that are part of the
      version you received) are completely retained in your package.} 
 
    \lsucmandatory{Incorporate a complete copy of the MS-PL license into your
      package.} 
  
    \lsucmandatory{If you do not want to publish your modifications under the
      MS-PL too, then cleanly separate your own sources and licensing documents
      from original elements of the adopted part(s).} 
  
    \lsucoptional{Mark your modifications in the sourcecode.} 
  
    \lsucoptional{It's a good tradition to let the documentation of your
      distribution or your additional material also contain a link to the
      original software (project) and its homepage (as far as this does not
      clashes with with the prohibitions stated below).} 
  \end{lsucrequires}

  \begin{lsucprohibits}
    \lsucitem{\noLogoOrTrademark}
  \end{lsucprohibits}
\end{lsuc}

% ------------------------------------------------------------------------------
\subsection{MS-PL-C6: Passing a modified library independently as binary}
\begin{lsuc}{MS-PL-C6}
  \linkosuc{08B}

  \lsucmeans{that you received an MS-PL licensed code snippet, module, library,
  or plugin (snimoli), that you modified it, and that you are now going to
  distribute this modified version to third parties in the form of binary files
  or as a binary package but without embedding it into another larger software
  unit.}

  \lsuccovers{OSUC-08B\footnote{For details $\rightarrow$ \oslic,
      pp.\ \pageref{OSUC-08B-DEF}}} 

  \begin{lsucrequires}
    \lsucoptional{Mark your modifications in the source code even if do not want
      to distribute it.} 
  
    \lsucoptional{It's a good tradition to let the documentation of your
      distribution or your additional material also contain a link to the
      original software (project) and its homepage (as far as this does not
      clashes with with the prohibitions stated below).} 
  \end{lsucrequires}

  \begin{lsucprohibits}
    \lsucitem{\noLogoOrTrademark} 
  \end{lsucprohibits}
\end{lsuc}

% ------------------------------------------------------------------------------
\subsection{MS-PL-C7: Passing a modified library as embedded source code}
\begin{lsuc}{MS-PL-C7}
  \linkosuc{10S}

  \lsucmeans{that you received an MS-PL licensed code snippet, module, library,
  or plugin (snimoli), that you modified it, and that you are now going to
  distribute this modified version to third parties in the form of source code
  files or as a source code package together with another larger software unit
  which contains this code snippet, module, library, or plugin as an embedded
  component.}

  \lsuccovers{OSUC-10S\footnote{For details $\rightarrow$ \oslic,
      pp.\ \pageref{OSUC-10S-DEF}}} 

  \begin{lsucrequires}
    \lsucoptional{Ensure that all licensing elements (particularly all
      copyright, patent, trademark, and attribution notices that are part of the
      version you received are completely retained in your package.} 
 
    \lsucoptional{Incorporate a complete copy of the MS-PL license into your
      package.} 
  
    \lsucoptional{If you do not want to publish your modifications or your
      overarching application under the MS-PL too, then cleanly separate your
      own sources and licensing documents from original elements of the adopted
      work.} 
  
    \lsucoptional{Mark your modifications in the sourcecode.}
  
    \lsucoptional{It's a good tradition to let the documentation of your
      distribution or your additional material also contain a link to the
      original software (project) and its homepage (as far as this does not
      clashes with with the prohibitions stated below).}

    \lsucoptional{It is a good practice of the open source community, to let the
      copyright notice shown by your overarching program also state that it is
      based on a component originally licensed under the MS-PL license, perhaps
      by linking the project homepage of the original (as far as this does not
      clashes with the prohibitions stated below).}  
  \end{lsucrequires}

  \begin{lsucprohibits}
    \lsucitem{\noLogoOrTrademark} 
  \end{lsucprohibits}
\end{lsuc}

% ------------------------------------------------------------------------------
\subsection{MS-PL-C8: Passing a modified library as embedded binary}
\begin{lsuc}{MS-PL-C8}
  \linkosuc{10B}

  \lsucmeans{that you received an MS-PL licensed code snippet, module, library,
  or plugin (snimoli), that you modified it, and that you are now going to
  distribute this modified version to third parties in the form of binary files
  or as a binary package together with another larger software unit which
  contains this code snippet, module, library, or plugin as an embedded component.}

  \lsuccovers{OSUC-10B\footnote{For details $\rightarrow$ \oslic,
      pp.\ \pageref{OSUC-10B-DEF}}} 

  \begin{lsucrequires}
    \lsucoptional{Mark your modifications in the source code even if do not want
      to distribute it.} 
  
    \lsucoptional{It's a good tradition to let the documentation of your
      distribution and/or your additional material also contain a link to the
      original software (project) and its homepage (as far as this does not
      clashes with with the prohibitions stated below).}
  
    \lsucoptional{It is a good practice of the open source community, to let the
      copyright notice shown by your own overarching program also state that it
      is based on a component originally licensed under the MS-PL license, 
      perhaps by linking the project homepage of the original (as far as this
      does not clashes with the prohibitions stated below).}
  \end{lsucrequires}

  \begin{lsucprohibits}
    \lsucitem{\noLogoOrTrademark} 
  \end{lsucprohibits}
\end{lsuc}

% ------------------------------------------------------------------------------
\subsection{Discussions and Explanations}
\label{MSPLDiscussion}

The MS-PL is also a very permissive and short license. It requires to do:
(a) You must preserve existing licensing elements. (b) You must distribute
the source code as whole or \enquote{portions} of the source code under the
MS-PL. (c) You must add a copy of the license if you distribute (parts of) the
source code. (d) If you distribute a binary package, you must distribute (the
parts of) the work under a license \enquote{that complies with this (MS-PL)
license}\footcite[cf.][\nopage wp]{MsplOsiLicense2013a}.

The most confusing clause is probably the condition, to \enquote{[\ldots]
distribute any portion of the software in compiled or object code form [\ldots]
only [\ldots] under a license that complies with this license}. But a closer
examination is lighting the situation: The only other conditions of the license
which refer to the context of distributing binaries are the requirements a) not
to abuse trademarks, b) not to bring a patent claim against any contributor, and
c) not to expect any warranties or guarantees with respect to the distributed
portion\footcite[cf.][\nopage wp.\ §3A, §3B, §3E]{MsplOsiLicense2013a}.

Based on these readings we decided \ldots

\label{MsplSourceBinHint} 
\begin{itemize}
  \item \ldots to let you incorporate a copy of the license into your
  distribution even if it only contains the binaries of the unmodified version:
  if you have not modified it, you do not lose any advantage if you add the
  license, too. So, this is the best method to fulfill the \emph{MSL-PL binary
  condition}.
  \item \ldots to erase all mandatory conditions in case of the binary
  distributions: the patent restriction of the MS-PL itself is already covered
  by the MS-PL patent section of the \oslic\footnote{$\rightarrow$ \oslic, p.\
  \patentpageref{MSPL}} and the no warranty clause of the MS-PL by
  the \oslic{} section concerning the power of the MS-PL\footnote{$\rightarrow$
  \oslic, p.\ \protectionpageref{MSPL}} while the trademark
  restrictions are explicitly added into the prohibition section.
  \item \ldots to erase the hints to a voluntarily updated copyright dialog in
  case of distributing a snimoli independently because the copyright dialog
  normally is designed by the overarching work which uses the library, not by
  the library itself.
\end{itemize}

\end{license}

%\bibliography{../../../bibfiles/oscResourcesEn}

% Local Variables:
% mode: latex
% fill-column: 80
% End:
}
{% Telekom osCompendium 'for being included' snippet template
%
% (c) Karsten Reincke, Deutsche Telekom AG, Darmstadt 2011
%
% This LaTeX-File is licensed under the Creative Commons Attribution-ShareAlike
% 3.0 Germany License (http://creativecommons.org/licenses/by-sa/3.0/de/): Feel
% free 'to share (to copy, distribute and transPGL)' or 'to remix (to adapt)'
% it, if you '... distribute the resulting work under the same or similar
% license to this one' and if you respect how 'you must attribute the work in
% the manner specified by the author ...':
%
% In an internet based reuse please link the reused parts to www.telekom.com and
% mention the original authors and Deutsche Telekom AG in a suitable manner. In
% a paper-like reuse please insert a short hint to www.telekom.com and to the
% original authors and Deutsche Telekom AG into your preface. For normal
% quotations please use the scientific standard to cite.
%
% [ Framework derived from 'mind your Scholar Research Framework' 
%   mycsrf (c) K. Reincke 2012 CC BY 3.0  http://mycsrf.fodina.de/ ]
%


%% use all entries of the bibliography
%\nocite{*}

\section{PostgreSQL License}
\begin{license}{PGL} % ends at end of file
\licensename{PostgreSQL}
\licensespec{PostgreSQL license}
\licenseabbrev{PGL}

Like the MIT License Postgres License is a very permissive licenses. Thus, the
PostgreSQL specific finder can be simplified:

\tikzstyle{nodv} = [font=\small, ellipse, draw, fill=gray!10, 
    text width=2cm, text centered, minimum height=2em]

\tikzstyle{nods} = [font=\footnotesize, rectangle, draw, fill=gray!20, 
    text width=1.2cm, text centered, rounded corners, minimum height=3em]

\tikzstyle{nodb} = [font=\footnotesize, rectangle, draw, fill=gray!20, 
    text width=2.2cm, text centered, rounded corners, minimum height=3em]

\tikzstyle{nodx} = [font=\footnotesize, rectangle, draw, fill=gray!20, 
    text width=2.4cm, text centered, rounded corners, minimum height=3em]
    
\tikzstyle{leaf} = [font=\tiny, rectangle, draw, fill=gray!30, 
    text width=2cm, text centered, minimum height=4em]

\tikzstyle{edge} = [draw, -latex']

\begin{tikzpicture}[]

\node[nodv] (l61) at ( 2.4, 9.2) {PostgreSQL};

\node[nodb] (l51) at ( 0.0, 7.8) {\textit{recipient:} \\ \textbf{4yourself}};
\node[nodb] (l52) at ( 4.8, 7.8) {\textit{recipient:} \\ \textbf{2others}};

\node[nodb] (l41) at ( 2.5, 6.2) {\textit{state:} \\ \textbf{unmodified}};
\node[nodb] (l42) at ( 7.0, 6.2) {\textit{state:} \\ \textbf{modified}};

\node[nodb] (l31) at ( 5.0, 4.6) {\textit{type:} \\ \textbf{proapse}};
\node[nodb] (l32) at ( 9.0, 4.6) {\textit{type:} \\ \textbf{snimoli}};

\node[nodx] (l21) at ( 7.5, 2.8) {\textit{context:} \\ \textbf{independent}};
\node[nodx] (l22) at (10.5, 2.8) {\textit{context:} \\ \textbf{embedded}};

\node[leaf] (l11) at ( 0.0, 0.0) {\textbf{PostgreSQL-C1} \textit{using software only for yourself}};
\node[leaf] (l12) at ( 2.5, 0.0) {\textbf{PostgreSQL-C2} \textit{distributing unmodified package}};
\node[leaf] (l13) at ( 5.0, 0.0) {\textbf{PostgreSQL-C3} \textit{distributing modified program}};
\node[leaf] (l14) at ( 7.5, 0.0) {\textbf{PostgreSQL-C4} \textit{distributing modified library as independent package}};
\node[leaf] (l15) at (10.5, 0.0) {\textbf{PostgreSQL-C5} \textit{distributing modified library as embedded package}};


\path [edge] (l61) -- (l51);
\path [edge] (l61) -- (l52);
\path [edge] (l51) -- (l11);
\path [edge] (l52) -- (l41);
\path [edge] (l52) -- (l42);
\path [edge] (l41) -- (l12);
\path [edge] (l42) -- (l31);
\path [edge] (l42) -- (l32);
\path [edge] (l31) -- (l13);
\path [edge] (l32) -- (l21);
\path [edge] (l32) -- (l22);
\path [edge] (l21) -- (l14);
\path [edge] (l22) -- (l15);

\end{tikzpicture}

%% =============================================================================
%% Common Building Blocks
%%

% ------------------------------------------------------------------------------
% Include license in package

\newcommand{\giveLicense}{Ensure that the complete PostgreSQL license including
  the copyright notice, the permission notices, and the PostgreSQL disclaimer
  are retained in your package in the form you have received them.}

% ------------------------------------------------------------------------------
% Add a link to the project's home page to the documentation

\newcommand{\linkToHomepage}{It's a good tradition to let the documentation of
  your distribution or your additional material also contain a link to the
  original software (project) and its homepage.}

% ------------------------------------------------------------------------------
% mark your modifications

\newcommand{\markModifications}{Mark your modifications in the source code,
  regardless whether you want to distribute the code or not.}

% ------------------------------------------------------------------------------
% add name and link to copyright dialog

\newcommand{\addToCopyrightDialog}{It is a good practice of the open source
  community to let the copyright notice, which is shown by the running program,
  also state that the program uses a component being licensed under the
  PostgreSQL license.  And it is a good tradition to insert links to the
  homepage or download page of this embedded component.}

% ------------------------------------------------------------------------------
% separate the components but keep component and license together

\newcommand{\separateComponents}{Arrange your distribution so that the original
  licensing elements (in particular the PostgreSQL license text containing the
  copyright notices of the original author(s), the permission notices and the
  PostgrSGL disclaimer) clearly refer only to the embedded library and do not
  affect the licensing of your own overarching work. Consider keeping embedded
  libraries, modules, snippets, or plugins in separate directories which also
  contain all their licensing elements.}

% ------------------------------------------------------------------------------
% add your own copyright notice to the copyright dialog

\newcommand{\addYourOwnCopyright}{You can add information about your own work or
  modifications to an existing copyright notice presented by the program.}

% ------------------------------------------------------------------------------
% acknowledge the original work in the copyright notice

\newcommand{\acknowledgeOriginalWork}{It is a good practice of the open source
  community to let the copyright notice, which is shown by the program, also
  state that it is based on a version originally licensed under the PostgreSQL
  license. Because you are already modifying the program, you may want to add
  such a hint, if the original copyright notice lacks such a statement.}

%% =============================================================================
%% Use Cases
%%

\subsection{PostgreSQL-C1: Using the software only for yourself}
\begin{lsuc}{PostgreSQL-C1}
  \linkosuc{01}
  \linkosuc{03L} 
  \linkosuc{03N} 
  \linkosuc{06L}
  \linkosuc{06N}
  \linkosuc{09L}
  \linkosuc{09N}
  
  \lsucmeans{that you received PostgreSQL licensed software, that you will use it
  only for yourself, and that you do not hand it over to any 3rd party in any
  sense.} 

  \lsuccovers{OSUC-01, OSUC-03L, OSUC-03N, OSUC-06L, OSUC-06N, OSUC-09L, and
  OSUC-09N\footnote{For details $\rightarrow$ \oslic, pp.\ \pageref{OSUC-01-DEF}
  - \pageref{OSUC-09N-DEF}}}

  \begin{lsucrequiresnothing}
    \lsucitem{You are allowed to use any kind of PostgreSQL licensed software in any
      sense and in any context without any other obligations if you do not give
      the software to third parties and if you do not modify the existing
      copyright notices or the existing permission notice.}
  \end{lsucrequiresnothing}

  \lsucprohibitsnothing 

\end{lsuc}

% ------------------------------------------------------------------------------
\subsection{PostgreSQL-C2: Passing the unmodified software}
\begin{lsuc}{PostgreSQL-C2}
  \linkosuc{02S} 
  \linkosuc{05S} 
  \linkosuc{07S} 
  \linkosuc{02B} 
  \linkosuc{05B} 
  \linkosuc{07B} 

  \lsucmeans{that you received PostgreSQL licensed software which you are now going to
  distribute to third parties in the form of unmodified binaries or as unmodifed
  source code files. In this case it makes no difference if you distribute a
  program, an application, a server, a snippet, a module, a library, or a plugin
  as an independent package.} 

  \lsuccovers{OSUC-02S,  OSUC-02B, OSUC-05S, OSUC-05B, OSUC-07S, OSUC-07B%
    \footnote{For details $\rightarrow$ \oslic, 
      pp.\ \pageref{OSUC-02S-DEF} -- \pageref{OSUC-07B-DEF}}}

  \begin{lsucrequires}
    \lsucmandatory{\giveLicense}
    \lsucoptional{\linkToHomepage}
  \end{lsucrequires}

  \lsucprohibitsnothing
\end{lsuc}

% ------------------------------------------------------------------------------
\subsection{PostgreSQL-C3: Passing a modified program}
\begin{lsuc}{PostgreSQL-C3}
  \linkosuc{04S} 
  \linkosuc{04B}

  \lsucmeans{that you received a PostgreSQL licensed program, application, or
  server (proapse), that you modified it, and that you are now going to distribute this
  modified version to third parties in the form binaries or as source code
  files.}
 
  \lsuccovers{OSUC-04S, OSUC-04B%
    \footnote{For details $\rightarrow$ \oslic, pp.\ \pageref{OSUC-04S-DEF}}}

  \begin{lsucrequires}
    \lsucmandatory{\giveLicense}
    \lsucoptional{\markModifications}
    \lsucoptional{\linkToHomepage}
    \lsucoptional{\addYourOwnCopyright}
    \lsucoptional{\acknowledgeOriginalWork}
  \end{lsucrequires}

  \lsucprohibitsnothing
\end{lsuc}

% ------------------------------------------------------------------------------
\subsection{PostgreSQL-C4: Passing a modified library independently}
\begin{lsuc}{PostgreSQL-C4}
  \linkosuc{08S}
  \linkosuc{08B}

  \lsucmeans{that you received a PostgreSQL licensed code snippet, module, library, or
  plugin (snimoli), that you modified it, and that you are now going to
  distribute this modified version to third parties in the form of binaries or
  as source code files together with another larger software unit which contains
  this code snippet, module, library, or plugin as an embedded component,
  regardless whether you distribute it in the form of binaries or as source code
  files.}

  \lsuccovers{OSUC-08S, OSUC-08B%
    \footnote{For details $\rightarrow$ \oslic, pp.\ \pageref{OSUC-08B-DEF}}}

  \begin{lsucrequires}
    \lsucmandatory{\giveLicense}
    \lsucoptional{\markModifications}
    \lsucoptional{\linkToHomepage}
  \end{lsucrequires}

  \lsucprohibitsnothing
\end{lsuc}

% ------------------------------------------------------------------------------
\subsection{PostgreSQL-C5: Passing a modified library as embedded component}
\begin{lsuc}{PostgreSQL-C5}
  \linkosuc{10S} 
  \linkosuc{10B}

  \lsucmeans{that you received a PostgreSQL licensed code snippet, module, library, or
  plugin (snimoli), that you modified it, and that you are now going to
  distribute this modified version to third parties in the form of binaries or
  as source code files together with another larger software unit which contains
  this code snippet, module, library, or plugin as an embedded component,
  regardless whether you distribute it in the form of binaries or as source code
  files.}

  \lsuccovers{OSUC-10S, OSUC-10B%
    \footnote{For details $\rightarrow$ \oslic, pp.\ \pageref{OSUC-10S-DEF}}}

  \begin{lsucrequires}
    \lsucmandatory{\giveLicense}
    \lsucoptional{\markModifications}
    \lsucoptional{\addToCopyrightDialog} 
    \lsucoptional{\linkToHomepage}
    \lsucoptional{\separateComponents}
  \end{lsucrequires}

  \lsucprohibitsnothing
\end{lsuc}

%% =============================================================================
%% Discussion
%%

\subsection{Discussions and Explanations}
\label{PGLDiscussion}

The PostgreSQL-License follows the structure of the MIT license: it, too, contains 
(1) a copyright notice, 
(2) a paragraph saying that you are allowed to do almost anything you want,
    followed 
(3) by the condition that the copyright notice, the permission notes, and the
    disclaimer \enquote{[\ldots] apperar in all copies}, and 
(4) the well known disclaimer.\citePGL{}
Moreover, like the MIT license, the PostgreSQL does not talk about the
difference between source code and object code. So, you can apply the analysis
of the MIT license\footnote{$\rightarrow$ \oslic, p. \pageref{MITDiscussion}} 
also to the PostgreSQL.

\end{license}
%\bibliography{../../../bibfiles/oscResourcesEn}

% Local Variables:
% mode: latex
% fill-column: 80
% End:
}
{% Telekom osCompendium 'for being included' snippet template
%
% (c) Karsten Reincke, Deutsche Telekom AG, Darmstadt 2011
%
% This LaTeX-File is licensed under the Creative Commons Attribution-ShareAlike
% 3.0 Germany License (http://creativecommons.org/licenses/by-sa/3.0/de/): Feel
% free 'to share (to copy, distribute and transmit)' or 'to remix (to adapt)'
% it, if you '... distribute the resulting work under the same or similar
% license to this one' and if you respect how 'you must attribute the work in
% the manner specified by the author ...':
%
% In an internet based reuse please link the reused parts to www.telekom.com and
% mention the original authors and Deutsche Telekom AG in a suitable manner. In
% a paper-like reuse please insert a short hint to www.telekom.com and to the
% original authors and Deutsche Telekom AG into your preface. For normal
% quotations please use the scientific standard to cite.
%
% [ Framework derived from 'mind your Scholar Research Framework' 
%   mycsrf (c) K. Reincke 2012 CC BY 3.0  http://mycsrf.fodina.de/ ]
%


%% use all entries of the bibliography
%\nocite{*}

\section{PHP-3.0 licensed software}

\begin{license}{PHP} % ends at end of file
\licensename{PHP-3.0}
\licensespec{PHP 3.0 License}
\licenseversion{3.0}
\licenseabbrev{PHP}


The PHP-3.0 license contains a few more conditions than the MIT license and
additionally distinguishes the \enquote{redistribution of source
code}\citePHP{} from the \enquote{redistribution in binary form}.\citePHP{}
Nevertheless, the PHP-3.0 license focusses only on the redistribution or---as we
call it in the \oslic---\emph{the 2others use
cases.} Thus, the PHP-3.0 specific finder can be simplified:

\tikzstyle{nodv} = [font=\small, ellipse, draw, fill=gray!10, 
    text width=2cm, text centered, minimum height=2em]


\tikzstyle{nods} = [font=\footnotesize, rectangle, draw, fill=gray!20, 
    text width=1.2cm, text centered, rounded corners, minimum height=3em]

\tikzstyle{nodb} = [font=\footnotesize, rectangle, draw, fill=gray!20, 
    text width=2.2cm, text centered, rounded corners, minimum height=3em]
    
\tikzstyle{leaf} = [font=\tiny, rectangle, draw, fill=gray!30, 
    text width=1.2cm, text centered, minimum height=6em]

\tikzstyle{edge} = [draw, -latex']

\begin{tikzpicture}[]

\node[nodv] (l71) at (3.5,10) {PHP-3.0};

\node[nodb] (l61) at (0,8.6) {\textit{recipient:} \\ \textbf{4yourself}};
\node[nodb] (l62) at (6.5,8.6) {\textit{recipient:} \\ \textbf{2others}};

\node[nodb] (l51) at (2.5,7) {\textit{state:} \\ \textbf{unmodified}};
\node[nodb] (l52) at (9.3,7) {\textit{state:} \\ \textbf{modified}};

\node[nods] (l41) at (1.8,5.4) {\textit{form:} \textbf{source}};
\node[nods] (l42) at (3.6,5.4) {\textit{form:} \textbf{binary}};
\node[nodb] (l43) at (6.5,5.4) {\textit{type:} \\ \textbf{proapse}};
\node[nodb] (l44) at (12,5.4) {\textit{type:} \\ \textbf{snimoli}};


\node[nods] (l31) at (5.4,3.8) {\textit{form:} \textbf{source}};
\node[nods] (l32) at (7.2,3.8) {\textit{form:} \textbf{binary}};
\node[nodb] (l33) at (10,3.8) {\textit{context:} \\ \textbf{independent}};
\node[nodb] (l34) at (13.5,3.8) {\textit{context:} \\ \textbf{embedded}};

\node[nods] (l21) at (9,2.2) {\textit{form:} \textbf{source}};
\node[nods] (l22) at (10.8,2.2) {\textit{form:} \textbf{binary}};
\node[nods] (l23) at (12.6,2.2) {\textit{form:} \textbf{source}};
\node[nods] (l24) at (14.4,2.2) {\textit{form:} \textbf{binary}};

\node[leaf] (l11) at (0,0) {\textbf{PHP-3.0-C1} \textit{using software only
for yourself}};

\node[leaf] (l12) at (1.8,0) { \textbf{PHP-3.0-C2} \textit{ distributing unmodified
software as sources}};

\node[leaf] (l13) at (3.6,0) { \textbf{PHP-3.0-C3}  \textit{ distributing unmodified
software as binaries}};

\node[leaf] (l14) at (5.4,0) { \textbf{PHP-3.0-C4}  \textit{ distributing modified
program as sources}};

\node[leaf] (l15) at (7.2,0) { \textbf{PHP-3.0-C5}  \textit{ distributing modified
program as binaries}};

\node[leaf] (l16) at (9,0) { \textbf{PHP-3.0-C6}  \textit{ distributing modified
library as independent sources}};

\node[leaf] (l17) at (10.8,0) { \textbf{PHP-3.0-C7} \textit{distributing modified
library as independent binaries}};

\node[leaf] (l18) at (12.6,0) { \textbf{PHP-3.0-C8}  \textit{distributing
modified library as embedded sources}};

\node[leaf] (l19) at (14.4,0) { \textbf{PHP-3.0-C9}  \textit{ distributing modified
library as embedded binaries}};

\path [edge] (l71) -- (l61);
\path [edge] (l71) -- (l62);
\path [edge] (l61) -- (l11);
\path [edge] (l62) -- (l51);
\path [edge] (l62) -- (l52);
\path [edge] (l51) -- (l41);
\path [edge] (l51) -- (l42);
\path [edge] (l52) -- (l43);
\path [edge] (l52) -- (l44);
\path [edge] (l41) -- (l12);
\path [edge] (l42) -- (l13);
\path [edge] (l43) -- (l31);
\path [edge] (l43) -- (l32);
\path [edge] (l44) -- (l33);
\path [edge] (l44) -- (l34);
\path [edge] (l31) -- (l14);
\path [edge] (l32) -- (l15);
\path [edge] (l33) -- (l21);
\path [edge] (l33) -- (l22);
\path [edge] (l34) -- (l23);
\path [edge] (l34) -- (l24);
\path [edge] (l21) -- (l16);
\path [edge] (l22) -- (l17);
\path [edge] (l23) -- (l18);
\path [edge] (l24) -- (l19);

\end{tikzpicture}

%% =============================================================================
%% Common building blocks
%%

% ------------------------------------------------------------------------------
% Preserve the license elements

\newcommand{\keepPHPLicense}{Ensure that the complete PHP-3.0 license
  (especially the copyright notice, the PHP-3.0 conditions, and the PHP-3.0
  disclaimer) are retained in your package in the form you have received them.}

% ------------------------------------------------------------------------------
% Reproduce the license elements
\newcommand{\reproducePHPLicense}{Ensure that the complete PHP-3.0 license
  (especially the copyright notice, the PHP-3.0 conditions, and the PHP-3.0
  disclaimer) are \emph{reproduced} by your package in the form you have
  received them. If you compile the binary file from the source code package and
  if this process does not also generate and integrate the licensing files then
  create the copyright notice, the PHP-3.0 conditions, and the PHP-3.0
  disclaimer in the form present in the source code package and insert these
  files into your distribution manually.}

% ------------------------------------------------------------------------------
% Mark modifications in the source

\newcommand{\markModifications}{Mark your modifications in the source code.}
\newcommand{\markUndistributedModifications}{Mark your modifications in the
  source code, even if you do not want to distribute the code.}

% ------------------------------------------------------------------------------
% Acknowledge PHP in documentation and link to homepage

\newcommand{\acknowledgePHPInDocumentation}{Let the documentation of your
  distribution or your additional material also contain a line of acknowledgment
  in the form:
  \enquote{This product includes PHP, freely available from http://www.php.net/}}

% ------------------------------------------------------------------------------
% Add license to documentation

\newcommand{\addLicenseToDocumentation}{Let the documentation of your
  distribution and/or your additional material also contain the original
  copyright notice, the PHP-3.0 conditions, and the PHP-3.0 disclaimer.}

% ------------------------------------------------------------------------------
% Create a copyright dialog or add elements to it

\newcommand{\auxCDIntro}[1]{It is a good practice of the open source community
  to let the copyright notice, which is shown by the running program, also state
  that the program #1 under the PHP-3.0 license.}
\newcommand{\auxCDElements}{\emph{reproduce} the complete PHP-3.0 license
  including the copyright notice, the PHP-3.0 conditions, and the PHP-3.0
  disclaimer (as it is required for binary distributions.)} 

\newcommand{\createCopyrightDialog}{%
  \auxCDIntro{is licensed}.
  Because you are already modifying the program you can also add such a hint if
  the original copyright notice lacks such a statement. If such a notice is
  missing in the copyright screen, consider, if it is possible to let it
  \auxCDElements} 

\newcommand{\listEmbeddedLibraryInCopyrightDialog}{%
  \auxCDIntro{uses a component licensed}
  So, let the copyright screen of the enclosing program \auxCDElements}

% ------------------------------------------------------------------------------
% Separate embedded library from enclosing program

\newcommand{\auxKeepSeparate}[1]{Arrange your #1 distribution so that the
  licensing elements (especially the PHP-3.0 license text, the specific
  copyright notice of the original author(s), and the PHP-3.0 disclaimer)
  clearly refer only to the embedded library and do not affect the licensing of
  your own overarching work. It's a good tradition to keep embedded components
  like libraries, modules, snippets, or plugins in separate directories, which
  contain also all additional licensing elements.}

\newcommand{\keepSourceSeparate}{\auxKeepSeparate{source code}}
\newcommand{\keepBinarySeparate}{\auxKeepSeparate{binary}}

% ------------------------------------------------------------------------------
% Forbid to use the name PHP

\newcommand{\toUseTheNamePHPForServices}{to endorse or promote any service you
  establish based on this software by the name `PHP.'}
\newcommand{\toUseTheNamePHP}{to endorse or promote your product by mentioning
  PHP, especially not by making the string `PHP' part of its name.}

%% =============================================================================
%% Use Cases

\subsection{PHP-3.0-C1: Using the software only for yourself}
\begin{lsuc}{PHP-3.0-C1}
  \linkosuc{01}
  \linkosuc{03L} 
  \linkosuc{03N} 
  \linkosuc{06L}
  \linkosuc{06N}
  \linkosuc{09L}
  \linkosuc{09N}
  
  \lsucmeans{that you received PHP-3.0 licensed software, that you will use it
    only for yourself, and that you do not hand it over to any third party in
    any sense.}

  \coversOsucs{OSUC-01, OSUC-03L, OSUC-03N, OSUC-06L, OSUC-06N, OSUC-09L, and
  OSUC-09N}{01}{09N}
  
  \begin{lsucrequiresnothing}
    \lsucitem{You are allowed to use any kind of PHP-3.0 software in any sense
      and in any context without any obligations as long as you do not give the
      software to third parties.}
  \end{lsucrequiresnothing}

  \begin{lsucprohibits}
    \lsucitem{\toUseTheNamePHPForServices}
  \end{lsucprohibits}
\end{lsuc}

% ------------------------------------------------------------------------------
\subsection{PHP-3.0-C2: Passing the unmodified software as source code}
\begin{lsuc}{PHP-3.0-C2}
  \linkosuc{02S}
  \linkosuc{05S} 
  \linkosuc{07S} 

  \lsucmeans{that you received PHP-3.0 licensed software which you are now going
    to distribute to third parties in the form of unmodified source code files
    or as unmodified source code package. In this case it makes no difference if
    you distribute a program, an application, a server, a snippet, a module, a
    library, or a plugin as an independent or as an embedded unit.}

  \coversOsucs{OSUC-02S, OSUC-05S, OSUC-07S}{02S}{07S}

  \begin{lsucrequires}
    \lsucmandatory{\keepPHPLicense}
    \lsucmandatory{\acknowledgePHPInDocumentation}
    \lsucoptional{\addLicenseToDocumentation}
  \end{lsucrequires}

  \begin{lsucprohibits}
    \lsucitem{\toUseTheNamePHP}
  \end{lsucprohibits} 
\end{lsuc}

% ------------------------------------------------------------------------------
\subsection{PHP-3.0-C3: Passing the unmodified software as binary}
\begin{lsuc}{PHP-3.0-C3}
  \linkosuc{02B} 
  \linkosuc{05B} 
  \linkosuc{07B} 

  \lsucmeans{that you received PHP-3.0 licensed software which you are now going
    to distribute to third parties in the form of unmodified binary files or as
    unmodified binary package. In this case it does not matter if you distribute
    a program, an application, a server, a snippet, a module, a library, or a
    plugin as an independent or an embedded unit.}

  \coversOsucs{OSUC-02B, OSUC-05B, OSUC-07B}{02B}{07B}

  \begin{lsucrequires}
    \lsucmandatory{\reproducePHPLicense}%
    \footnote{Because you are distributing an unmodified binary, you could
      assume that the copright screens of the application do already what they
      have to do.}%
    \passingFilesCorrectly
    \lsucmandatory{\acknowledgePHPInDocumentation}
    \lsucoptional{\addLicenseToDocumentation}
  \end{lsucrequires}

  \begin{lsucprohibits}
    \lsucitem{\toUseTheNamePHP}
  \end{lsucprohibits}
\end{lsuc}

% ------------------------------------------------------------------------------
\subsection{PHP-3.0-C4: Passing a modified program as source code}
\begin{lsuc}{PHP-3.0-C4}
  \linkosuc{04S}

  \lsucmeans{that you received a PHP-3.0 licensed program, application, or
    server (proapse), that you modified it, and that you are now going to
    distribute this modified version to third parties in the form of source code
    files or as a source code package.}

  \mapsToOsuc{04S}

  \begin{lsucrequires}
    \lsucmandatory{\keepPHPLicense}
    \lsucmandatory{\acknowledgePHPInDocumentation}
    \lsucoptional{\addLicenseToDocumentation}
    \lsucoptional{\createCopyrightDialog}% 
    \footnote{Following distributors of compiled versions will appreciate your
      prepatory work.} 
    \lsucoptional{\markModifications}
  \end{lsucrequires}

  \begin{lsucprohibits}
    \lsucitem{\toUseTheNamePHP}
  \end{lsucprohibits}
\end{lsuc}

% ------------------------------------------------------------------------------
\subsection{PHP-3.0-C5: Passing a modified program as binary}
\begin{lsuc}{PHP-3.0-C5}
  \linkosuc{04B}

  \lsucmeans{that you received a PHP-3.0 licensed program, application, or
    server (proapse), that you modified it, and that you are now going to
    distribute this modified version to third parties in the form of binary
    files or as a binary package.}

  \mapsToOsuc{04B}

  \begin{lsucrequires}
    \lsucmandatory{\acknowledgePHPInDocumentation}
    \lsucmandatory{\addLicenseToDocumentation}
    \lsucoptional{\reproducePHPLicense}
    \lsucoptional{\markUndistributedModifications}
  \end{lsucrequires}

  \begin{lsucprohibits}
    \lsucitem{\toUseTheNamePHP}
  \end{lsucprohibits}
\end{lsuc}

% ------------------------------------------------------------------------------
\subsection{PHP-3.0-C6: Passing a modified library as independent source code}
\begin{lsuc}{PHP-3.0-C6}
  \linkosuc{08S}

  \lsucmeans{that you received a PHP-3.0 licensed code snippet, module, library,
    or plugin (snimoli), that you modified it, and that you are now going to
    distribute this modified version to third parties in the form of source code
    files or as a source code package, but without embedding it into another
    larger software unit.}

  \mapsToOsuc{08S}

  \begin{lsucrequires}
    \lsucmandatory{\keepPHPLicense}
    \lsucmandatory{\acknowledgePHPInDocumentation}
    \lsucoptional{\addLicenseToDocumentation}
    \lsucoptional{\markModifications}
  \end{lsucrequires}

  \begin{lsucprohibits}
    \lsucitem{\toUseTheNamePHP}
  \end{lsucprohibits}
\end{lsuc}

% ------------------------------------------------------------------------------
\subsection{PHP-3.0-C7: Passing a modified library as independent binary}
\begin{lsuc}{PHP-3.0-C7}
  \linkosuc{08B}

  \lsucmeans{that you received a PHP-3.0 licensed code snippet, module, library,
    or plugin (snimoli), that you modified it, and that you are now going to
    distribute this modified version to third parties in the form of binary files
    or as a binary package but without embedding it into another larger software
    unit.}

  \mapsToOsuc{08B}

  \begin{lsucrequires}
    \lsucmandatory{\acknowledgePHPInDocumentation}
    \lsucmandatory{\addLicenseToDocumentation}
    \lsucoptional{\reproducePHPLicense}
    \lsucoptional{\markUndistributedModifications}
  \end{lsucrequires}

  \begin{lsucprohibits}
    \lsucitem{\toUseTheNamePHP}
  \end{lsucprohibits}
\end{lsuc}

% ------------------------------------------------------------------------------
\subsection{PHP-3.0-C8: Passing a modified library as embedded source code}
\begin{lsuc}{PHP-3.0-C8}
  \linkosuc{10S}

  \lsucmeans{that you received a PHP-3.0 licensed code snippet, module, library,
    or plugin (snimoli), that you modified it, and that you are now going to
    distribute this modified version to third parties in the form of source code
    files or as a source code package together with another larger software unit
    which contains this code snippet, module, library, or plugin as an embedded
    component.}

  \mapsToOsuc{10S}

  \begin{lsucrequires}
    \lsucmandatory{\keepPHPLicense}
    \lsucmandatory{\acknowledgePHPInDocumentation}
    \lsucmandatory{\addLicenseToDocumentation}
    \lsucoptional{\listEmbeddedLibraryInCopyrightDialog}%
    \footnote{Following distributors of compiled versions will appreciate your
      prepatory work.} 
    \lsucoptional{\markModifications}
    \lsucoptional{\keepSourceSeparate}
  \end{lsucrequires}

  \begin{lsucprohibits}
    \lsucitem{\toUseTheNamePHP}
  \end{lsucprohibits}
\end{lsuc}

% ------------------------------------------------------------------------------
\subsection{PHP-3.0-C9: Passing a modified library as embedded binary}
\begin{lsuc}{PHP-3.0-C9}
  \linkosuc{10B}

  \lsucmeans{that you received a PHP-3.0 licensed code snippet, module, library,
    or plugin (snimoli), that you modified it, and that you are now going to
    distribute this modified version to third parties in the form of binary
    files or as a binary package together with another larger software unit
    which contains this code snippet, module, library, or plugin as an embedded
    component.}

  \mapsToOsuc{10B}

  \begin{lsucrequires}
    \lsucmandatory{\acknowledgePHPInDocumentation}
    \lsucmandatory{\addLicenseToDocumentation}
    \lsucoptional{\reproducePHPLicense}
    \lsucoptional{\markUndistributedModifications}
    \lsucoptional{\keepBinarySeparate}
  \end{lsucrequires}

  \begin{lsucprohibits}
    \lsucitem{\toUseTheNamePHP}
  \end{lsucprohibits}
\end{lsuc}

% ------------------------------------------------------------------------------
\subsection{Discussions and Explanations}
\label{PHPDiscussion}

First of all, it might surprise some readers that the \oslic{} also describes
the open source use cases which concern the distribution of binary files
although it deals with the PHP-3.0 license. PHP is a script language. Thus,
delivering the source code seems to be a must. But one has to consider that the
PHP-3.0 license could also be applied to works which are based on other
languages constituted on the compiler paradigm. Or there might a PHP compiler be
used. 

It might also surprise some readers that in case of the binary distribution of
modifications the condition to repoduce the php license in the documentation is
a \emph{must,} while its reproduction in a copyright screen of the program is a
\emph{should.} This is directly caused by the binary-condition of the php license
which expressly requires that \enquote{Redistributions in binary form must
reproduce the above copyright notice, this list of conditions and the following
disclaimer in the documentation and/or other materials provided with the
distribution.}\citePHP{} But of course, implementing the \emph{must} and the
\emph{should} is best. 

% ------------------------------------------------------------------------------
\end{license}

%\bibliography{../../../bibfiles/oscResourcesEn}

% Local Variables:
% mode: latex
% fill-column: 80
% End:
}


%%%%%%%%%%%%%%%
% % Telekom osCompendium 'for being included' snippet template
%
% (c) Karsten Reincke, Deutsche Telekom AG, Darmstadt 2011
%
% This LaTeX-File is licensed under the Creative Commons Attribution-ShareAlike
% 3.0 Germany License (http://creativecommons.org/licenses/by-sa/3.0/de/): Feel
% free 'to share (to copy, distribute and transmit)' or 'to remix (to adapt)'
% it, if you '... distribute the resulting work under the same or similar
% license to this one' and if you respect how 'you must attribute the work in
% the manner specified by the author ...':
%
% In an internet based reuse please link the reused parts to www.telekom.com and
% mention the original authors and Deutsche Telekom AG in a suitable manner. In
% a paper-like reuse please insert a short hint to www.telekom.com and to the
% original authors and Deutsche Telekom AG into your preface. For normal
% quotations please use the scientific standard to cite.
%
% [ File structure derived from 'mind your Scholar Research Framework' 
%   mycsrf (c) K. Reincke CC BY 3.0  http://mycsrf.fodina.de/ ]
%

% Chapter Abstract
% ----------------
\chapter{Open Source Licenses and Their Legal Environments [tbd]}

\footnotesize
\begin{quote}\itshape
In this chapter we analyze why to know a license alone is not enough. At the end
you will know that open source licenses are embedded into the legal environment
of a state. And you will know in which sense the German legal environment
predetermines your readings of open source licenses.
\end{quote}
\normalsize{}


% Local Variables:
% mode: latex
% fill-column: 80
% End:


%%%%%%%%%%%%%%%
% Telekom osCompendium 'for being included' snippet template
%
% (c) Karsten Reincke, Deutsche Telekom AG, Darmstadt 2011
%
% This LaTeX-File is licensed under the Creative Commons Attribution-ShareAlike
% 3.0 Germany License (http://creativecommons.org/licenses/by-sa/3.0/de/): Feel
% free 'to share (to copy, distribute and transmit)' or 'to remix (to adapt)'
% it, if you '... distribute the resulting work under the same or similar
% license to this one' and if you respect how 'you must attribute the work in
% the manner specified by the author ...':
%
% In an internet based reuse please link the reused parts to www.telekom.com and
% mention the original authors and Deutsche Telekom AG in a suitable manner. In
% a paper-like reuse please insert a short hint to www.telekom.com and to the
% original authors and Deutsche Telekom AG into your preface. For normal
% quotations please use the scientific standard to cite.
%
% [ File structure derived from 'mind your Scholar Research Framework' 
%   mycsrf (c) K. Reincke CC BY 3.0  http://mycsrf.fodina.de/ ]
%

% Chapter Abstract
% ----------------
\chapter{Conclusion}

During the last 4 years, we have developed this \textbf{O}pen \textbf{S}ource
\textbf{Li}cense \textbf{C}ompendium. We had the honor and the pleasure to
discuss our ideas with many open source experts, for example with those, who
visit the European Legal and Licensing Workshop, organized by the FSFE. We were
invited to present our work on different conferences, in Germany, in Europe, and
even in Asia. We got a very encouring feedback. Today we know what we only
supposed when we started: We could indeed close an important gap by offering a
simple and reliable way to ascertain what one has to do for using open source
software compliantly. We are proud of having gone this long way. And we pride
ourselves on the fact that -- today -- the OSLiC is officially listed by the OSI
as one of those tools by which one can manage the open source
compliance\footnote{$\rightarrow$
http://osi.xwiki.com/bin/Projects/Process+and+Compliance+Resources}.

But, we also got adjusting feedback: Namely our initial premise was justifiably
not really accepted by the community. We were told that the software developers
themselves would never use our OSLiC. They would never read a book of more than
300 pages full of lists and tables -- as long as this book was not a
specification of a computer language. The OSLiC would be too large and too
complex for simplifying the daily life of the open source users. It would be an
excellent foundation for becoming an open source license expert -- but not a
tool for the desk. And indeed, it was simply silly to assume that software
developers, project managers, or IT managers can directly understand and use the
OSLiC: reading the OSLiC the first time has a discouraging shock effect. Today,
also we know this.

Nevertheless, it was very important for us to fall for the charme of this
illusion. Without this error, we never would have started the development of the
OSLiC. And thus, we never would have find the idea to organize the issue in form
of finders and a 5 question form. Without this error, we today would never have
a work which justifies and proves each single assertion by quoting the licenses
and the experts. And without this frightening feedback we received, we never
would have got one of our best and encouraging experiences: 

When we had accepted the feedback, we directly decided to develop an online
version of the OSLiC, the Open Source Compliance Advisor, also know as
OSCAd\footnote{$\rightarrow$ http://opensource.telekom.net/oscad/}.
We distributed it under the terms of the AGPL. Then, the company Amadeus decided
to take over the development of this online tool. We, on our side, inserted an
export interface into the OSLiC. They, on their side, rewrote the OSCAd and
integrated an import interface. So -- finally -- we both were able to focus on
only one specific aspect:  they took the responsibility for computing and
maintaining the online tool\footnote{$\rightarrow$
https://github.com/AmadeusITGroup/oscad}, we took the responsibility maintaining
for the fundamental analysis of the open source licenses\footnote{$\rightarrow$
https://github.com/dtag-dbu/oslic/}.

Thus, we concretely experienced the advantages of sharing ideas and sources,
which were so often emphazied: Playing the open source game actively means
giving a bit and getting back a lot. Playing the open source game actively means
saving the own resources.
 
Therefore, you may also take the fact that we finally could indeed publish the
version 1.0 of the OSLiC as a thankful profound curtsey to the open source
community!


%%%%%%%%%%%%%%%
\chapter{Appendices}

% Telekom osCompendium 'for being included' snippet template
%
% (c) Karsten Reincke, Deutsche Telekom AG, Darmstadt 2011
%
% This LaTeX-File is licensed under the Creative Commons Attribution-ShareAlike
% 3.0 Germany License (http://creativecommons.org/licenses/by-sa/3.0/de/): Feel
% free 'to share (to copy, distribute and transmit)' or 'to remix (to adapt)'
% it, if you '... distribute the resulting work under the same or similar
% license to this one' and if you respect how 'you must attribute the work in
% the manner specified by the author ...':
%
% In an internet based reuse please link the reused parts to www.telekom.com and
% mention the original authors and Deutsche Telekom AG in a suitable manner. In
% a paper-like reuse please insert a short hint to www.telekom.com and to the
% original authors and Deutsche Telekom AG into your preface. For normal
% quotations please use the scientific standard to cite.
%
% [ Framework derived from 'mind your Scholar Research Framework' 
%   mycsrf (c) K. Reincke 2012 CC BY 3.0  http://mycsrf.fodina.de/ ]
%

\section{Some Additional Remarks on the OSLiC Quotation Style}\label{sec:QuotationAppendix}

We have already characterized the general tone of our
footnotes\footnote{$\rightarrow$ p.\ \pageref{QuotationPrinciple} }. Let us now
briefly explain a little peculiarity of our bibliography:

Modern times have also changed the humanities. Formerly a book or an article
must be printed for being ripe to be quoted. Our statements relied on static,
readily prepared works. Nowadays even university libraries sometimes offer those
books and articles as PDF files which are printed in the original. As a scholar,
now you must rely on the equality of the printed version and the PDF file -- at
least with respect to the page numbers and the appearance. You can not verify the
equivalence -- at least to a certain degree.

Moreover: in case of such 'e-books' and 'e-articles' the libraries often do not
offer the pdf files themselves but links to the download pages of the publisher.
Formerly as a scholar you could trust that your readers would be able to
retrieve the quoted work if they want to verify your citations. It's one task of
our libraries to hold available our scientific sources. But now they do not buy
any longer the books, but the right to download files over the university net.
In this case these PDF files are not stored on the serves of the university
library. By using the link provided by the publisher each student or each reader
downloads his own file -- case by case. Therefore -- as a scholar -- you now have
to trust that the publisher, who provides the link, will not change that pdf
file that you have cited.

But it gets even worse: While it might be that publishers modify their work
secretly (even it is not very likely that they do it), it's a definite feature
of the web that its pages are fre\-quen\-tly changed. Hence we must ask
ourselves: Can we seriously argue on the basis of statements and documents which
might disappear? Can we quote such possibly volatile sources? The problem is: we
must do it, especially if we write about an internet topic -- and even if we want
to write a really reliable compendium.

So, what can we do? First, we must confide in our readers, that they either
will retrieve our sources or -- if they can not find them -- that they
believe that we really have found and read what we have written and
quoted. Second, we store all these e-wares\footnote{Take this little word as
(new) generalization of 'e-book', 'e-article', 'e-paper' and so on.} we
read\footnote{But because of the copyright we ourselves are naturally not
allowed to offer a download link for them or to send a copy of it to those who
want to verify our quotes.}. And thirdly we should lay open to our readers the
different levels of reliableness of our sources. Therefore we use
the following markers in our bibliographic data\footnote{And another hint: Nowadays sometimes
even scientific libraries don't offer exact 'e-copies' of the original. In
some cases one can only get html-versions of articles which formerly were
printed as part of journals. In these case the scholar has to use sources which
lost their original page-numbers. The same can happen to articles of proceedings
etc.\ which are now only offered as autonomous pdf files with an internal paging.
If we quote such kind of articles we try to specify the number of the quoted
article in the original row of articles, added -- if possible -- by an internal
page number. But naturally we also try to follow the bibliographic data
delivered by that organization which distributes these kind of copies.}:

\begin{itemize}
  \item Print / Copy:- The source is printed and we saw either the printed work
  really or we get an official copy by our library. Hence you should also be able
  to get the work in a library, at least in those we used (UB Frankfurt or ULB
  Darmstadt).
  \item BibWeb/[PDF/\ldots] :- The source might be printed, but we read only the
  electronic version (PDF or other type of format), offered by and over the
  net of our university libraries (UB Frankfurt or ULB Darmstadt).
  \item FreeWeb/[PDF/\ldots] :- We read the electronic version offered by the
  free web. In this case we add the url\footnote{Please note: Long urls often
  destroy the pleasing appearance of a text because it's difficult to wrap the
  lines acceptably. Hence we wished to make it easier for LaTeX to do this job.
  Therefor we sometimes split the urls and inserted blanks. So you have to erase
  all blanks if you want to verify our urls.} and the date when we downloaded /
  saw the text.
\end{itemize}


%\bibliography{../../../bibfiles/oscResourcesEn}

% Local Variables:
% mode: latex
% fill-column: 80
% End:


% Telekom osCompendium 'for being included' snippet template
%
% (c) Karsten Reincke, Deutsche Telekom AG, Darmstadt 2011
%
% This LaTeX-File is licensed under the Creative Commons Attribution-ShareAlike
% 3.0 Germany License (http://creativecommons.org/licenses/by-sa/3.0/de/): Feel
% free 'to share (to copy, distribute and transmit)' or 'to remix (to adapt)'
% it, if you '... distribute the resulting work under the same or similar
% license to this one' and if you respect how 'you must attribute the work in
% the manner specified by the author ...':
%
% In an internet based reuse please link the reused parts to www.telekom.com and
% mention the original authors and Deutsche Telekom AG in a suitable manner. In
% a paper-like reuse please insert a short hint to www.telekom.com and to the
% original authors and Deutsche Telekom AG into your preface. For normal
% quotations please use the scientific standard to cite.
%
% [ File structure derived from 'mind your Scholar Research Framework' 
%   mycsrf (c) K. Reincke CC BY 3.0  http://mycsrf.fodina.de/ ]

%


%% use all entries of the bibliography
%\nocite{*}


\section{Some Widespread Open Source Myths}

From the viewpoint of an internet student we have to consider that the web
offers a mass of rumors concerning the nature of open source software
(Licenses). Here are some of the myths\footcite[At least one time even a
scientific legally discussing book is talking about the \enquote{myth around open
source licenses} -- although only as part of  the title: cf][1ff,
especially 209ff]{GuiOvd2006a} we met:
 
\textbf{BE CAREFUL: THIS SECTION MUST THOROUGHLY BE REVIEWED AND REWRITTEN. 
IT'S ONLY AN OUTLINE!!! Do not quote part of it. It must be verified.}

\begin{description}
  \item[open source tries to improve the world ethically] :- No, there's a clear
  ban to exclude persons, groups, purposes. Thus, there is no chance to exclude
  anyone from using open source software because he is an ethical or moralic
  malefactor.
  \item[Changed open source software must be re-published] :- No, in a double
  sense! There are OS licenses which allow the proprietarization of the
  modified code. And even the LGPL and the GPL, which clearly try to prevent
  the proprietarization, do not require generally that a modified code must be
  (re-)published. Only if you give your modfied (L)GPL licensed application as
  binary to anybody, then you have to handover the modified code, too.
  \item[Modified open source software must be given back to the whole community]
  :- No. Again, there are OS licenses which allow the proprietarization of the
  modified code. And even the LGPL and the GPL -- which clearly require, that you
  also publish the modified code, if you give the modified binary to anybody --
  do not require that you distribute your modification around the world. LGPL and
  GPL clearly say that you have to hand over the code to those persons you
  give the binary to. And if you only give your improvement only one person or a
  group of persons, then you must handover your code only to that persons or
  only to all members of that group.
  \item[Published open source software is open for ever] :- No, if this myth
  says that also all future versions will have to be distributed under an open
  source license. The copyright holder ever holds the copyright. They can change
  the licence of next release of its software -- but only for the following
  release, not for the current or for former versions. Those releases, which
  already have been distributed under an open source license, indeed remain
  open.
  \item[Software can either be open source software or proprietary software] :-
  No. The copyright holders themselves can additionally distribute the code
  under other conditions when ever they want to do it. That's not a question of
  the licence, but of the copyright.  
  \item[The opposite of open source software is commercial Software] :- No.
  First, you are also allowed to use the open source software in any commercial
  purpose. There's only one point which is excluded in OSS: you are not allowed
  to ask for a licence fee if you distribute 'open source software'. Second,
  there are many other forms like freeware, public domain software or anything
  else which is neither open source software nor Commercial Software. It's
  pointless to take the question of money as a criterion for distinguish open
  source software and its opposite. Moreover: Proprietary Software as opposite
  of open source software should be defined ex negativo: all kind of software,
  which does not fit the OSD is proprietary.
  \item[open source software prohibits to earn money] :- No, you are allowed to
  invent each business model you want. There's only one exception: you are not
  allowed to ask for a licence fee if you distribute open source software. This
  limitation is based on the open source definition which clearly states that a
  license -- which wants to become an open source license -- \enquote{shall not
  restrict any party from selling or giving away the software as a component of
  an aggregate software distribution containing programs from several different
  sources} and that the license under this circumstances \enquote{[\ldots] shall
  not require a royalty or other fee for such sale}\footcite[cf.][§1]{OSI2012a}.
  If you combine this constraint with the requirements that an open source
  license \enquote{[\ldots] must not restrict anyone from making use of the
  program [\ldots]}\footcite[cf.][§6]{OSI2012a} and that it \enquote{[\ldots]
  must allow distribution in source code as well as compiled form
  [\ldots]}\footcite[cf.][§2]{OSI2012a}, you can generally conclude that none of
  the open source licenses may require a fee for using and/or distributing the
  program. But being paid for the service to install the program, to collect
  and compile a customer specific version, and/or to monitor the environment is
  of course not excluded by this condition.
  
  Historically this mistake might be evoked by Debian: The GNU project missed
  its kernel while the Linux kernel was already distributed as part of
  collections which also include GNU software. Then, in 1983? Ian Murdock was
  supported by RMS and its FSF to build a really free distribution (Debian)
  containg GNU software and the Linux kernel. But Ian Murdock states also, that
  Debian does not want to earn money.
% TODO find sources for indirect citations
% TODO: check, whether OSD requires license fee free distribution
  \item[Modifications of open source software must be marked] :- No. This is not
  a defining postulation of the OSD. The OSD allows licenses to require the mark
  of modifications. But it does not require from all licenses to require the mark
  modifications for being an open source license.
  \item[Modifications of open source software must be marked by your personal
  data] :- No, it is only required to mark modifications so that a reader could
  distinguish the modifications from the original code. It's required for saving
  the integrity of the original author. And therefore it is not required as a
  constitutive criterion by the OSD. It might be that a license additionally
  requires your name. But that is not feature of open source software in general.
  And at least the licenses discussed by us do not require to insert your name.
% TODO: check whether any of our licenses reuire that you mark modifications by
% your personal data / real name  
  \item[The open source Definition determines the conditions to use open source
  software] :- No. The \emph{Open Source Definition} determines which licenses
  are open source licenses, nothing more. The OSD is a set of necessary
  conditions to be an open source license. It determines the freedom and the
  responsibilities of a user as a set of more or less abstract rules. But it
  does not constitute a set of sufficient tasks which a user has to perform for
  fulfilling any open source license. Open source licenses may differ by
  instantiating the OSD criteria. So, if you want to know what you have to do to
  fulfill a license, you have to go back to the real license of that software
  you are using.
\end{description}

%\bibliography{../bibfiles/oscResourcesEn}

% Local Variables:
% mode: latex
% fill-column: 80
% End:


% Telekom osCompendium 'for being included' snippet template
%
% (c) Karsten Reincke, Deutsche Telekom AG, Darmstadt 2011
%
% This LaTeX-File is licensed under the Creative Commons Attribution-ShareAlike
% 3.0 Germany License (http://creativecommons.org/licenses/by-sa/3.0/de/): Feel
% free 'to share (to copy, distribute and transmit)' or 'to remix (to adapt)'
% it, if you '... distribute the resulting work under the same or similar
% license to this one' and if you respect how 'you must attribute the work in
% the manner specified by the author ...':
%
% In an internet based reuse please link the reused parts to www.telekom.com and
% mention the original authors and Deutsche Telekom AG in a suitable manner. In
% a paper-like reuse please insert a short hint to www.telekom.com and to the
% original authors and Deutsche Telekom AG into your preface. For normal
% quotations please use the scientific standard to cite.
%
% [ File structure derived from 'mind your Scholar Research Framework' 
%   mycsrf (c) K. Reincke CC BY 3.0  http://mycsrf.fodina.de/ ]
%

% Chapter Abstract
% ----------------

\footnotesize \begin{quote}\itshape This section outlines reflections by which
we initially focused ourselves on the question why we need an OSLiC and how its
content and form should be derivated from these needs.
\end{quote}
\normalsize{}

\subsection{Why}

Do we need another book about open source? Do \emph{you} need another book about
open source software? Let us address this question from the viewpoint of what we
already know, what we instinctively believe and what we may have heard. For
example you may presume one or more of the following statements are correct. Or
you may even have experienced similar perceptions from your peers or managers.
Or you have been told they describe 'open source':

\begin{itemize}
  \item The \emph{Open Source Definition} offers rules to use open source software.
  \item Modified open source software must be published.
  \item Modified open source software must be given back to the community.
  \item All generations of open source software will remain open for ever.
  \item Software can either be open source software or proprietary software.
  \item The opposite of open source software is commercial software.
  \item open source software prohibits to earn money.
  \item Modifications of open source software must be marked explicitly.
  \item Modifiers of open source software must identify themselves.
  \item When distributing an open source binary it’s enough point to a download
  page to obtain the source code.
  \item The aim of open source software is to improve the world ethically.
  \item open source software is viral and infectious.
\end{itemize}

Do these conceptions sound familiar to you? Unfortunately, whatever we might
believe or wish for, these concepts are incorrect. Naturally we will discuss
this issue later on. For the moment let us assume they are indeed
incorrect\footnote{For those who want directly verify our argumentation, we have
generated a condensed summary of the arguments and citations. You can find this
summary in our appendices.}.

So, again: Do \emph{we} need another book about open source software? \emph{We},
that is -- in this case and at least initially -- the large German company
\textit{Deutsche Telekom AG}. Arguing from the perspective of a large company
requires not only identifying the common misconceptions, but catering for the
unique needs of a large Enterprise. And indeed the very size of the company
brings its own problems.

Large companies use more open source software in more varied contexts than small
companies. There is an important question that every company should ask:
\emph{'Are we sure that we respect all those requirements of open source
software we have to respect?'}. But large companies cannot answer this question
as easily as small companies: the large number of diverse open source
deployments in different contexts mean that case by case governance, a model
that may work in small concerns, is far from appropriate for our needs. This
leads to wasting both time and money. Further, the chances of success are small:
training at least one employee in each software team as an open source software
License expert is unrealistic in terms of cost-efficiency and reliability.

Nevertheless even large companies want to and try to fulfill the rules of open
source software thoroughly -- especially \emph{Deutsche Telekom AG}. When this
company realized that the question \textit{Are we sure that we respect all those
rules of open source software correctly which we have to respect} could be
problematic, it directly asked some of its employees known as open source
enthusiasts to establish a service and a process for answering this question.

So, it is no surprise that we, the initial authors of this \textit{Open Source
License Compendium}, were asked by our employer \emph{Deutsche Telekom AG}.
Naturally we were proud to work on an open source topic officially. But while we
were doing our job we had to ask ourselves if \emph{we} perhaps needed another
book on open source. Our answer was \textit{Yes, we do!} Let us shortly explain,
why:

First, we already knew that there exists supporting software. These
meta-pro\-grams take the code of any other application and try to list those
open source components being 'covered' by that application\footnote{As general
examples let us mention Palamida (\texttt{http://www.palamida.com/}) and
BlackDuck (\texttt{http://www.blackducksoftware.com/}).}. But we had also
already realised that this supporting software did not always match the way we
thought the problem should be solved. Second, we recognized fairly quickly that
we need a reliable guide. We personally were asked to give the \emph{ok} for
projects of our company. We could not answer such requests on the base of
\textit{'Oh yes, I read this in the \emph{Heise-Ticker} a few days ago'} -- even
if the \emph{Heise-Ticker} had described the situation completely correctly. We
ourselves had to be more reliable than this\footnote{But of course, we have to
do ourselves the honor of conceding that we -- like many many other German open
source enthusiasts -- love using the \emph{Heise-Ticker} as main IT information
source. Unfortunately, its reputation is stil not high enough that its news can
directly be cited.}. Naturally we already knew a great deal about open source
software. Even so, our knowledge was not as systematic as necessary. We looked
for an open source compendium which adequately described what a project or
product development team had to do to fulfill the criteria of its open source
licenses. We wanted to use that compendium to the basis of our recommendations.

We were very thorough but we did not find what we were looking for. Our 'little'
bibliography attest our seriousness. What we found was a lot of information
releated to individual issues spread over many sources. We did not find answers
to our question even in the specific literature. Let us describe three little
steps to increase the understanding of the issue:

Without open source licenses there is no open source movement. Nevertheless in
dealing with open source licenses, this is sometimes neglected. Take the
\emph{Apache Web Server} as an example: No doubt, it is one of the most important
pieces of open source software\footnote{To prove that the \textit{Apache} is
really a piece of open source software one must execute a set of steps: First,
you have to note, that \emph{Apache} is something like a meta project, covered
by the \emph{Apache Software Foundation}, also known as \emph{ASF} (cf.
\texttt{http://www.apache.org/}, wp). Thus, you can not directly jump into
the \emph{Apache License}. First of all you have to visit the project site (cf.
\texttt{http://httpd.apache.org/}, wp) even if at the end its license link
leads you back to the general \emph{Apache License sub site} (cf.
\texttt{http://www.apache.org/licenses/}, wp) which announces, that \enquote{all
software produced by The Apache Software Foundation or any of its projects or
subjects is licensed according to the terms of the documents listed
below}. Only now you can use the offered link for switching to the
\emph{Apache License}, Version 2.0, if you want to check your rights and duties.
But that is difficult. There does not exist any simple list what you have to do
for fulfilling the license. Even the faq (cf.
\texttt{http://httpd.apache.org/docs/2.2/faq/}, wp) -- meanwhile being moved to
a wiki -- only says that the server \enquote{[\ldots] comes with an unrestrictive
license} and that you are allowed to put the code on a CD (cf.
\texttt{http://wiki.apache.org/httpd/FAQ}, wp). Hence, from the viewpoint of
the ASF the license itself shall answer all questions. [Reference download for
all urls: 2011-08-31] } with a specific license\footcite[cf.][\nopage
wp]{AsfApacheLicense20a}. Moreover: the success of the open source movement
in the commercial world depends directly on the decision of IBM to replace its
corresponding own component in the \textit{IBM WebSphere Application Server}
with the free \textit{Apache Web Server}\footcite[cf.][287ff]{Moody2001a}.
Meanwhile many companies use the \textit{Apache Web Server} to act as a web
provider. Currently the \emph{Apache http server} -- as it has to be named
correctly -- is used more than twice as much as all the other http server
software together\footcite[cf.][\nopage wp]{Netcraft2011a}. Hence many business
models depend on the Apache License. Another aspect is that even the famous
\emph{Apache Cookbook}, which explains the installation, the configuration, and
the maintainance of an Apache Web Server in details\footcite[cf.][\nopage et
passim]{CoaBow2004a}, does not mention anything about the license which allows
for installation, configuration and maintenance. Neither the index lists the
word 'license'\footcite[cf.][245ff, esp.\ p.\ 250]{CoaBow2004a}, nor the chapters
'Installation'\footcite[cf.][1ff]{CoaBow2004a} or the chapter
'Miscellaneous'\footcite[cf.][219ff]{CoaBow2004a} mentions the license question
in a serious way. There's only one short hint as to the advantage of open source
software, i.e.\ that everybody is allowed to install it\footcite[cf.][1: \enquote{%
\ldots einer der Vorzüge von open source software besteht darin, dass
je\-der\-mann die Erlaubnis zur Erzeugung eines eigenen Installationskits hat
}]{CoaBow2004a}. Can you be sure that you are allowed to do what you are
doing on the base of such a phrase?

Naturally, the \emph{Apache Cookbook} is not a book for lawyers, it is a book for
administrators and developers. They do not want to get bogged down by
legalities, they want to set up an Apache Web Server as fast as possible and get
down to work. Indeed, the Apache Cookbook offers a good support. But not only as
a company you have to ask yourself whether you are really allowed to do what you
are doing. Can you find the answer in the \emph{Apache Cookbook}? No. Can you
find it in the license itself? Yes, but it is difficult\footnote{And do we
really want our developers and maintainers to read the original licenses? Do we
really want them to discover that they also have to check the licenses of the
used modules?}. So again: Can you find your answer in another book, which is
\emph{Amazon's} current top recommendation for the search term \emph{'apache
server'}\footnote{Tested on \texttt{http://www.amazon.de/} at 2011-08-31.}? Not
really: Sascha Kersken's Apache 2.2 Handbook offers a license chapter, but it is
only two pages long\footcite[cf.][111f]{Kersken2009a}. Moreover, the rights and
duties are condensed into just 5 bullet points which taken together do not
explain when the software and the license have to be handed over to a customer
and when you are allowed to hide your
improvements\footcite[cf.][112]{Kersken2009a}.

This brings us to the question of what prevents us from using something like a
\emph{'general license cookbook'} which explains all the necessary details and
which offers  quick access to the relevant points:

Of course we also browsed the internet. At least for German speaking people
there is an excellent site concerning the topic \emph{open source licenses}.
offered by \textit{iffross}, which, loosely translated, means an
\textit{Institute for Legal Aspects of the Free and open source
software}\footnote{originally: \enquote{Institut für Rechtsfragen der Freien und
open source software}. Main entry point for its site is the URL
\texttt{http://www.ifross.org/}.}, founded in 2000 as a private institute to
track the phenomenon 'free software' from the viewpoint of (German)
lawyers\footcite[cf.][\nopage wp]{ifross2011b}. Besides many other
aspects this site offers a very well and thoroughly elaborated
FAQ\footcite[cf.][\nopage wp]{ifross2011c} and a large list of open
source licenses and other related licenses: moreover, evidently it is
classifying the open source licenses in those 'without copyleft-effect' (BSD),
in those with 'strict copyleft-effect' (GPL) and in those with 'restricted
copyleft-effect' (LGPL)\footcite[cf.][\nopage wp]{ifross2011a}.

However, even this excellent site does not fulfill our needs. It does not offer
those context specific to-do lists which companies, developers or project
managers can use to ensure their open source software is used in a regular
manner.

We therefore evaluated that standard book which is listed in the most legal
bibliographies\footnote{at least in that German judicial literature dealing with
open source}: the book of Jaeger and Metzger which concerns -- loosely translated
-- \textit{the judicial framework requirement for open source
software}\footcite[cf.][V -- It can not be any surprise that both authors,
Mr. Jaeger and Mr. Metzger are members of ifross (cf.
\texttt{http://www.ifross.org/personen/}, wp)]{JaeMet2002a}. Even the most
earliest edition of this book already had a clear structure in its chapter
'copyright': For each license mentioned (or at least for each license cluster)
it offered a subchapter for the rights and a subchapter for the
duties\footcite[cf.][30ff]{JaeMet2002a} of the software user\footcite[For
getting a good survey of the structure and the line of thought see the contents
cf.][VIIIf]{JaeMet2002a}. Many other important aspects of the topic
\textit{open source} are discussed, too\footcite[pars pro toto: have a
look at the chapter concerning the liability: cf.][137ff]{JaeMet2002a}.

But we needed more than this. Despite the quality of the book we were certain
that we could not hand over this book to our programmers with the recommendation
\textit{check your touched licenses and follow the instructions of the relevant
subchapters\ldots}. This book did not contain simply checkable to-do lists,
neither in the first edition\footcite[cf.][VIff]{JaeMet2002a} and in the second
edition\footcite[cf.][VIIff]{JaeMet2006a} nor in the recently published third
edition\footcite[cf.][VIIIff. Naturally we use this latest edition for adopting
or discussing systematical aspects]{JaeMet2011a}. So, how can a company or a
developer or a project manager be sure of fulfilling the requirements of the
open source licenses sufficiently if he/she does not have a verified list
telling him \textit{'do this, and in case of that, do that, and finally do also
this'}? Why should he himself implicitly become an open source licenses expert
who has to extract the necessary steps out of the literature?

While we were searching for an existing open source compendium, we found an
article with the title 'Compendium for the Publication of open source
software'\footnote{approximately translated}. It aims to be a 'pragmatic
guidebook' and an 'assistance' for 'publishing software under the conditions of
an open source license'\footcite[cf.][166f (originally: ein
\enquote{pragmatischer Ratgeber} zur \enquote{Veröffentlichung einer Software
unter den Rahmenbedingungen einer Open-Source-Lizenz}) ]{BreGlaGra2008a}.
Moreover, at the end of this article, its authors formulate ambitiously that
their 'guide' should be carried out, section by section -- for getting a legally
water tight process of publishing open source software\footcite[cf.][186
(originally: ein \enquote{Ratgeber}, der es erlaubt \enquote{ (\ldots) die zu
berücksichtigende Aspekte (strukturiert abzuarbeiten) (\ldots) } und einen
\enquote{rechtlich nicht angreifbaren Veröffentlichungsprozess} zu
ermöglichen) ]{BreGlaGra2008a}.

The authors of this article describe something close to what we were looking
for. Indeed, the article lists important aspects which have to be taken in
consideration if you want to deal open source software correctly: It announces
that no obligation exists to publish code either if you embed GPL code into your
proprietary code or if you modify the GPL code. It is only if you hand over your
binary to other persons that you have to distribute the code too, but only to
them and not to the general public\footcite[cf.][170 and 181]{BreGlaGra2008a}.
Additionally the articles explains exactly that software -- at least in Germany --
can only be acknowledged as open source software by transferring the rights to
use -- the \emph{'Nutzungsrechte'} -- to other people, while the copyright itself
-- the \emph{'Urheberpersönlichkeitsrecht'} -- is not transferable and belongs to
the author\footcite[cf.][173]{BreGlaGra2008a}. Moreover, besides other aspects
the articles briefly and deeply discusses the problem of the No-Warranty-Clauses
which are not valid in Germany and which will therefore automatically be
replaced by the liability rules for a
donation\footcite[cf.][177]{BreGlaGra2008a}. And last but not least this article
actually summarizes the idea of Copyleft and the differences between LGPL and
GPL\footcite[cf.][181]{BreGlaGra2008a}.

However some gaps remain. The article does not analyze in which cases a
University or a company perhaps \emph{must} publish its developments based
on open source software. It does not discern between different licenses
and conditions. It also does not discuss what Universities or companies,
which (re-)use and/or distribute open source software (internally), must do to
fulfill the touched open source licenses. And finally this article
does not offer the step by step list as promised.

We did, however, feel supported by this article, in two ways. First, it was a
well written summary of some main problems. Second, it stated the necessity to
have a compendium for being able to establish a legally 'water-tight' process of
publishing open source software\footcite[cf.][186]{BreGlaGra2008a}. We
seemed to be justified in our assumptions. But the open source compendium we
were looking for had to be more practical, more processable, more distinguishing
and more elaborated.

So again: Did we need a new book about open source software? We had looked for a
reliable integrated open source compendium. But we found separate pieces of
information and -- as we know today -- some rumors. Our answer was clear:
naturally we did not need a new general book about open source. But what was
lacking was a description of what responsible developers, project managers or
product developers require to fulfill open source licenses. We needed an
\textit{Open Source License Compendium}.

At the best such an \textit{Open Source License Compendium} would contain a set
of simply to process \textit{'For-Fulfilling-The-License-To-Do-Lists'}.
Additionally it should offer an intuitively user-friendly search option for
these lists. In any case, it should share developers and project managers the
effort of having to become open source license experts. For the other users, it
should also clearly explain why one has to do this and not that. Hence a
reliable \textit{Open Source License Compendium} should not only list what one
has to do, but should offer both, thoroughly verified reliable details and
clearly condensed guidance.

Although we did not find such an open source compendium we were familiar with
the spirit of the open source community. Hence we followed one of its most
simple rules: \emph{'what you miss you must develop on your own'}. Some
principles should help us to achieve our targets:

\begin{description}
  \item[To-do lists as the core, discussions around them]: Our work should be
  split into two parts. As its core we wanted to offer a
  set of to-do-Lists. Each of these lists should be relevant to one specific
  open source license and should be clustered by the open source specific use
  cases. Around this all those aspects of open source software which influence the
  interpretation of the licenses and the rules core should be precisely
  characterized. Nevertheless, the users should be able to skip
  details and go directly to the section they require.
  \item[Quotations with thoroughly specified sources]: Even if our users should
  not be obliged to read every part of the compendium they should not be
  required to believe us. We wanted to be revisable. Because our sources and our
  conclusions should be easily verifiable, we decided to use the academic
  citations and list bibliographic data extensively on the basis that our task
  should be to collect information, not to invent new 'facts'.
  \item[Not the internet alone, also books and articles]: We wanted to go back
  to the originals even if the internet was full of more or less modified
  copies. We wished to get reliable facts and descriptions. Therefore we decided
  to evaluate not only the internet but also scientific sources -- for example --
  offered by university libraries.
  \item[Not clearing out the forest land, but cutting out a swathe]: Even if we
  had to deal with licenses and their legal aspects we did not want to get lost
  in detailed discussions. It should not be our task to find out whether a
  specific kind of handling would still be legal or already forbidden.
  We did not want to fight against the licenses. We did not want to stretch
  their ambit or to test their boundary. We wished to accept open source
  licenses as they are: rules written from developers for developers. And even
  if some parts of these licenses would not be valid with respect to a legal
  system\footcite[And indeed for example for the GPL one can argue in this way:
  Even if you take the GPL as a contract of the type 'donation' respectively
  \enquote{Schenkung}, it is presented in the form of AGBs respectively
  \enquote{Allgemeine Geschäftsbedingungen} and must therefore follow the
  general AGB rules.'Regrettably' in Germany these general AGB rules do not
  allow to exclude each type of warranty. If we follow Oberhem, §11 and §12 of
  the GPL must be invalid in Germany because of these general AGB rules.
  Moreover, for Oberhem even §5 -- the important clause of the GPL by which you
  can only get the right to use and to distribute GPL software if you respect
  the rules of the GPL -- seems also to be invalid respectively
  \enquote{unwirksam}. But the good message is that the GPL as whole is not
  invalid even if it contains invalid clauses.][128, 133ff, 150ff, esp.\ 146,
  159]{Oberhem2008a}, we wanted to take them as our guideline -- at least while
  they do not violate more general laws\footnote{what they clearly do not do!}.
  We simply wanted to \emph{find one proven way} to cross the maybe slightly
  unsure forest of open source licenses. Even if indeed some clauses of the
  licenses finally were not enforceable against us we wanted to respect them
  'voluntarily'. We wanted to deliver a set of rules which support users and
  remove the possibility of becoming involved in license disputes with open
  source developers or the Free Software Foundation.
  \item[Take the text seriously]: On the other side we wanted to take our
  license texts as they were. If they lacked anything\footcite[The systematical
  underdetermination of licenses is a problem being also known in the open
  source respectively Free Software movement. Following the biography of RMS his
  main judicial counselor Moglen has stated, that \enquote{there is uncertainty
  in every legal process (\ldots) } and that it seemed to be silly to try
  \enquote{(\ldots) to take out all the bugs (\ldots)}. Nevertheless -- so
  Moglen resp.\ Williams -- the goal of Richard Stallman was \enquote{the complete
  opposite}: He tried \enquote{(\ldots) to remove uncertainty which is
  inherently impossible}. But -- and that's the nub of this analysis --
  Moglen had to follow Stallmann because of RMS character. And he had to
  summarize their work so, that \enquote{(\ldots) the resulting elegance (of the
  GPL; KR.), the resulting simplicity (of the GPL; KR.) in design almost
  achieves what it has to achieve}. Hence we are asked to take the license
  texts themselves seriously. cf.][177f]{Williams2002a}, we would interpret the
  open issues in the spirit of the open source idea. But where the text was
  clear and definite we wanted to take its propositions as a definite decision --
  even if that meaning stood against well known open source 'facts'.
  \item[Trust the swarm]: We did not want to use our own research alone as a
  basis. We knew that the swarm is ever stronger than a set of some randomly
  selected experts. Therefore we decided to publish our text as a still
  unfinished work, starting with an early release 0.2. And then we wanted to
  invite the community to complete the compendium together with us. We would
  elaborate our open source compendium as a set of LaTeX- and BibTeX files which
  could be developed and managed in GIT or any other version control system. And
  finally we would publish our text under a Creative Commons Attribution-Share
  Alike German 3.0 license, to allow other people to correct us, to help us or
  even to take our results for their own purposes.
\end{description}

And so we did. Here is the result. Feel free to use it -- according to our
licensing.

\subsection{What}

Now we can briefly explain how one should be able to use the compendium:

% TODO adopt real chapter structure into the prolegomena 
\begin{description}
  \item[The Same Idea, Different Licenses] :- Here you will find background
  information to help you interpret open source licenses in the sense of the
  \emph{Free Software movement}\footcite[At least at this place you are perhaps
  expecting that we use the logograms FLOSS, F/OSS, F/LOSS, or whatever. As you
  will read later on the word \textit{Free} is ambiguous and has strained the
  use of the concept \textit{Free Software}. Later on we will also talk about
  the invention of the concept \textit{open source} designed as a 'replacement'
  and acting as a 'splitter'. The mentioned logograms are introduced to
  re-establish or -- at least -- to underline the common history and the common
  center of 'both' movements, whereby the word \textit{Libre} shall resolve the
  ambiguity of the word \textit{Free}. For a first survey cf.] [\nopage
  wp]{wpFloss2011a}, the \emph{open source software movement}\footcite[For
  another brief and informative introduction cf.][231ff esp.\ p.\ 
  232f]{Fogel2006a}, or the GNU-Project\footnote{ We ourselves will stay with the
  concept \textit{open source} because the OSD specifies the scope of our
  analysis. But we do it with a deep obeisance to Stallmann and the FSF -- even
  if we know that this will not protect us from the thunderbolt of RMS.}. We discuss
  different ways to cluster open source licenses. Finally we present our own
  taxonomy based on the labels 'protecting the developer', 'protecting the
  licensed code' and 'protecting the on-top-developments'. If you are familiar
  with the methods of grouping different open source licenses and particular
  if you know that you can not authorize your doings on the base of descriptions
  of such license groups, then it is enough, in order to understand our line of
  thought, to briefly note our taxonomy and its wording.
  \item[The Problem of Derivated Works] :- This chapter is important. In the
  spirit of software developers we try to explain which kinds of programming
  evoke a derivated work and which not. Our to-do lists will refer to this
  analysis.
  \item[The Problem of Combining Different Licenses] :- You should
  not ignore this chapter. We will explain why and how combining software
  of different licenses is not as dangerous as it is often told. The results of
  this chapter influence the structure of our to-do lists.
  \item[open source software and Money] :- Here we will shortly
  discuss ways in which money is no problem. If you already know that it is only
  prohibited to require payment for the act of licensing a piece of open source
  software to second or third parties and if you already know that this is only
  forbidden by some licenses, and not by all, than you can postpone the reading
  of this chapter.
  \item[The Problem of Implicitly Freeing Patents] :- Here we
  will illuminate some aspects of software patents and how the are handled by
  some open source licenses. You should know what licenses implicitly do with
  your patents. But it is not our intention to write a software patent
  compendium.
  \item[Open Source Use Cases as Principle of Classification] :- This is an
  important chapter. We explain our categories 'Use as it is', 'Modify the
  Code', 'With Redistribution', 'Without Redistribution', 'Isolated Initial
  Development', 'On-Top-Development': we develop and discuss our taxonomy with
  respect to the side effects of 'combining different licenses' and 'generating
  derivated works'. This taxonomy will determine the following chapters.
  \item[open source licenses: Find Your Specific To-do Lists] :- This is a kind
  of summary which joins the relevant aspects and elaborates the 'finder
  for your to-do lists'. This is the chapter which you probably will reuse
  frequently, even if you do not want to read any of our explanations.
  \item[open source license Fulfillment: Classified To-do Lists] :- This chapter
  offers all classified to-do lists. The structure of its subchapters will
  match the structure of our finder and the structure of our taxonomy.
  \item[open source licenses and Their Legal Environments] :- Here we discuss
  why using open source software in a regular manner is not only a question of
  the licenses themselves but of the kind of the surrounding legal system.
  \item[Appendices: Some Widespread Open Source Myths] :- Here we make good on
  our promise to explain why all the propositions mentioned at the beginning of
  this chapter are wrong. You might read this chapter as a special introduction
  or a reminder epilogue whenever you want to do.
\end{description}


%\bibliography{../../../bibfiles/oscResourcesEn}

% Local Variables:
% mode: latex
% fill-column: 80
% End:



\small
%\theendnotes


\footnotesize
% Telekom osCompendium English Nomenclation Tokens Include Module 
%
% (c) Karsten Reincke, Deutsche Telekom AG, Darmstadt 2011
%
% This LaTeX-File is licensed under the Creative Commons Attribution-ShareAlike
% 3.0 Germany License (http://creativecommons.org/licenses/by-sa/3.0/de/): Feel
% free 'to share (to copy, distribute and transmit)' or 'to remix (to adapt)'
% it, if you '... distribute the resulting work under the same or similar
% license to this one' and if you respect how 'you must attribute the work in
% the manner specified by the author ...':
%
% In an internet based reuse please link the reused parts to www.telekom.com and
% mention the original authors and Deutsche Telekom AG in a suitable manner. In
% a paper-like reuse please insert a short hint to www.telekom.com and to the
% original authors and Deutsche Telekom AG into your preface. For normal
% quotations please use the scientific standard to cite.
%
% [ File structure derived from 'mind your Scholar Research Framework' 
%   mycsrf (c) K. Reincke CC BY 3.0  http://mycsrf.fodina.de/ ]


%\abbr[aaO]{a.a.O.}{am angegebenen Ort}
%\abbr[ds]{ds.}{kollektiv für ders., dies., \ldots}
\abbr[etseqq]{et seqq.}{and the following ones}
\abbr[id]{id.}{idem = latin for 'the same', be it a man, woman or a group\ldots}
\abbr[ibid]{ibid.}{ibidem = latin for 'at the same place'}
\abbr[ifross]{ifross}{Institut für Rechtsfragen der Freien und Open Source
Software}
\abbr[lc]{l.c.}{loco citato = latin for 'in the place cited'}
\abbr[np]{np.}{no page numbering}
\abbr[wp]{wp.}{webpage / webdocument without any internal (page)numbering}
\abbr[nst]{n.st.}{not stated}
\abbr[njear]{n.y.}{year not stated / no year}
\abbr[nlocation]{n.l.}{location not stated / no location}
\abbr[ub]{UB}{'Universitätsbibliothek' = library of university X}
\abbr[ulb]{ULB}{'Universitäts- \& Landesbibliothek' = library of university and state X}
\abbr[apl]{ApL}{Apache License}
\abbr[bsd]{BSD}{Berkeley Software Distrobution (License)}
\abbr[mit]{MIT}{Massachusetts Institute of Technology (License)}
\abbr[mspl]{Ms-PL}{Microsoft Public License}
\abbr[pgl]{PgL}{Postgres License}
\abbr[php]{PHP}{PHP (License)}
\abbr[epl]{EPL}{Eclipse Public License}
\abbr[eupl]{EUPL}{European Union Public License}
\abbr[lgpl]{LGPL}{GNU Lesser General Public License}
\abbr[mpl]{MPL}{Mozilla Public License}
\abbr[gpl]{GPL}{GNU General Public License}
\abbr[agpl]{AGPL}{GNU Affero General Public License}
\abbr[nabbr]{n.abbr.}{no abbreviation (known)}

% Local Variables:
% mode: latex
% fill-column: 80
% End:

% Telekom osCompendium English Nomenclation Tokens Include Module 
%
% (c) Karsten Reincke, Deutsche Telekom AG, Darmstadt 2011
%
% This LaTeX-File is licensed under the Creative Commons Attribution-ShareAlike
% 3.0 Germany License (http://creativecommons.org/licenses/by-sa/3.0/de/): Feel
% free 'to share (to copy, distribute and transmit)' or 'to remix (to adapt)'
% it, if you '... distribute the resulting work under the same or similar
% license to this one' and if you respect how 'you must attribute the work in
% the manner specified by the author ...':
%
% In an internet based reuse please link the reused parts to www.telekom.com and
% mention the original authors and Deutsche Telekom AG in a suitable manner. In
% a paper-like reuse please insert a short hint to www.telekom.com and to the
% original authors and Deutsche Telekom AG into your preface. For normal
% quotations please use the scientific standard to cite.
%
% [ Derived from 'mykeds Scholar Research Framework' 
%   mykeds-CSR-framework (c) K. Reincke CC BY 3.0  http://www.mykeds.net/ ]

%\abbr[]{[n.abbr.]}{ }
\abbr[zge]{ZGE / IPJ}{Zeitschrift für geistiges Eigentum [ISSN: 1867-237x]}
\abbr[itrb]{ITRB}{Der IT-Rechtsberater [ISSN: 1617-1527]}
\abbr[cri]{CRi}{Computer Law Review international [ISSN: 1610-7608]}
\abbr[btlj]{[n.abbr.]}{Berkeley Technology Law Journal}
\abbr[eclr]{E.C.L.R.}{European Competition Law Review}
\abbr[iesw]{[n.abbr.]}{IEEE Software [ISSN: 0740-7459]}
\abbr[cuitj]{[n.abbr.]}{Cutter IT Journal [ISSN: 1048-5600]}
\abbr[uoclr]{[n.abbr.]}{University of Chicago Law Review}
\abbr[uoilr]{[n.abbr.]}{University of Illinois Law Review}
\abbr[uoplr]{[n.abbr.]}{University of Pittsburgh Law Review}
\abbr[ddt]{DDT}{Drug Discovery Today [ISSN: 1359-6446]}
\abbr[rdm]{[n.abbr.]}{R\&D Management [ISSN: 1467-9310]}
\abbr[jleo]{JLEO}{Journal of Law, Economics, \& Organization [ISSN: 1465-7341]}
\abbr[ijomi]{[n.abbr.]}{International Journal of Medical Informatics [ISSN: 1386-5056]}
\abbr[slr]{[n.abbr.]}{Stanford Law Review [ISSN: 00389765]}
\abbr[bise]{BISE}{Business \& Information Systems Engineering [ISSN: 1867-0202]}
\abbr[joals]{[n.abbr.]}{Journal of Academic Librarianship [ISSN: 0099-1333]}
\abbr[eait]{[n.abbr.]}{Ethics and Information Technology [ISSN: 1388-1957]}
\abbr[jais]{JAIS}{Journal of the Association for Information Systems [ISSN:
1536-9323]}
\abbr[josas]{[n.abbr.]}{Journal of Systems and Software [ISSN: 0164-1212]}
\abbr[iialr]{[n.abbr.]}{International Information and Library Review [ISSN: 1057-2317]}
\abbr[sthv]{STHV}{Science, Technology \& Human Values [ISSN: 0162-2439]}
\abbr[cue]{[n.abbr.]}{Computers \& Education [ISSN: 0360-1315]}
\abbr[eer]{EER}{European Economic Review [ISSN: 0014-2921]}
\abbr[icc]{ICC}{Industrial and Corporate Change [ISSN: 0960-6491]}
\abbr[ca]{[n.abbr.]}{Cultural Anthropology [ISSN: 1548-1360]}
\abbr[sqj]{[n.abbr.]}{Software Qualilty Journal [ISSN: 0963-9314]}
\abbr[jmir]{JMIR}{Journal of Medical Information Research [ISSN: 1438-8871]}
\abbr[joce]{[n.abbr.]}{Journal of Comparative Economics [ISSN: 0147-5967]}
\abbr[orgsci]{[n.abbr.]}{Organization Science [ISSN: 1047-7039]}
\abbr[iam]{[n.abbr.]}{Information \& Management [ISSN: 0378-7206]}
\abbr[rp]{RP}{Research Policy [ISSN: 0048-7333]}
\abbr[jsis]{JSIS}{Journal of Strategic Information Systems [ISSN: 0963-8687]}
\abbr[isj]{ISJ}{Information Systems Journal [ISSN: 1365-2575]}
\abbr[jise]{JISE}{Journal of Information Science and Engineering [ISSN:
1016-2364]}
\abbr[dss]{DSS}{Decision Support Systems [ISSN: 0167-9236]}
\abbr[cihp]{CiHB}{Computers in Human Behavior [ISSN: 0747-5632]}
\abbr[iep]{IEaP}{Information Economics and Policy [ISSN: 0167-6245]}
\abbr[tosem]{ToSEM}{Transactions on Software Engineering Methodology [ISSN:
1049-331X]}
\abbr[commacm]{CotACM}{Communications of the ACM [ISSN: 0001-0782]}
\abbr[interactions]{[n.abbr.]}{interactions[ISSN: 1072-5520]}
\abbr[jcsc]{JCSC}{Journal of Computing Sciences in [Small] Colleges [ISSN:
1937-4771]}
\abbr[linuxjournal]{LJ}{Linux Journal [ISSN: 1075-3583]}
\abbr[networker]{[n.abbr.]}{netWorker [ISSN: 1091-3556]}
\abbr[queue]{[n.abbr.]}{Queue [ISSN: 1542-7730]}
\abbr[sigmisdb]{SIGMIS Database}{ACM SIGMIS - The Data Base for Advances in
Information Systems [ISSN: 0095-0033]}
\abbr[sigcas]{SIGCAS}{ACM SIGCAS Computers and Society [ISSN: 0095-2737]}
\abbr[sigsoft]{SIGSOFT SEN}{SIGSOFT Software Engineering Notes [ISSN:
0163-5948]}
\abbr[toit]{ToIT}{Transaction on Internet Technology [ISSN: 1533-5399]}
\abbr[sigbul]{SIGCSE Bulletin}{SIGCSE Bulletin [ISSN: 0097-8418]}
\abbr[ubiquity]{Ubiquity}{Ubiquity - The ACM IT Magazine and Forum [ISSN:
1530-2180]}
\abbr[bwv]{BWV}{Berliner Wissenschafts-Verlag GmbH}
\abbr[cr]{CR}{Computer und Recht. Zeitschrift für die Praxis des Rechts der
Informationstechnologien}

% Local Variables:
% mode: latex
% fill-column: 80
% End:

\printnomenclature

\bibliography{bibfiles/oscResourcesEn,bibfiles/oscCopiedButNotRead,bibfiles/oscNextActions}

\end{document}

% Local Variables:
% mode: latex
% fill-column: 80
% End:
